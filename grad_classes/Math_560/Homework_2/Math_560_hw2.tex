\documentclass[leqno]{article}
\usepackage[utf8]{inputenc}
\usepackage[T1]{fontenc}
\usepackage{amsfonts}
\usepackage{fourier}
\usepackage{heuristica}
\usepackage{enumerate}
\author{Colin Roberts}
\title{MATH 560, Homework 2}
\usepackage[left=3cm,right=3cm,top=3cm,bottom=3cm]{geometry}
\usepackage{amsmath}
\usepackage[thmmarks, amsmath, thref]{ntheorem}
\usepackage{kbordermatrix}
\usepackage{mathtools}

\theoremstyle{nonumberplain}
\theoremheaderfont{\itshape}
\theorembodyfont{\upshape:}
\theoremseparator{.}
\theoremsymbol{\ensuremath{\square}}
\newtheorem{proof}{Proof}
\theoremsymbol{\ensuremath{\square}}
\newtheorem{lemma}{Lemma}
\theoremsymbol{\ensuremath{\blacksquare}}
\newtheorem{solution}{Solution}
\theoremseparator{. ---}
\theoremsymbol{\mbox{\texttt{;o)}}}
\newtheorem{varsol}{Solution (variant)}

\newcommand{\tr}{\mathrm{tr}}

\begin{document}
\maketitle
\begin{large}
\begin{center}
Solutions
\end{center}
\end{large}
\pagebreak

%%%%%%%%%%%%%%%%%%%%%%%%%%%%%%%%%%%%%%%%%%%%%%%%%%%%%%%%%%%%%%%%%%%%%%%%%%%%%%%%%%%%%%%%%%%%%%%%%%%%%%%%%%%%%%%%%%%%%
%%%%%%%%%%%%%%%%%%%%%%%%%PROBLEM 1%%%%%%%%%%%%%%%%%%%%%%%%%%%%%%%%%%%%%%%%%%%%%%%%%%%%%%%%%%%%%%%%%%%%%%%%%%%%%%%%%%%%%%%%%%%%%%%%%%%%%%%%%%%%%%%%%%%%%%%%%%%%%%%%%%%%%%%%%%%%%%%%%%%%%%%%%%%%%%%%%%%%%%%%%%%%%%%%%%%%%%%%%%%%%%%%%%%%%%%%
\noindent\textbf{Problem 1.} The singular value decomposition of a real $m\times n$ matrix is written
\[
A=U\Sigma V^T
\] 
where $U^T U = I_{m\times m}$, $V^T V=I_{n\times n}$ and $\Sigma_{m\times n}$ has zero entries aside from the $n\times n$ block diagonal with entries $(\sigma_1,...,\sigma_r)$. We will assume, without loss of generality, that $m\geq n$. 

\begin{enumerate}[(a)]
\item Show exactly the structure of $\Sigma$ as a matrix, populating this matrix with the $r$ non-zero singular values.
\item Show that the left singular vectors can be found by solving an $m\times m$ eigenvector problem. Explicitly construct this problem.
\item Show that the right singular vectors can be found by solving an $n\times n$ eigenvector problem.  Explicitly construct this problem.
\item Show that these eigenvector problems are for symmetric matrices in each case.
\item Show that the left singular vectors associated with non-zero singular values may be computed in terms of $A,\Sigma$ and $V$. Write down the formula.
\end{enumerate}

\noindent\rule[0.5ex]{\linewidth}{1pt}

\begin{solution}[Part (a)]
\[
\Sigma = 
\kbordermatrix{&1 &2 &\cdots & n-1 & n\\
1 & \sigma_1 & & & \\
2 & & \ddots & & & \\
\vdots & & & \sigma_r &\\
m-1 & & & & 0 \\
m & & & & & \ddots
}
\]
With $m$ rows $n$ columns and the off diagonals all zero.
\end{solution}

\begin{solution}[Part (b)]
\begin{align*}
A&=U\Sigma V^T\\
AA^T&= U\Sigma V^T (V \Sigma^T U^T)\\
AA^T&=U(\Sigma \Sigma^T)U^T
\end{align*}
Which is an $m\times m$ eigenvalue problem. It gives us the following,
\begin{align*}
AV_i=\sigma_i U_i
\end{align*}
Where the $V_i$ and $U_i$ are the $i^{th}$ columns of the matrices. with $U_i$ being the left singular vectors and $V_i$ being the right singular vectors.
\end{solution}

\begin{solution}[Part (c)]
\begin{align*}
A&=U\Sigma V^T\\
A^TA&= (V\Sigma^T U^T) (U \Sigma V^T)\\
A^TA&=V(\Sigma^T \Sigma)V^T
\end{align*}
Which is an $n\times n$ eigenvalue problem. It gives us the following,
\begin{align*}
A^T U_i=\sigma_i V_i
\end{align*}
Where the $V_i$ and $U_i$ are the $i^{th}$ column vectors with $U_i$ being the left singular vectors and $V_i$ being the right singular vectors.
\end{solution}

\begin{solution}[Part (d)]
Since $U^T U = I_{m \times m}$ and $V^T V = I_{n \times n}$ then we have that $U$ and $V$ are symmetric matrices.
\end{solution}

\begin{solution}[Part (e)]
\begin{align*}
A&=U\Sigma V^T\\
AV &= U \Sigma\\
\frac{1}{\det(\Sigma)}\Sigma AV&=U
\end{align*}
Which allows us to find $U$ in terms of $A, \Sigma$ and $V$.
\end{solution}

\pagebreak

%%%%%%%%%%%%%%%%%%%%%%%%%%%%%%%%%%%%%%%%%%%%%%%%%%%%%%%%%%%%%%%%%%%%%%%%%%%%%%%%%%%%%%%%%%%%%%%%%%%%%%%%%%%%%%%%%%%%%
%%%%%%%%%%%%%%%%%%%%%%%%%PROBLEM 2%%%%%%%%%%%%%%%%%%%%%%%%%%%%%%%%%%%%%%%%%%%%%%%%%%%%%%%%%%%%%%%%%%%%%%%%%%%%%%%%%%%%%%%%%%%%%%%%%%%%%%%%%%%%%%%%%%%%%%%%%%%%%%%%%%%%%%%%%%%%%%%%%%%%%%%%%%%%%%%%%%%%%%%%%%%%%%%%%%%%%%%%%%%%%%%%%%%%%%%%


\noindent\textbf{Problem 2.} This problem concerns finding bases for the four fundamental subspaces in terms of the SVD of a matrix.
\begin{enumerate}[(a)]
\item Reconstruct the argument in class to find a basis for $\mathcal{R}(A)$. What is the column rank?
\item Reconstruct the argument in class to find a basis for $\mathcal{R}(A^T)$. What is the row rank?
\item Find a basis for $\mathcal{N}(A)$. Prove that is is a basis. What is the dimension of the null space?
\item Find a basis for $\mathcal{N}(A^T)$. Prove that this is a basis. What is the dimension of the left null space?
\end{enumerate} 

\noindent\rule[0.5ex]{\linewidth}{1pt}

\begin{solution}[Part (a)]
The column rank is $r$. Since we have a basis $\{v_1,...,v_n\}$ for $\mathbb{R}^n$ and $\{Av_1,...,Av_n\}$ forms the range.  But $Av_i$ for $i=r,...n$ is zero. Thus our basis for the range is $\{u_1,...,u_r\}$.
\end{solution}

\begin{solution}[Part (b)]
The column rank is $r$. Since we have a basis $\{u_1,...,u_m\}$ for $\mathbb{R}^m$ and $\{A^T u_1,...,A^T u_m\}$ forms the range.  But $Av_i$ for $i=r,...m$ is zero. Thus our basis for the range is $\{v_1,...,v_r\}$.
\end{solution}

\begin{solution}[Part (c)]
The $\dim\mathcal{N}(A)=n-r$. Then with a basis $\{v_1,...v_n\}$ for $\mathbb{R}^n$ we have that $\{v_{r+1},...,v_{n}\}$ is the basis for $\mathcal{N}(A)$ by the argument in part (a).
\end{solution}

\begin{solution}[Part (d)]
The $\dim\mathcal{N}(A)=m-r$. Then with a basis $\{u_1,...u_m\}$ for $\mathbb{R}^m$ we have that $\{u_{r+1},...,v_{m}\}$ is the basis for $\mathcal{N}(A^T)$ by the argument in part (b).
\end{solution}

\pagebreak


%%%%%%%%%%%%%%%%%%%%%%%%%%%%%%%%%%%%%%%%%%%%%%%%%%%%%%%%%%%%%%%%%%%%%%%%%%%%%%%%%%%%%%%%%%%%%%%%%%%%%%%%%%%%%%%%%%%%%
%%%%%%%%%%%%%%%%%%%%%%%%%PROBLEM 3%%%%%%%%%%%%%%%%%%%%%%%%%%%%%%%%%%%%%%%%%%%%%%%%%%%%%%%%%%%%%%%%%%%%%%%%%%%%%%%%%%%%%%%%%%%%%%%%%%%%%%%%%%%%%%%%%%%%%%%%%%%%%%%%%%%%%%%%%%%%%%%%%%%%%%%%%%%%%%%%%%%%%%%%%%%%%%%%%%%%%%%%%%%%%%%%%%%%%%%%


\noindent\textbf{\S 1.6 Problem 35.} Let $W$ be a subspace of a finite-dimensional vector space $V$, and consider the basis $\{u_1,u_2,...,u_k\}$ for $W$. Let $\{u_1,...,u_k,u_{k+1}, ..., u_n\}$ be an extension of this basis to a basis for $V$.
\begin{enumerate}[(a)]
\item Prove that $\{u_{k+1}+W,u_{k+2}+W,...,u_n+W\}$ is a basis for $V/W$.
\item Derive a formula relating $\dim(V)$, $\dim(W)$, and $\dim(V/W)$.
\end{enumerate}

\noindent\rule[0.5ex]{\linewidth}{1pt}

\begin{proof}[Part (a)]
Consider the following,
\[
a_{k+1}(u_{k+1}+W)+...+a_n(u_n+W)=0+W.
\]
Which implies $a_{k+1}u_{k+1}+...+a_n u_n=0$. But these vectors linearly independent, thus we would have that each $a_i$ is 0. Finally, consider $x+W \in V/W$ be arbitrary and we have that $x=a_1 u_1+...+a_n u_n$ so that $x+W=(a_1 u_1 + ... + a_n u_n)+W=(a_{k+1}u_{k+1}...a_n u_n + W$. Thus any arbitrary element is in the span of these linearly independent vectors.  So we have $\{u_{k+1}+W,u_{k+2}+W,...,u_n+W\}$ is a basis.
\end{proof}

\begin{proof}[Part (b)]
We have that $\dim(V)=n$, $\dim(W)=k$ and we know that $V/W=\textrm{span}\{u_{k+1}+W,...,u_n+W\}$ Thus we have that
\[
\dim(V/W)=\dim(V)-\dim(W).
\]
\end{proof}

\pagebreak



%%%%%%%%%%%%%%%%%%%%%%%%%%%%%%%%%%%%%%%%%%%%%%%%%%%%%%%%%%%%%%%%%%%%%%%%%%%%%%%%%%%%%%%%%%%%%%%%%%%%%%%%%%%%%%%%%%%%%
%%%%%%%%%%%%%%%%%%%%%%%%%PROBLEM 4%%%%%%%%%%%%%%%%%%%%%%%%%%%%%%%%%%%%%%%%%%%%%%%%%%%%%%%%%%%%%%%%%%%%%%%%%%%%%%%%%%%%%%%%%%%%%%%%%%%%%%%%%%%%%%%%%%%%%%%%%%%%%%%%%%%%%%%%%%%%%%%%%%%%%%%%%%%%%%%%%%%%%%%%%%%%%%%%%%%%%%%%%%%%%%%%%%%%%%%%


\noindent\textbf{\S 2.1 Problem 3.}  $T\colon \mathbb{R}^2 \to \mathbb{R}^3$ defined by $T(a_1,a_2)=(a_1+a_2,0,2a_1-a_2)$. Prove that $T$ is linear and find bases for both $\mathcal{N}(T)$ and $\mathcal{R}(T)$. The compute the nullity and rank of $T$, and verify the dimension theorem. Finally, use the appropriate theorems in this section to determine whether $T$ is injective or surjective.

\noindent\rule[0.5ex]{\linewidth}{1pt}

\begin{proof}
Consider
\begin{align*}
T(a(x_1,x_2)+(y_1,y_2))&=T(ax_1+y_1,ax_2+y_2)\\
&=(ax_1+x_2+y_1+y_2,0,2ax_1+2y_1-2ax_2-y_2)\\
&=(a(x_1+x_2)+(y_1+y_2),0,a(2x_1-x_2)+(2y_1-y_2))\\
&=aT(x_1,x_2)+T(y_1,y_2)
\end{align*}
So $T$ is linear. To find the basis for $\mathcal{N}(T)$ we find what elements are mapped to the zero vector. Thus we need to satisfy 
\begin{align*}
a_1+a_2&=0\\
2a_1-a_2=0
\end{align*}
Which implies that $a_1=a_2=0$. So the basis for $\mathcal{N}(T)$ is $\{0\}$. A basis for $\mathcal{R}(T)$ is given by $\{(1,0,0),(0,0,1)\}$. $\textrm{nullity}(T)=0$, $\textrm{rank}(T)=2$ and we have $\dim(V)=2=\textrm{nullity}(T)+\textrm{rank}(T)=0+2$.  Since $\textrm{nullity}(T)=0$ we have that $T$ is injective.  But since $\dim(\mathbb{R}^3)>\textrm{rank}(T)$ we have that $T$ is not surjective.
\end{proof}

\pagebreak


%%%%%%%%%%%%%%%%%%%%%%%%%%%%%%%%%%%%%%%%%%%%%%%%%%%%%%%%%%%%%%%%%%%%%%%%%%%%%%%%%%%%%%%%%%%%%%%%%%%%%%%%%%%%%%%%%%%%%
%%%%%%%%%%%%%%%%%%%%%%%%%PROBLEM 5%%%%%%%%%%%%%%%%%%%%%%%%%%%%%%%%%%%%%%%%%%%%%%%%%%%%%%%%%%%%%%%%%%%%%%%%%%%%%%%%%%%%%%%%%%%%%%%%%%%%%%%%%%%%%%%%%%%%%%%%%%%%%%%%%%%%%%%%%%%%%%%%%%%%%%%%%%%%%%%%%%%%%%%%%%%%%%%%%%%%%%%%%%%%%%%%%%%%%%%%


\noindent\textbf{\S 2.1 Problem 4.} $T\colon M_{2\times 3}(F) \to M_{2\times 2}(F)$ defined by 
\[T
\begin{bmatrix}
a_{11} & a_{12} & a_{13}\\
a_{21} & a_{22} & a_{23}
\end{bmatrix}
=
\begin{bmatrix}
2a_{11}-a_{12} & a_{13}+2a_{12}\\
0 & 0
\end{bmatrix}
.\] Prove that $T$ is linear and find bases for both $\mathcal{N}(T)$ and $\mathcal{R}(T)$. The compute the nullity and rank of $T$, and verify the dimension theorem. Finally, use the appropriate theorems in this section to determine whether $T$ is injective or surjective.

\noindent\rule[0.5ex]{\linewidth}{1pt}

\begin{proof}
To show that $T$ is linear, we want to show $T(aA+B)=aT(A)+T(B)$.  So we have,
\begin{align*}
T(aA+B)&=T\left(
\begin{bmatrix}
aA_{11}+B_{11} & aA_{12}+B_{12} & aA_{13}+B_{13}\\
aA_{21}+B_{21} & aA_{22}+B_{22} & aA_{23}+B_{23}
\end{bmatrix}
\right)\\
&=
\begin{bmatrix}
2aA_{11}+B_{12}-aA_{12}-B_{12} & aA_{13}+B_{13}+2aA_{12}+2B_{12}\\
0 & 0
\end{bmatrix}\\
&=a
\begin{bmatrix}
2A_{11}-A_{12} & A_{13}+2A_{12}\\
0 & 0
\end{bmatrix}+
\begin{bmatrix}
B_{12}-B_{12} & B_{13}+2B_{12}\\
0 & 0
\end{bmatrix}\\
&=aT(A)+T(B)
\end{align*}
So $T$ is linear.  A basis for $\mathcal{N}(T)$ is given by \[\left\{
\begin{bmatrix}
0 & 0 & 0\\
1 & 0 & 0
\end{bmatrix},
\begin{bmatrix}
0 & 0 & 0\\
0 & 1 & 0
\end{bmatrix},
\begin{bmatrix}
0 & 0 & 0\\
0 & 0 & 1
\end{bmatrix},
\begin{bmatrix}
2 & 1 & -2\\
0 & 0 & 0
\end{bmatrix}
\right\}
\]
A basis for $\mathcal{R}(T)$ is given by
\[\left\{
\begin{bmatrix}
1 & 0\\
0 & 0
\end{bmatrix},
\begin{bmatrix}
0 & 1\\
0 & 0
\end{bmatrix}
\right\}
\]
Then we have $\textrm{nullity}(T)=4$ and $\textrm{rank}(T)=2$.  And $\dim(M_{2\times 3}(\mathbb{F}))=6=\textrm{nullity}(T)+\textrm{rank}(T)=4+2$.  $T$ is not injective since $\textrm{nullity}(T)\neq 0$ and not surjective since $\textrm{rank}(T)<\dim(M_{2\times 2}(\mathbb{F}))$.
\end{proof}


\pagebreak


%%%%%%%%%%%%%%%%%%%%%%%%%%%%%%%%%%%%%%%%%%%%%%%%%%%%%%%%%%%%%%%%%%%%%%%%%%%%%%%%%%%%%%%%%%%%%%%%%%%%%%%%%%%%%%%%%%%%%
%%%%%%%%%%%%%%%%%%%%%%%%%PROBLEM 6%%%%%%%%%%%%%%%%%%%%%%%%%%%%%%%%%%%%%%%%%%%%%%%%%%%%%%%%%%%%%%%%%%%%%%%%%%%%%%%%%%%%%%%%%%%%%%%%%%%%%%%%%%%%%%%%%%%%%%%%%%%%%%%%%%%%%%%%%%%%%%%%%%%%%%%%%%%%%%%%%%%%%%%%%%%%%%%%%%%%%%%%%%%%%%%%%%%%%%%%


\noindent\textbf{\S 2.1 Problem 11.} Prove that there exists a linear transformation $T\colon \mathbb{R}^2 \to \mathbb{R}^3$ such that $T(1,1)=(1,0,2)$ and $T(2,3)=(1,-1,4)$. What is $T(8,11)$?

\noindent\rule[0.5ex]{\linewidth}{1pt}

\begin{solution}
\begin{align*}
(8,11)&=a(1,1)+b(2,3)\\
\implies a=2&,b=3
\end{align*}
Thus we have
\begin{align*}
T(8,11)&=2T(1,1)+3T(2,3)\\
&=2(1,0,2)+3(1,-1,4)\\
&=(5,-3,16)
\end{align*}
\end{solution}
\pagebreak


%%%%%%%%%%%%%%%%%%%%%%%%%%%%%%%%%%%%%%%%%%%%%%%%%%%%%%%%%%%%%%%%%%%%%%%%%%%%%%%%%%%%%%%%%%%%%%%%%%%%%%%%%%%%%%%%%%%%%
%%%%%%%%%%%%%%%%%%%%%%%%%PROBLEM 7%%%%%%%%%%%%%%%%%%%%%%%%%%%%%%%%%%%%%%%%%%%%%%%%%%%%%%%%%%%%%%%%%%%%%%%%%%%%%%%%%%%%%%%%%%%%%%%%%%%%%%%%%%%%%%%%%%%%%%%%%%%%%%%%%%%%%%%%%%%%%%%%%%%%%%%%%%%%%%%%%%%%%%%%%%%%%%%%%%%%%%%%%%%%%%%%%%%%%%%%


\noindent\textbf{\S 2.1 Problem 15.} Recall the definition of $P(\mathbb{R})$ on page 10. Define
\[T\colon P(\mathbb{R})\to P(\mathbb{R}) \textrm{~ by ~} T(f(x))=\int_0^x f(t)dt.
\]
Prove that $T$ is linear and injective, but not surjective.

\noindent\rule[0.5ex]{\linewidth}{1pt}

\begin{proof}
To show that $T$ is linear we show that $T(af(x)+g(x))=aT(f(x))+T(g(x))$. So
\begin{align*}
T(af(x)+g(x))&=\int_0^x(af(t)+g(t)dt\\
&=a\int_0^xf(t)dt+\int_0^x g(t)dt
&=aT(f(x))+T(g(x))
\end{align*}
by properties of integrals. 

Suppose that $f(x)=a_0+a_1x+...+a_nx^n\in \mathcal{N}(T)$. Thus
\begin{align*}
T(f(x))&=\int_0^x (a_0+...+a_nx^n)dt\\
&=a_0 \int_0^x 1 dt + ... + a_n \int_0^x x^n dt
\end{align*}
Thus since no integrand evaluates to $0$, we have that $a_i=0$ $\forall i$. So $\mathcal{N}(T)=\{0\}$. So $T$ is injective.

Consider $c\in P(\mathbb{R})$. Then let $f(x)=a_0+a_1x+...+a_nx^n \in P(\mathbb{R})$ Thus
\begin{align*}
c=T(f(x))&=\int_0^x (a_0+...+a_nx^n)dt\\
&=a_0 \int_0^x 1 dt + ... + a_n \int_0^x x^n dt\\
c&=a_0x+...+a_nx^{n+1}
\end{align*}
Which has no solution.  Thus $T$ is not surjective since there exists an element of $P(\mathbb{R})$ not in $\mathcal{R}(T)$.
\end{proof}

\pagebreak


%%%%%%%%%%%%%%%%%%%%%%%%%%%%%%%%%%%%%%%%%%%%%%%%%%%%%%%%%%%%%%%%%%%%%%%%%%%%%%%%%%%%%%%%%%%%%%%%%%%%%%%%%%%%%%%%%%%%%
%%%%%%%%%%%%%%%%%%%%%%%%%PROBLEM %%%%%%%%%%%%%%%%%%%%%%%%%%%%%%%%%%%%%%%%%%%%%%%%%%%%%%%%%%%%%%%%%%%%%%%%%%%%%%%%%%%%%%%%%%%%%%%%%%%%%%%%%%%%%%%%%%%%%%%%%%%%%%%%%%%%%%%%%%%%%%%%%%%%%%%%%%%%%%%%%%%%%%%%%%%%%%%%%%%%%%%%%%%%%%%%%%%%%%%%


\noindent\textbf{\S 2.1 Problem 17.} Let $V$ and $W$ be finite-dimensional vector spaces and $T\colon V \to W$ be linear.
\begin{enumerate}[(a)]
\item Prove that if $\dim(V)<\dim(W)$, then $T$ cannot be surjective.
\item Prove that if $\dim(V)>\dim(W)$, then $T$ cannot be injective.
\end{enumerate}

\noindent\rule[0.5ex]{\linewidth}{1pt}

\begin{proof}[Part (a)]
We have that
\[
\dim(W)>\dim(V)\geq \textrm{rank}(T)
\]
Since $\textrm{rank}(T)$ is less than $\dim(W)$, $T$ is not surjective.
\end{proof}

\begin{proof}[Part (b)]
We have that
\[
\textrm{rank}(T)\leq \dim(W)<\dim(V)
\]
So we have
\[
\dim(V)-\textrm{rank}(T)>0
\]
Which means that $\textrm{nullity}(T)>0$ by the dimension theorem. This means that $T$ is not injective.
\end{proof}

\pagebreak


%%%%%%%%%%%%%%%%%%%%%%%%%%%%%%%%%%%%%%%%%%%%%%%%%%%%%%%%%%%%%%%%%%%%%%%%%%%%%%%%%%%%%%%%%%%%%%%%%%%%%%%%%%%%%%%%%%%%%
%%%%%%%%%%%%%%%%%%%%%%%%%PROBLEM %%%%%%%%%%%%%%%%%%%%%%%%%%%%%%%%%%%%%%%%%%%%%%%%%%%%%%%%%%%%%%%%%%%%%%%%%%%%%%%%%%%%%%%%%%%%%%%%%%%%%%%%%%%%%%%%%%%%%%%%%%%%%%%%%%%%%%%%%%%%%%%%%%%%%%%%%%%%%%%%%%%%%%%%%%%%%%%%%%%%%%%%%%%%%%%%%%%%%%%%


\noindent\textbf{\S 2.1 Problem 35.} Let $V$ be a finite-dimensional vector space and $T\colon V \to V$ be linear.
\begin{enumerate}[(a)]
\item Suppose that $V=\mathcal{R}(T)+\mathcal{N}(T)$. Prove that $V=\mathcal{R}(T)\oplus \mathcal{N}(T)$.
\item Suppose that $\mathcal{R}(T)\cap \mathcal{N}(T)=\{0\}$. Prove that $V=\mathcal{R}(T)\oplus \mathcal{N}(T)$.
\end{enumerate}

\noindent\rule[0.5ex]{\linewidth}{1pt}

\begin{proof}[Part (a)]
Suppose that $V=\mathcal{R}+\mathcal{N}(T)$ and that we have $v\in \mathcal{R}(T)\cap \mathcal{N}(T)$.  Then we have $T(v)=0$ since $v\in \mathcal{N}(T)$, which means that $v=0$ since $v\in \mathcal{R}(T)$. Thus $\mathcal{R}(T)\cap\mathcal{N}(T)=\{0\}$ and thus $V=\mathcal{R}(T)\oplus \mathcal{N}(T)$.
\end{proof}

\begin{proof}[Part (b)]
Suppose that $\mathcal{R}(T)\cap \mathcal{N}(T)=\{0\}$. Suppose we have $v\in V$ so that $T(v)\notin \mathcal{R}(T)+\mathcal{N}(T)$. Thus we know that $T(v)\neq 0$ since $0\in \mathcal{R}(T)+\mathcal{N}(T)$. But then if $T(v)\neq 0$ then $T(v)\in \mathcal{R}(T)$ and we contradict $T(v)\notin \mathcal{R}(T)+\mathcal{N}(T)$. So $V=\mathcal{N}(T)\oplus \mathcal{R}(T)$.
\end{proof}

\pagebreak


%%%%%%%%%%%%%%%%%%%%%%%%%%%%%%%%%%%%%%%%%%%%%%%%%%%%%%%%%%%%%%%%%%%%%%%%%%%%%%%%%%%%%%%%%%%%%%%%%%%%%%%%%%%%%%%%%%%%%
%%%%%%%%%%%%%%%%%%%%%%%%%PROBLEM %%%%%%%%%%%%%%%%%%%%%%%%%%%%%%%%%%%%%%%%%%%%%%%%%%%%%%%%%%%%%%%%%%%%%%%%%%%%%%%%%%%%%%%%%%%%%%%%%%%%%%%%%%%%%%%%%%%%%%%%%%%%%%%%%%%%%%%%%%%%%%%%%%%%%%%%%%%%%%%%%%%%%%%%%%%%%%%%%%%%%%%%%%%%%%%%%%%%%%%%


\noindent\textbf{\S 2.1 Problem 40.} Let $V$ be a vector space and $W$ be a subspace of $V$. Define the mapping $\eta \colon V \to V/W$ by $\eta(v)=v+W$ for $v\in V$.
\begin{enumerate}[(a)]
\item Prove that $\eta$ is a linear transformation from $V$ onto $V/W$ and that $\mathcal{N}(\eta)=W$.
\item Suppose that $V$ is finite-dimensional. Use (a) and the dimension theorem to derive a formula relating $\dim(V)$, $\dim(W)$, and $\dim(V/W)$.
\item Read the proof of the dimension theorem.  Compare the method of solving (b) with the method of deriving the same result as outlined in Exercise 35 of Section 1.6.
\end{enumerate}


\noindent\rule[0.5ex]{\linewidth}{1pt}

\begin{proof}[Part (a)]
Let $u,v\in V$ and $a\in \mathbb{F}$. Then
\begin{align*}
\eta(av+u)&=(av+u)+W\\
&=(av+W)+(u+W)\\
&=a(v+W)+(u+W)\\
&=a\eta(v)+\eta(u)
\end{align*}
So $\eta$ is linear. Then let $v+W\in V/W$ be arbitrary and note that $\eta(v)=v+W$ for $v\in V$ and thus $\eta$ is surjective.
\end{proof}

\begin{proof}[Part (b)]
We have
\begin{align*}
\dim(V)&=\dim(\mathcal{R}(\eta))+\dim(\mathcal{N}(\eta))\\
&=\dim(V/W)+\dim(W) &\textrm{since $\eta$ is onto}\\
\implies \dim(V/W)&=\dim(V)-\dim(W)
\end{align*}
\end{proof}

\begin{solution}[Part (c)]
(b) uses an onto linear transformation to allow us to utilize the dimension theorem. But Ex. 35 of \S 1.6 uses an argument which involves constructing bases for $\mathcal{R}(T)$ and $\mathcal{N}(T)$.
\end{solution}

\pagebreak



\end{document}

