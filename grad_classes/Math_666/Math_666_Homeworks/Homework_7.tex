\documentclass[leqno]{article}
\author{Colin Roberts}
\usepackage{preamble}

\begin{document}

\begin{center}
  \begin{huge}
    MATH 666: Advanced Algebra I
  \end{huge}
\end{center}

\section*{Homework assignment 7 -- due 10/24/2019}
\noindent \emph{I worked with Brittany, Shannon, and Brenden.}
\setcounter{problem}{28}

\begin{problem}
Let $V$ be a $\C G$-module with corresponding character $\chi$. According to problem 14, we can decompose $V\otimes V = W_S \oplus W_A$. Let $\chi_S$ (respectively $\chi_A$) be the characters that belong to $W_S$ (respectively $W_A$). Show that
\[
\chi_S(g) = \frac{1}{2} \left( \chi(g)^2+\chi(g^2)\right) \qquad \textrm{and} \qquad \chi_A(g)=\frac{1}{2}\left( \chi(g)^2-\chi(g^2)\right).
\]
\end{problem}
\begin{solution}
Since we are over $\C$, we can diagonalize each matrix in the rerpesentation and so we can say $\varphi(g)$ is diagonal. This gives us the eigenbasis $\{v_1,\dots,v_n\}$ for $V$ with corresponding eigenvalues $\{\lambda_1,\dots,\lambda_n\}$. Then a basis for $W_S$ is given by $\{v_i \otimes v_j + v_j \otimes v_i\}_{i\leq j}$ and similarly a basis for $W_A$ is given by $\{v_i \otimes v_j - v_j \otimes v_i\}_{i<j}$. Then
\begin{align*}
    \frac{1}{2}\left( \chi(g)^2+\chi(g^2)\right)&= \frac{1}{2} \left( \left( \sum_{i} \lambda_i \right)^2 + \sum_{j} \lambda_j^2 \right)\\
    &= \frac{1}{2} \cdot 2 \sum_{i\leq j} \lambda_i \lambda j\\
    &= \sum_{i\leq j}\lambda_i \lambda_j\\
    &= \chi_S(g).
\end{align*}
We also have,
\begin{align*}
    \frac{1}{2}\left( \chi(g)^2-\chi(g^2)\right)&= \frac{1}{2} \left( \left( \sum_{i} \lambda_i \right)^2 - \sum_{j} \lambda_j^2 \right)\\
    &= \frac{1}{2} \cdot 2 \sum_{i<j} \lambda_i \lambda j\\
    &= \sum_{i< j}\lambda_i \lambda_j\\
    &= \chi_S(g).
\end{align*}
\end{solution}


\newpage
\begin{problem} 
Show that the row sums in a character table are nonnegative integers.
\end{problem}
\begin{solution}
If we consider $\chi$ to be the permutation character for the conjugation action and $\psi$ another irreducible character we have
\[
\langle \psi, \chi \rangle = \sum_{\textrm{representatives}} \frac{1}{|C_G(g)|} \psi(g) \cdot \overline{\chi(g)},
\]
where the sum is over representatives of conjugacy classes.  Since $\chi$ is the permutation character, we have that $\chi(g)=|C_G(g)|$ since the trace of the representation of $g$ is the number of elements fixed by $g$. Hence, we have
\[
\langle \psi,\chi\rangle = \sum_{\textrm{representatives}} \psi(g)
\]
which is the row sum of $\psi$.  Then note that for any characters we have
\[
\langle \chi_1, \chi_2\rangle = \dim(\hom(V,W)),
\]
and so the row sums must be positive.
\end{solution}

\newpage
\begin{problem}
Let $G$ be a finite group.
\begin{enumerate}[(a)]
    \item Let $\chi$ be a character of $G$ associated to the representation $\delta$ and $g\in G$ with $|g|=2$. Show that $\chi(g)\in \Z$ and that $\chi(g)\equiv\chi(1) (\textrm{mod}2)$.
    \item Show that simple groups cannot have irreducible $\C$-representations of degree 2. (Hint: Assume that $\delta$ is irreducible of degree 2 and $|g|=2$, conclude that $g^\delta\in Z(G^\delta)$.)
\end{enumerate}
\end{problem}
\begin{solution}~
\begin{enumerate}[(a)]
    \item By Problem 33 we know that $\chi(g)$ is a sum of second roots of unity which are $1$ and $-1$ and so $\chi(g)\in \Z$.  Again, by Problem 33 we have $\chi(g^2)\equiv \chi(g)^2 ~(\textrm{mod}~p)$ and we have that $g^2=1$ and thus $\chi(g)\equiv \chi(1) ~(\textrm{mod}~p)$. 
    \item For a contradiction, let $\delta$ be an irreducible representation of degree 2 with an element $g$ such that $|g|=2$. The fact that this representation is not of degree 1 means that $G$ must not be abelian and so we can consider $G/G'$. By supposition $G$ is simple, and so $G'$ is trivial or $G'=G$ and since we have that $G$ is abelian, $G'=G$ is the only option.  Thus we have that the number of characters is equal to $[G:G']=1$ and so $g^\delta = -I_{2\times 2}$ since $g$ has order two and thus $g^{\delta}\in Z(G^\delta)$. This contradicts $G$ being simple as $\langle g\rangle$ generates a nontrivial subgroup of $G$.
\end{enumerate}
\end{solution}

\newpage
\begin{problem}
The character table of the symmetric group $S_5$ is given as
\[
    \begin{tabular}{c|ccccccc}
        ~ & 1a & 2a & 3a & 5a & 2b & 4a & 6a \\
        \hline
        $\chi_1$ & 1 & 1 & 1 & 1 & 1 & 1 & 1\\
        $\chi_2$ & 1 & 1 & 1 & 1 & -1 & -1 & -1\\
        $\chi_3$ & 6 & -2 & . & 1 & . & . & .\\
        $\chi_4$ & 4 & . & 1 & -1 & 2 &. & 1\\
        $\chi_5$ & 4 & . & 1 & -1 & -2 & . & 1\\
        $\chi_6$ & 5 & 1 & -1 & . & 1 & -1 & 1\\
        $\chi_7$ & 5 & 1 & -1 & . & -1 & 1 & -1
    \end{tabular}
\]
where dot (.) indicates a zero entry and the class names give the element orders, the letters are assigned to distinguish classes with the same element order.
\begin{enumerate}[(a)]
    \item Determine representatives for the conjugacy classes (for example from element orders and centralizer orders - the latter are obtainable via the 2nd orthogonality relation).
    \item Write the permutation character $\chi_\pi$ for the natural permutation representation of $S_5$ as a sum of irreducible characters. \textbf{Hint:} The \GAP commands
    \begin{verbatim}
        c:= CharacterTable("S5"); mat:=List(Irr(c), i->List(i,j->j);
    \end{verbatim}
    can be used to obtain a matrix with the character values so you do not need to solve linear equations by hand.
    \item Let $\tau$ be the tensor product of $\chi_5$ with $\chi_6$. Write $\tau$ as a sum of irreducible characters.
    \item Let $\phi$ be the permutation representation of $S_5$ on the cosets of $A_4$ (of degree 10). Write the corresponding permutation character as a sum of irreducible characters. \textbf{Hint:} The coset $A_4h$ remains fixed under $g$ if $hgh^{-1}\in A_4$. As the \GAP  command \begin{verbatim}
        RightTransversal
    \end{verbatim}
    can be used to obtain a set of representatives for the cosets, the following command thus determines representatives of those cosets, which remain fixed under the group element $g$.
    \begin{verbatim}
        Filtered(RightTransversal(s5,a4),h->h*g/h in a4);
    \end{verbatim}
\end{enumerate}
\end{problem}
\begin{solution}~
\begin{enumerate}[(a)]
    \item We can break up $S_5$ into the cycles of different type. There is the trivial cycle given by $()$ corresponding to $1a$, a three cycle $(123)$ corresponding to $3a$, a four cycle $(1234)$ corresponding to $4a$, a five cycle $(12345)$ corresponding to $5a$, and a six cycle corresponding to, for example, $(12)(1234)$. Then, there are two types of two cycles, we have $2a$ can be given by $(12)(34)$ which is an even signed permutation, and $2b$ given by $(12)$ which is an odd signed permutation which agrees with the character table.
    
    \item Then we use the above representatives and take the trace of their permutation matrix to find that we get $\chi_\pi(())=5$, $\chi_\pi((12)(34))=1$, $\chi_\pi((123))=2$, $\chi((12345)=0$, $\chi_\pi((12)(34))=3$, $\chi_\pi((1234))=1$, and $\chi_\pi((12)(1234))=0$, which we can know by knowing how many elements are fixed by the permutation. Then we can get these results for $\chi_\pi$ by noting $\chi_\pi = \chi_1 + \chi_4$.
    
    \item We have $\tau(g)=\chi_5(g)\cdot \chi_6(g)$ and so in this table we would have 
    \[
    \begin{tabular}{c|cccccccc}
        ~ & 1a & 2a & 3a & 5a & 2b & 4a & 6a  \\
        \hline
        $\tau$ & 20 & 0 & -1 & 0 & -2 & 0  & 1 
    \end{tabular}.
    \]
    We can solve this as a linear system to get $\tau = \chi_3 + \chi_5 + \chi_6 + \chi_7$.
\end{enumerate}
\end{solution}


\newpage
\begin{problem}
Let $G$ be a finite group and $\chi$ a character of $G$. Let $g\in G$ and $p$ be a prime. Show: If $\chi$ is rational (i.e. has only rational values), then $\chi(g^p)\equiv \chi(g)^p (\textrm{mod}~p)$. \textbf{Hint:} Show that $\chi(g^p)-\chi(g)^p=p\cdot a$ for an algebraic integer $a$.
\end{problem}
\begin{solution}
    Suppose that $|g|=d$ then $\chi(g)$ is the sum of $d$th roots of unity.  Suppose that $\chi(g)=\lambda_1 + \cdots + \lambda_n$ then
    \begin{align*}
        \chi(g^p)-\chi(g)^p &= \sum_{i} \lambda_i^p - \left( \sum_{i} \lambda_i\right)^p\\
        &= \sum_{i} \lambda_i^p - \sum_{\sum k_t = p} \frac{p!}{k_1!\cdot k_m!} \prod_{1\leq t \leq m} \lambda_t^{k_t}\\
        &\equiv 0 \textrm{~ mod}~ p.
    \end{align*}
\end{solution}



\end{document}