\documentclass[leqno]{article}
\author{Colin Roberts}
\usepackage{preamble}

\begin{document}

\begin{center}
  \begin{huge}
    MATH 666: Advanced Algebra I
  \end{huge}
\end{center}

\section*{Homework assignment 3 -- due 9/20/2019}
\noindent \emph{I worked with Brittany and Brenden.  Brenden also helped me a lot with the GAP code.}
\setcounter{problem}{9}
\begin{problem}
Let $G$ be a finite group and $K$ a field. What is the relation between the 1-dimensional $K$-representations of $G$, and those of $G/G'$. Describe these representations for the case of $G/G'$ cyclic.
\end{problem}
\begin{solution}
We let $G'=\{[g,h]~\vert~ g,h\in G\}$.  Let $\rho \colon G \to K$ and $\rho'\colon G/G' \to K$ be 1-dimensional representations. We also have the projection map $\pi \colon G \to G/G'$ so that $\im \pi = G/G'$ and $\ker \pi = G'$ by letting $\pi(g)=g+G'$. 

Note that if we have $\ker \rho = G'$, then we have $\rho'=\rho \circ \pi$.  If $G/G'$ is cyclic, then the image of $\rho'$ is cyclic and hence the image of $\rho'$ must be the roots of unity in $K$.
\end{solution}

\newpage
\begin{problem}
(\textsf{GAP}) 
\begin{enumerate}
    \item Let $G=A_5$ and $F=\mathbb{F}_9=GF(9)$. Determine with the MeatAxe (but without using \textsf{MTX.CollectedFactors} command) the irreducible $FG$-modules of dimension 3 up to isomorphism. 
    \item Also determine (you may use the \textsf{MTX.CollectedFactors} command here) the irreducible $F_4G$ modules of dimension 3 up to isomorphism. Compare with the result of (a).
\end{enumerate}
\end{problem}
\begin{solution}
\begin{enumerate}[(a)]
    \item Here is \textsf{GAP} code.
    \begin{verbatim}
    G := AlternatingGroup(5);;
F := GF(9);;
A := PermutationGModule(G,F);;
a1 := MTX.Generators(A)[1];;
a2 := MTX.Generators(A)[2];;
omega := a1-a2;;
#omega := a1*a2-a2;;
nr := NullspaceMat(omega);;
y := nr[1];
Rank([y, y*a1, y*a1^2, y*a1^3, y*a1^4]);

nc := NullspaceMat(TransposedMat(omega));;
x := nc[1];;
Rank([x, x*a1, x*a1^2, x*a1^3, x*a1^4]);

#according to my findings, we have irreducible representations of dimensions 1 and 4, none of dimensions w3.
\end{verbatim}

\item And for (b),
\begin{verbatim}
MTX.CollectedFactors(A);
\end{verbatim}
which agrees with (a).
\end{enumerate}
\end{solution}


\newpage
\begin{problem}
Let $G=A_4$ and $K=\finitef_2$. Let $\varphi$ be the 3-dimensinonal representation of $A_4$ on $\finitef_2^4/S$ from problem 3. Calculate matrices for generators of $A_4$ under $\varphi \otimes \varphi$, and determine the irreducible constituents of this representation.
\end{problem}
\begin{solution}
Here is some \textsf{GAP} code.
\begin{verbatim}
A := PermutationGModule(AlternatingGroup(4),GF(2));;
bas := MTX.BasesSubmodules(A)[2];;
A2 := MTX.InducedActionFactorModule(A,bas);; 
MTX.Dimension(A2);
a1 := MTX.Generators(A2)[1];;
a2 := MTX.Generators(A2)[2];;
#Now the tensor product and its generators
AA := TensorProductGModule(A2,A2);
aa1 := MTX.Generators(AA)[1];; Display(aa1);
aa2 := MTX.Generators(AA)[2];; Display(aa2);
#note that aa1 is the kronecker product of a1 with itself, and similarly for aa2.
#Use the meataxe to find the irreducible constituents of AA
Display(MTX.CollectedFactors(AA));
\end{verbatim}
\end{solution}

\setcounter{problem}{13}
\newpage
\begin{problem}
Let $G$ be a finite group and $K$ a field with $\Char K\neq 2$. Consider a $KG$ module $V$ with basis $\boldv_1,\dots,\boldv_n$ and let $W=V\otimes V$. We define a linear map $\alpha \colon W \to W$ by prescribing images of the basis vectors as $\alpha \boldv_i\otimes \boldv_j\mapsto \boldv_j\otimes \boldv_i$ and set $W_S = \{w\in W ~\vert~ w^\alpha = w\}$ and $W_A=\{w\in W~\vert~ w^\alpha = - w\}$. (These are called the \emph{symmetric}, respectively \emph{antisymmetric} tensors.) Show:
\begin{enumerate}[(a)]
    \item $W=W_S \oplus_K W_A$ (direct sum of vector spaces.)
    \item Both $W_S$ and $W_A$ are $KG$-submodules of $W$. (Thus $W=W_S \oplus_{KG} W_A$ as well.)
    \item $\dim(W_S)=\frac{1}{2}n(n+1)$ and $\dim(W_A)=\frac{1}{2}n(n-1)$.
\end{enumerate}
\end{problem}
\begin{solution}~
\begin{enumerate}[(a)]
    \item First, we can show that these are indeed subspaces of $W$. Let $\omega, \eta \in W_A$ and $\lambda,\mu \in K$ and consider
    \begin{align*}
        (\lambda \omega + \mu \eta)^\alpha &= \lambda \omega^\alpha + \mu \eta^\alpha\\
        &= -\lambda \omega - \mu \eta\\
        &= -(\lambda \omega + \mu \eta),
    \end{align*}
    meaning that $W_A$ is a subspace.  Then take $g,h \in W_S$ and $\lambda,\mu \in K$ and consider
    \begin{align*}
        (\lambda g+ \mu h)^\alpha &= \lambda g^\alpha + \mu h^\alpha \\
        &= \lambda g + \mu g,
    \end{align*}
    hence $W_S$ is a subspace.
    
    Next, we can show that $W_S\cap W_A=\{0\}$. To see this, note that if we have $w\in W_S\cap W_A$ then $w^\alpha = w = -w$.  Since $w=-w$ it must be that $w=0$ and hence $W_S\cap W_A=\{0\}$.  
    
    Lastly, we can prove (c) which will allow us to show the result for (a).  To see how many linearly independent elements we have in $W_S$ and $W_A$, it suffices to show the list of equations needed to satisfy for elements in $W_S$ and $W_A$.  Note that $W$ is $n^2$-dimensional and elements in $W_A$ must satisfy
    \[
    g_{ij}\boldv_i \otimes \boldv_j = g_{ji}\boldv_j\otimes \boldv_i.
    \]
    In other words, the coefficients $g_{ij}$ must satisfy
    \[
    g_{ij}=g_{ji}.
    \]
    This is $n^2-n=n(n-1)$ equations (since the $n$ diagonal elements are arbitrary) which, since we are setting coefficients equal to each other, give us that $\frac{1}{2}n(n-1)$ elements are fixed by these equations.  We then have that the dimension of $W_S$ is given by the dimension of $W$ minus how many elements are fixed, $n^2-\frac{1}{2}n(n-1)=\frac{1}{2}n(n+1)$. Hence $\dim(W_S)=\frac{1}{2}n(n+1)$.
    
    The argument for $\dim(W_A)$ is similar.  In this case we have the equations
    \[
    \omega_{ij} \boldv_i \otimes \boldv_j = -\omega_{ji} \boldv_j \otimes \boldv_i.
    \]
    This gives us $n^2-n$ equations for the off diagonal elements by noting
    \[
    \omega_{ij}=-\omega_{ji} \qquad i\neq j
    \]
    but also $n$ equations for the diagonal elements since they must be identically zero.  That is,
    \[
    \omega_{ii}=0.
    \]
    In total, we have $\frac{1}{2}n(n-1)+n=\frac{1}{2}n(n+1)$ independent equations. Then the dimension of $W_A$ is found to be
    \[
    \dim(W_A)=n^2-\frac{1}{2}n(n+1)=\frac{1}{2}n(n-1).
    \]
    To prove (a), note that we have $W_S\cap W_A=\{0\}$ and \[
    \dim(W_S)+\dim(W_A)=\frac{1}{2}n(n+1)+\frac{1}{2}n(n-1)=n^2=\dim(W)
    \]
    and so $W_S\oplus_K W_A=W$.
    \item To see that $W_S$ and $W_A$ are $KG$-submodules, take a $g\in KG$ and a $w\in W$.  Now, write $w=w_{ij}\boldv_i\otimes \boldv_j$ (with Einstein summation assumed)
    \begin{align*}
        (wg)^\alpha &= (w_{ij}(\boldv_ig)\otimes (\boldv_j g))^\alpha\\
        &= w_{ij}(\boldv_j g)\otimes (\boldv_i g)\\
        &= w_{ji}(\boldv_ig)\otimes (\boldv_j g)\\
        &= w_{ji}\boldv_i \otimes \boldv_j.
    \end{align*}
    Now if $w\in W_S$, then $w_{ij}=w_{ji}$ and thus $(wg)^\alpha=wg$. Similarly, if $w\in W_A$, then $w_{ij}=-w_{ji}$ and hence $(wg)^\alpha = -wg$. So these are both indeed $KG$-submodules.
    \item See part (a) for a proof.
\end{enumerate}
\end{solution}

\newpage
\begin{problem}
Let $K$ be a field and $G=\SL_2(K)$ and $V=K[x,y]$ the bivariate polynomial ring over $K$ in indeterminates $x$ and $y$. 
\begin{enumerate}[(a)]
    \item For $g=\begin{pmatrix} a & b\\ c& d\end{pmatrix}\in G$, we define $x.g\coloneqq ax+by$ and $y.g\coloneqq cx+dy$. Show that this makes $V$ a $KG$-module. 
    \item For positive $m$, let $V_m\leq V$ be the subspace of homogeneous polynomials of degree $m-1$ (i.e., all monomials have degree $m-1$). Show that $V_m\leq_{KG}V$ and determine $\dim_K V_m$.
    \item Show that for prime $p$ and $K=\finitef_p$, we have that
    \[
    G=\left\langle \begin{pmatrix} 1 & 1 \\ 0 & 1\end{pmatrix}, \begin{pmatrix} 1 & 0 \\ 1 & 1 \end{pmatrix}\right\rangle.
    \]
\end{enumerate}
\end{problem}
\begin{enumerate}[(a)]
    \item We take the action of $G$ on $V$ and extend linearly to get an action of $KG$ on $V$.  We then have for $u,v\in V$ and $A,B\in KG$ that
    \[
    v \mathbb{I} = 1,
    \]
    and
    \[
    v(AB)=(vA)B,
    \]
    since $G$ was a group action. Since we extend linearly, we have
    \[
    (u+v)A=uA+vA
    \]
    and we define the $+$ in $KG$ so that
    \[
    v(A+B)=vA+vB.
    \]
    \item Without loss of generality, consider a monomial $x^iy^j$ such that $i+j=m-1$.  It suffices to see how $g$ acts on this element and if it produces another set of monomials of degree $m-1$.  So take
    \[
    (x^iy^j)g=(ax+by)^i(cx+dy)^j.
    \]
    Now, we can write
    \begin{align*}
        (ax+by)^i =\sum_{k=0}^i \binom{n}{k} x^{i-k}y^k
    \end{align*}
    which is a sum of monomials of degree $i$. Similarly, we have that $(cx+dy)^j$ will be a sum of monomials of degree $j$.  Hence, when we multiply
    \[
    (ax+by)^i(cx+dy)^j
    \]
    we will have a sum of monomials of degree $m-1$ which means it is homogeneous of degree $m-1$.  
    
    Now, we can determine the degree by noting that the basis monomials are of the form $x^iy^j$ with $i+j=m-1$.  So, we have $m$ choices for the power of $x$, and the power of $y$ is determined from that. Thus, we have $m$ basis elements and $\dim_K(V_m)=m$.
    
    dimension is $m$, we can show for a monomial $x^i y^j$ where $i+j=m-1$ since linear. Use binomial thing and you'll have it.
    \item It's trivial that $G\leq \SL_2(K\finitef_p)$ seeing as the determinant of the generators are both one and the determinant of the product of matrices is the product of determinants. So the elements of $G$ must form a subgroup of $\SL_2(\finitef_p)$.
    
    To see that $G = \SL_2(\finitef_p)$, it suffices to show that the order of the two groups are the same.  One could enumerate this for any (reasonable) $p$ on \textsf{GAP}.
\end{enumerate}
\end{document}
