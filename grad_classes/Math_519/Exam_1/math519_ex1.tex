\documentclass[leqno]{article}
\usepackage[utf8]{inputenc}
\usepackage[T1]{fontenc}
\usepackage{amsfonts}
%\usepackage{fourier}
%\usepackage{heuristica}
\usepackage{enumerate}
\author{Colin Roberts}
\title{MATH 519, Exam 1}
\usepackage[left=3cm,right=3cm,top=3cm,bottom=3cm]{geometry}
\usepackage{amsmath}
\usepackage[thmmarks, amsmath, thref]{ntheorem}
%\usepackage{kbordermatrix}
\usepackage{mathtools}
\usepackage{color}
\usepackage{hyperref}

\theoremstyle{nonumberplain}
\theoremheaderfont{\itshape}
\theorembodyfont{\upshape:}
\theoremseparator{.}
\theoremsymbol{\ensuremath{\square}}
\newtheorem{proof}{Proof}
\theoremsymbol{\ensuremath{\square}}
\newtheorem{lemma}{Lemma}
\theoremsymbol{\ensuremath{\blacksquare}}
\newtheorem{solution}{Solution}
\theoremseparator{. ---}
\theoremsymbol{\mbox{\texttt{;o)}}}
\newtheorem{varsol}{Solution (variant)}

\newcommand{\id}{\mathrm{Id}}
\newcommand{\R}{\mathbb{R}}
\newcommand{\N}{\mathbb{N}}
\newcommand{\Z}{\mathbb{Z}}
\newcommand{\C}{\mathbb{C}}

\begin{document}
\maketitle
\begin{large}
\begin{center}
Solutions
\end{center}
\end{large}

%%%%%%%%%%%%%%%%%%%%%%%%%%%%%%%%%%%%%%%%%%%%%%%%%%%%%%%%%%%%%%%%%%%%%%%%%%%%%%%%%%%%%%%%%%%%%%%%%%%%%%%%%%%%%%%%%%%%%
%%%%%%%%%%%%%%%%%%%%%%%%%PROBLEM%%%%%%%%%%%%%%%%%%%%%%%%%%%%%%%%%%%%%%%%%%%%%%%%%%%%%%%%%%%%%%%%%%%%%%%%%%%%%%%%%%%%%%%%%%%%%%%%%%%%%%%%%%%%%%%%%%%%%%%%%%%%%%%%%%%%%%%%%%%%%%%%%%%%%%%%%%%%%%%%%%%%%%%%%%%%%%%%%%%%%%%%%%%%%%%%%%%%%%%%%%

\noindent\textbf{Problem 1.}  Use the Cauchy Integral Formula to evaluate $\int_C \frac{\cos(z)}{z}dz$ where $C$ is the unit circle.

\begin{proof}
We have 
\begin{align*}
\int_C \frac{\cos(z)}{z}dz &= \int_C \frac{\exp(iz)-\exp(-iz)}{2z}dz\\
&= \frac{1}{2} \left( \int_C \frac{\exp(iz)}{z-0}dz + \int_C \frac{\exp(-iz)}{z-0}dz \right)\\
&= \frac{1}{2} \left( 2\pi i \exp(i\cdot 0)+ 2\pi i \exp(-i\cdot 0)\right) &&\textrm{by Cauchy's Integral Formula}\\
&= 2\pi i.
\end{align*}
\end{proof}

\vspace*{1cm}


\noindent\textbf{Problem 2.} Let $f(z)=\frac{1}{p(z)}$, where $p(z)$ is some degree $k$ polynomial. What is the maximum number of different values for the integral of $f(z)$ around various closed, simple, positively-oriented contours $C$ that do not pass through any of the roots of $p(z)$? (\underline{NOTE:} If you can handle the combinatorics and write down the explicit answer, do so. Otherwise, describe how you might go about counting all the possible values.)

\begin{proof}
Letting ${n\choose m} = \frac{n!}{m!(n-m)!}$, we have that there are ${k\choose 0}=1$ ways for a simple closed curve to inclose zero roots of $p(z)$, ${k\choose 1}$ ways for a simple closed curve to inclose a single root of $p(z)$, and in general we have ${k\choose n}$ ways for a simple closed curve to inclose $n\leq k$ roots of $p(z)$. Since we only allow for positively-oriented curves, we then have that the total number of values for a contour integral around our simple closed positively-oriented curve is given by
\begin{align*}
\sum_{n=0}^k {k \choose n}.
\end{align*}
\end{proof}

\vspace*{1cm}

\noindent\textbf{Problem 3.} ~
\begin{enumerate}[(a)]
\item Evaluate $\int_C \frac{\cos(z)}{z^5-1}dz$ where $C$ is the circle $|z-2i|=\frac{1}{2}$.
\item Evaluate $\int_C \frac{e^z}{z^2+4}dz$ where $C$ is the circle $|z-2i|=\frac{1}{2}$.
\end{enumerate}

\begin{proof}
\begin{enumerate}[(a)]
\item Note that $\displaystyle{\frac{\cos(z)}{z^5-1}}$ is holomorphic on $C$ and within the interior of $C$. Thus 
\begin{align*}
\int_C\frac{\cos(z)}{z^5-1}dz = 0.
\end{align*}
\item We have that $z^2+4=(z-2i)(z+2i)$.  Then note that we have
\begin{align*}
\int_C \frac{e^z}{z+2i}\cdot \frac{1}{z-2i} dz
\end{align*}
allows for the use of Cauchy's integral formula.  Namely, $\displaystyle{\frac{e^z}{z+2i}}$ is holomorphic on $C$ and in the interior of $C$ as well, meaning that if we let $\displaystyle{f(z)=\frac{e^z}{z+2i}}$ then we have
\begin{align*}
\int_C \frac{e^z}{z^2+4}dz &= \int_C \frac{f(z)}{z-2i}dz\\
&= 2\pi i f(2i)\\
&= 2\pi i \frac{e^{2i}}{4i}\\
&=\frac{\pi e^{2i}}{2}.
\end{align*}
\end{enumerate}
\end{proof}

\vspace*{1cm}

\noindent\textbf{Problem 4.} Suppose $f\colon \C \to \C$ is entire and $|f(z)|<e^{-|z|}$ for all $z\in \C$. What can you say about the image of $\C$ under $f(z)$?


\begin{proof}
The image of $f$ is a singleton since $f$ must be constant.  We have this by Liouville's theorem since $f$ is entire and $f$ is bounded. We're given $f$ is entire, and to see that $f$ is bounded, note that $\sup_{z\in \C} e^{-|z|}=1$ and hence $|f(z)|<1$ which shows $f$ is bounded.
\end{proof}


\vspace*{1cm}

\noindent\textbf{Problem 5.}  Each of the following functions has an isolated singularity at $z=0$. Determine which type of isolated singularity each on is AND
\begin{itemize}
\item if it is removable, define $f(0)$ so that $f(z)$ is analytic;
\item if it is a pole, find the residue of $f(z)$ at $z=0$; and
\item if it is essential, decide which value (if any) is neglected from the range in a (any) neighborhood of $z=0$.
\end{itemize}

\begin{enumerate}[(a)]
\item $\frac{\cos(z)}{z}$
\item $\frac{\cos(z)-1}{z}$
\item $e^{\frac{1}{z}}-5$
\item $\frac{z^2+1}{z(z-1)}$.
\end{enumerate}

\begin{proof}~
\begin{enumerate}[(a)]
\item At $z=0$ residue should be $1$. We show this by computing $a_{-1}$ of the laurent expansion where
\begin{align*}
a_n=\frac{1}{2\pi i} \int_\gamma \frac{f(z)}{(z-c)^{n+1}}dz.
\end{align*}
Here $\gamma$ is a closed path around the point we are doing the expansion on, $c$. In our case, we will choose $\gamma$ to be the unit circle with the typical orientation. Then we have
\begin{align*}
a_{-1}&=\frac{1}{2\pi i} \int_\gamma \frac{\cos(z)/z}{z^{-1+1}}dz\\
&= \frac{1}{2\pi i}\int_\gamma \frac{\cos(z)}{z}\\
&=1.
\end{align*}
This result is from Problem 1.
\item At $z=0$ residue should be $0$. We compute by using $\gamma$ as the unit circle with typical orientation again, and we find
\begin{align*}
a_{-1}&=\frac{1}{2\pi i} \int_\gamma \frac{\cos(z)}{z}-\frac{1}{z}dz\\
&= \frac{1}{2\pi i} \int_\gamma \frac{\cos(z)}{z}dz - \frac{1}{2\pi i} \int_\gamma \frac{1}{z}dz\\
&=1-1.
\end{align*}
Note that the $\cos(z)/z$ integral is Problem 1 again and that the $1/z$ integral is easily seen by Cauchy's integral formula.
\item This has an essential singularity at $z=0$ and the neglected value is $z=-5$.
\item At $z=0$ residue should be $-1$. Here we choose $\gamma$ to be the circle $|z|=\frac{1}{2}$ with the typical orientation so that we avoid the singularity at $z=1$.  Then letting $f(z)=\frac{z^2+1}{z-1}$ we have
\begin{align*}
a_{-1}&= \int_\gamma \frac{z^2+1}{z(z-1)}dz\\
&= \int_\gamma f(z)\cdot \frac{1}{z}dz\\
&= f(0) = -1.
\end{align*}
\end{enumerate}
\end{proof}



\end{document}



