\documentclass[leqno]{article}
\usepackage[utf8]{inputenc}
\usepackage[T1]{fontenc}
\usepackage{amsfonts}
\usepackage{fourier}
\usepackage{heuristica}
\usepackage{enumerate}
\author{Colin Roberts}
\title{MATH 570, Homework 2}
\usepackage[left=3cm,right=3cm,top=3cm,bottom=3cm]{geometry}
\usepackage{amsmath}
\usepackage[thmmarks, amsmath, thref]{ntheorem}
%\usepackage{kbordermatrix}
\usepackage{mathtools}

\theoremstyle{nonumberplain}
\theoremheaderfont{\itshape}
\theorembodyfont{\upshape:}
\theoremseparator{.}
\theoremsymbol{\ensuremath{\square}}
\newtheorem{proof}{Proof}
\theoremsymbol{\ensuremath{\square}}
\newtheorem{lemma}{Lemma}
\theoremsymbol{\ensuremath{\blacksquare}}
\newtheorem{solution}{Solution}
\theoremseparator{. ---}
\theoremsymbol{\mbox{\texttt{;o)}}}
\newtheorem{varsol}{Solution (variant)}

\newcommand{\tr}{\mathrm{tr}}

\begin{document}
\maketitle
\begin{large}
\begin{center}
Solutions
\end{center}
\end{large}
\pagebreak

%%%%%%%%%%%%%%%%%%%%%%%%%%%%%%%%%%%%%%%%%%%%%%%%%%%%%%%%%%%%%%%%%%%%%%%%%%%%%%%%%%%%%%%%%%%%%%%%%%%%%%%%%%%%%%%%%%%%%
%%%%%%%%%%%%%%%%%%%%%%%%%PROBLEM 1%%%%%%%%%%%%%%%%%%%%%%%%%%%%%%%%%%%%%%%%%%%%%%%%%%%%%%%%%%%%%%%%%%%%%%%%%%%%%%%%%%%%%%%%%%%%%%%%%%%%%%%%%%%%%%%%%%%%%%%%%%%%%%%%%%%%%%%%%%%%%%%%%%%%%%%%%%%%%%%%%%%%%%%%%%%%%%%%%%%%%%%%%%%%%%%%%%%%%%%%

\noindent\textbf{Problem 1.} Prove that a map between topological spaces is continuous if and only if the preimage of every closed subset is closed.

\noindent\rule[0.5ex]{\linewidth}{1pt}

\begin{proof}
For the forward direction suppose that we have a continuous map $f\colon X\to Y$.  Let $C\subseteq X$ be closed in $X$.  Thus we have that $C=Y\setminus O$ for some open set $O\subseteq Y$.  Then $f^{-1}(O)$ is open and $f^{-1}(0)\setminus f^{-1}(C)=X\setminus f^{-1}(C)$.  Thus we have that $f^{-1}(C)$ is closed.

Suppose that every preimage of a closed set is closed.  Then consider $C\subseteq X$ closed in $X$ and note that we can write $C=X\setminus 0$ for $0$ open in $Y$.  Then $f^{-1}(C)=f^{-1}(Y\setminus 0)=X\setminus f^{-1}(O)$.  Then we have that $f^{-1}(C)$ is closed which implies that $f^{-1}(O)$ must be open.  So $f$ is continuous.
\end{proof}

\pagebreak

%%%%%%%%%%%%%%%%%%%%%%%%%%%%%%%%%%%%%%%%%%%%%%%%%%%%%%%%%%%%%%%%%%%%%%%%%%%%%%%%%%%%%%%%%%%%%%%%%%%%%%%%%%%%%%%%%%%%%
%%%%%%%%%%%%%%%%%%%%%%%%%PROBLEM 2%%%%%%%%%%%%%%%%%%%%%%%%%%%%%%%%%%%%%%%%%%%%%%%%%%%%%%%%%%%%%%%%%%%%%%%%%%%%%%%%%%%%%%%%%%%%%%%%%%%%%%%%%%%%%%%%%%%%%%%%%%%%%%%%%%%%%%%%%%%%%%%%%%%%%%%%%%%%%%%%%%%%%%%%%%%%%%%%%%%%%%%%%%%%%%%%%%%%%%%%


\noindent\textbf{Problem 2.} Let $D$ be a discrete topological space, let $T$ be a space with the trivial (indiscrete) topology, let $H$ be a Hausdorff space, and let $A$ be an arbitrary topological space.
\begin{enumerate}[(a)]
\item Show that every function $f \colon D\to A$ is continuous.
\item Show that every function $f\colon A \to T$ is continuous.
\item Show that $f\colon T\to H$ is continuous if and only if it is a constant map.
\end{enumerate}

\noindent\rule[0.5ex]{\linewidth}{1pt}

\begin{proof}[Part (a)]
Let $f\colon D \to A$.  Then $O \subseteq A$ be an open set. Then consider $f^{-1}(O)\subseteq D$.  Since any subset of $D$ is open, we have that $f^{-1}(O)$ is open and thus $f$ is continuous.
\end{proof}

\begin{proof}[Part (b)]
Let $f\colon A \to T$.  Then let $O \subseteq T$ thus $O=\emptyset$ or $O=T$.  If $O=\emptyset$ then $f^{-1}(O)=\emptyset$ which is open.  Then if $O=T$ we have $f^{-1}(O)=T$ which is also open.  Thus $f$ is continuous.
\end{proof}

\begin{proof}[Part (c)]
For the forward direction, suppose that $f\colon T\to H$ is continuous. Let $x_1,x_2\in T$ be unique. Then $f(x_1),f(x_2)\in H$. Suppose that $f(x_1)\neq f(x_2)$, then $\exists O_1 \ni f(x_1)$ and $O_2 \ni f(x_2)$ with $O_1$ and $O_2$ open and $O_1\cap O_2=\emptyset$.  Then $f$ being continuous implies that $f^{-1}(O_1)=T=f^{-1}(O_2)$ Since we said we had two unique elements $x_1,x_2$ we have that $f^{-1}(O_1)\neq\emptyset \neq f^{-1}(O_2)$.  Then note that $f(f^{-1}(O_1))\subseteq O_1$ and $f(f^{-1}(O_2))\subseteq O_2$.  Thus we have that $O_1\cap O_2 \neq \emptyset$ and we contradict $H$ being Hausdorff.  Thus $f$ is a constant map.

Suppose $f\colon T \to H$ is a constant map.  Let $O\subseteq H$ be open.  But $f(x)=h\in H$ $\forall x$ thus we have $f^{-1}(O)=T$ $\forall O\subseteq H$ that are open.  So $f$ is continuous.

\emph{Note:} I worked on this problem with Zach and Tarun.
\end{proof}




\pagebreak


%%%%%%%%%%%%%%%%%%%%%%%%%%%%%%%%%%%%%%%%%%%%%%%%%%%%%%%%%%%%%%%%%%%%%%%%%%%%%%%%%%%%%%%%%%%%%%%%%%%%%%%%%%%%%%%%%%%%%
%%%%%%%%%%%%%%%%%%%%%%%%%PROBLEM 3%%%%%%%%%%%%%%%%%%%%%%%%%%%%%%%%%%%%%%%%%%%%%%%%%%%%%%%%%%%%%%%%%%%%%%%%%%%%%%%%%%%%%%%%%%%%%%%%%%%%%%%%%%%%%%%%%%%%%%%%%%%%%%%%%%%%%%%%%%%%%%%%%%%%%%%%%%%%%%%%%%%%%%%%%%%%%%%%%%%%%%%%%%%%%%%%%%%%%%%%


\noindent\textbf{Problem 3.} True or false:
\begin{enumerate}[(a)]
\item The intervals $[0,1)$ and $(0,\infty)$ in the real line, equipped with the Euclidean topology.\
\item The subsets $\{1,2,3,4,...\}$ and $\{1,\frac{1}{2},\frac{1}{3},\frac{1}{4},...\}$, equipped with the Euclidean topology.
\item The rationals $\mathbb{Q}$ with the discrete topology and the rationals $\mathbb{Q}$ with the Euclidean topology.
\item $S^2\setminus\{(0,0,1)\}$ and $\mathbb{R}^2$. 
\item $S^2\setminus\{(0,0,1),(0,0,-1)\}$ and $\{x\in \mathbb{R}^2 \vert 1 < \|\|x\|\|<3\}$.
\item $S^1$ and $S^1\cup \{(x,0)\in \mathbb{R}^2 \vert 1\leq x \leq 2\}$.
\item $\mathbb{R}^n$ and $\mathbb{R}^m$ for $n\neq m$.
\end{enumerate}

\noindent\rule[0.5ex]{\linewidth}{1pt}

\begin{solution}
\begin{enumerate}[(a)]
\item false
\item true
\item false
\item true
\item true
\item false
\item false
\end{enumerate}
\end{solution}

\pagebreak



%%%%%%%%%%%%%%%%%%%%%%%%%%%%%%%%%%%%%%%%%%%%%%%%%%%%%%%%%%%%%%%%%%%%%%%%%%%%%%%%%%%%%%%%%%%%%%%%%%%%%%%%%%%%%%%%%%%%%
%%%%%%%%%%%%%%%%%%%%%%%%%PROBLEM 4%%%%%%%%%%%%%%%%%%%%%%%%%%%%%%%%%%%%%%%%%%%%%%%%%%%%%%%%%%%%%%%%%%%%%%%%%%%%%%%%%%%%%%%%%%%%%%%%%%%%%%%%%%%%%%%%%%%%%%%%%%%%%%%%%%%%%%%%%%%%%%%%%%%%%%%%%%%%%%%%%%%%%%%%%%%%%%%%%%%%%%%%%%%%%%%%%%%%%%%%


\noindent\textbf{Problem 4.}  Let $X$ be a topological space and let $A\subseteq X$.
\begin{enumerate}[(a)]
\item Prove that a point $x\in X$ is in $\bar{A}$ if and only if every neighborhood of $x$ contains a point of $A$.
\item Suppose $\{x_i\}$ is a sequence of points in $A$ that converges to a point $x\in X$. Prove that $x\in A$.
\item Show that there exists a sequence for Figure 2.1(c) that converges to more than one limit point.
\end{enumerate}

\noindent\rule[0.5ex]{\linewidth}{1pt}

\begin{proof}[a]
For the forward direction, let $x\in \bar{A}$ and let $N_x \cap A=\emptyset$.  Then for $N_x$ a neighborhood of $x$ we have $X\setminus N_x$ is closed and since $N_x\cap A=\emptyset$, $X\setminus N_x\supset \bar{A}\supset A$. This implies that $x\notin \bar{A}$, so that implies that $N_x\cap A\neq \emptyset$. Thus every neighborhood of $x$ contains a point of $A$.

Suppose that every neighborhood of $x$ contains a point of $A$. Then, for a contradiction, suppose that every open neighborhood $N_x \ni x$, $N_x \cap A\neq \emptyset$. Suppose that $x\notin \bar{A}$. Then we have $x\in X\setminus \bar{A}$ is open.  This implies that $\exists N_x \subseteq X\setminus \bar{A}$ so then $N_x\cap A =\emptyset$.
\end{proof}

\begin{proof}[b]
Since $x_i\in A$ $\forall i$, then by definition of convergence $\forall N_x$, $N_x\cap A\neq \emptyset$. So by (a), $x\in \bar{A}$.
\end{proof}

\begin{proof}[c]
Let $x_i=1$ $\forall i$. Then $\{x_i\}\to 1$. since $x_i\in N_1 \forall i$. Then let $N_2=\{1,2\}$.  Note that $\forall i$, $1\in N_2$. Thus $\{x_2\}\to 2$ as well. 
\end{proof}

\pagebreak


%%%%%%%%%%%%%%%%%%%%%%%%%%%%%%%%%%%%%%%%%%%%%%%%%%%%%%%%%%%%%%%%%%%%%%%%%%%%%%%%%%%%%%%%%%%%%%%%%%%%%%%%%%%%%%%%%%%%%
%%%%%%%%%%%%%%%%%%%%%%%%%PROBLEM 5%%%%%%%%%%%%%%%%%%%%%%%%%%%%%%%%%%%%%%%%%%%%%%%%%%%%%%%%%%%%%%%%%%%%%%%%%%%%%%%%%%%%%%%%%%%%%%%%%%%%%%%%%%%%%%%%%%%%%%%%%%%%%%%%%%%%%%%%%%%%%%%%%%%%%%%%%%%%%%%%%%%%%%%%%%%%%%%%%%%%%%%%%%%%%%%%%%%%%%%%


\noindent\textbf{Problem 5.} Prove that a second countable space $X$ contains a countable dense subset.

\noindent\rule[0.5ex]{\linewidth}{1pt}

\begin{proof}
Let $X$ be a second countable space.  Thus we have a basis for the topology on $X$ given by open sets $U_i$ $\forall i\in \mathbb{N}$.  Then let $A=\{x_i \vert i \in \mathbb{N}\}$ so that each $x_i \in U_i$.  Notice that $X\setminus \bar{A}$ is open since $\bar{A}$ is closed.  Then suppose that $\exists x \in X\setminus \bar{A}$ and that $\exists N_x$ with $N_x\subseteq X\setminus \bar{A}$.  But since $U_i$ form a basis, we have that for $\alpha \subseteq \mathbb{N}$, $N_x=\cup_{i\in\alpha}U_i$.  Thus $N_x\cap A \neq \emptyset$ since $N_x$ must contain at least $x_i\in U_i$ for some $i$. This contradicts $N_x\subseteq X\setminus \bar{A}$.  So no $x\in X\setminus \bar{A}$ so $X\setminus \bar{A}=\emptyset$.
\end{proof}

\pagebreak



\end{document}

