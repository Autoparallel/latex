\documentclass[leqno]{article}
\usepackage[utf8]{inputenc}
\usepackage[T1]{fontenc}
\usepackage{amsfonts}
%\usepackage{fourier}
%\usepackage{heuristica}
\usepackage{enumerate}
\author{Colin Roberts}
\title{MATH 570, Homework 10}
\usepackage[left=3cm,right=3cm,top=3cm,bottom=3cm]{geometry}
\usepackage{amsmath}
\usepackage[thmmarks, amsmath, thref]{ntheorem}
%\usepackage{kbordermatrix}
\usepackage{mathtools}
\usepackage{tikz-cd}
\usepackage{ragged2e}

\theoremstyle{nonumberplain}
\theoremheaderfont{\itshape}
\theorembodyfont{\upshape:}
\theoremseparator{.}
\theoremsymbol{\ensuremath{\square}}
\newtheorem{proof}{Proof}
\theoremsymbol{\ensuremath{\square}}
\newtheorem{lemma}{Lemma}
\theoremsymbol{\ensuremath{\blacksquare}}
\newtheorem{solution}{Solution}
\theoremseparator{. ---}
\theoremsymbol{\mbox{\texttt{;o)}}}
\newtheorem{varsol}{Solution (variant)}

\newcommand{\tr}{\mathrm{tr}}
\newcommand{\Int}{\ensuremath{\mathrm{Int}}}
\newcommand{\N}{\ensuremath{\mathbb{N}}}
\newcommand{\Q}{\ensuremath{\mathbb{Q}}}
\newcommand{\R}{\ensuremath{\mathbb{R}}}
\newcommand{\Z}{\ensuremath{\mathbb{Z}}}
\newcommand{\cB}{\ensuremath{\mathcal{B}}}
\newcommand{\cF}{\ensuremath{\mathcal{F}}}
\newcommand{\obj}{\ensuremath{\mathrm{Obj}}}
\newcommand{\im}{\ensuremath{\mathrm{im}}}
\newcommand{\Id}{\ensuremath{\mathrm{Id}}}



\begin{document}
\maketitle
\begin{large}
\begin{center}
Solutions
\end{center}
\end{large}
\pagebreak

%%%%%%%%%%%%%%%%%%%%%%%%%%%%%%%%%%%%%%%%%%%%%%%%%%%%%%%%%%%%%%%%%%%%%%%%%%%%%%%%%%%%%%%%%%%%%%%%%%%%%%%%%%%%%%%%%%%%%
%%%%%%%%%%%%%%%%%%%%%%%%%PROBLEM 1%%%%%%%%%%%%%%%%%%%%%%%%%%%%%%%%%%%%%%%%%%%%%%%%%%%%%%%%%%%%%%%%%%%%%%%%%%%%%%%%%%%%%%%%%%%%%%%%%%%%%%%%%%%%%%%%%%%%%%%%%%%%%%%%%%%%%%%%%%%%%%%%%%%%%%%%%%%%%%%%%%%%%%%%%%%%%%%%%%%%%%%%%%%%%%%%%%%%%%%%

\noindent\textbf{Problem 1.}  Let $X$ be the abstract simplicial complex
\[
\{\{0\},\{1\},\{2\},\{3\},\{4\},\{5\},\{0,1\},\{0,2\},\{0,3\},\{0,4\},\{1,2\},\{3,4\}\}.
\]
\begin{enumerate}[(a)]
\item Draw the geometric realization of $X$.
\item Compute the simplicial homology group $H_0(X)$.
\item Compute the simplicial homology group $H_1(X)$.
\item Compute the simplicial homology group $H_2(X)$.
\end{enumerate}

\noindent\rule[0.5ex]{\linewidth}{1pt}

\begin{proof}
For now I'm noting that $Z_p(X)=\ker(\partial_p)$, $B_p(X)=\im(\partial_{p+1})$, and $H_p(X)=Z_p(X)/B_p(X)$. Also we define $\partial_p \colon C_p(X) \to C_{p-1}(X)$ by
\[
\partial_p([x_0,\dots,x_p])=\sum_{i=0}^p (-1)^i [x_0,\dots,\hat{x_i},\dots,x_p].
\]
\begin{enumerate}[(a)]
\item We have $X$:
\vspace*{5cm}

\item Now $H_0(X)=Z_0(X)/B_0(X)=\ker(\partial_0)/\im(\partial_1)$, so we find $Z_0(X)$ and $B_0(X)$.  First we have $C_1(X)=\{a[0,1]+b[0,2]+c[0,3]+d[0,4]+e[1,2]+f[3,4]~\vert~ a,b,c,d,e,f\in \Z\}$ and $C_0(X)=\{a[0]+b[1]+c[2]+d[3]+e[4]+g[5]~\vert~ a,b,c,d,e,f\in \Z\}$. Then 
\begin{align*}
B_0(X)&=\im(\partial_1)\\
&=\{a([1]-[0])+b([2]-[0])+c([3]-[0])+d([4]-[0])+e([2]-[1])+f([4]-[3])\\
&~\vert~ a,b,c,d,e,f\in \Z\}\\
&=\{a([1]-[0])+b([2]-[0])+c([3]-[0])+d([4]-[0])~\vert~ a,b,c,d\in \Z\}\cong \Z^4.
\end{align*}
Notice that $e([2]-[1])$ and $f([4]-[3])$  are $\Z$ linear combinations of the other three. 

Now $\partial_0 = 0$ so we have that $Z_0(X)=\ker(\partial_0)=C_0(X)\cong \Z^6$. Then we have $H_0(X)=\Z^6/\Z^4=\Z^2$. This tells us that there are two connected components. 

\item Now $H_1(X)=Z_1(X)/B_1(X)=\ker(\partial_1)/\im(\partial_2)$ and we have $C_2(X)=0$ and $C_1(X)=\{a[0,1]+b[0,2]+c[0,3]+d[0,4]+e[1,2]+f[3,4]~\vert~a,b,c,d,e,f\in \Z\}$. Note that $B_1(X)=\im(\partial_2)=\langle e\rangle$, as the trivial group. Now from above we have $\im(\partial_1)\cong \Z^3$ and since $C_1(X)\cong \Z^5$ we have $Z_1(X)=\ker(\partial_1)\cong \Z^2$. So it follows $H_1(X)\cong \Z^2$.

\item We have no 2-simplices, so $H_2(X)=\langle e \rangle$.
\end{enumerate}
\end{proof}

\pagebreak

%%%%%%%%%%%%%%%%%%%%%%%%%%%%%%%%%%%%%%%%%%%%%%%%%%%%%%%%%%%%%%%%%%%%%%%%%%%%%%%%%%%%%%%%%%%%%%%%%%%%%%%%%%%%%%%%%%%%%
%%%%%%%%%%%%%%%%%%%%%%%%%PROBLEM 2%%%%%%%%%%%%%%%%%%%%%%%%%%%%%%%%%%%%%%%%%%%%%%%%%%%%%%%%%%%%%%%%%%%%%%%%%%%%%%%%%%%%%%%%%%%%%%%%%%%%%%%%%%%%%%%%%%%%%%%%%%%%%%%%%%%%%%%%%%%%%%%%%%%%%%%%%%%%%%%%%%%%%%%%%%%%%%%%%%%%%%%%%%%%%%%%%%%%%%%%


\noindent\textbf{Problem 2.} Let $X$ be the simplicial complex which is the boundary of a tetrahedron. That is, $X$ has 4 vertices (say labeled 0,1,2,3), all ${{4}\choose{2}}=6$ possible edges, all ${{4}\choose{3}} =4$ possible 2-simplices, and no tetrahedra.
\begin{enumerate}[(a)]
\item Draw the geometric realization of $X$.
\item Compute the simplicial homology group $H_1(X)$. What group is $Z_1(X)$ isomorphic to?
\item Compute the simplicial homology group $H_2(X)$. 
\end{enumerate}


\noindent\rule[0.5ex]{\linewidth}{1pt}
\begin{enumerate}[(a)]
\item We have $X$:
\vspace*{5cm}

\item Now $H_1(X)=Z_1(X)/B_1(X)=\ker(\partial_1)/\im(\partial_2)$ and we have $C_2(X)=\{a[0,1,2]+b[0,1,3]+c[0,2,3]+d[1,2,3]~\vert~ a,b,c,d\in \Z\}$ and $C_1(X)=\{a[0,1]+b[0,2]+c[0,3]+d[1,2]+e[1,3]+f[2,3]~\vert~a,b,c,d,e,f\in \Z\}$. Now 
\begin{align*}
B_1(X)&=\im(\partial_2)\\
&=\{a([1,2]-[0,2]+[0,1])+b([1,3]-[0,3]+[0,1])+c([2,3]-[0,3]+[0,2])+\\
&d([2,3]-[1,3]+[1,2])~\vert~ a,b,c,d \in \Z\}
\end{align*}
From the extra work below, we have $\im(\partial_1)\cong \Z^3$ and since $C_1(X)\cong \Z^6$ we have $Z_1(X)=\ker(\partial_1)\cong \Z^3$. This can be seen by letting the ordered basis vectors be $\{[0,1],[0,2],[0,3],[1,2],[1,3],[2,3]\}$ and augmenting a matrix (really the matrix for $\partial_2$) as follows:
\begin{align*}
\begin{bmatrix}
1 & 1 & 0 & 0\\
-1 & 0 & 1 & 0\\
0 & -1 & -1 & 0\\
1 & 0 & 0 & 1\\
0 & 1 & 0 & -1\\
0 & 0 & 1 & 1
\end{bmatrix} ~~\textrm{ which reduces to }~~ 
\begin{bmatrix}
1 & 0 & 0 & 1\\
0 & 1 & 0 & -1\\
0 & 0 & 1 & 1\\
0 & 0 & 0 & 0\\
0 & 0 & 0 & 0\\
0 & 0 & 0 & 0
\end{bmatrix}.
\end{align*}
This above row reduction shows that we have 3 linearly independent vectors, which shows that $Z_1(X)\cong \Z^3$. So it follows $H_1(X)\cong \Z^3/\Z^3\cong \langle 3 \rangle$, the trivial group. 

\item We have $C_2(X)=\{a[0,1,2]+b[0,1,3]+c[0,2,3]+d[1,2,3]~\vert~ a,b,c,d\in \Z\}$. $H_2(X)=Z_2(X)/B_2(X)=\ker(\partial_2)/\im(\partial_3)$, and we have that $\im(\partial_3)=0$ since there are no 3-simplices. Then $\ker(\partial_2)=\Z$ since $\im(\partial_2)=\Z^3$. Thus we have $H_2(X)\cong \Z$.

\item \emph{Extra work:} We have $C_1(X)=\{a[0,1]+b[0,2]+c[0,3]+d[1,2]+e[1,3]+f[2,3]~\vert~ a,b,c,d,e,f\in \Z\}$ and $C_0(X)=\{a[0]+b[1]+c[2]+d[3]~\vert~ a,b,c,d,e,f\in \Z\}$. Then 
\begin{align*}
B_0(X)&=\im(\partial_1)\\
&=\{a([1]-[0])+b([2]-[0])+c([3]-[0])+d([2]-[1])+e([3]-[1])+f([3]-[2])\\
&~\vert~ a,b,c,d,e,f\in \Z\}\\
&=\{a([1]-[0])+b([2]-[0])+c([3]-[0])~\vert~ a,b,c\in \Z\}\cong \Z^3.
\end{align*}
Notice that $d([2]-[1])$ and $e([3]-[1])$  are $\Z$ linear combinations of the other three. 

Now $\partial_0 = 0$ so we have that $Z_0(X)=\ker(\partial_0)=C_0(X)\cong \Z^4$. Then we have $H_0(X)=\Z^4/\Z^3=\Z$. This registers the one connected component.

\end{enumerate}

\pagebreak


%%%%%%%%%%%%%%%%%%%%%%%%%%%%%%%%%%%%%%%%%%%%%%%%%%%%%%%%%%%%%%%%%%%%%%%%%%%%%%%%%%%%%%%%%%%%%%%%%%%%%%%%%%%%%%%%%%%%%
%%%%%%%%%%%%%%%%%%%%%%%%%PROBLEM 3%%%%%%%%%%%%%%%%%%%%%%%%%%%%%%%%%%%%%%%%%%%%%%%%%%%%%%%%%%%%%%%%%%%%%%%%%%%%%%%%%%%%%%%%%%%%%%%%%%%%%%%%%%%%%%%%%%%%%%%%%%%%%%%%%%%%%%%%%%%%%%%%%%%%%%%%%%%%%%%%%%%%%%%%%%%%%%%%%%%%%%%%%%%%%%%%%%%%%%%%


\noindent\textbf{Problem 3.} Choose any old homework or exam problem, or a portion thereof. Clearly state both the problem and the homework/exam number. Write out a solution that is as clear as possible, with no extraneous steps.

\noindent\textbf{Problem 2. Homework 8:} Let $S^1$ be the unit circle and let $C=S^1\times [-1,1]$ be a cylinder. Prove that $S^1\cong C$.

\noindent\rule[0.5ex]{\linewidth}{1pt}

\begin{proof}
Define the maps $f\colon S^1\to C$ and $g \colon C \to S^1$ with $f(x)=(x,0)$ and $g(x,s)=x$.  Then we show that $f\circ g \simeq \Id_{C}$ and $g\circ f \simeq \Id_{S^1}$. Clearly we have $g\circ f = \Id_{S^1}$ which shows $g\circ f \simeq \Id_{S^1}$. Now we have $H\colon C\times I \to C$ defined by $H((x,s),t)=(x,st)$ is continuous and satisfies $H((x,s),0)=f(x)$ and $H((x,s),1)=\Id_{C}(x,s)$ which shows that $f\circ g \simeq \Id_{C}$. Hence, $S^1\simeq C$.
\end{proof}

\pagebreak






\end{document}

