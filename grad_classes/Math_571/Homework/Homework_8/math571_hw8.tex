\documentclass[leqno]{article}
\usepackage[utf8]{inputenc}
\usepackage[T1]{fontenc}
\usepackage{amsfonts}
%\usepackage{fourier}
%\usepackage{heuristica}
\usepackage{enumerate}
\author{Colin Roberts}
\title{MATH 571, Homework 8}
\usepackage[left=3cm,right=3cm,top=3cm,bottom=3cm]{geometry}
\usepackage{amsmath}
\usepackage[thmmarks, amsmath, thref]{ntheorem}
%\usepackage{kbordermatrix}
\usepackage{mathtools}
\usepackage{color}
\usepackage{hyperref}
\usepackage{tikz-cd}

\theoremstyle{nonumberplain}
\theoremheaderfont{\itshape}
\theorembodyfont{\upshape:}
\theoremseparator{.}
\theoremsymbol{\ensuremath{\square}}
\newtheorem{proof}{Proof}
\theoremsymbol{\ensuremath{\square}}
\newtheorem{lemma}{Lemma}
\theoremsymbol{\ensuremath{\blacksquare}}
\newtheorem{solution}{Solution}
\theoremseparator{. ---}
\theoremsymbol{\mbox{\texttt{;o)}}}
\newtheorem{varsol}{Solution (variant)}

\newcommand{\id}{\mathrm{Id}}
\newcommand{\im}{\mathrm{im}}
\newcommand{\R}{\mathbb{R}}
\newcommand{\N}{\mathbb{N}}
\newcommand{\Z}{\mathbb{Z}}

\begin{document}
\maketitle
\begin{large}
\begin{center}
Solutions
\end{center}
\end{large}

%%%%%%%%%%%%%%%%%%%%%%%%%%%%%%%%%%%%%%%%%%%%%%%%%%%%%%%%%%%%%%%%%%%%%%%%%%%%%%%%%%%%%%%%%%%%%%%%%%%%%%%%%%%%%%%%%%%%%
%%%%%%%%%%%%%%%%%%%%%%%%%PROBLEM%%%%%%%%%%%%%%%%%%%%%%%%%%%%%%%%%%%%%%%%%%%%%%%%%%%%%%%%%%%%%%%%%%%%%%%%%%%%%%%%%%%%%%%%%%%%%%%%%%%%%%%%%%%%%%%%%%%%%%%%%%%%%%%%%%%%%%%%%%%%%%%%%%%%%%%%%%%%%%%%%%%%%%%%%%%%%%%%%%%%%%%%%%%%%%%%%%%%%%%%%%

\noindent\textbf{Problem 1.} 
Pick a $\Delta$-complex structure on the pair of spaces $(S^1\times S^1,S^1\times \{1\})$ -- probably the first $\Delta$-complex structure you think of on $S^1\times S^1$ will work. Compute the simplicial relative homology $H_n(S^1\times S^1,S^1\times \{1\})$ for all $n$.


\begin{proof}
Let me draw the $\Delta$-complex for $S^1\times S^1$ and for $S^1\times \{1\}$ below.
\vspace*{5cm}\\
Now we look at the chain complex
\begin{align*}
\cdots \rightarrow \Delta_3(X)/\Delta_3(A) \xrightarrow{\partial_3} \Delta_2(X)/\Delta_2(A) \xrightarrow{\partial_2} \Delta_1(X)/\Delta_1(A) \xrightarrow{\partial_1}  \Delta_0(X)/\Delta_0(A) \xrightarrow{\partial_0} 0.
\end{align*}
Note that for $i\geq 3$,  $\Delta_i(X)/\Delta_i(A)\cong 0$ since there are no simplicies of dimension $3$ or higher.  We have that $\Delta_2(X)/\Delta_2(A)\cong \Z^2$ is generated by $T,U$ and that under $\partial_2$ we have 
\begin{align*}
T&\mapsto b-c+a = b-c\\
U&\mapsto b-c+a = b-c.
\end{align*}
Then $\Delta_1(X)/\Delta_1(A)\cong \Z^2$ is generated by $b$ and $c$ and under $\partial_1$ we have
\begin{align*}
b&\mapsto 0\\
c\mapsto 0.
\end{align*}
Then since $\Delta_0(X)\cong \Z$ and $\Delta_0(A) \cong Z$ we have $\Delta_0(X)/\Delta_0(A)\cong 0$.  We then compute homology to find that 
\begin{align*}
H_0(X,A)&\cong 0\\
H_1(X,A)\cong \Z
\end{align*}
since we have that $\ker \partial_1$ is generated by $\{b,b-c\}$ with a change of basis and $\im \partial_1$ is generated by $\{b-c\}$. Then
\begin{align*}
H_2(X,A)&\cong \Z.
\end{align*}
Finally for $i\geq 3$ we have $H_i(X,A)\cong 0$.
\end{proof}

\vspace*{1cm}


\noindent\textbf{Problem 2.} 
Hatcher exercise 9(a) on page 155: Compute the homology groups of the quotient of $S^2$ obtained by identifying the north and south poles to a point.\\

\noindent \emph{Remark:} I recommend using the long exact sequence for the singular homology of a pair $(S^2,S^0)$.

\begin{proof}
First note that we have $S^2/S^0$ as our desired space since $S^0$ as a subspace of $S^2$ can be taken to be the north and south poles.  Then also we have that $S^2$ and $S^0$ form a good pair since $S^0$ is a closed subspace that is also a deformation retract of open neighborhoods about the north and south pole of $S^2$.  Now, this means $\tilde{H}_i(S^2/S^0)\cong H_i(S^2,S^0)$ so we can use the long exact sequence in the above remark. Namely, we have
\begin{align*}
\cdots \rightarrow H_3(S^0) \cong 0 \rightarrow H_3(S^2)\cong 0 \rightarrow H_3(S^2,S^0) \\
\rightarrow H_2(S^0) \cong 0\rightarrow H_2(S^2)\cong \Z \rightarrow H_2(S^2,S^0)\\
\rightarrow H_1(S^0)\cong 0 \rightarrow H_1(S^2) \cong 0 \rightarrow H_1(S^2,S^0)\\
\rightarrow H_0(S^0)\cong \Z^2\rightarrow H_0(S^2)\cong \Z \rightarrow H_0(S^2,S^0)\cong 0.   
\end{align*}
Note that we have $H_0(S^2,S^0)$ by our previous homework problem since $S^0$ meets the connected component of $S^2$. Now, this exact sequence gives us that for $i\geq 3$, $H_i(S^2,S^0)\cong 0$. We also have $H_2(S^2,S^0)\cong \Z$ by exactness above. Exactness again implies that $H_1(S^2,S^0)\cong H_0(S^0)/H_0(S^2)$ and hence $H_1(S^2,S^0)\cong \Z$.
\end{proof}



\end{document}



