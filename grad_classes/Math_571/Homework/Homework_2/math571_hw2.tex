\documentclass[leqno]{article}
\usepackage[utf8]{inputenc}
\usepackage[T1]{fontenc}
\usepackage{amsfonts}
%\usepackage{fourier}
%\usepackage{heuristica}
\usepackage{enumerate}
\author{Colin Roberts}
\title{MATH 571, Homework 2}
\usepackage[left=3cm,right=3cm,top=3cm,bottom=3cm]{geometry}
\usepackage{amsmath}
\usepackage[thmmarks, amsmath, thref]{ntheorem}
%\usepackage{kbordermatrix}
\usepackage{mathtools}
\usepackage{color}
\usepackage{hyperref}

\theoremstyle{nonumberplain}
\theoremheaderfont{\itshape}
\theorembodyfont{\upshape:}
\theoremseparator{.}
\theoremsymbol{\ensuremath{\square}}
\newtheorem{proof}{Proof}
\theoremsymbol{\ensuremath{\square}}
\newtheorem{lemma}{Lemma}
\theoremsymbol{\ensuremath{\blacksquare}}
\newtheorem{solution}{Solution}
\theoremseparator{. ---}
\theoremsymbol{\mbox{\texttt{;o)}}}
\newtheorem{varsol}{Solution (variant)}

\newcommand{\id}{\mathrm{Id}}
\newcommand{\R}{\mathbb{R}}
\newcommand{\N}{\mathbb{N}}

\begin{document}
\maketitle
\begin{large}
\begin{center}
Solutions
\end{center}
\end{large}

%%%%%%%%%%%%%%%%%%%%%%%%%%%%%%%%%%%%%%%%%%%%%%%%%%%%%%%%%%%%%%%%%%%%%%%%%%%%%%%%%%%%%%%%%%%%%%%%%%%%%%%%%%%%%%%%%%%%%
%%%%%%%%%%%%%%%%%%%%%%%%%PROBLEM%%%%%%%%%%%%%%%%%%%%%%%%%%%%%%%%%%%%%%%%%%%%%%%%%%%%%%%%%%%%%%%%%%%%%%%%%%%%%%%%%%%%%%%%%%%%%%%%%%%%%%%%%%%%%%%%%%%%%%%%%%%%%%%%%%%%%%%%%%%%%%%%%%%%%%%%%%%%%%%%%%%%%%%%%%%%%%%%%%%%%%%%%%%%%%%%%%%%%%%%%%

\noindent\textbf{Problem 1.} Give an example of a connected space $X$, a subspace $A\subseteq X$, and a map $r\colon X\to A$ that is a retract but which does not come from a deformation retraction. Prove your answer is correct.

\begin{proof}
Let $X=S^1$ and let $A=\{p\}\in S^1$ be a single point. Then let $r\colon X \to A$ be a retraction.  Note this map is continuous since $r^{-1}(p)=S^1$ and is a retraction since $r\vert_A = \id_A$ and $r(X)=A$.  Yet, this map is not a deformation retraction since $S^1 \not\simeq \{p\}$.
\end{proof}

\vspace*{1cm}


\noindent\textbf{Problem 2.} The \emph{Euler characteristic} of a finite CW complex $X$ is $\chi(X)=\sum_i (-1)^i c_i$, where $c_i$ is the number of $i$-cells of $X$. Alternatively see also the definition on page 6 of our book.
\begin{enumerate}[(a)]
\item If $X$ is $S^n$ with a CW structure of a single 0-cell $e^0$ and a single $n$-cell $e^n$, then what is $\chi(X)$? 
\item If $X$ is $S^n$ with a CW structure of two 0-cells, two 1-cells, \ldots, two $n$-cells, then what is $\chi(X)$?
\item Explain why Theorem~2.44 on page 146 of our book, which we haven't covered yet, says that you should have expected to get the same answer in (a) and (b).
\item Any simplicial complex has a CW complex structure with one $i$-cell for each $i$-simplex. Find the Euler characteristic of the $n$-simplex $\Delta^n$ by counting $c_i$ for each $i$, and then computing the alternating sum $\chi(\Delta^n)=\sum_i (-1)^i c_i$. How in the world do you simplify that long alternating sum?
\item Instead, now find the Euler characteristic of the $n$-simplex by remarking that $\Delta^n$ is homotopy equivalent to a simpler space (no proof needed), and then computing the Euler characteristic of that simpler space.
\end{enumerate}

\begin{proof}~
\begin{enumerate}[(a)]
\item $\chi(X)=\sum_i (-1)^i c_i = (-1)^0+(-1)^n = 1 +(-1)^n$. So $\chi(X)=0$ if $n$ is even and $\chi(X)=2$ if $n$ is odd.
\item $\chi(X)=\sum_i (-1)^i c_i = 2(-1)^0 + 2(-1)^1 + \cdots + 2(-1)^{n-1}+2(-1)^n$.  Then if $n$ is even we can pair off these quantities by $2(-1)^0+2(-1)^1 = 0$, $2(-1)^2+2(-1)^3=0$, up to $2(-1)^{n-1}+2(-1)^n=0$ and hence $\chi(X)=0$ if $n$ is even.  If $n$ is odd, then we can pair off and cancel out terms up to the $n$th term, which will give us $\chi(X)=2(-1)^n=2$.
\item Since $S^n$ has a given homology, we don't expect the way we construct $S^n$ to change the rank of any homology group of $S^n$.  Otherwise, we would not have a consistent homology theory.
\item Notice that we have $c_0=n+1$ for $\Delta^n$.  Also, we have that $c_i = {{n+1}\choose{i+1}}$ for $\Delta^n$ as well.  This gives that $\chi(X)=n+1 + \sum_{i=1}^n (-1)^i{{n+1}\choose{i+1}}$. I think this will have the sum collpase into giving us $\chi(X)=1$.
\item Since each $\Delta^n$ is contractible, we have that $\Delta^n \simeq \Delta^0$.  Then the long alternating sum just becomes $c_0=1$.  
\end{enumerate}
\end{proof}

\vspace*{1cm}


\noindent\textbf{Problem 3.} Show that $S^\infty$ is contractible. I encourage you to consult Example 1B.3 on page 88 in our book! 

\begin{proof}
First consider the homotopy $f_t \colon \R^\infty \to \R^\infty$ given by $f_t(x_1,x_2,\dots)=(1-t)(x_1,x_2,\dots)+t(0,x_1,x_2,\dots)$ and note that for all $t\in [0,1]$, $f_t$ takes nonzero vectors to nonzero vectors.  This means that we can let $\frac{f_t}{|f_t|}\colon S^\infty \to S^\infty$ be a homotopy.  We can then define $g_t \colon \R^\infty \to \R^\infty$ by $g_t(x_1,x_2,\dots)=(1-t)(0,x_1,x_2,\dots)+(1,0,0,\dots)$, which is again nonzero for all $t\in [0,1]$.  Hence, we can define a homotopy $\frac{g_t}{|g_t|} \colon S^\infty \to S^\infty$ where $g_1$ a constant map.  Now, if we consider $h_t = \frac{g_{2t-1}}{|g_{2t-1}|} \circ \frac{f_{2t}}{|f_{2t}|}$ we have a homotopy from $S^\infty$ to a point, showing that $S^\infty$ is contractible.
\end{proof}








\end{document}



