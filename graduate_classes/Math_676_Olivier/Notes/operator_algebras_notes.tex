\documentclass[leqno]{article}
\usepackage[utf8]{inputenc}
\usepackage[T1]{fontenc}
\author{Colin Roberts}
\title{Operator Algebras: Notes}
\usepackage[left=3cm,right=3cm,top=3cm,bottom=3cm]{geometry}
\usepackage{amssymb, amsmath ,cleveref ,thmtools, amsthm, mathtools}
\usepackage{enumerate}
\usepackage{hyperref}
\usepackage{color}

%%fonts
%??


\makeatletter
\def\thmhead@plain#1#2#3{%
  \thmname{#1}\thmnumber{\@ifnotempty{#1}{ }\@upn{#2}}%
  \thmnote{ {\the\thm@notefont#3}}}
\let\thmhead\thmhead@plain
\makeatother
\theoremstyle{definition}
\newtheorem{definition}{Definition}[section]
\newtheorem*{remark}{Remark}
\newtheorem*{example}{Example}
\newtheorem*{question}{Question}
\newtheorem*{exercise}{Exercise}

\theoremstyle{remark}
\newtheorem*{solution}{Solution}

\theoremstyle{theorem}
\newtheorem{theorem}{Theorem}[section]
\newtheorem{corollary}{Corollary}
\newtheorem{proposition}{Proposition}
\newtheorem{axiom}{Axiom}

\newcommand{\R}{\mathbb{R}}
\newcommand{\Q}{\mathbb{Q}}
\newcommand{\F}{\mathbb{F}}
\newcommand{\A}{\mathcal{A}}
\newcommand{\N}{\mathbb{N}}

\begin{document}
\maketitle
\tableofcontents
\pagebreak

\section{Notes From Papers}	

\subsection{Notes from Operator Algebras: An Informal Overview:}  
There is a lot of info on Von Neumann algebras which are more important in quantum theory.  Von Neumann algebras deal with operators on Hilbert spaces.
	
	All Von Neumann algebras are $C^*$ algebras. Any commutative Von Neumann algebra is isomorphic to the algebra of continuous functions $C(X)$ where $X$ is an \emph{extremely disconnected compact Hausdorff space}. 
	
	There are only two examples of commutative (abelian) $C^*$ algebras and Von Neumann algebras. These are $(C_0(X),\|\cdot \|)$ and $L^\infty (Z, d\mu)$ respectively.  $(C_0(X),\|\cdot \|)$ is the algebra of continuous functions which vanish at infinity on a locally compact Hausdorff space. The involution is complex conjugation and the norm $\|f\|:=\sup_{x\in X} |f(x)|$. The latter, $L^\infty (Z,d\mu)$ is the algebra of essentially bounded measurable functions for some $\sigma$-finite measure space $(Z,d\mu)$. The operators are understood to be multiplication operators on the complex Hilbert space.
	
	It is possible to recover the topological space $X$ from $(C_0(X),\|\cdot\|)$.  This gives some meaning to a non-commutative topological space $X$ when considering non-abelian $C^*$ algebras. Similarly you can find an association between non-commutative Von Neumann algebras and non-commutative measure spaces. These can be useful in non-commutative geometry, non-commutative $L^p$ spaces, and quantum groups.
	
	Operator algebras are also the ``natural universe" for spectral theory. There is a proposition that implies that any von Neumann algebra is generated as a norm closed subspace by the set of the spectral projections corresponding to its self-adjoint elements.
	
	One way of explaining the spectral theorem is that for bounded and unbounded normal operators on a Hilbert space, we have that the operator is unitarily equivalent to a multiplication operator.  I guess a classic example is the Fourier transform of the derivative operator.
	
	There is an analogous way of decomposing operators on infinite dimensional Hilbert spaces much like we do for operators on $\mathbb{C}^n$. This is of the form of \emph{direct integrals}.  Let $(Z, d\mu)$ be a suitable measure space. Denote by $H := Z H(z) dz$, the direct integral of the family of separable Hilbert spaces $\{H(z)\}_{z\in Z}$ in-dexed by points in $Z$ and with the corresponding measurability and convergence restrictions. 
	
	Let $T$ be a closed unbounded operator. We say that $T$ on $H$ is \emph{affiliated} to a von Neumann algebra $\mathcal{M} \subset L(H)$ if $U T U^{-1} = T$ for every unitary $U \in \mathcal{M}'$ . In this context we have the following natural characterization: if $T = V \cdot |T |$ is the corresponding polar decomposition of the closed operator, then $T$ is affiliated to $\mathcal{M}$ iff $V \in \mathcal{M} \subset \{E |T | (B) ~\vert~ B \subset \R^+ , \textrm{Borel} \}$. Moreover, it can be shown that an (unbounded) operator is normal on a Hilbert space $H$ iff it is affiliated to an Abelian von Neumann algebra $\mathcal{A}$.  A symmetric operator affiliated with a finite factor is automatically self-adjoint.
	
	See 4.3.1 for some usefulness in quantum mechanics and 4.3.2 for local quantum mechanics.  They are at the end of the paper.
	
\subsection{Notes from The $C^*$-Algebraic Formalism of Quantum Mechanics}
	 The notion of a state and the notion of an observable are important in both classical and quantum physics. In Hamiltonian mechanics, we describe the state of a system by an point $(q, p)$ in a symplectic manifold $M$ , known as phase space. Now, in all real physical
systems, a particles position and momentum must remain bounded, and hence, for the remainder of the paper, we shall assume that $M$ is compact.

	Classical observables can be measured with arbitrary precision.  A physical requirement is that observables depend on the state of the system: position $q$ and momentum $p$. Secondly, we better require that these functions be real-valued.
Thirdly, we must require that there is some way to make the error, when we measure an observable in the laboratory, arbitrarily small. Let $f$ be a real-valued function of $q$ and $p$, and we want to measure $f$ with error less
than some $\epsilon > 0$. Now, I know that I can make the error in $q$ and $p$ arbitrarily small, so, if there is some maximum error in $q$ and $p$, call it $\delta$, so that when I plug in my measured values of $q$ and $p$ into $f$ , my experimental value of $f$ will be within $\epsilon$ of the true value of $f$ , then $f$ will be observable. But of course, this is just the definition of a continuous function! Thus, the natural definition for an observables in classical mechanics can be stated as follows:

\begin{definition}
The \emph{classical observables} are exactly the continuous real-valued functions on $M$.
\end{definition}

\begin{theorem}
The set of observables $\mathcal{O}$ are exactly the self-adjoint elements of a commutative unital $C^*$-Algebra $\mathcal{A}$.
\end{theorem}

\begin{remark}
This above theorem will source as an axiom in the quantum theory.
\end{remark}

It is useful to think of classical states as a linear functional on an algebra $\mathcal{A}$.  So that we have for a state $S$ and $f\in \mathcal{A}$ that
\[
f(S)=S(f).
\]

\begin{theorem}
Let $S\in M$ be a state.  Then $S\colon \mathcal{A} \to \mathbb{C}$ is a normalized positive linear functional on $\mathcal{A}$.
\end{theorem}

\begin{theorem}[(Riesz-Markov Theorem)] Let X be a locally compact Hausdorff
space, and let $S$ be a state on $C^0 (X, \R)$. Then, there exists a unique Borel proba-
bility measure $\mu_S$ on $X$ such that, for all $f \in C^0 (X, \R)$
\[
S(f)=\int_X f d\mu_S
\]
\end{theorem}

\begin{remark}
When viewed in the light of the Riesz-Markov Theorem, it makes sense to view
$S(f )$ as the expected value of the observable $f$ in the state $S$. Physically, if we
measure f many times in the laboratory, and our particle is in the state $S$, then
our results should average to the value $S(f )$. 
\end{remark}

Given the previous definition, it makes sense to define the variance of an observable as well. We do that as follows:

\begin{definition}
Let $S\in M$ be a state and let $f \in \mathcal{O}$.  Then the \emph{variance of $f$ with respect to $S$} is defined as
\[
\sigma_S (f)^2 = S\left[ (f-S(f))^2 \right].
\]
\end{definition}

We find that for classical systems, the variance will be zero.  So this doesn't make sense based on what we know with quantum mechanics, i.e., 
\[
\sigma_S(p)\sigma_S(q)\geq \frac{\hbar}{2}.
\]

However, if we take the observables to be a non-commutative algebra, with the commutator \[[q,p]=i\hbar \mathbf{1}\], we recover the above equation.

Now we can create the axioms for quantum mechanics from the above ideas:

\begin{axiom}[(Quantum Observables)]
The set of observables $\mathcal{O}$ of a quantum system are exactly the self-adjoint elements of a separable (noncommutative) unital $C^*$-algebra $\mathcal{A}$.
\end{axiom}

\begin{axiom}[(Quantum States)]
The set of states $\mathcal{S}$ of a quantum system is the set of all positive linear functionals $\psi$ on $\mathcal{A}$ such that $\psi (\mathbf{1}) = 1$.
\end{axiom}

\begin{proposition}[(Bounded Operators on a Hilbert Space Form a $C^*$-Algebra)]
Let $H$ be a Hilbert space. Then, $\mathcal{BL}(H)$ (bounded linear operators on $H$) with pointwise addition, scalar multiplication, multiplication, the unary operation of “adjointing” for an involution, and the operator norm make $\mathcal{BL}(H)$ into a $C^*$-algebra.
\end{proposition}

\begin{proposition}
Let $H$ be a separable Hilbert space, let $\{e_n ~\vert~ n \in N\}$ and $\{f_n ~\vert~ n \in N\}$ be countable orthonormal bases for $H$, which exist by virtue of $H$ being seperable, and let $A \in \mathcal{L}(H)$ be trace-class. Then, 
\[
\sum_{n\in \N} (e_n \vert Ae_n ) = \sum_{n\in \N} (f_n \vert Af_n ) < \infty.
\]
\end{proposition}

\begin{proposition}
Let $V$ be a normed vector space, let $W$ be a Banach space, and let $U$ be a dense subset of $V$. Then for $\mathbf{A}\in \mathcal{BL}(U,W)$, there exists a unique $\tilde{\mathbf{A}} \in \mathcal{BL}(V,W)$ such that $\tilde{\mathbf{A}}\vert_U = \mathbf{A}$.
\end{proposition}

\color{blue}{Need more info on:
\begin{enumerate}[1.]
\item Weak-* topology (wikipedia)
\item Nets (wikipedia)
\item Gelfand-Naiman-Segal Theorem (wikipedia)
\end{enumerate} 
Left off note taking at section 6}\color{black}

\subsection{Notes from Algebraic Quantum Field Theory and Operator Algebras}

\color{blue}{Need more info on:
\begin{enumerate}[1.]
\item Weyl Algebras (operator algebra, specific von Neumann algebra)
\item Superselection (Wightman and Wigner), connection between different spins
\item LQP, specifically what does this all mean:
``In fact one could define LQP as being the theory of spacetime dynamics of local densities of superselected
charges. It turns out that the localized version of these charges constitute the backbone of the observable algebra
which, as we will see in more detail in the next section, is described by a map (a net) of spacetime regions into $C^*$-
algebras."
\item Quantum mechanics on a circle.  What the fuck
\item von Neumann wave packet collapse
\item Gelfand-Naimark-Segal is very important everywhere
\item The norm of a $C^*$-algebra is determined by the algebraic structure.
\end{enumerate} 
}\color{black}

\begin{theorem}
Every commutative unital $C^*$-algebra is isomorphic to the multiplication algebra of continuous
complex-valued functions on an $L^2(M, \mu)$ where $M$ is an appropriately chosen measure space with measure $\mu$.
\end{theorem}


\begin{theorem}
Every $C^*$-algebra is isomorphic to a norm-closed $~^*$-algebra of operators in a Hilbert space.
\end{theorem}

\begin{definition}
A state $\psi$ is a linear functional on $\A$ with the following properties.
\begin{enumerate}[(1)]
\item $\psi(A^*A)\geq 0$ $\forall A \in \A$ (positivity)
\item $\psi(\mathbf{1})=1$ (normalization).
\end{enumerate}
\end{definition}

\color{blue}{left off on state in appendix (b) for notes}\color{black}

\subsection*{Notes from An Introduction into the Mathematical Structure of Quantum Mechanics}

\begin{enumerate}[1.]
\item Section 1.2
\item I like the interpretation of uncertainty in beginning of chapter 2
\item GNS theory for classical and quantum will be good
\item Are observables (operators) for classical mechanics always bounded? I think I read that in here.
\item Unbounded operators in quantum case, definitely get to these. Theorem 2.3.1
\end{enumerate}

\subsection*{Notes from Wikipedia}

\begin{theorem}[Gefland-Naimark Theorem]
Any $C^*$-algebra $\A$ is isometrically $~^*$-isomorphic to a $C^*$-algebra of bounded operators on a Hilbert space.
\end{theorem}

\begin{proposition}[Gelfand-Naimark-Segal Construction]


\end{proposition}

\begin{proposition}[Gelfand Representation/Isomorphism]

\end{proposition}

\begin{enumerate}[1.]
\item Pure states, vector states, and more about states.  Find how they relate to states in QM.  Intro to Mathematical Structure of Quantum Mechanics may have some. \url{https://en.wikipedia.org/wiki/State_(functional_analysis)#Pure_states}
\end{enumerate}



\pagebreak

\section{Preliminary Definitions}

\begin{definition}
An bounded linear operator $A$ over a seperable Hilbert space $H$ is said to be in the \emph{trace class} if for some orthonormal basis $\{e_k\}_k$ we have
\[
\|A\|_1 = \mathrm{Tr}|A| \coloneqq \sum_{k} \langle (A^*A)^{1/2} e_k,e_k \rangle
\]
is finite. In this case we have
\[
\mathrm{Tr}A \coloneqq \sum_k \langle A e_k,e_k \rangle.
\]
\end{definition}

\begin{definition}
An \emph{isometry} between metric spaces is a map that preserves distance. In other words, if $T\colon X \to Y$ is such that 
\begin{align*}
\|Tx\|_Y = \|x\|_X, ~\textrm{for all $x\in X$}.
\end{align*}
\end{definition}

\begin{definition}
If $X$ and $Y$ are normed spaces, $X$ and $Y$ are \emph{isometrically isomorphic} if there is a surjective linear isometry from $X$ onto $Y$.
\end{definition}

\begin{definition}
An \emph{isomorphism} between Banach spaces $X$ and $Y$ is a linear bijection $T\colon X \to Y$ that is also a homeomorphism.
\end{definition}

\begin{definition}
If $H$ and $K$ are Hilbert spaces, an \emph{isomorphism} between $H$ and $K$ is a linear surjection $U\colon H \to K$ such that
\[
\langle Uh, Ug \rangle = \langle h, g\rangle.
\]
\end{definition}


\section{Banach Algebras}

\subsection{Basic Definitions and Properties}

\begin{definition}
An \emph{algebra} over $\F$ is a vector space $\A$ over $\F$ that also has a multiplication defined on it that makes $\A$ into a ring such that if $\alpha \in \F$ and $a,b\in \A$ then $\alpha(ab)=(\alpha a)b=a(\alpha b)$.
\end{definition}

\begin{definition}
A \emph{Banach algebra} is an algebra $\A$ over $\F$ that has a norm $\|\cdot \|$ relative to which $\A$ is a Banach space and such that for all $a,b\in \A$ we have
\[
\|ab\|\leq \|a\|\|b\|.
\]
\end{definition}

Note that $\A$ does not always have to have an identity.  If $\A$ does contain an identity $e$, then the map $\alpha \to \alpha e$ is an isomorphism of $\F$ into $\A$ and $\|\alpha e \|=|\alpha|$. Of course when $\A$ does have an identity we have $\|e\|=1$. However, we can create a new Banach algebra containing an identity from $\A$.

\begin{proposition}
If $\A$ is a Banach algebra without an identity, let $\A_1=\A\times \F$. Then define algebraic operations on $\A_1$ by
\begin{enumerate}[\textnormal{(i)}]
\item $(a,\alpha)+(b,\beta)=(a+b,\alpha+\beta)$;
\item $\beta(a,\alpha)=(\beta a,\beta \alpha)$;
\item $(a,\alpha)(b,\beta)=(ab+\alpha b+\beta a, \alpha \beta)$.
\end{enumerate}
Define $\|(a,\alpha)\|=\|a\|+|\alpha|$. Then $\A_1$ with this norm and the algebraic operations defined in (i), (ii), and (iii) is a Banach algebra with identity $(0,1)$ and $a\mapsto (a,0)$ is an isometric isomorphism of $\A$ into $\A_1$.
\end{proposition}

\subsection{Theory for Banach Algebras}





\section{$C^*$-Algebras}

\subsection{Basic Definitions}
\begin{definition}
An \emph{involution} is a function $f$ that is its own inverse.  In other words, $f(f(x))=x$ for every $x\in \mathrm{domain}(f)$.
\end{definition}

\begin{remark}
In our case, the adjoint is the involution we care about.  Since we have that $(A^*)^*=A$.
\end{remark}


\subsection{Theory for $C^*$-Algebras}


\pagebreak
\section{Bibliography}
Put stuff here for now:

\begin{enumerate}[1.]
\item Conway, A Course in Functional Analysis.
\item \url{http://www.math.uchicago.edu/~may/VIGRE/VIGRE2009/REUPapers/Gleason.pdf}\\
\item \url{https://www.imsc.res.in/~sunder/psc.pdf}
\end{enumerate}

\end{document}