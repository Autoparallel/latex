\documentclass[leqno]{article}
\usepackage[utf8]{inputenc}
\usepackage[T1]{fontenc}
\usepackage{amsfonts}
%\usepackage{fourier}
%\usepackage{heuristica}
\usepackage{enumerate}
\author{Colin Roberts}
\title{MATH 617, Homework 3}
\usepackage[left=3cm,right=3cm,top=3cm,bottom=3cm]{geometry}
\usepackage{amsmath}
\usepackage[thmmarks, amsmath, thref]{ntheorem}
%\usepackage{kbordermatrix}
\usepackage{mathtools}
\usepackage{color}

\theoremstyle{nonumberplain}
\theoremheaderfont{\itshape}
\theorembodyfont{\upshape:}
\theoremseparator{.}
\theoremsymbol{\ensuremath{\square}}
\newtheorem{proof}{Proof}
\theoremsymbol{\ensuremath{\square}}
\newtheorem{lemma}{Lemma}
\theoremsymbol{\ensuremath{\blacksquare}}
\newtheorem{solution}{Solution}
\theoremseparator{. ---}
\theoremsymbol{\mbox{\texttt{;o)}}}
\newtheorem{varsol}{Solution (variant)}

\newcommand{\tr}{\mathrm{tr}}
\newcommand{\R}{\mathbb{R}}
\newcommand{\N}{\mathbb{N}}


\usepackage{amssymb}
\usepackage{graphics}

\textheight=9.0in
\textwidth=6.5in
\oddsidemargin=0in
\topmargin=-0.50in

\pagestyle{empty}


\begin{document}

\begin{center}
  \textsc{\large ColoState ~~ Spring 2018 ~~ MATH 617 ~~ Assignment 2}
\end{center}

\begin{center}
  \textrm{Due Fri. 03/02/2018}
\end{center}

\vglue 0.10in

\bigskip
\noindent
\textsc{Name:} \underline{Colin Roberts\hglue 1.5in} ~~
\textsc{CSUID:} \underline{829773631\hglue 1.5in}

\vskip 0.15in

\bigskip
\noindent
(20 points) \textit{Problem 1}. \quad
Define $\mathcal{A}=\{A\subseteq \R ~\colon~ \textrm{Either $A$ or $A^c$ is countable}\}$. For $A\in \mathcal{A}$, define $\mu(A)=0$ if $A$ is countable and $\mu(A)=1$ if $A^c$ is countable.
\begin{enumerate}[(i)]
\item Show that $\mu$ is a measure on $\mathcal{A}$.
\item Consider the outer measure $\mu^*$ on $\mathcal{P}(\R)$ induced by $\mu$. Show that $\mu^*$ is not finitely additive.
\end{enumerate}

\bigskip
\bigskip
\noindent
(20 points) \textit{Problem 2}. \quad
Assume $\mu$ is a measure defined on an algebra $\mathcal{A}$ consisting of subsets of a fixed nonempty set $X$. Let $(X,\mathcal{S}^*,\mu^*)$ be the measure space obtained through outer measure and the Caratheodory condition. Let $E\subset \mathcal{S}^*$ with $\mu^*(E)<+\infty$. Prove that for any $\epsilon>0$, there exists $A_\epsilon \in \mathcal{A}$ such that $\mu^*(E\Delta A_\epsilon)<\epsilon$.

\bigskip
\bigskip
\noindent
(20 points) \textit{Problem 3}. \quad
Prove the following regarding the Lebesgue outer measure $\lambda^*$:
\begin{enumerate}[(i)]
\item For $E\in \R$, $\lambda^*(E)=0$ iff $E$ is a null set.
\item For $E\in \R$, if $\lambda^*(E)=0$, then $E$ has an empty interior.
\end{enumerate}

\bigskip
\bigskip
\noindent
(20 points) \textit{Problem 4}. \quad
Let $E\subset \R$ be bounded. Prove that there exists a Borel set $F$ such that 
\begin{enumerate}[(i)]
\item $E\subseteq F$ and $\lambda^*(E)=\lambda(F)$.
\item For any Borel subset $G\subseteq (F\setminus E)$, we have $\lambda(G)=0$.
\end{enumerate}
Here $\lambda^*$ is the Lebesgue outer measure and $\lambda$ is the Lebesgue measure.

\bigskip
\bigskip
\noindent
(20 points) \textit{Problem 5}. \quad
Suppose that $A\subset \R$ is a Lebesgue nonmeasurable set and $0<\lambda^*(A)<\infty$. Prove that $\lambda(E)<\lambda^*(A)$ for any Lebesgue measurable set $E\subset A$.



\pagebreak

%%%%%%%%%%%%%%%%%%%%%%%%%%%%%%%%%%%%%%%%%%%%%%%%%%%%%%%%%%%%%%%%%%%%%%%%%%%%%%%%%%%%%%%%%%%%%%%%%%%%%%%%%%%%%%%%%%%%%
%%%%%%%%%%%%%%%%%%%%%%%%%PROBLEM%%%%%%%%%%%%%%%%%%%%%%%%%%%%%%%%%%%%%%%%%%%%%%%%%%%%%%%%%%%%%%%%%%%%%%%%%%%%%%%%%%%%%%%%%%%%%%%%%%%%%%%%%%%%%%%%%%%%%%%%%%%%%%%%%%%%%%%%%%%%%%%%%%%%%%%%%%%%%%%%%%%%%%%%%%%%%%%%%%%%%%%%%%%%%%%%%%%%%%%%%%

\noindent\textbf{Problem 1.} \quad
Define $\mathcal{A}=\{A\subseteq \R ~\colon~ \textrm{Either $A$ or $A^c$ is countable}\}$. For $A\in \mathcal{A}$, define $\mu(A)=0$ if $A$ is countable and $\mu(A)=1$ if $A^c$ is countable.
\begin{enumerate}[(i)]
\item Show that $\mu$ is a measure on $\mathcal{A}$.
\item Consider the outer measure $\mu^*$ on $\mathcal{P}(\R)$ induced by $\mu$. Show that $\mu^*$ is not finitely additive.
\end{enumerate}

\noindent\rule[0.5ex]{\linewidth}{1pt}

\begin{proof}~
\begin{enumerate}[(i)]
\item First we see that $\mu(\emptyset)=0$ since $\emptyset$ is countable. Next, we must show that $\mu$ is countably additive. Let $\{A_n\}_{n\in \N}$ be a countable collection of disjoint sets. Then if $\bigcup_{n\in \N} A_n$ is countable we have that each $A_n$ is countable and hence
\begin{align*}
\mu\left( \bigcup_{n\in \N} A_n \right)&=0=\sum_{n \in \N} \mu(A_n).
\end{align*}
If $\bigcup_{n\in \N}A_n$ is such that $\R\setminus \bigcup_{n\in \N}A_n$ is countable, then in order for each $A_n$ to be disjoint, we must have that only a single $A_i$ is so that $\R \setminus \bigcup_{n\in \N}A_i$ is countable and all other sets are countable by the construction of $\mathcal{A}$.  Hence
\begin{align*}
\mu\left(\bigcup_{n\in \N} A_n\right) &=1= \sum_{n\in \N}\mu(A_n) = \mu(A_i). 
\end{align*}
So we have that $\mu$ is a measure.

\item Consider a counter example of a finite collection of two sets $E_1=[0,1]$ and $E_2=[2,3]$.  Then note that
\begin{align*}
\mu^*(E_1\cup E_2)=1
\end{align*}
since there is no countable covering of $E_1\cup E_2$. But we also have, by the same logic, 
\begin{align*}
\mu^*(E_1)+\mu^*(E_2)=2.
\end{align*}
Hence, $\mu^*$ is not finitely additive.
\end{enumerate}
\end{proof}



\pagebreak

%%%%%%%%%%%%%%%%%%%%%%%%%%%%%%%%%%%%%%%%%%%%%%%%%%%%%%%%%%%%%%%%%%%%%%%%%%%%%%%%%%%%%%%%%%%%%%%%%%%%%%%%%%%%%%%%%%%%%
%%%%%%%%%%%%%%%%%%%%%%%%%PROBLEM%%%%%%%%%%%%%%%%%%%%%%%%%%%%%%%%%%%%%%%%%%%%%%%%%%%%%%%%%%%%%%%%%%%%%%%%%%%%%%%%%%%%%%%%%%%%%%%%%%%%%%%%%%%%%%%%%%%%%%%%%%%%%%%%%%%%%%%%%%%%%%%%%%%%%%%%%%%%%%%%%%%%%%%%%%%%%%%%%%%%%%%%%%%%%%%%%%%%%%%%%%


\noindent\textbf{Problem 2.} \quad
Assume $\mu$ is a measure defined on an algebra $\mathcal{A}$ consisting of subsets of a fixed nonempty set $X$. Let $(X,\mathcal{S}^*,\mu^*)$ be the measure space obtained through outer measure and the Caratheodory condition. Let $E\subset \mathcal{S}^*$ with $\mu^*(E)<+\infty$. Prove that for any $\epsilon>0$, there exists $A_\epsilon \in \mathcal{A}$ such that $\mu^*(E\Delta A_\epsilon)<\epsilon$.

\noindent\rule[0.5ex]{\linewidth}{1pt}


\begin{proof}
Note that the infimum definition of $\mu^*$ allows us to find a countable collection of pairwise disjoint sets $A_n \in \mathcal{A}$ so that $E\subseteq \bigcup_{n=1}^\infty A_n$ so that we have
\begin{align*}
\mu^*(E)+\epsilon\geq \sum_{n=1}^\infty \mu^*(A_n)\geq \mu^*(E).
\end{align*}
Then of course we have $\sum_{n=1}^\infty \mu^*(A_n) < \infty$ which weans that we have some $N\in \N$ so that 
\begin{align*}
\sum_{n=N+1}^\infty \mu^*(A_n) <\epsilon/2.
\end{align*}
We denote $A_\epsilon = \bigcup_{n=1}^N A_n$ and we find that
\begin{align*}
E\setminus A_\epsilon = E\setminus \left( \bigcup_{n=1}^N A_n \right) \subseteq \left( \bigcup_{n=1}^\infty A_n\right) \setminus \left( \bigcup_{n=1}^N A_n \right) = \bigcup_{n=N+1}^\infty A_n.
\end{align*}
It follows that we have
\begin{align*}
\mu^*(E\setminus A_\epsilon \leq \sum_{n=N+1}^\infty \mu^*(A_n)<\epsilon/2
\end{align*}
and that
\begin{align*}
A_\epsilon \setminus E \subseteq \left( \bigcup_{n=1}^\infty A_n \right) \setminus E.
\end{align*}
Then we have
\begin{align*}
\mu^*(A_\epsilon \setminus E ) \leq \sum_{n=1}^\infty \mu^*(A_n)-\mu^*(E).
\end{align*}
Lastly, the countable additivity of $\mu^*$ implies that
\begin{align*}
\mu^*(E\Delta A_\epsilon) = \mu^*(E\setminus A_\epsilon)+\mu^*(A_\epsilon \setminus E ) \leq \epsilon.
\end{align*}

\noindent \emph{Note: I saw a very similar solution in the text. I tried to clean it up and make it as nice as possible.}
\end{proof}


\pagebreak

%%%%%%%%%%%%%%%%%%%%%%%%%%%%%%%%%%%%%%%%%%%%%%%%%%%%%%%%%%%%%%%%%%%%%%%%%%%%%%%%%%%%%%%%%%%%%%%%%%%%%%%%%%%%%%%%%%%%%
%%%%%%%%%%%%%%%%%%%%%%%%%PROBLEM%%%%%%%%%%%%%%%%%%%%%%%%%%%%%%%%%%%%%%%%%%%%%%%%%%%%%%%%%%%%%%%%%%%%%%%%%%%%%%%%%%%%%%%%%%%%%%%%%%%%%%%%%%%%%%%%%%%%%%%%%%%%%%%%%%%%%%%%%%%%%%%%%%%%%%%%%%%%%%%%%%%%%%%%%%%%%%%%%%%%%%%%%%%%%%%%%%%%%%%%%%

\noindent\textbf{Problem 3.} \quad
Prove the following regarding the Lebesgue outer measure $\lambda^*$:
\begin{enumerate}[(i)]
\item For $E\subseteq \R$, $\lambda^*(E)=0$ iff $E$ is a null set.
\item For $E\subseteq \R$, if $\lambda^*(E)=0$, then $E$ has an empty interior.
\end{enumerate}

\noindent\rule[0.5ex]{\linewidth}{1pt}

\begin{proof}~
\begin{enumerate}[(i)]
\item Let $E\subseteq \R$ and that $\lambda^*(E)=0$.  Then 
\begin{align*}
&\lambda^*(E)=0\\
&\iff \inf\left\{\sum_{i=1}^\infty \lambda(I_i) ~\vert~ I_i \in \mathcal{I} \forall i, I_i \cap I_j = \emptyset ~\textrm{for } i\neq j \textrm{ and } E\subseteq \bigcup_{i=1}^\infty I_i\right\}=0\\
&\iff E \textrm{ for all $\epsilon>0$, can be covered by a countable family of intervals with $\sum_{n=1}^\infty \lambda(I_n)\leq \epsilon$}\\
&\iff E \textrm{ is a null set.}
\end{align*}
\item Suppose that $E\subseteq \R$ with $\lambda^*(E)=0$.  Suppose that $\textrm{int}(E) \neq \emptyset$. Then we have that there exists an open interval $I \subset E$ and we necessarily have that $\lambda^*(I)\geq 0$. Hence, $E$ must have an empty interior.
\end{enumerate}
\end{proof}

\pagebreak



%%%%%%%%%%%%%%%%%%%%%%%%%%%%%%%%%%%%%%%%%%%%%%%%%%%%%%%%%%%%%%%%%%%%%%%%%%%%%%%%%%%%%%%%%%%%%%%%%%%%%%%%%%%%%%%%%%%%%
%%%%%%%%%%%%%%%%%%%%%%%%%PROBLEM%%%%%%%%%%%%%%%%%%%%%%%%%%%%%%%%%%%%%%%%%%%%%%%%%%%%%%%%%%%%%%%%%%%%%%%%%%%%%%%%%%%%%%%%%%%%%%%%%%%%%%%%%%%%%%%%%%%%%%%%%%%%%%%%%%%%%%%%%%%%%%%%%%%%%%%%%%%%%%%%%%%%%%%%%%%%%%%%%%%%%%%%%%%%%%%%%%%%%%%%%%

\noindent\textbf{Problem 4.} \quad
Let $E\subset \R$ be bounded. Prove that there exists a Borel set $F$ such that 
\begin{enumerate}[(i)]
\item $E\subseteq F$ and $\lambda^*(E)=\lambda(F)$.
\item For any Borel subset $G\subseteq (F\setminus E)$, we have $\lambda(G)=0$.
\end{enumerate}
Here $\lambda^*$ is the Lebesgue outer measure and $\lambda$ is the Lebesgue measure.

\noindent\rule[0.5ex]{\linewidth}{1pt}


\begin{proof}~
\begin{enumerate}[(i)]
\item By definition of $\lambda^*$, we know that $\lambda^*(E)$ is the infimum of the measure of all possible coverings of $E$ by disjoint intervals in $\R$.  Hence, for any $m\in \N$ and $\frac{1}{m}>0$ we have a countable collection of pairwise disjoint intervals $\{I_n^m\}_{n\in \N}$ so that $E\subseteq \bigcup_{n\in \N} I_n^m$. Since $E$ is bounded, $\lambda^*(E)<\infty$ and we have
\begin{align*}
\lambda\left(\bigcup_{n\in \N} I_n^m \right) - \lambda^*(E)<\frac{1}{m}.
\end{align*}
Since this is true for all $\frac{1}{m}>0$, we necessarily have $F\subseteq \bigcap_{m\in \N}\bigcup_{n\in \N} I_n^m$. Then with this, we see 
\begin{align*}
\lambda(F) - \lambda^*(E)\leq \lambda \left( \bigcap_{m\in \N}\bigcup_{n\in \N} I_n^m \right) - \lambda^*(E)=0\\
\implies \lambda^*(E)&=\lambda(F).
\end{align*}


\item Let $G\subseteq (F\setminus E)$ be such that $G=\bigcup_{n\in \N} J_n$ where $J_n$ are Borel sets. Note that since $G$ is a Borel set we have
\begin{align*}
\lambda(G)=\lambda^*(G).
\end{align*}
We then have
\begin{align*}
\lambda^*(F\setminus E)&=0
\end{align*}
which, since $G\subseteq F\setminus E$, means
\begin{align*}
\lambda^*(G)=0
\end{align*}
\end{enumerate}
\end{proof}

\pagebreak



%%%%%%%%%%%%%%%%%%%%%%%%%%%%%%%%%%%%%%%%%%%%%%%%%%%%%%%%%%%%%%%%%%%%%%%%%%%%%%%%%%%%%%%%%%%%%%%%%%%%%%%%%%%%%%%%%%%%%
%%%%%%%%%%%%%%%%%%%%%%%%%PROBLEM%%%%%%%%%%%%%%%%%%%%%%%%%%%%%%%%%%%%%%%%%%%%%%%%%%%%%%%%%%%%%%%%%%%%%%%%%%%%%%%%%%%%%%%%%%%%%%%%%%%%%%%%%%%%%%%%%%%%%%%%%%%%%%%%%%%%%%%%%%%%%%%%%%%%%%%%%%%%%%%%%%%%%%%%%%%%%%%%%%%%%%%%%%%%%%%%%%%%%%%%%%

\noindent\textbf{Problem 5.} \quad
Suppose that $A\subset \R$ is a Lebesgue nonmeasurable set and $0<\lambda^*(A)<\infty$. Prove that $\lambda(E)<\lambda^*(A)$ for any Lebesgue measurable set $E\subset A$.

\noindent\rule[0.5ex]{\linewidth}{1pt}

\begin{proof}
First, note that $\lambda^*(E)=\lambda(E)$ since $E$ is measurable. Now, monotonicity of $\lambda^*$ implies the following:
\begin{align*}
\lambda^*(E)\leq \lambda^*(A).
\end{align*}
Now, by the Caratheodory condition, we can do the following:
\begin{align*}
\lambda^*(A)=\lambda^*(E\cap A)+\lambda^*(E^c\cap A).
\end{align*}
Then suppose for a contradiction that we in fact have $\lambda^*(E)=\lambda^*(A)$.  This means that
\begin{align*}
0&=\lambda^*(E)-\lambda^*(A)\\
\iff 0&= \lambda^*(E)-\lambda^*(E\cap A)-\lambda^*(E^c \cap A)\\
\iff 0&= \lambda^*(E^c\cap A), && \textrm{since $\lambda^*(E)=\lambda^*(E\cap A)$ because $E\subset A$}\\
\iff A &&&\textrm{is measurable}.
\end{align*}
But this is a contradiction since we supposed that $A$ is non-measurable. Hence, we must have that $\lambda(E)<\lambda^*(A)$ since the case for equality provides a contradiction.
\end{proof}

\pagebreak

\end{document}



