\documentclass[leqno]{article}
\usepackage[utf8]{inputenc}
\usepackage[T1]{fontenc}
\usepackage{amsfonts}
%\usepackage{fourier}
%\usepackage{heuristica}
\usepackage{enumerate}
\author{Colin Roberts}
\title{MATH 519, Homework 3}
\usepackage[left=3cm,right=3cm,top=3cm,bottom=3cm]{geometry}
\usepackage{amsmath}
\usepackage[thmmarks, amsmath, thref]{ntheorem}
%\usepackage{kbordermatrix}
\usepackage{mathtools}
\usepackage{color}
\usepackage{hyperref}

\theoremstyle{nonumberplain}
\theoremheaderfont{\itshape}
\theorembodyfont{\upshape:}
\theoremseparator{.}
\theoremsymbol{\ensuremath{\square}}
\newtheorem{proof}{Proof}
\theoremsymbol{\ensuremath{\square}}
\newtheorem{lemma}{Lemma}
\theoremsymbol{\ensuremath{\blacksquare}}
\newtheorem{solution}{Solution}
\theoremseparator{. ---}
\theoremsymbol{\mbox{\texttt{;o)}}}
\newtheorem{varsol}{Solution (variant)}

\newcommand{\id}{\mathrm{Id}}
\newcommand{\R}{\mathbb{R}}
\newcommand{\N}{\mathbb{N}}
\newcommand{\Z}{\mathbb{Z}}
\newcommand{\C}{\mathbb{C}}

\begin{document}
\maketitle
\begin{large}
\begin{center}
Solutions
\end{center}
\end{large}

%%%%%%%%%%%%%%%%%%%%%%%%%%%%%%%%%%%%%%%%%%%%%%%%%%%%%%%%%%%%%%%%%%%%%%%%%%%%%%%%%%%%%%%%%%%%%%%%%%%%%%%%%%%%%%%%%%%%%
%%%%%%%%%%%%%%%%%%%%%%%%%PROBLEM%%%%%%%%%%%%%%%%%%%%%%%%%%%%%%%%%%%%%%%%%%%%%%%%%%%%%%%%%%%%%%%%%%%%%%%%%%%%%%%%%%%%%%%%%%%%%%%%%%%%%%%%%%%%%%%%%%%%%%%%%%%%%%%%%%%%%%%%%%%%%%%%%%%%%%%%%%%%%%%%%%%%%%%%%%%%%%%%%%%%%%%%%%%%%%%%%%%%%%%%%%

\noindent\textbf{Problem 1. (S \& S 3.1.)} Using Euler's formula
\[
\sin \pi z = \frac{e^{i\pi z}-e^{-i\pi z}}{2i},
\]
show that the complex zeros of $\sin \pi z$ are exactly at the integers, and that they are each of order 1. 

Calculate the residue of $1/\sin \pi z$ at $z=n\in \Z$.

\begin{proof}
First, we have that 
\begin{align*}
\sin \pi z &= \frac{e^{i\pi z}-e^{-i \pi z}}{2i}=0\\
\iff \frac{e^{-i\pi z}(e^{2i\pi z}-1)}{2i}=0,
\end{align*}
and so we must have that $e^{2i\pi z}=1$ which means that $z$ must be an integer. Clearly these are also zeros of order 1.

To calculate the residue, we take 
\begin{align*}
\lim_{z\to n} \frac{z-n}{\sin \pi z} & = \frac{1}{\pi} && \textrm{via L'Hopital's rule.}
\end{align*}
\end{proof}

\vspace*{1cm}


\noindent\textbf{Problem 2.(S \& S 3.2)} Evaluate the integral
\[
\int_{-\infty}^\infty \frac{dx}{1+x^4}.
\]
Where are the poles of $1/(1+z^4)$?

\begin{proof}
\end{proof}

\vspace*{1cm}

\noindent\textbf{Problem 3. (S \& S 3.6.)} Show that
\[
\int_{-\infty}^\infty \frac{dx}{(1+x^2)^{n+1}} = \frac{1\cdot 3 \cdot 5 \cdots (2n-1)}{2\cdot 4\cdot 6 \cdots (2n)}\cdot \pi.
\]


\begin{proof}
\end{proof}

\vspace*{1cm}

\noindent\textbf{Problem 4. (S \& S 3.14.)} Prove that all entire functions that are also injective take the form $f(z)=az+b$ with $a,b\in \mathbb{C}$ and $a\neq 0$.\\
\noindent\emph{Hint:} Apply the Casorati-Weierstrass theorem to $f(1/z)$.]


\begin{proof}
\end{proof}


\vspace*{1cm}

\noindent\textbf{Problem 5. (S \& S 3.21ab)}  Certain sets have geometric properties that guarantee they are simply connected
\begin{enumerate}[(a)]
\item An open set $\Omega\in \mathbb{C}$ is \textbf{convex} if for any two points in $\Omega$, the straight line segment between them is contained in $\Omega$. Prove that a convex open set is simply connected.
\item More generally, an open set $\Omega \in \mathbb{C}$ is \textbf{star-shaped} if there exists a point $z_0\in \Omega$ such that for any $z\in \Omega$, the straight line segment between $z$ and $z_0$ is contained in $\Omega$. Prove that a star shaped open set is simply connected. Conclude that the slit plane $\mathbb{C}\setminus \{(-\infty,0]\}$ (and more generally any sector, convex or not) is simply connected.
\end{enumerate}

\begin{proof}
\end{proof}

\vspace*{1cm}

\noindent\textbf{Problem 6.} Find the residues at the (obvious) singularities:
\begin{enumerate}[(a)]
\item $\frac{1}{z+z^2}$.
\item $z\cos \left( \frac{1}{z}\right)$.
\item $\frac{z-\sin z}{z}$.
\item $z^2e^{1/z}$.
\end{enumerate} 


\begin{proof}
\end{proof}


\end{document}



