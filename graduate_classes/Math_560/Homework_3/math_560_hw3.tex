\documentclass[leqno]{article}
\usepackage[utf8]{inputenc}
\usepackage[T1]{fontenc}
\usepackage{amsfonts}
\usepackage{fourier}
\usepackage{heuristica}
\usepackage{enumerate}
\author{Colin Roberts}
\title{MATH 560, Homework 3}
\usepackage[left=3cm,right=3cm,top=3cm,bottom=3cm]{geometry}
\usepackage{amsmath}
\usepackage[thmmarks, amsmath, thref]{ntheorem}
\usepackage{kbordermatrix}
\usepackage{mathtools}

\theoremstyle{nonumberplain}
\theoremheaderfont{\itshape}
\theorembodyfont{\upshape:}
\theoremseparator{.}
\theoremsymbol{\ensuremath{\square}}
\newtheorem{proof}{Proof}
\theoremsymbol{\ensuremath{\square}}
\newtheorem{lemma}{Lemma}
\theoremsymbol{\ensuremath{\blacksquare}}
\newtheorem{solution}{Solution}
\theoremseparator{. ---}
\theoremsymbol{\mbox{\texttt{;o)}}}
\newtheorem{varsol}{Solution (variant)}

\newcommand{\tr}{\mathrm{tr}}

\begin{document}
\maketitle
\begin{large}
\begin{center}
Solutions
\end{center}
\end{large}
\pagebreak

%%%%%%%%%%%%%%%%%%%%%%%%%%%%%%%%%%%%%%%%%%%%%%%%%%%%%%%%%%%%%%%%%%%%%%%%%%%%%%%%%%%%%%%%%%%%%%%%%%%%%%%%%%%%%%%%%%%%%
%%%%%%%%%%%%%%%%%%%%%%%%%PROBLEM%%%%%%%%%%%%%%%%%%%%%%%%%%%%%%%%%%%%%%%%%%%%%%%%%%%%%%%%%%%%%%%%%%%%%%%%%%%%%%%%%%%%%%%%%%%%%%%%%%%%%%%%%%%%%%%%%%%%%%%%%%%%%%%%%%%%%%%%%%%%%%%%%%%%%%%%%%%%%%%%%%%%%%%%%%%%%%%%%%%%%%%%%%%%%%%%%%%%%%%%%%
\noindent\textbf{Problem 1.} The discrete Fourier transform of a fector $f\in \mathcal{C}^n$ may be written
\begin{equation}
\hat{f}_j = \sum_{k=0}^{n-1} f_k \exp(-2\pi ijk/n), \textrm{~~~} j=0,...,n-1
\end{equation}
while the inverse transform is given by
\begin{equation}
f_k=\frac{1}{n}\sum_{j=0}^{n-1} \hat{f}_j \exp(2\pi ijk/n), \textrm{~~~} k=0,...,n-1
\end{equation}
Define the Fourier basis vector to be
\[
v_j=(1,z^j,...,z^{(n-1)j})^T
\]
The Fourier basis expansion can be written
\begin{equation}
\hat{f} = \sum_{k=0}^{n-1} f_k \bar{v}_k
\end{equation}
and the inverse
\begin{equation}
f=\frac{1}{n} \sum_{j=0}^{n-1} \hat{f}_j v_j
\end{equation}
where $z=\exp(2\pi i/n)$ and $\bar{v}_k$ is the complex conjugate of $v$.

Show that formulas $(1)$ and $(2)$ can be obtained from $(3)$ and $(4)$, respectively.

\noindent\rule[0.5ex]{\linewidth}{1pt}

\begin{proof}
For $(1)$ to $(3)$ we have
\begin{align*}
\hat{f_j} &= \sum_{k=0}^{n-1} f_k \exp(-2\pi ijk/n), \textrm{~~~} j=0,...,n-1\\
&=\sum_{k=0}^{n-1} f_k \bar{v}_k \textrm{~~~} j=0,...,n-1\\
\implies \hat{f}&=\sum_{k=0}^{n-1} f_k \bar{v}_k .
\end{align*}  

For $(2)$ to $(4)$ we have
\begin{align*}
f_k &= \frac{1}{n}\sum_{k=0}^{n-1} \hat{f_j} \exp(2\pi ijk/n), \textrm{~~~} k=0,...,n-1\\
&=\frac{1}{n}\sum_{j=0}^{n-1} \hat{f_j} v_j \textrm{~~~} k=0,...,n-1\\
\implies f&=\frac{1}{n}\sum_{j=0}^{n-1} \hat{f_j} v_j .
\end{align*}  
\end{proof}

\pagebreak

%%%%%%%%%%%%%%%%%%%%%%%%%%%%%%%%%%%%%%%%%%%%%%%%%%%%%%%%%%%%%%%%%%%%%%%%%%%%%%%%%%%%%%%%%%%%%%%%%%%%%%%%%%%%%%%%%%%%%
%%%%%%%%%%%%%%%%%%%%%%%%%PROBLEM%%%%%%%%%%%%%%%%%%%%%%%%%%%%%%%%%%%%%%%%%%%%%%%%%%%%%%%%%%%%%%%%%%%%%%%%%%%%%%%%%%%%%%%%%%%%%%%%%%%%%%%%%%%%%%%%%%%%%%%%%%%%%%%%%%%%%%%%%%%%%%%%%%%%%%%%%%%%%%%%%%%%%%%%%%%%%%%%%%%%%%%%%%%%%%%%%%%%%%%%%%


\noindent\textbf{Problem 2.} Compute the Discrete Fourier Transform of $f$ where 
\begin{enumerate}[(a)]
\item $f=v_3$ and $n=8$.
\item $f=(1,2,-1,4)$.
\end{enumerate}

\noindent\rule[0.5ex]{\linewidth}{1pt}

\begin{solution}[Part (a)]
Using $v_j^T\bar{v_k}=n\delta_{jk}$ we have that 
\begin{align*}
\hat{f}&=\sum_{k=0}^{7}f_k \bar{v_k}\\
&= \sum_{k=0}^{7}v_3 \bar{v_k}\\
&=(0,0,1,0,0,0,0,0)
\end{align*}
\end{solution}

\begin{solution}[Part (b)]
\begin{align*}
\hat{f}&=\sum_{k=0}^{3}f_k \bar{v_k}\\
&= \bar{v_0}+2\bar{v_1}-\bar{v_2}+4\bar{v_3}
\end{align*}
\end{solution}

\pagebreak


%%%%%%%%%%%%%%%%%%%%%%%%%%%%%%%%%%%%%%%%%%%%%%%%%%%%%%%%%%%%%%%%%%%%%%%%%%%%%%%%%%%%%%%%%%%%%%%%%%%%%%%%%%%%%%%%%%%%%
%%%%%%%%%%%%%%%%%%%%%%%%%PROBLEM%%%%%%%%%%%%%%%%%%%%%%%%%%%%%%%%%%%%%%%%%%%%%%%%%%%%%%%%%%%%%%%%%%%%%%%%%%%%%%%%%%%%%%%%%%%%%%%%%%%%%%%%%%%%%%%%%%%%%%%%%%%%%%%%%%%%%%%%%%%%%%%%%%%%%%%%%%%%%%%%%%%%%%%%%%%%%%%%%%%%%%%%%%%%%%%%%%%%%%%%%%


\noindent\textbf{Problem 3. (\S 2.2 Problem 2a.)} Let $\beta$ and $\gamma$ be the standard ordered bases for $\mathbb{R}^n \to \mathbb{R}^m$, compute $[T]_\beta^\gamma$.

\noindent\rule[0.5ex]{\linewidth}{1pt}

\begin{solution}
$[T]_\beta^\gamma$ can be found by,
\begin{align*}
\begin{bmatrix}
T_{11} & T_{12}\\
T_{21} & T_{22}\\
T_{31} & T_{32}
\end{bmatrix}
\begin{bmatrix}
a_1\\
a_2
\end{bmatrix}
&=
\begin{bmatrix}
T_{11}a_1+T_{12}a_2\\
T_{21}a_1+T_{22}a_2\\
T_{31}a_1+T_{32}a_2
\end{bmatrix}
=
\begin{bmatrix}
2a_1-a_2\\
3a_1+4a_2\\
a_1
\end{bmatrix}\\
\implies
\begin{bmatrix}
2 & -1\\
3 & 4\\
1 & 0
\end{bmatrix}
&= [T]_\beta^\gamma
\end{align*}
\end{solution}

\pagebreak



%%%%%%%%%%%%%%%%%%%%%%%%%%%%%%%%%%%%%%%%%%%%%%%%%%%%%%%%%%%%%%%%%%%%%%%%%%%%%%%%%%%%%%%%%%%%%%%%%%%%%%%%%%%%%%%%%%%%%
%%%%%%%%%%%%%%%%%%%%%%%%%PROBLEM%%%%%%%%%%%%%%%%%%%%%%%%%%%%%%%%%%%%%%%%%%%%%%%%%%%%%%%%%%%%%%%%%%%%%%%%%%%%%%%%%%%%%%%%%%%%%%%%%%%%%%%%%%%%%%%%%%%%%%%%%%%%%%%%%%%%%%%%%%%%%%%%%%%%%%%%%%%%%%%%%%%%%%%%%%%%%%%%%%%%%%%%%%%%%%%%%%%%%%%%%%


\noindent\textbf{Problem 4. (\S 2.2 Problem 4.)}  Define
\begin{align*}
T\colon M_{2\times 2}(\mathbb{R})\to P_2(\mathbb{R}) &\textrm{by} &T
\begin{bmatrix}
a & b\\
c & d
\end{bmatrix}
=(a+b)+(2d)x+bx^2
\end{align*}
Let
\begin{align*}
\beta = \left\{
\begin{bmatrix}
1 & 0\\
0 & 0
\end{bmatrix},
\begin{bmatrix}
0 & 1\\
0 & 0
\end{bmatrix},
\begin{bmatrix}
0 & 0\\
1 & 0
\end{bmatrix},
\begin{bmatrix}
0 & 0\\
0 & 1
\end{bmatrix}\right\} \textrm{~~~~~and~~~~~} \gamma=\{1,x,x^2\}.
\end{align*}
Compute $[T]_\beta^\gamma$.

\noindent\rule[0.5ex]{\linewidth}{1pt}

\begin{solution}
\begin{align*}
\begin{bmatrix}
T_{11} & T_{12} & T_{13} & T_{14}\\
T_{21} & T_{22} & T_{23} & T_{24}\\
T_{31} & T_{32} & T_{33} & T_{34}
\end{bmatrix}
\begin{bmatrix}
a\\
b\\
c\\
d
\end{bmatrix}
&=
\begin{bmatrix}
a+b\\
2d\\
b
\end{bmatrix}\\
\implies 
\begin{bmatrix}
1 & 1 & 0 & 0 \\
0 & 0 & 0 & 2 \\
0 & 2 & 0 & 0
\end{bmatrix}
&=[T]_\beta^\gamma
\end{align*}
\end{solution}

\pagebreak


%%%%%%%%%%%%%%%%%%%%%%%%%%%%%%%%%%%%%%%%%%%%%%%%%%%%%%%%%%%%%%%%%%%%%%%%%%%%%%%%%%%%%%%%%%%%%%%%%%%%%%%%%%%%%%%%%%%%%
%%%%%%%%%%%%%%%%%%%%%%%%%PROBLEM%%%%%%%%%%%%%%%%%%%%%%%%%%%%%%%%%%%%%%%%%%%%%%%%%%%%%%%%%%%%%%%%%%%%%%%%%%%%%%%%%%%%%%%%%%%%%%%%%%%%%%%%%%%%%%%%%%%%%%%%%%%%%%%%%%%%%%%%%%%%%%%%%%%%%%%%%%%%%%%%%%%%%%%%%%%%%%%%%%%%%%%%%%%%%%%%%%%%%%%%%%


\noindent\textbf{Problem 5. (\S 2.2 (Problem 8.)} Let $V$ be $n$-dimensional vector space with an ordered basis $\beta$. Define $T\colon V\to \mathbb{F}^n$ by $T(x)=[x]_\beta$. Prove that $T$ is linear.

\noindent\rule[0.5ex]{\linewidth}{1pt}

\begin{proof}
To show $T$ is linear, consider $u,v\in V$ and $a \in \mathbb{F}$.  Then
\begin{align*}
T(au+v)&=[au+v]_\beta\\
&=[au]_\beta+[v]_\beta\\
&=a[u]_\beta +[v]_\beta\\
&=aT(u)+T(v)
\end{align*}
\end{proof}


\pagebreak


%%%%%%%%%%%%%%%%%%%%%%%%%%%%%%%%%%%%%%%%%%%%%%%%%%%%%%%%%%%%%%%%%%%%%%%%%%%%%%%%%%%%%%%%%%%%%%%%%%%%%%%%%%%%%%%%%%%%%
%%%%%%%%%%%%%%%%%%%%%%%%%PROBLEM%%%%%%%%%%%%%%%%%%%%%%%%%%%%%%%%%%%%%%%%%%%%%%%%%%%%%%%%%%%%%%%%%%%%%%%%%%%%%%%%%%%%%%%%%%%%%%%%%%%%%%%%%%%%%%%%%%%%%%%%%%%%%%%%%%%%%%%%%%%%%%%%%%%%%%%%%%%%%%%%%%%%%%%%%%%%%%%%%%%%%%%%%%%%%%%%%%%%%%%%%%


\noindent\textbf{Problem 6. (\S 2.2 Problem 15.)} Let $V$ and $W$ be vector spaces, and let $S$ be a subset of $V$. Define $S^0=\{T \in \mathcal{L}(V,W) \vert T(x)=0 \textrm{ for all } x\in S\}$. Prove the following statements.
\begin{enumerate}[(a)]
\item $S^0$ is a subspace of $\mathcal{L}(V,W)$.
\item If $S_1$ and $S_2$ are subsets of $V$ and $S_1\subseteq S_2$, then $S_2^0 \subseteq S_1^0$.
\item If $V_1$ and $V_2$ are subspaces of $V$, then $(V_1+V_2)^0 = V_1^0\cap V_2^0$.
\end{enumerate}

\noindent\rule[0.5ex]{\linewidth}{1pt}

\begin{proof}[Part (a)]
To show $S^0$ is a subspace we need to show closure under addition and scalar multiplication as well as the existence of the zero vector.  

Surely $T=0$ is in $S^0$ as $0(x)=0$ for any $x\in V$ so for any $y\in S$, $0(y)=0$. Then let $T_1,T_2 \in S^0$ and $x\in S$, then $(T_1+T_2)(x)=T_1(x)+T_2(x)=0+0=0$.  So $S^0$ is closed under addition.  Finally, let $a\in \mathbb{F}$, $T\in S^0$, and $x\in S$, then $(aT)(x)=aT(x)=a0=0$. So $S^0$ is closed under scalar multiplication.
\end{proof}

\begin{proof}[Part (b)]
First, suppose that $S_2^0 \supset S_1^0$. Then we have that $\exists T \in S_2^0$ so that $T\notin S_1^0$.  Thus $\exists x \in S_1$ so that $T(x)=0$.  But $S_1\subseteq S_2$ which implies that $x\in S_2$ and we know that $T\in S_2^0$ so $T(x)=0$ which is a contradiction.  Thus we have that $S_2^0\subseteq S_1^0$.
\end{proof}

\begin{proof}[Part (c)]
For $(V_1+V_2)^0\subseteq V_1^0\cap V_2^0$ we let $v\in (V_1+V_2)^0=V_1^0+V_2^0$. Suppose for a contradiction that $v\notin V_1^0\cap V_2^0$ and thus $v$ is an element in $V_1\cup V_2$ so that $T(v)\neq 0$.  But this contradicts $v\in (V_1+V_2)^0$ and thus $v\in V_1^0\cap V_2^0$.
For the other inclusion, let $v\in V_1^0\cap V_2^0$. Thus we have $v\in V_1^0$ and $v\in V_2^0$. Thus suppose we have that $v\notin V_1^0 + V_2^0$. Then we can write $v=v_1+v_2$ with $v_1\in V_1$ and $v_2\in V_2$ with $T(v_1+v_2)\neq 0$. But we have that $T(v)=T(v_1+v_2)=0$ since $v\in V_1^0\cap V_2^0$. Thus $V_1^0\cap V_2^0\subseteq (V_1+V_2)^0$. So both containments imply that $(V_1+V_2)^0=V_1^0\cap V_2^0$.
\end{proof}
\pagebreak




%%%%%%%%%%%%%%%%%%%%%%%%%%%%%%%%%%%%%%%%%%%%%%%%%%%%%%%%%%%%%%%%%%%%%%%%%%%%%%%%%%%%%%%%%%%%%%%%%%%%%%%%%%%%%%%%%%%%%
%%%%%%%%%%%%%%%%%%%%%%%%%PROBLEM%%%%%%%%%%%%%%%%%%%%%%%%%%%%%%%%%%%%%%%%%%%%%%%%%%%%%%%%%%%%%%%%%%%%%%%%%%%%%%%%%%%%%%%%%%%%%%%%%%%%%%%%%%%%%%%%%%%%%%%%%%%%%%%%%%%%%%%%%%%%%%%%%%%%%%%%%%%%%%%%%%%%%%%%%%%%%%%%%%%%%%%%%%%%%%%%%%%%%%%%%%


\noindent\textbf{Problem 7. (\S 2.5 Problem 2d.)} For each of the following pairs of ordered bases $\beta$ and $\beta'$ for $\mathbb{R}^2$, find the change of coordinate matrix that changes $\beta'$-coordinates into $\beta$-coordinates.
\[
\beta = \{(-4,3),(2,-1)\} \textrm{ and } \beta' = \{(2,1),(-4,1)\}
\]

\noindent\rule[0.5ex]{\linewidth}{1pt}

\begin{proof}
\begin{align*}
(-4,3)=a(2,1)+b(-4,1)\\
(2,-1)=c(2,1)+d(-4,1)\\
\implies a=\frac{8}{6},b=\frac{10}{6},c=\frac{1}{3},d={-2}{3}
\end{align*}
So we have
\begin{align*}
Q=
\begin{bmatrix}
\frac{4}{3} & \frac{1}{3}\\
\frac{5}{3} & \frac{-2}{3}
\end{bmatrix}
\end{align*}
\end{proof}

\pagebreak


%%%%%%%%%%%%%%%%%%%%%%%%%%%%%%%%%%%%%%%%%%%%%%%%%%%%%%%%%%%%%%%%%%%%%%%%%%%%%%%%%%%%%%%%%%%%%%%%%%%%%%%%%%%%%%%%%%%%%
%%%%%%%%%%%%%%%%%%%%%%%%%PROBLEM %%%%%%%%%%%%%%%%%%%%%%%%%%%%%%%%%%%%%%%%%%%%%%%%%%%%%%%%%%%%%%%%%%%%%%%%%%%%%%%%%%%%%%%%%%%%%%%%%%%%%%%%%%%%%%%%%%%%%%%%%%%%%%%%%%%%%%%%%%%%%%%%%%%%%%%%%%%%%%%%%%%%%%%%%%%%%%%%%%%%%%%%%%%%%%%%%%%%%%%%


\noindent\textbf{Problem 8. (\S 2.5 Problem 6a.)} For each matrix $A$ and ordered basis $\beta$, find $[L_A]_\beta$. Also, find an invertible matrix $Q$ such that $[L_A]_\beta = Q^{-1} A Q$. 
\[
A=
\begin{bmatrix}
1 & 3\\
1 & 1
\end{bmatrix} \textrm{ and } 
\beta=
\left\{
\begin{bmatrix}
1\\
1
\end{bmatrix},
\begin{bmatrix}
1\\
2
\end{bmatrix}
\right\}
\]

\noindent\rule[0.5ex]{\linewidth}{1pt}

\begin{proof}
\begin{align*}
Q=
\begin{bmatrix}
1 & 1\\
1 & 2
\end{bmatrix}\\
Q^{-1}=
\begin{bmatrix}
2 & -2\\
-1 & 1
\end{bmatrix}
\end{align*}
So we have
\begin{align*}
[L_A]_\beta &=
Q^{-1}AQ=\begin{bmatrix}
2 & -2\\
-1 & 1
\end{bmatrix}
\begin{bmatrix}
1 & 3\\
1 & 1
\end{bmatrix}
\begin{bmatrix}
1 & 1\\
1 & 2
\end{bmatrix}\\
&=
\begin{bmatrix}
6 & 11\\
-2 & 4
\end{bmatrix}
\end{align*}
\end{proof}

\pagebreak


%%%%%%%%%%%%%%%%%%%%%%%%%%%%%%%%%%%%%%%%%%%%%%%%%%%%%%%%%%%%%%%%%%%%%%%%%%%%%%%%%%%%%%%%%%%%%%%%%%%%%%%%%%%%%%%%%%%%%
%%%%%%%%%%%%%%%%%%%%%%%%%PROBLEM %%%%%%%%%%%%%%%%%%%%%%%%%%%%%%%%%%%%%%%%%%%%%%%%%%%%%%%%%%%%%%%%%%%%%%%%%%%%%%%%%%%%%%%%%%%%%%%%%%%%%%%%%%%%%%%%%%%%%%%%%%%%%%%%%%%%%%%%%%%%%%%%%%%%%%%%%%%%%%%%%%%%%%%%%%%%%%%%%%%%%%%%%%%%%%%%%%%%%%%%


\noindent\textbf{Problem 9. (\S 2.5 Problem 10.)} Prove that if $A$ and $B$ are similar $n\times n$ matrices, then $\tr(A)=\tr(B)$. \emph{Hint:} Use Exercise of \S 2.3.


\noindent\rule[0.5ex]{\linewidth}{1pt}

\begin{proof}
We have from Exercise 13, $\tr(AB)=\tr(BA)$ so now
\begin{align*}
\tr(B)&=\tr(Q^{-1}AQ)\\
&=\tr((Q^{-1}Q)A)\\
&=\tr(A)
\end{align*}
\end{proof}

\pagebreak



\end{document}

