\documentclass[conf]{new-aiaa}
%\documentclass[journal]{new-aiaa} for journal papers
\usepackage[utf8]{inputenc}

%\usepackage{enumerate}
%\usepackage{graphicx}
%\usepackage{amsmath}
\usepackage[version=4]{mhchem}
\usepackage{siunitx}
\usepackage{longtable,tabularx}
\setlength\LTleft{0pt} 
\usepackage{preamble}

\setlength{\marginparwidth}{2.5cm}
\usepackage[colorinlistoftodos]{todonotes}
%\usepackage[disable]{todonotes} % This will disable viewing Todos
\setuptodonotes{size=\scriptsize,backgroundcolor=red!15!white} 


\newcommandx{\cameron}[2][1=]{\todo[linecolor=cyan,backgroundcolor=cyan!25,bordercolor=cyan,#1]{\tiny Cameron: #2}}
\newcommandx{\colin}[2][1=]{\todo[linecolor=green,backgroundcolor=green!25,bordercolor=green,#1]{\tiny Colin: #2}}

\title{Spinor Equations of Fluid Plasmas and the Coordinate Free Vlasov Equation}

\author{Colin P. Roberts\footnote{Ph.D. Candidate, Colorado State University, 1874 Campus Delivery Fort Collins, CO 80523-1874}}
\affil{Colorado State University, Fort Collins, Colorado, 80523}
\author{Alan G. Hylton\footnote{Engineer, LCN}}
\affil{NASA Glenn Research Center, Cleveland, Ohio, 44133, USA}

\begin{document}

\maketitle

\begin{abstract}
The kinematics and self interaction of charges and neutral particles is described by combining Maxwell's equations of electromagnetism with the Boltzmann equation of gas/fluid kinematics. In tandem, we refer to the set of these equations as the Vlasov equations. One can note that the theory of electromagnetism is purely topological and, in fact, through the lens of relativity, Maxwell's equations are purely topological. This perspective is immensely powerful, but it does not immediately ascend to providing us with a topological version of Vlasov's equation due to Boltzmann's equation. There have been versions of Vlasov's equations which are, at least, coordinate free. 

Our goals are to understand the coordinate free Vlasov equations and seek to determine a topological version, if possible. We take two novel approaches. First, we consider relativistic phase space and construct the Boltzmann equations and investigate the symplectic and contact structures therein. Second, we take an approach to describing the relativistic motion of single charged particles via differential equations of spinors and attempt to extend this to a fluid of charges. An analogy forms -- the equations of motion of lone charges immersed in an electromagnetic field follows paths in the Poincar\'e group much like the motion of rigid body can be seen as a path (in fact, geodesics) in the Euclidean group as described by Vladimir Arnol'd. In the same vein, Arnol'd determined that incompressible fluids follow geodesics on the infinite dimensional group of volume preserving diffeomorphisms. This begs the question of whether ideal charged fluids behave similarly but on, perhaps, some other group.
\end{abstract}

%\section{Nomenclature}
%
%{\renewcommand\arraystretch{1.0}
%\noindent\begin{longtable*}{@{}l @{\quad=\quad} l@{}}
%$A$  & amplitude of oscillation \\
%$a$ &    cylinder diameter \\
%$C_p$& pressure coefficient \\
%$Cx$ & force coefficient in the \textit{x} direction \\
%$Cy$ & force coefficient in the \textit{y} direction \\
%c   & chord \\
%d$t$ & time step \\
%$Fx$ & $X$ component of the resultant pressure force acting on the vehicle \\
%$Fy$ & $Y$ component of the resultant pressure force acting on the vehicle \\
%$f, g$   & generic functions \\
%$h$  & height \\
%$i$  & time index during navigation \\
%$j$  & waypoint index \\
%$K$  & trailing-edge (TE) nondimensional angular deflection rate
%\end{longtab*}}

\section{Introduction}
Plasma dynamics is a complicated problem with many facets of interest in many communities. Since plasmas consist of freely moving charged particles,the evolution of a plasma is tightly coupled to its self generated electromagnetic (EM) field. A consistent theory should provide a coupling of kinemetic (or fluid) equations for the plasma to the self induced field via the Lorentz force under the constraints of Maxwell's equations. Examples of these equations are given by, but not limited to, magnetohydrodynamics and the Vlasov equation. 

If we take for example, the Vlasov equation, we find that it is, in essence, a combination of Maxwell's equations (which describe the field produced by charges) with the collisionless Boltzmann equation (which describes the kinematics of these charges). Fundamentally, Maxwell's equations are topological. They solely require that the spacetime manifold $M^4$ admits a $3+1$-foliation and do not hinge on a metric structure on spacetime \cite{delphenich_axioms_2005, hehl_foundations_2003}. It is a worthy question to ask if the Vlasov equations admit a purely topological understanding as well. Hence, we really seek to find a topological version of the collisionless Boltzmann equation.

Kinematics on spacetime is a touchy subject. Sadly, there is an inability to describe the worldlines of more than one particle \colin{cite something here}. Sarbach and Zannias have produced a series of papers \cite{sarbach_relativistic_2013, sarbach_tangent_2014, sarbach_geometry_2014} that provide a relativistic version of Boltzmann's equation for a single particle which they claim is useful for describing the average properties of a gas. In fact, they even provide the equations for a charged and massive gas that includes self generated EM field and gravitation. In our case, we can ignore gravitational interactions since we wish to consider plasma on the small and low mass scale. Almost 50 years prior to Sarbach and Zannias, Bichteler produced the paper \cite{bichteler_cauchy_1967} which outlines a similar approach in the first few sections.

\section{Preliminaries}

\subsection{Clifford Algebras and Analysis}

\subsubsection{Clifford Algebras, Multivectors, and Rotors}

Clifford (or geometric) algebras are $\Z$- and $\Z/2\Z$- graded algebras with elements we refer to as multivectors. Formally, To see this, let us take the quadratic space $(V,q)$ and construct the Clifford algebra $C\ell(V,Q)$ by
\begin{equation}
C\ell(V,Q) \coloneqq \mathcal{T}(V) ~ / ~ \langle \blade{v} \otimes \blade{v} - Q(\blade{v}) \rangle
\end{equation}
with the induced addition and multiplication from this quotient. There are many wonderful sources on Clifford algebras but I will primarily use \cite{doran_geometric_2003} as a source for geometric and physical insight and the source \cite{chisolm_geometric_2012} for the vast amount of identities and clear notation. 

These algebras extend the exterior algebra $\bigwedge(V)$ by including the quadratic form $Q$ in the quotient which implies that $\bigwedge(V)\subset C\ell(V,Q)$ and, moreover, the product of vectors splits into a grade lowering term and grade raising term
\begin{equation}
    \blade{v}\blade{w}= \underbrace{\blade{v}\cdot \blade{w}}_{\textrm{grade lowering}} +\underbrace{\blade{v}\wedge \blade{w}}_{\textrm{grade raising}},
\end{equation}
where $\wedge$ is indeed the exterior product in $\bigwedge(V)$. Hence, we see that $C\ell(V,Q)$ gains an additional term $\cdot$ between vectors and, as with the exterior algebra, the higher graded elements are generated from taking exterior products of vectors
\begin{equation}
    \blade{A_k}\blade{v}_1 \wedge \cdots \blade{v}_k,
\end{equation}
If $\blade{v}_1,\dots,\blade{v}_k$ are linearly independent, we refer to $\blade{A_k}$ as a $k$-blade and sums of $k$-blades form the more general $k$-vectors which are referred to as grade-$k$ as well. The vector subspace of all $k$-vectors is denoted by $C\ell^k(V,Q)$. Given any multivector, we have the reverse operation $\dagger$ which is extended from the action on a $k$-blade by
\begin{equation}
    \blade{A_k}^\dagger = \blade{v}_k\wedge \cdots \blade{v}_1 = (-1)^{k(k-1)/2} \blade{A_k}
\end{equation}

The Clifford algebras are most special when the quadratic form is inherited from an inner product $Q(-)=g(-,-)$, since $g$ will be clear from context, we just put $\G$ to denote this algebra. When $V$ has pseudo-euclidean inner product with $p$ vectors that square to $-1$ and and $q$ vectors that square to $1$, we will put $\G_{p,q}$. In the case we are given $\G$, there are natural subgroups contained, namely $V\subset C\ell(V,Q)$ as well as $\sping(V) \subset C\ell(V,Q)$. The elements of $R\in \sping(V)$ are of even grade (multivectors consisting of only even grade elements) and have unit norm so that 
\begin{equation}
|R|^2 \coloneqq (R,R) \coloneqq \proj{0}{R^\dagger R}= R^\dagger R\pm 1.
\end{equation}
Here the notation $\proj{k}{A}$ tells us to select only the grade $k$-components of a multivector $A$ and it is important to note that $\dagger$ acts as the adjoint in the inner product $(-,-)$ on $\G$. If $R^\dagger R = +1$, we refer to this element as an \emph{rotor}. Briefly, let us first investigate the linear transformation induced by a rotor $\mathsf{R}({\blade{v}})=R \blade{v} R^\dagger$. 
\begin{proposition}
\label{prop:orthogonal_transformation}
The transformation above, $\mathsf{R}(\blade{v}) \mapsto R\blade{v}R^\dagger$ with $R=\sping(V)$ is an isometry and hence, $\mathsf{R}\in \mathrm{O}(V)$.
\end{proposition}
\begin{proof}
The proof is immediate. By definition we have that $R$ satisfies $RR^\dagger = \pm 1$. Hence,
\begin{equation}
    (R\blade{v}R^\dagger,R\blade{v}R^\dagger) = (\blade{v},R^\dagger R \blade{v} RR^\dagger) = (\blade{v},\blade{v}).
\end{equation}
\end{proof}

Finally, we mention that the top grade elements are scaled copies of the unit pseudoscalar $\blade{I}$. In particular, the volume element in some basis $\blade{e}_i$ is given by
\begin{equation}
    \blade{e}_1 \wedge \cdots \blade{e}_1 = \mu \blade{I} = \sqrt{\pm \det g} \blade{I}.
\end{equation}


\subsubsection{Clifford Analysis}

Semi-Riemannian manifolds $M$ can be given a Clifford algebra structure since each tangent space is equipped with an inner product $g$. This is done in an analogous way to the exterior algebra of smooth differential $k$-forms. We put $\G(M)$ to represent this algebra bundle and the sections use the previous terminology with addition of the word ``field". For another reference, see \cite{schindler_geometric_2020}. 

Given the Levi-Civita connection $\nabla$ and a vector field $\blade{v}$, we have the covariant derivative $\nabla_{\blade{v}}$ that can act on vector fields. In local coordinates on $M$ $x^i$ we have the induced basis in the tangent space $\blade{e}_i$ so that $\blade{e}_i\cdot \blade{e}_j = g_{ij}$ and this allows us to construct the reciprocal basis $\blade{e}^i$ so that $\blade{e}^i\cdot \blade{e}_j = \delta_{ij}$. This then allows us to define the gradient (or Dirac operator) $\grad$ given in these coordinates
\begin{equation}
    \grad \coloneqq \blade{e}^i \frac{\partial}{\partial x^i},
\end{equation}
where Einstein summation is implied. This derivative acts algebraically as a vector in the algebra and so we have
\begin{equation}
\grad A = \grad \cdot A + \grad \wedge A
\end{equation}
on any multivector field. 

Also in these coordinates we have the measures $dx^i$ which, when combined with a reciprocal vector yield directed measures $d\blade{x}^i = \blade{e}^i dx^i$. Hence, we can recover a differential form from a $k$-vector by taking the \emph{$k$-dimensional directed measure} given locally by 
\begin{equation}
    dX_k \coloneqq \frac{1}{r!} d\blade{x}^{i_1}\wedge \cdots \wedge d\blade{x}^{i_r}.
\end{equation}
Hence a $k$-form $\alpha_k \in \Omega^k(M)$ is given locally by  $\alpha_k = \alpha_{i_1 \cdots i_r} dx^{i_1}\wedge \cdots dx^{i_r}$ and in terms of a $k$-vector $A_k$ we have
\begin{equation}
\alpha_k = A_k \cdot dX_k^\dagger,
\end{equation}
where
\begin{equation}
A_r = \alpha_{i_1 \cdots i_r} \blade{v}^{i_1} \wedge \cdots \wedge \blade{v}^{i_r}.
\end{equation}
We refer to $A_r$ as the \emph{multivector equivalent} of $\alpha_r$ and, for example, the multivector equivalent of the Riemannian volume form $\mu$ is $\pseudoscalar^{-1 \dagger}$ which one can think of as defining the tangent space at some point. This provides an isomorphism between $k$-forms and $k$-vectors via a contraction with the $k$-dimensional volume directed measure. For example, there is the Riesz (or musical) isomorphism $\flat \mathfrak{X}(M)\to \Omega(M)$ by taking a vector field $\blade{v} \mapsto \blade{v}^\flat = \blade{v} \cdot dX_1$. In coordinates,
\begin{equation}
\label{eq:line_element}
 \blade{v} \cdot dX_1 = v_i  \blade{v}_i \cdot d\blade{x}^j = v_i dx^i.
\end{equation}
The algebraic operations of addition $+$, exterior multiplication $\wedge$, and contractions $\rfloor$ carry over to the familiar products on $\Omega(M)$ to $\G(M)$. Likewise, the differential operations of the exterior derivative $d$ take the form of the grade raising action of $\grad$ on multivector equivalents
\begin{equation}
d \alpha_r = (\grad \wedge A_r) \cdot dX_{r+1}^\dagger,
\end{equation}
and the codifferential $\delta$ (which is adjoint to $d$) by the grade lowering
\begin{equation}
\delta \alpha_r = (\grad \cdot A_r)\cdot dX_{r-1}^\dagger,
\end{equation} 
which gives us that $\grad \cdot$ is adjoint to $\grad \wedge$. Thus, the Hodge-Dirac operator $d+\delta$ on forms coincides with $\grad$ on multivectors and since $d^2=\delta^2=0$ we have $\grad \wedge^2 = \grad \cdot^2=0$ so we can build chain and cochain complexes as well as the Laplace-Beltrami operator $\Delta = \grad^2$. This final remark points to the Clifford analysis of $\grad$ as a refinement of the $\Delta$ of harmonic analysis. Finally, there is a mapping $\star \colon \G^k(M)\to \G^{n-k}(M)$ defined by
\begin{equation}
\alpha_k \wedge \star \beta_k = (A_k \wedge B_k^{\star})\cdot dX_n^\dagger = (A_k,B_k) \mu.
\end{equation}
so that it captures the action of Hodge star of forms on their multivector equivalents. 

\subsubsection{de Rham (Co)homology}

\label{subsubsec:derham}

Clifford analysis on manifolds can be used to extract the \emph{$k$th de Rham cohomology ring}
\begin{equation}
H^\bullet_{dR}(M) = \bigwedge_{k \in \mathbb{N}} H^k_{dR} = \bigwedge_{k \in \mathbb{N}} \ker \grad \wedge_k ~/~ \im \grad \wedge_{k-1}
\end{equation}
where $\im \grad \wedge_k$ are the \emph{exact} and $\ker \grad \wedge_k$ are the \emph{closed} $k$-vectors and we have identified the wedge product as the cup product $\wedge \colon H^k_{dR}(M) \times H^\ell_{dR}(M) \to H^{k+\ell}_{dR}(M)$. To see this is a cup product, let $A_k \in H^k_{dR}(M)$ and $B_\ell \in H^k_{dR}(M)$ and note $A_k \wedge B_\ell$ is a closed $k+\ell$-form
\begin{align}
\grad \wedge (A_k \wedge B_\ell) = (\grad \wedge A_k) \wedge B_\ell + (-1)^k A_k \wedge (\grad \wedge B_\ell) = 0,
\end{align}
since both $A_k$ and $B_\ell$ are closed. For the sake of this work moving forward, let us assume that we are speaking solely about forms with compact support when we mention homology or cohomology.

Dual to the de Rham cohomology is the homology of de Rham currents. A (compact) current is an element $T\in \Omega^*(M)$ where $\Omega^*(M)$ is the space of linear functionals on smooth (compact) forms $\Omega(M)$, so that $T\colon \Omega(M)\to \R$ has compact support. Examples of currents include chains on $M$ for which we can take the $k$-chain $C^k$ and note
\begin{equation}
    C^k[\alpha_k] = \int_{C^k} \alpha_k
\end{equation}
defines a current and take an $k$-vector $B_{k}$ on $M$ and note
\begin{equation}
    B^{k}[\alpha_k] = \int_M (A_k,B_k) \mu,
\end{equation}
is also a current. In fact, if the $k$-chain $C^k$ is a smooth submanifold $C^k=K\subset M$, then we can define the distributional $k$-vector $\delta_K$ so that
\begin{equation}
\delta_K[\alpha_k] = \int_{K} \alpha_k = \int_{M} (A_k, \pseudoscalar_K)\mu,
\end{equation}
where $\pseudoscalar_K$ is the unit pseudoscalar representing the tangent space at points of $K$. Likewise, if we took a point (a $0$-chain) $x\in M$ then for a $0$-form $\alpha_0$,
\begin{equation}
\delta_x[\alpha_0] = \int_M (A_0,\delta_x) \mu = A_0(x).
\end{equation}

With currents, we can build a homology theory. For more details, see \cite{iversen_cauchy_1989}. We define the boundary operator $\partial$ on $k+1$ currents $T^{k+1}$ by
\begin{equation}
\partial T^{k+1}[\alpha_k] \coloneqq T^{k+1}[d\alpha_k],
\end{equation}
defines a boundary map and we have the de Rham homology
\begin{equation}
H_\bullet^{dR} = \bigoplus_{n\in \mathbb{N}} H_k^{dR} \coloneqq \bigoplus_{n\in \mathbb{N}} \ker \partial_k ~/~ \im \partial_{k+1}. 
\end{equation}
Given Stokes' theorem, we realize for $k$-chains $C^k$ that
\begin{equation}
\label{eq:stokes}
\partial C^k[\alpha_{k-1}] = C^k[d\alpha_{k-1}] = \int_{C^k} d\alpha_{k-1} = \int_{\partial C^k} \alpha_{k-1},
\end{equation}
which gives us back the usual notion of a boundary which necessitated our choice of compact support. This in fact leads us to the following theorem.
\begin{theorem}[de Rham Theorem for Homology and Cohomology]
    Let $H_\bullet(M)$ and $H^\bullet(M)$ be the singular homology and cohomology on $M$ over the ring $\R$, repsectively. Then,
\begin{equation}
    H_\bullet(M)\cong H_\bullet^{dR}(M) \cong H_\bullet^\infty(M) \qquad \textrm{and} \qquad H^\bullet(M) \cong H_\bullet^{dR}(M) \cong H^\bullet_\infty(M),
\end{equation}
where $\infty$ denotes that we are taking smooth simplexes (submanifolds) and forms.
\end{theorem}
Given de Rham's theorem, we only put $H_\bullet(M)$ and $H^\bullet(M)$. In fact, we have another related complex on $k$-vectors induced built from $\grad \cdot$ acting on multivectors. 
\begin{proposition}
\label{prop:grad_dot}
    Let $K$ be a smooth submanifold with corresponding $k$-current $\delta_K$ and corresponding pseudoscalar $\blade{I}_K$. Then $\partial K$ corresponds to $\grad \cdot \blade{I}_K$.
\end{proposition}
\begin{proof}
Fix a $k-1$-form $\alpha_{k-1}$ and note that
\begin{align}
    \partial \delta_K [\alpha_{k-1}] = \int_K d\alpha_{k-1}= \int_M(\grad \wedge A_{k-1},\blade{I}_K)\mu = \int_M(A_{k-1},\grad \cdot \blade{I}_K) \mu.
\end{align}
\end{proof}
Henceforth, we can refer to $k$-currents in the kernel of $\partial$ as co-closed since their corresponding $k$-vector field must be co-closed. A corollary follows.
\begin{corollary}
    The chain complex induced by $\grad \cdot$ on multivectors is isomorphic to the singular homology. 
\end{corollary}
The proof for the corollary is immediate given that each equality of the proof of \cref{prop:grad_dot} shows the equivalences of smooth, de Rham, and $\grad \cdot$ homologies and we find the equivalence of singular homology comes via Stokes' theorem in \cref{eq:stokes}.

We have a cap product $\frown \colon H_k(M)\times H^\ell(M) \to H_{k-\ell}(M)$, which can be realized as an integral when we take a $k$-chain $\delta_K$ and a $\ell$-form $\alpha_\ell$ to get $T^{k-\ell} = \delta_K \frown  \alpha_\ell$ defined by
\begin{equation}
T^{k-\ell}[\beta_{\ell - k}] = \int_{K} \alpha_\ell \wedge \beta_{k-\ell}.
\end{equation}
Another way to realize the cap product is by noting that we can instead take a smooth $k$-dimensional submanifold $K$
\begin{equation}
T^{k-\ell}[-] = \int_K \alpha_\ell \wedge - = \int_M (-,A_\ell \rfloor \blade{I}_K) \mu,
\end{equation}
is an element of $\Omega^{k-\ell*}(M)$ and, therefore, a map $H^{k-\ell}(M)\to \R$. Then, For posterity, let $\ell = k$ we see $T^0 = \delta_K \frown \alpha_k = \delta_{X} \in H_0(M)$ so that $X$ is a $0$-cycle (i.e., a collection of points $\partial X = 0$ so that each $x\in X$ is from one connected component of $K$). Then
\begin{equation}
T^0[\beta_0] = \delta_{X}[\beta_0] = \int_{K} (B_0 , \delta_X ) \mu_{K} = \sum_{x\in X} B_0(x).
\end{equation}
Since $B_0$ is closed, it is constant on each connected component. The cap product above leads us to an important duality theorem of the $\R$ homology and cohomology of smooth manifolds.
\begin{theorem}[Poincar\'e Duality]
    The $k$th homology is isomorphic to the $n-k$th cohomology, that is 
\begin{equation}
H_k(M) \cong H^{n-k}(M).
\end{equation}
\end{theorem}
In fact, for closed manifolds, it is even true that $H_k(M)\cong H_{n-k}(M)$ and $H^k(M)\cong H^{n-k}(M)$ since the Hodge star maps harmonic $k$-forms (not defined here) to harmonic $n-k$-forms (see \cite{cappell_cohomology_2006} for more). At any rate, Poincar\'e duality yields the isomorphism through the cap product by taking the fundamental class $\delta_M$ and a $k$-form $\alpha_k \in H^k(M)$ to get the Poincar\'e dual $\overline{\alpha_k} \in H_{n-k}(M)$
\begin{equation}
\overline{\alpha_k} = (\delta_M \frown \alpha_k)[-] = \int_M (-,A_k \rfloor \blade{I})\mu
\end{equation}
which we see is well defined since
\begin{align}
\partial \overline{\alpha_k}[\beta_{k-1}] = \int_M (B_{k-1}, \grad \cdot (A_k \rfloor \blade{I}))\mu = \int_M (B_{k-1},(\grad \wedge A_k)\blade{I}) \mu =0,
\end{align}
since $A_k$ is a closed $k$-vector field. It is a worthy remark to mention that the $n-k$-vector associated to $\overline{\alpha_k}$ is given by $A_k \rfloor \blade{I}$ which is often referred to as the dual and we can put $A_k^\perp = A_k \rfloor \blade{I}$.

The pairing between co-closed currents and closed forms captured through integration is a (co)homological invariant that has ramifications in analysis.
\begin{proposition}
\label{def:period}
For any closed $k$-form $\alpha_k$ and each co-closed $k$-current $K$, a the \emph{period of the form $\alpha_k$} is the real number
\begin{equation}
\delta_K(\alpha_k)=\int_K \alpha_k.
\end{equation}
The period is invariant over both the homology class of $K$ and cohomology class of $\alpha_k$.
\end{proposition}
\begin{proof}
Let $K \in H_k(M)$ with current $\delta_K$ and $\alpha_k \in H^k(M)$. Then
\begin{align}
(\delta_K+\partial \delta_L)[\alpha_k + d \beta_{k-1}] &= \int_{K+\partial L} \alpha_k + d\beta_{k-1}\\
&= \int_K \alpha_k + \int_{\partial K} \beta_{k-1} + \int_L d \alpha_k + \int_L d(d\beta_{k-1})\\
&= \int_K \alpha_k.
\end{align}
\end{proof}
As it turns out, we have the following analytical statement.
\begin{proposition}
\label{prop:periods}
    If all periods of a form $\alpha_k$ vanish, then $\alpha_k$ has a potential $\beta_{k-1}$ such that $d\beta_{k-1}=\alpha_k$.
\end{proposition}
Thus, the topology of the domain is intimately connected with partial differential equations. We will use this in the subsequent section.

\subsection{Electromagnetism}

The work done in \cref{subsubsec:derham} can be compared with the preliminaries of \cite{delphenich_axioms_2005}. We will use this source as well as \cite{hehl_introduction_2003} as motivation for the topological theory of electromagnetism and the source \cite{gross_electromagnetic_2004} is also wonderful. Classically, electromagnetism is taught through the guise of analysis yet this is quite superfluous as the theory requires far less rigidity. There are four important physical postulates (each backed by experimentation) for electromagnetism:
\begin{enumerate}
    \item Conservation of charge;
    \item Conservation of magnetic flux;
    \item Constitutive law;
    \item Lorentz force.
\end{enumerate}
We take $M^4$ to be the foliated manifold of global spacetime with the Lorentz metric $g$ of signature $(-1, ~+1,~+1,~+1)$ so that we ignore any curvature or gravitation. When necessary, we take the local coordinates $x^\mu$ with $\mu=0,1,2,3$ and the induced orthonormal tangent vector fields satisfy $\blade{e}_0^2 = -1$ and $\blade{e}_i^2=+1$ for $i=1,2,3$. We will typically use Greek indices when running over the full spacetime and Latin when running over only the spatial indices.

\subsubsection{Conservation of Charge}
First, let $j_3$ be a 3-form field on spacetime $M^4$. In order for charge to be conserved, we require that any charge entering or exiting a region $N^4 \subset M^4$ must happen due to the charge passing through the boundary $\partial N^4$. Hence, we can state the physical postulate \emph{charge conservation} by
\begin{equation}
    \label{eq:topological_charge_conservation}
    \int_{\partial N^4} j_3 = \int_{N^4} dj_3= \int_{N^4} (\grad \wedge J_3) \cdot dX_4 = \int_{N^4} \grad \cdot (J_3 \rfloor \blade{I}) \mu = 0    ~~\iff~~ \grad \wedge J_3=\grad \cdot (J_3 \rfloor \blade{I}) = 0.
\end{equation}
The \emph{$4$-vector current} is then the dual $J_1 = J_3^\perp=J_3 \rfloor \blade{I}$ and we remark that $J_3 \in H^3(M^4)$ gives us $J_1 \in H_1(M^4)$ via Poincar\'e duality. Then, since we have determined that $j$ is closed, we can note that for any co-closed $3$-current $\delta_{N^3}$
\begin{equation}
    \delta_{N^3}[j_3] = 0
\end{equation}
and by \cref{prop:periods} we realize that $j_3$ has a potential $h_2$ which we refer to as the \emph{electromagnetic excitation 2-form}. That is, 
\begin{equation}
    j_3=dh_2 \qquad \textrm{or} \qquad J_3 = \grad \wedge H_2
\end{equation}
and dually
\begin{equation}
\label{eq:current_from_excitation}
J_1 = \grad \cdot H_2^\perp.
\end{equation}
This axiom can be understood in two ways and it owes to the belief of John Wheeler that ``charge is topology". First, we have seen the physical/analytical requirement of a conservation law of the field(s) $J$ correspond to a (co)homological statement on the field(s) as well.  Second, we realize this is a statement about the topological nature of spacetime itself in that $H_1(M)\cong H^3(M)$ must vanish. It turns out that conservation laws and topology are deeply related \cite{westenholz_topological_nodate}.

\subsubsection{Magnetic flux conservation}

For a co-closed $N^2$, we have another conservation law assuming a similar form. Namely, the the constraint on the electromagnetic bivector field $F_2$
\begin{equation}
    \label{eq:magnetic_flux_conservation}
    \oint_{N^2} f_2 = 0  ~~\iff~~ \grad \wedge F_2=0.
\end{equation}
is a physical postulate which we regard as \emph{magnetic flux conservation}. However, our starting point differs slightly from that of charge conservation. We do note require $N^2$ to be co-exact. 

Thus, depending on the (co)homology of $M^4$, we may be able to say something about $N^2$. In particular, if $H^2(M^4)=H_2(M^4)$ is trivial (i.e., all periods of 2-forms vanish), then $N^2=\partial N^3$ shows a potential for $f_2$ exists. Specifically, we can put
\begin{equation}
da_1 = f_2 \qquad \textrm{or} \qquad \grad \wedge \blade{A}_1 = F_2,
\end{equation}
and we refer to $\blade{A}_1$ as the \emph{$4$-vector potential}. We do not postulate the existence of a global potential $\blade{A}_1$, but if we work locally, this is always true since small enough local patches have trivial second (co)homology. In the case $F_2$ does have a potential we realize that $F_2=\blade{F}_2$ is a 2-blade (i.e., it represents a 2-dimensional subspace at each point).

\subsubsection{Constitutive Law and Maxwell's Equations}

At this point, we nearly have a set of equations that can be worked with. However, we need to determine a relationship between the electromagnetic field $F_2$ and the electromagnetic excitation $H_2$. This relationship is referred to as the \emph{constitutive law} and the simplest possible choice is linear so that $F_2 = H_2^\perp$. Thus, we note \cref{eq:current_from_excitation,magnetic_flux_conservation} yield the relativistic Maxwell equations
\begin{align}
	\grad \wedge F_2 &= 0  &&\textrm{(homogeneous)}\\
	\grad \cdot F_2 &= J_1 && \textrm{(inhomogeneous)}.
\end{align}
In fact, we can put this most succinctly as a multivector statement
\begin{align}
	\grad F_2 = J_1.
\end{align}
Supposing as well that $F_2$ has a potential $\blade{A}_1$, we can choose the Lorenz gauge so that $\grad \cdot \blade{A}_1 = 0$ and note that 
\begin{equation}
\Delta \blade{A}_1 = J_1
\end{equation}
since we have
\begin{align}
\Delta \blade{A}_1 = \grad \wedge \grad \cdot \blade{A}_1 + \grad \cdot \grad \wedge {A}_1= \grad \cdot F_2 = J_1.
\end{align}

Working locally, $F_2$ can be split into constituents, $E_2$ and $B_2$ since
\begin{align}
	F_2 = \underbrace{E_2^1 \blade{e}_0 \blade{e}_1 + E_2^2 \blade{e}_0 \blade{e}_2 + E_2^3 \blade{e}_0 \blade{e}_3}_{\textrm{electric field } E_2} + \underbrace{B_2^{3} \blade{e}_1 \blade{e}_2 + B_2^{2} \blade{e}_3 \blade{e}_1 + B_2^{1} \blade{e}_2 \blade{e}_3}_{\textrm{magnetic field } B_2}
\end{align}
Using this decomposition and noting that $\blade{e}^0 \nabla_{\blade{e}_0}$ is the time derivative and letting $\vec{\grad} = \blade{e}^i \nabla_{\blade{e}_i}$ be the spatial gradient,  we find the Heaviside version of Maxwell's equations splits into components containing $\blade{e}_0$ (i.e., have time derivatives) and those that do not by
\begin{align}
\label{eq:gauss_faraday}
	\grad \wedge F_2=0 ~\implies~ \underbrace{\vec{\grad} \wedge B_2 = 0}_{\textrm{spatial}} ~~~\textrm{and}~~~ \underbrace{\vec{\grad} \wedge E_2 + \blade{e}^0 \nabla_{\blade{e}_0} \wedge B_2 = 0}_{\textrm{spatio-temporal}}
\end{align}
are Gauss's law for magnetism and Faraday's law, respectively, and
\begin{align}
\label{eq:spacetime_split}
	\grad \cdot F_2=J_1 ~\implies~ \underbrace{\blade{e}^0 \cdot \vec{\grad} \cdot E_2 = \blade{e}^0 \cdot J_1}_{\textrm{spatial}}~~~\textrm{and}~~~ \underbrace{\blade{e}^0 \wedge (\blade{e}^0 \nabla_{\blade{e}_0} \cdot E_2 + \vec{\grad} \cdot E_2) = \blade{e}^0 \wedge J_1}_{\textrm{spatio-temporal}}
\end{align}
are Gauss's law for electricity and Ampere's law respectively. Multiplication by $\blade{e}^0$ seen in \cref{eq:spacetime_split} is often called the spacetime split and since \cref{eq:gauss_faraday} is homogeneous, we do not see this as a necessary step. The equations for the electric and magnetic potential can be found this way as well.

\subsubsection{Lorentz force}

Recall that a major motivator of this project is the plasma dynamics and, as such, we must also posulate the coupling between charges and fields. For a particle with charge $q$, mass $m$, and velocity $4-vector$ $V_1$, we know via experimentation that this particle undergoes acceleration due to the \emph{Lorentz force} 
\begin{equation}
    \label{eq:lorentz_force}
    \nabla_{\blade{V}_1}\blade{V}_1 = \frac{q}{m} \blade{V}_1 \rfloor F_2.
\end{equation}
This equation, by virtue of $\nabla$, is coordinate independent as well as Lorentz invariant. This equation will be revisited later in \cref{sec:spinor_equations}.


\section{Relativistic Kinematics}

\subsection{Boltzmann's Equation}

\subsection{Vlasov Equation}

\section{Spinor Equations of Fluid Plasmas}
\label{sec:spinor_equations}

In order to develop a deeper understanding of the motion of a fluid plasma, it will be worthwhile to dive into the motion of a single particle immersed in a field in spacetime. In some ways, the motion of a particle in spacetime can be thought of as an analog of the motion of a rigid body in Euclidean space. Specifically, whereas a the configuration space of a non-relativistic rigid body corresponds to the semi-product Lie group $\mathrm{A}(3)=\R^3 \rtimes \sping(3)$ called the \emph{Euclidean group} (really, this is the universal cover of that group), we can show that configuration of a massive relativistic particle lies on the cover of the \emph{Poincar\'e group} $\mathrm{A}(1,3) = \R^{1,3} \rtimes \sping^+(1,3)$ which we refer to as the \emph{Fermi transport group}.

\subsection{The group $\A(V)$}

The groups mentioned before can be discussed in broad generality.  If $R\in \sping(V)$ and $RR^\dagger = 1$, then we say that $R\in \sping^+(V)$ and refer to such $R$ is an \emph{rotor}. which leads us to realize the set $V\rtimes \sping^+(V) \subset C\ell(V,Q)$ as well. Moreover, this group inherits its structure from the Clifford algebra and we find the Lie algebra does as well. 

\begin{definition}
Fix a quadratic space $(V,Q)$ where $Q$ is a non-degenerate quadratic form, then we define the \emph{transport group} as the set
\begin{equation}
\A(V)\coloneqq V\rtimes \sping^+(V).
\end{equation}
To realize this as a group, we note that $\sping^+(V)$ acts on $V$ via conjugation which yields the multiplication in the semi-direct product
\begin{equation}
\label{eq:product_in_A}
(v,R)(v',R')\coloneqq(v+Rv'R^\dagger, RR')
\end{equation}
with inverse
\begin{equation}
(v,R)^{-1} = (R^\dagger v R, R^\dagger).
\end{equation}
\end{definition}

Next, let us visit the classical example in 3-dimensional space where we can realize that the group $\A(3)$ serves as the configuration space of a rigid body.

\begin{example}
Take for example the motion of a rigid body in 3-dimensional space. There are two components of this motion each which three degrees of freedom. First, is the 3-dimensional position of the center of mass of the body (i.e., the linear momentum) and the second is rotation about the center of mass (i.e., the angular momentum) which also has three degrees of freedom corresponding to the three planes in $\R^3$. We claim that the configuration of a rigid body must lie in the group $\A(3)=\R^3 \rtimes \sping^+(3)$. 

First, let $\blade{v}(0)$ be the initial position of the center of mass of the body which makes up the first component of the semi-direct product. By \cref{eq:product_in_A}, we see that this initial center of mass vector $\blade{v}(0)$ can be translated to some new position at a short time later, $\epsilon$, by 
\begin{equation}
\blade{v}(\epsilon) = \blade{v}(0)+R\blade{u} R^\dagger,
\end{equation}
where $\blade{u}\in \R^3$ and $R\in \sping(3)$. One should note that there is not necessarily one single choice of $\blade{u}$ and $R$.

To see the interpretation of the second component of the semi-direct product, fix an initial orthonormal frame that describes the rotational state of the body, $\mathscr{F}=(\blade{e}_1,\blade{e}_2,\blade{e}_3)$ and let $R(0)$ be such that $\mathscr{F}(0)=R(0)\mathscr{F}R^\dagger(0)$ corresponds to the initial rotational orientation of the body. At a time $\epsilon$ later, we have a new orientation
\begin{equation}
\mathscr{F}(\epsilon) = R(\epsilon) \mathscr{F} R^\dagger(\epsilon),
\end{equation}
which is a new orthonormal frame by \cref{prop:orthogonal_transformation}. This shows that the rotors $R(\epsilon)$ themselves encapsulate the rotation of the principal axes of inertia of the rigid body. Hence, it suffices to just consider the value of $R(\epsilon)$ as opposed to the frame itself.
\end{example}
We keep the notions of this example moving forward and posit that the group $\A(V)$ represents the configuration of a generalized notion of a rigid body.

This all begs the question as to what the infinitesimal motions on $\A(V)$ correspond to, i.e., what is the Lie algebra to $\A(V)$? Note that the Lie algebra to $V$ is itself a trivial Lie algebra since $V$ is a commutative group. The Lie algebra of $\sping^+(V)$ is the algebra of bivectors $\spina(V)=C\ell^2(V,Q)$ along with the commutator $[-,-]$ which we inherit from $C\ell(V,Q)$ as well. We denote the Lie algebra of $\A(V)$ by $\liealg(V)$ and note that we have the orthogonal decomposition
\begin{equation}
\liealg(V)=V\oplus \spina(V),
\end{equation}
which allows us to write any element in $\liealg(V)$ as a sum of a vector $\blade{v}$ and bivector $b$. 

\begin{proposition}
The commutator bracket of $\liealg(V)$, $[-,-]_{\liealg(V)}$ can be written in terms of the commutator for the Clifford algebra $[-,-]$.
\end{proposition}
\begin{proof}
Let $\blade{v}_1,\blade{v}_2\in V$ and $b_1,b_2 \in \spina(V)$, we have that 
\begin{equation}
[\blade{v}_1+b_1,\blade{v}_2+b_2]_{\liealg(V)} = [\blade{v}_1,\blade{v}_2]_{V}+\mathrm{ad}_{b_1}\blade{v}_2 - \mathrm{ad}_{b_2}\blade{v}_1 + [b_1,b_2]_{\spina(V)}.
\end{equation}
Then, by \cite[Lemma 5.7]{gracia-bondia_elements_2001}, 
\begin{equation}
\mathrm{ad}_{b_i}\blade{v_j}=[b_i,\blade{v}_j].
\end{equation}
Likewise, the commutator $[-,-]_{\spina(V)}=[-,-]$ and $[\blade{v}_1,\blade{v}_2]_V=0$ hence
\begin{align}
[\blade{v}_1 + b_1,\blade{v}_2+b_2]_{\liealg(V)} &= [b_1,\blade{v}_2] + [\blade{v}_1,b_2] + [b_1,b_2]\\
&= [\blade{v}_1+b_1,\blade{v}_2+b_2]-[\blade{v}_1,\blade{v}_2].
\end{align}
\end{proof}

\subsection{Relativistic motion of a massive charged particle}

To set the stage, let $M^4$ be global spacetime and let $N^4\subset M^4$ be a (small) local region of spacetime. Since $M^4$ is foliated, there exists a function $\tau \colon N^4 \to \R$ such that $d\tau \neq 0$ anywhere on $N^4$. Let $\blade{e}_\tau$ be the corresponding vector field corresponding to $d\tau$, i.e., $\blade{e}_\tau \cdot dX_1 = d\tau$ (or, in other words, $\blade{e}_\tau = d\tau^\flat$).  In particular, choose this $\tau$ so that $\blade{e}_\tau\cdot \blade{e}_\tau = -1$ everywhere in $N^4$. Let $T \subset \R$ be an open set and refer to some $t\in T$ as the \emph{time}. Next, define \emph{space at time $t$} by $N^3(t)=\tau^{-1}(t)$ and we can see space forms the $3$-dimensional leaves of the $3+1$-foliation of spacetime. Let $\iota\colon N^3(t)\hookrightarrow N^4$ be inclusion then define the coordinates $x_{i}$ be local coordinates on $N^3(t)$ with corresponding $1$-forms, $dx_i = \blade{e}_i \cdot dX_1$, where $\blade{e}_1,\blade{e}_2,\blade{e}_3\in \G_{3}^1(N^3(t))$ and $\blade{e}_i\cdot \blade{e}_j = \delta_{ij}$. Let $\blade{I}_4$ be the unit pseudoscalar field on $N^4$, then at the point $x \in N^3(t)$, the tangent unit pseudoscalar is defined by $\blade{I}_3(x) = \blade{e_\tau}(x,t)\rfloor \blade{I}_4(x,t)$ is the pseudoscalar for $N^3(t)$.

\subsubsection{The 4-Momentum, 4-Current, and 4-Velocity Field Decomposition}

Consider $\gamma \colon T \to N^4$ be the time parameterization of the worldline of a massive particle and let $\blade{p} = \dot{\gamma} \coloneqq \in \G_{1,3}^1(N^4)$ be the \emph{4-momentum field} of the particle. Since $\gamma$ is massive, it must be that $\blade{p}^2<0$. Hence, we assume that this can be decomposed as
\begin{equation}
    \blade{p} = m \blade{v},
\end{equation}
where $m\in \G_{1,3}^0(N^4)$ and $\blade{v} \in \G_{1,3}^1(N^4)$ with $\blade{v}^2=-1$ everywhere in $N^4$. We refer to $m$ as the \emph{mass energy field} and $\blade{v}$ as the \emph{massive 4-velocity field}. From $\blade{p}$ we can always deduce $m$ via the expression $\blade{p}^2 = -m^2$. Finally, if the particle were charged, we let $q\colon N^4 \to \R$ be the \emph{charge field}, then 
\begin{equation}
\blade{j}=q\blade{v}
\end{equation}
defines the \emph{4-current field}. If we assume the mass of the particle is unchanging, $\blade{p}^2=-m^2$ for some $m>0$. Likewise, if the particle is statically charged, then $\blade{j}=q\blade{v}$ would be the 4-current associated to this particle. 

%the \emph{4-current vector field of this particle} if $\grad \cdot \blade{j} = 0$. Note that
%\begin{equation}
%    \grad \cdot \blade{j} = \grad \cdot (q\blade{v}) = (\grad q) \cdot \blade{v} + q\grad \cdot \blade{v} = 0
%\end{equation}
Since $\blade{v}^2=-1$, we realize a differential constraint on ${\blade{v}}$ by
\begin{equation}
    \grad ({\blade{v}} \cdot {\blade{v}}) = 0 
\end{equation}
which implies that
\begin{equation}
    \label{eq:velocity_covariant_derivative}
    \nabla_{\blade{v}} {\blade{v}} - {\blade{v}}\cdot (\grad \wedge {\blade{v}}) = 0.
\end{equation}
One interpretation of this could be that the advection of the velocity field depends solely on ${\blade{v}} \cdot (\grad \wedge {\blade{v}})$. Or, said another way, optimal transport of ${\blade{v}}$ (when $\nabla_{\blade{v}} {\blade{v}} = 0$) occurs when ${\blade{v}}$ lies solely outside of the plane defined by the 2-blade (the \emph{relativistic vorticity}) $\grad \wedge {\blade{v}}$ or if the vorticity itself equals zero, $\grad \wedge {\blade{v}} = 0$. 

Let $F\in G_{1,3}^2(N^4)$ be the \emph{electromagnetic bivector field}. Since this is done locally in $M^4$, $F=\grad A$ where $A\in \G_{1,3}^1(N^4)$ and this implies that $F=\blade{F}$ is a $2$-blade. We say that this particle $\gamma$ obeys the Lorentz force law if 
\begin{equation}
    \label{eq:lorentz_force_geometric_algebra}
    \nabla_{\blade{v}} \blade{p} = \blade{j}\rfloor \blade{F}.
\end{equation}
 On the other hand, when $m$ is constant
\begin{equation}
    \label{eq:lorentz_force_static_mass}
    \nabla_{\blade{v}} \blade{v} = \frac{q}{m} \blade{v} \rfloor \blade{F}.
\end{equation}
Given that $\blade{v}$ is a vector, then we also note that for a 2-blade $\blade{B}$ that
\begin{equation}
    \blade{v} \rfloor \blade{B} = \frac{1}{2}[\blade{B},\blade{v}] = \frac{1}{2}(\blade{B}\blade{v} - \blade{v}\blade{B}),
\end{equation}
which means that we use \cref{eq:lorentz_force_static_mass} in order to get
\begin{equation}
    \label{eq:static_charge_mass_faraday_transport}
    \boxed{\nabla_{\blade{v}} \blade{v} = \frac{q}{2m} [\blade{F},\blade{v}].}
\end{equation}
This \cref{eq:static_charge_mass_faraday_transport} is the \emph{static mass and charge Faraday transport equation}.

\subsubsection{Configuration space of a massive charged particle}

For the moment, let us identify our flat spacetime $N^4$ with Minkowski space $\R^{1,3}$ with the coordinates $(t,x_1,x_2,x_3)$. As seen in \cref{eq:lorentz_force_static_mass}, motion of a massive charged particle depends on the (now vector) 4-position $\blade{\gamma}$ and the 4-velocity $\blade{v}$ as well as the charge to mass ratio $\frac{q}{m}$. Without loss of generality, let us assume that $\frac{q}{m}=1$. Let $\tau$ be the proper time parameter of the particle, then note that $\nabla_{\blade{v}}\blade{v}=\frac{d}{d\tau} \blade{v}(\tau)$ and hence
\begin{equation}
    \label{eq:faraday_transport}
    \frac{d\blade{v}}{d\tau} (\tau) = \frac{1}{2} [\blade{F},\blade{v}].
\end{equation}
Suppose that the initial position is $\blade{\gamma}(0)=\blade{\gamma}_0$ and the initial velocity is $\blade{v}(0)=\blade{v}_0$ and note that the 4-position $\blade{\gamma}$ and 4-velocity  $\blade{v}$ along with the electromagnetic field $\blade{F}$ completely determines the trajectory of the particle. At an infinitesimal increment of proper time $\epsilon$ later,
\begin{align}
\label{eq:linearization_position}
\blade{\gamma}(\epsilon) &\approx \blade{\gamma}(0)+\epsilon\blade{v}(0)\\
\blade{v}(\epsilon) &=  R(\epsilon)\blade{v}_0 R^\dagger(\epsilon),
\end{align}
where the latter equation is required since it must be that $\blade{v}(\tau)^2=-1$ which means that $R\in \sping^+(1,3)$ and we refer to such an $R$ as a \emph{spacetime rotor}. Note that these equations hold true for any time $\tau$, not just $\tau = 0$. By choosing some fixed $\blade{v}_0$, we can note that the configuration of the particle lies in the group $\A(1,3)$. Then, 
\begin{equation}
\label{eq:linearization_rotor}
R(\epsilon) \approx R(0)+\epsilon\frac{d}{d\tau}R(0).
\end{equation}
It will be worth investigating the equations of motion for the rotor $R$ later on.


%Finally, using \cref{eq:lorentz_force_geometric_algebra,eq:velocity_covariant_derivative}, we could also write
%\begin{equation}
%    {\blade{v}}\cdot (\grad \wedge {\blade{v}}) = \frac{q}{m} {\blade{v}} \rfloor \blade{F} = \frac{q}{m} \blade{v} \rfloor (\grad \wedge \blade{A}).
%\end{equation}

\subsubsection{Lie Algebras of Bivectors and Spacetime Rotors}

Since $\blade{F}$ is a bivector, it is a section of the $\spina(1,3)$ bundle. To this end, let us investigate this Lie algebra with the commutator bracket $[-,-]$. Using our basis, we realize there is an orthogonal decomposition 
\begin{equation}
    \label{eq:spin_algebra_split}
    \spina(1,3) = \mathcal{T} \oplus \mathcal{S},
\end{equation}
where 
\begin{align}
    \mathcal{T} &\coloneqq \mathrm{span}(\{\blade{e}_0 \blade{e}_i ~\vert~ i = 1,2,3\})\\
    \mathcal{S} &\coloneqq \mathrm{span}(\{\blade{e}_i \blade{e}_j ~\vert~ i,j = 1,2,3, ~ i\neq j\}).
\end{align}
Each space $\mathcal{T}$ and $\mathcal{S}$ are 3-dimensional and the space $\mathcal{S}\cong \mathfrak{spin}(3)$. Orthogonality is realized by the fact
\begin{align}
    (\blade{e}_0 \blade{e}_i, \blade{e}_j \blade{e}_k) &= \proj{}{(\blade{e}_0 \blade{e}_i)^\dagger \blade{e}_j \blade{e}_k} = 0.
\end{align}
The space $\mathcal{T}$ does not form a Lie subalgebra since it is not closed under the bracket
\begin{equation}
    [\blade{e}_0\blade{e}_i,\blade{e}_0 \blade{e}_j] = \frac{1}{2} (\blade{e}_0 \blade{e}_i \blade{e}_0 \blade{e}_j - \blade{e}_0 \blade{e}_j \blade{e}_0 \blade{e}_i) = \frac{1}{2}(\blade{e}_j \blade{e}_i-\blade{e}_i \blade{e}_j).
\end{equation}
However, we can note that elements in $\mathcal{T}$ and $\mathcal{S}$ commute since
\begin{equation}
[\blade{e}_0 \blade{e}_i,\blade{e}_j \blade{e}_k] = 0.
\end{equation}

It follows from the splitting in \cref{eq:spin_algebra_split} that a spacetime rotor $R$ can be decomposed further into the decomposition $R=LU$. Physically, we care about the vector ${\blade{v}}$ and its evolution in space which is governed by ${\blade{v}}=R\blade{v}_0 R^\dagger$. Relative to $\blade{v}_0$ we build an orthonormal frame $\mathscr{F}_0=(\blade{v}_0,\blade{y}_1(0),\blade{y}_2(0),\blade{y}_3(0))$ and, as before, we can note that this whole frame is is transformed over time by $R\mathscr{F}_0 R^\dagger$. The transformation of the frame vectors $\blade{y}_i$ is not physical which represents a freedom in choice of the spatial reference frame for an observer. We note that,
\begin{equation}
    R(\tau) = \exp(B) = \exp(B_\mathcal{T}+B_\mathcal{S})=\exp(B_\mathcal{T})\exp(B_\mathcal{S})=L(\tau)U(\tau),
\end{equation}
which follows from the fact $[\mathcal{T},\mathcal{S}]=0$. 

\subsubsection{Rotor equations and trajectory for a single particle in a constant field}

For a single particle, assumption of static mass and charge is reasonable to make. Letting $\epsilon$ be an infinitesimal increase in proper time, we have the linearizations given by \cref{eq:linearization_position,eq:linearization_rotor} and in particular we want that
\begin{equation}
    L(\tau+\epsilon) = 1+ \frac{1}{2}\epsilon \frac{d \blade{v}}{d\tau}(\tau) {\blade{v}}(\tau),
\end{equation}
which can be seen in \cite{doran_geometric_2003}. Finally, equating $R(\tau+\epsilon)=L(\tau+\epsilon)$ yields 
\begin{equation}
    \label{eq:fermi_transport}
    \frac{dR}{d\tau} R^\dagger = \frac{1}{2}\frac{d{\blade{v}}}{d\tau}{\blade{v}}.
\end{equation}
and we refer to \cref{eq:fermi_transport} as the \emph{Fermi transport equation}. These, in particular, are special since they reflect the transport equations given by pure boosts. Noting that $\frac{d\blade{v}}{d\tau}\cdot \blade{v}=0$ since $\tau$ is the arclength parameter we have that Fermi transport due to Faraday transport gives us
\begin{equation}
\frac{d\blade{v}}{d\tau} = -2 \frac{dR}{d \tau} R^\dagger \blade{v}=\blade{F}\lfloor \blade{v}
\end{equation}
which yields an equation purely for the rotor in terms of the electromagnetic field (while reintroducing the charge-to-mass ratio)
\begin{equation}
     \label{eq:fermi_transport_rotor}   
    \boxed{\frac{dR}{d\tau} = \frac{q}{2m} \blade{F}R.}
\end{equation}

Let $\blade{F}$ be constant and non-null (i.e., that $\blade{F}^2\neq 0$), then we can put
\begin{equation}
\blade{F}^2 = \proj{0}{\blade{F}^2}+\proj{4}{\blade{F}^2}=\rho \exp(\blade{I}\theta)
\end{equation}
which allows us to write 
\begin{equation}
\blade{F}=\rho^{1/2}\exp(\blade{I}\theta/2)\hat{\blade{F}}=\alpha \hat{\blade{F}}+\beta\blade{I} \hat{\blade{F}}.
\end{equation}
Given an initial rotor $R(0)=R_0$, we then have
\begin{equation}
R(\tau)=\exp\left(\frac{q}{2m}\alpha \hat{\blade{F}}\tau\right)\exp\left(\frac{q}{2m}\beta\blade{I}\hat{\blade{F}}\tau\right)R_0.
\end{equation}
Likewise, the position of the particle can be recovered as well by noting $\blade{v}_0=R_0\blade{e}_0 R_0^\dagger$ and using the Faraday transport equation \cref{eq:faraday_transport} to get
\begin{equation}
\blade{\gamma}(\tau) = \blade{\gamma}(0)+\frac{\exp\left(\frac{q}{m}\alpha \hat{\blade{F}}\right)-1}{q\alpha/m} \hat{\blade{F}}\cdot \blade{v}_0 - \frac{\exp\left(\frac{q}{m}\beta \blade{I}\hat{\blade{F}}\right)-1}{q\beta/m}(\blade{I}\hat{\blade{F}})\cdot \blade{v}_0.
\end{equation}


\subsection{Collection of particles}

Let us begin with a collection of particles at $t=0$ to define our spatial manifold $N^3(0)$. As the collection of particles evolves in time, we will produce a new manifold $N^3(t)$. This is a Lagrangian description of the particles and, as such, we specify $X(\blade{x}_0,t)$ to be the location of the particle we refer to as $\blade{x}_0$ such that $X(0,\blade{x}_0)=(0,\blade{x}_0)\in N^4$. For example, if $N^3(0)$ represents a single point particle, then $N^3(0)=\blade{x}_0$ and given $\blade{v}_0=\frac{\partial X}{\partial t}(0,\blade{x}_0)$ we have a constant electromagnetic field, then
\begin{equation}
X(t,\blade{x}_0)=(0,\blade{x}_0)+\frac{\exp\left(\frac{q}{m}\alpha \hat{\blade{F}}\right)-1}{q\alpha/m} \hat{\blade{F}}\cdot \blade{v}_0 - \frac{\exp\left(\frac{q}{m}\beta \blade{I}\hat{\blade{F}}\right)-1}{q\beta/m}(\blade{I}\hat{\blade{F}})\cdot \blade{v}_0,
\end{equation}
as before.

We could also imagine ${\blade{v}}$ as a velocity field on $N^4$ that describes a fluid of charged particles. In that case, we would not be able to impose that $m$ is static and instead
\begin{equation}
    m \nabla_{\blade{v}} \blade{v} + (\nabla_{\blade{v}} m)\blade{v} = q\blade{v} \rfloor \blade{F}
\end{equation}
where we use linearity on the right hand side.
By smoothness, if $R\in \sping^+(1,3)$ at some point, then $R$ will be a section of the $\sping^+(1,3)$ bundle, i.e., a spacetime rotor field. We can deduce differential constraints for the spacetime rotor field $R$ via transport. Let us assume that we have some fixed tangent vector $\blade{u}$ at a point in $N^4$ with $\blade{u}^2 = -1$, then the tangent vector at another point is $\blade{v}=R\blade{u}R^\dagger$ where $R$ is allowed to vary in space and we assume we parallel translate $\blade{u}$ from point to point. Thus,
\begin{align}
    \nabla_{\blade{v}} {\blade{v}} &= \nabla_{\blade{v}} R {\blade{u}} R^\dagger + R{\blade{u}} \nabla_{\blade{v}} R^\dagger \\
    &= \nabla_{\blade{v}} R(R^\dagger {\blade{v}} R)R^\dagger + R(R^\dagger {\blade{v}}R)\nabla_{\blade{v}} R^\dagger\\
    &= (\nabla_{\blade{v}} R) R^\dagger {\blade{v}} + {\blade{v}} R \nabla_{\blade{v}} R^\dagger
\end{align}
Then, since $R$ is a spacetime rotor field, $RR^\dagger = 1$ everywhere and it must be that
\begin{align}
    0 = \nabla_{\blade{v}} (RR^\dagger) = \nabla_{\blade{v}} R R^\dagger + R\nabla_{\blade{v}}R^\dagger
\end{align}
and by the previous work 
\begin{equation}
    \nabla_{\blade{v}} {\blade{v}} = [(\nabla_{\blade{v}} R) R^\dagger,{\blade{v}}]
\end{equation}
In the same vein, ${\blade{v}}^2={\blade{v}}\cdot {\blade{v}}=-1$ so
\begin{align}
    0=\nabla_{\blade{v}} ({\blade{v}}^2)=2(\nabla_{\blade{v}} {\blade{v}})\cdot {\blade{v}},
\end{align}
which is a requirement for the transport along a geodesic, hence
\begin{align}
    (\nabla_{\blade{v}} {\blade{v}}) {\blade{v}} = [(\nabla_{\blade{v}} R)R^\dagger,{\blade{v}}] {\blade{v}}.
\end{align}

Letting $\delta r = (\delta t,\delta x_1,\delta x_2, \delta x_3)$ be an infinitesimal displacement in Minkowski space, then we have to first order
\begin{align}
    {\blade{v}}(r + \delta r) &= {\blade{v}}(r)+\delta r \nabla_{\blade{v}} {\blade{v}}\\
    R(r+\delta r) &= (1+\delta r (\nabla_{\blade{v}} R) R^\dagger)R.
\end{align}
Hence, we want that
\begin{equation}
    L(r + \delta r) = 1+ \frac{1}{2}\delta r (\nabla_{\blade{v}} {\blade{v}}){\blade{v}}.
\end{equation}
(see \cite{doran_geometric_2003}) and equating $R=L$ yields
\begin{equation}
    (\nabla_{\blade{v}} R)R^\dagger = \frac{1}{2}(\nabla_{\blade{v}} {\blade{v}}){\blade{v}}
\end{equation}
and using the bracket
\begin{equation}
    \label{eq:fermi_transport}
    \boxed{[\nabla_{\blade{v}} R, R^\dagger] = \frac{1}{2} [\nabla_{\blade{v}} {\blade{v}},{\blade{v}}].}
\end{equation}
We refer to the above \cref{eq:fermi_transport} as the \emph{Fermi transport equation}. A charged particle's 4-velocity undergoes Faraday transport and we can combine this with the equations of Fermi transport to get
\begin{equation}
    \boxed{[\nabla_{\blade{v}} R,R^\dagger]= \frac{q}{4m} \left[ [\blade{F},{\blade{v}}],{\blade{v}}\right].}
\end{equation}

\section{Conclusion}



\section*{Appendix}



\section*{Acknowledgments}
An Acknowledgments section, if used, \textbf{immediately precedes} the References. Sponsorship information and funding data are included here. The preferred spelling of the word ``acknowledgment'' in American English is without the ``e'' after the ``g.'' Avoid expressions such as ``One of us (S.B.A.) would like to thank\ldots'' Instead, write ``F.~A.~Author thanks\ldots'' Sponsor and financial support acknowledgments are also to be listed in the ``acknowledgments'' section.

\bibliography{final_report_bib}

\end{document}
