%%%%%%%%%%%%%%%%%%%%%%%%%%%%%%%%%%%%%%%%%%%%%%%%%%%%%%%%%%%%%%%%%%%%%%%%%%%%%%%%%%%%
% Document data
%%%%%%%%%%%%%%%%%%%%%%%%%%%%%%%%%%%%%%%%%%%%%%%%%%%%%%%%%%%%%%%%%%%%%%%%%%%%%%%%%%%%
\documentclass[12pt]{article} %report allows for chapters
%%%%%%%%%%%%%%%%%%%%%%%%%%%%%%%%%%%%%%%%%%%%%%%%%%%%%%%%%%%%%%%%%%%%%%%%%%%%%%%%%%%%
\usepackage{preamble}

\begin{document}

\begin{center}
   \textsc{\large MATH 271, Worksheet 5}\\
   \textsc{Power Series}
\end{center}
\vspace{.5cm}

\begin{problem}
Find the radius of convergence for the following power series.
\begin{enumerate}[(a)]
    \item $\displaystyle{\sum_{n=1}^\infty \frac{x^n}{n}}$;
    \item $\displaystyle{\sum_{n=1}^\infty \frac{x^n}{n^2}}$;
    \item $\displaystyle{\sum_{n=0}^\infty (-1)^n x^n}$;
    \item Repeat (c) but for $z\in \C$ instead of $x\in \R$.
    \item How does the convergence in (a) and (b) compare with the convergence of the typical $p$-series?
\end{enumerate}
\end{problem}

\begin{problem}
A \emph{Taylor series centered at $a$} is a power series for a function $f$ is computed by 
\[
f(x) = \sum_{n=0}^\infty \frac{f^{(n)}(x)}{n!}(x-a)^n.
\]
When $a=0$, we call this a \emph{Maclaurin series}.  
\begin{enumerate}[(a)]
    \item Compute the Maclaurin series for $e^x$.
    \item Compute the Taylor series for $e^x$ centered at $a=1$.
    \item Compute the Maclaurin series for $\cos(x)$.
    \item Compute the Maclaurin series for $\sin(x)$.
\end{enumerate}
\end{problem}

\begin{problem}
Consider the recursive sequence generated by $a_n = \frac{a_{n-1}}{n}$ with $a_0=1$.  We can generate a power series for $e^x$ we have seen before from this data
\[
e^x=\sum_{n=0}^\infty a_n x^n
\]
from this data. 
\begin{enumerate}[(a)]
    \item Write down the first five terms of the sequence $\{a_n\}_{n=0}^\infty$ to show that it is the same as the sequence $\left\{\frac{1}{n!}\right\}_{n=0}^\infty$.
    \item Using some test for series convergence, show that the power series for $e^x$ converges for any $x\in \R$. This defines the function $e^x$ on all real numbers.
\end{enumerate}
\end{problem}

\begin{problem}
How might you arrive at the above recursive sequential definition for $e^x$? Consider defining $e^x$ to be the solution to the initial value problem
\[
\frac{d}{dx} e^x = e^x \qquad \textrm{and} \qquad e^0=1.
\]
We have seen that we can differentiate a power series term by term and this will help us here.
\begin{enumerate}[(a)]
    \item Let $e^x = \sum_{n=0}^\infty a_n x^n$ where we are pretending (for the moment) that we don't know what $a_n$ is.  Using the series definition of $e^x$, write out the differential equation above.
    \item Now, by matching coefficients on the same powers of $x$, can you find the relationship on $a_n$ posed in Problem 1?
\end{enumerate}
\end{problem}

\begin{problem}
Recall that we can also take $e^z$ with complex numbers $z\in \C$. In the last worksheet, we found we can write
\begin{align*}
e^{ix} = \sum_{n=0}^\infty \frac{(ix)^n}{n!}&= \sum_{n=0}^\infty \frac{(-1)^n x^{2n}}{(2n)!} + i \sum_{n=0}^\infty \frac{(-1)^n x^{2n+1}}{(2n+1)!}\\
&= \cos(x) + i \sin(x).
\end{align*}
Using the above fact and work from Problem 1, show that $e^z$ converges for any $z\in \C$.
\end{problem}

\begin{problem}
We have differentiated series term by term which will prove to be useful for solving ODEs.  However, we can also integrate them term by term which proves useful for computing integrals of functions that have no ``closed-form" anti-derivative.  Take for example, the \emph{Gaussian function} (or \emph{normal distribution})
\[
f(x;\mu,\sigma)=\frac{1}{\sqrt{2\pi \sigma^2}}e^{-\frac{(x-\mu)^2}{2\sigma^2}}.  
\]
This describes a probability distribution on the real line with mean $\mu$ and standard deviation $\sigma$. The semi-colon notation for $f(x;\mu,\sigma)$ just means $\mu,\sigma$ are parameters (as opposed to variables) for the function.
\begin{enumerate}[(a)]
    \item Using the power series for $e^x$, write the power series for the Gaussian $f(x;\mu,\sigma)$.
    \item Now, suppose we want to find the probability of an event occurring between $\mu \pm \sigma$ (that is, within one standard deviation of the mean).  For the sake of ease, let $\mu=0$ and $\sigma=1$ and integrate the series term by term.
    \item Using the anti-derivative you made in (b), compute an approximation to the definite integral on the region $[-1,1]$ which will tell us the probability of an event occuring within one standard deviation of the mean.
    \item Look at a $z$-score table to see how close your approximation is.
\end{enumerate}
\end{problem}


\end{document}
