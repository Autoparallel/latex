%%%%%%%%%%%%%%%%%%%%%%%%%%%%%%%%%%%%%%%%%%%%%%%%%%%%%%%%%%%%%%%%%%%%%%%%%%%%%%%%%%%%
% Document data
%%%%%%%%%%%%%%%%%%%%%%%%%%%%%%%%%%%%%%%%%%%%%%%%%%%%%%%%%%%%%%%%%%%%%%%%%%%%%%%%%%%%
\documentclass[12pt]{article} %report allows for chapters
%%%%%%%%%%%%%%%%%%%%%%%%%%%%%%%%%%%%%%%%%%%%%%%%%%%%%%%%%%%%%%%%%%%%%%%%%%%%%%%%%%%%
\usepackage{preamble}

\begin{document}

\begin{center}
   \textsc{\large MATH 271, Homework 5}\\
   \textsc{Due October 11$^\textrm{th}$}
\end{center}
\vspace{.5cm}

\begin{problem} $p$-series are actually related to a very important function called the \emph{Riemann zeta function}.  This function is involved in a million dollar math problem! If you're interested in other million dollar problems, look up the Clay Institute Millennium Problems. The Riemann zeta function is given by
\[
\zeta (s) = \sum_{n=1}^\infty \frac{1}{n^s}.
\]
\begin{enumerate}[(a)]
    \item Use the integral test to show that the $p$-series
    \[
    \sum_{n=1}^\infty \frac{1}{n^2}
    \]
    converges.  Look up what this series converges to and write it down. This is $\zeta(2)$.
    \item Use the comparison test to show that the $p$-series
    \[
    \sum_{n=1}^\infty \frac{1}{n^3}
    \]
    converges. This converges as well to $\zeta(3)$. Look up what this approximate value is.
\end{enumerate}
\end{problem}

\begin{problem}
How can we approximate a (possibly complicated) function by using a power series? Why is this useful (specifically for computation on a computer)?
\end{problem}

\begin{problem} Consider the function
\[
f(x)=\frac{1}{1-x}.
\]
\begin{enumerate}[(a)]
    \item Compute the Maclaurin series for the function.
    \item Find the integral $\int \frac{dx}{1-x}$ using the Maclaurin series for $f(x)$ found in (a).  
    \item Write down the Macluarin series for $\ln(1-x)$ and compare to your answer in (b).
\end{enumerate}
\end{problem}

\begin{problem} 
Compute the Taylor series centered at $a=0$ for $f(x)=e^{-\frac{x^2}{2}}$. Then, instead use the Taylor series for $e^x$ and modify it to work for $f(x)$.  For each of these power series, plot the original function $f(x)$ compared to the four term approximation on the same graph.
\end{problem}

\begin{problem}
Find the radius of convergence for the following power series
\begin{enumerate}[(a)]
    \item $\displaystyle{\sum_{n=1}^\infty \frac{x^n}{n(n+1)}}$;
    \item $\displaystyle{\sum_{n=0}^\infty \frac{x^{2n+1}}{(2n+1)!}}$.
\end{enumerate}
\end{problem}


\end{document}