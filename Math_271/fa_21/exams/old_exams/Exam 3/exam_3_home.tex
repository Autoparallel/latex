\documentclass[12pt]{amsbook}
\usepackage{preamble}


\begin{document}
\pagenumbering{gobble}       % This kills the page numbering

\begin{center}
   \textsc{\large MATH 271, Exam 3}\\
   \textsc{Take Home Portion}\\
   \textsc{Due December 4$^\textrm{th}$ at the start of class}
\end{center}
\vspace{1cm}

\noindent\textbf{Name} \; \underline{\hspace{8cm}}

\vspace{1cm}

\noindent\textbf{Instructions} \; You are allowed a textbook, homework, notes, worksheets, material on our Canvas page, but no other online resources (including calculators or WolframAlpha). for this portion of the exam.  \textbf{Do not discuss any problem any other person.} All of your solutions should be easily identifiable and supporting work must be shown.  Ambiguous or illegible answers will not be counted as correct. \textbf{Print out this sheet and staple your solutions to it. Use a new page for each problem.}


\vspace{1cm}

\begin{center}\textbf{Problem 1} \; \underline{\hspace{1cm}}/15 \qquad \qquad \textbf{Problem 2} \; \underline{\hspace{1cm}}/10\end{center}

\vspace{1cm}

\hrule

\vspace*{1cm}
\noindent\emph{Note, these problems span two pages.}

\newpage
\begin{problem}
Consider the vectors $\vecu = \xhat + \yhat$, $\vecv = \xhat+\zhat$, and $\vecw = \yhat + \zhat$.  
\begin{enumerate}[(a)]
    \item \textbf{(2 pts.)} Write the matrix $[A]$ whose columns are $\vecu$, $\vecv$, and $\vecw$ and compute the determinant. Specifically, let
    \[
    [A]=\begin{pmatrix} \vert & \vert & \vert \\ \vecu & \vecv & \vecw \\ \vert & \vert & \vert \end{pmatrix}.
    \]
    then find $\det([A])$.
    \item \textbf{(1 pt.)} What is the volume of the parallelepiped generated by the vectors $\vecu$, $\vecv$, and $\vecw$?
    \item \textbf{(2 pts.)} Are the vectors $\vecu$, $\vecv$, and $\vecw$ linearly independent? Explain.
    \item \textbf{(2 pts.)} Explain geometrically why any inhomogeneous equation 
    \[
    [A]\vecx = \vecy
    \]
    (where $\vecy\neq \zerovec$) has a unique solution.  \emph{Hint: your answer from (c) may help you.}
    \item \textbf{(2 pts.)} Without computing the eigenvalues, argue why $[A]$ cannot have an eigenvalue of zero.
    \item \textbf{(2 pts.)} Is $[A]$ a symmetric matrix? Explain.
    \item \textbf{(2 pts.)} Without computing the eigenvalues, argue why all the eigenvalues must be real.
    \item \textbf{(2 pts.)} Show that $\det([A]-[I])=0$ and argue why $[A]$ must have at least one eigenvalue that is equal to 1.
\end{enumerate}
\end{problem}

\newpage
\begin{problem}
\emph{Spin} is an observable quantum phenomenon which describes intrinsic angular momentum of particles.  For example, electrons and positrons are spin-1/2 particles whereas photons are spin-1 particles.  An example of a massive spin-1 particle is \emph{triplet oxygen} in its ground state (which you can read more about here: \url{https://en.wikipedia.org/wiki/Triplet_oxygen}). The following matrix
\[
[S]_y=\frac{i\hbar}{\sqrt{2}}\begin{pmatrix} 0 & -1 & 0 \\ 1 & 0 & -1 \\ 0 & 1 & 0 \end{pmatrix},
\]
is Hermitian and it describes the measurement of spin aligned along the $y$-axis of a \emph{Stern-Gerlach apparatus}.

If we pass a spin-1 particle through the Stern-Gerlach apparatus then the possible observed states are the eigenvectors of $[S]_y$ and the angular momentum aligned with the $y$-axis is given by the respective eigenvalue. 
\begin{enumerate}[(a)]
    \item \textbf{(2 pts.)} Compute the eigenvalues of $[S]_y$.
    \item \textbf{(3 pts.)} Compute the eigenvectors of $[S]_y$.
    \item \textbf{(3 pts.)} Show that the eigenvectors are orthogonal with respect to the Hermitian inner product
    \[
    \langle \vecu,\vecv \rangle \coloneqq \sum_{i=1}^3 u_i v_i^*,
    \]
    where $~^*$ denotes the complex conjugate.
    \item \textbf{(3 pts.)} We can prepare a spin-1 particle in the state
    \[
    \boldsymbol{\vec{\Psi}} = \frac{1}{\sqrt{3}}\begin{pmatrix} 1 \\ 1 \\ 1 \end{pmatrix}.
    \]
    Then, we can compute the \emph{expected value} of the spin angular momentum of the particle $\boldsymbol{\vec{\Psi}}$ by computing
    \[
    E([S]_y) \coloneqq \langle \boldsymbol{\vec{\Psi}}, [S]_y \boldsymbol{\vec{\Psi}} \rangle. 
    \]
    Compute this expected value.
\end{enumerate}
\end{problem}








\end{document}  
