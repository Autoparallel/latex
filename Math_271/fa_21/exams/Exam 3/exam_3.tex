\documentclass[12pt]{amsbook}
\usepackage{preamble}
%\newcommand{\vecy}{\boldsymbol{\vec{y}}}
%\newcommand{\vecu}{\boldsymbol{\vec{u}}}
%\newcommand{\vecv}{\boldsymbol{\vec{v}}}
%\newcommand{\vecw}{\boldsymbol{\vec{w}}}
\newcommand{\R}{\mathbb{R}}


\begin{document}
\pagenumbering{gobble}       % This kills the page numbering

\begin{center}
   \textsc{\large MATH 271, Exam 3}\\
   \textsc{Oral Examination Problems}\\
   \textsc{Due one hour before your exam time slot.}
\end{center}

\vspace{1cm}

\noindent\textbf{Instructions} \; You are allowed a textbook, homework, notes, worksheets, material on our Canvas page.  You can use online tools such as Desmos and Wolfram Alpha to check your work, but you will need to explain how you arrived at your answers.  You can work with other students and this is, in fact, encouraged! However, I will not be giving out direct help for these problems but can answer questions about previous problems and notes, for example. Ambiguous or illegible answers will not be counted as correct. Scan your solutions and submit them as a pdf on Canvas under Oral Exam 3.


\vspace{1cm}


\hrule

\vspace*{1cm}
\noindent\emph{Note, there are four total problems.}

\newpage
\begin{problem}
Consider the following vectors in $\R^2$:
\[
\vecu = \xhat -3\yhat \quad \vecv = -2\xhat + 2\yhat \quad \vecw = -\xhat -\yhat.
\]
\begin{enumerate}[(a)]
    \item Draw all vectors $\vecu$, $\vecv$, and $\vecw$ in the plane. Draw $\vecu + \vecv$ in the plane as well.
    \item Are any of these vectors orthogonal? Explain.
    \item Explain why $\vecu$ and $\vecv$ form a basis for $\R^2$.
    \item Given the vector $\vecy = 13\xhat + \yhat$, write $\vecy$ as a linear combination of $\vecu$ and $\vecv$.
\end{enumerate}
\end{problem}

\newpage
\begin{problem}
Consider the vectors $\vecu, \vecv \in \R^3$ given by
\[
\vecu = -\xhat - 2\yhat \qquad \textrm{and} \qquad \vecv = 3\zhat.
\]
\begin{enumerate}[(a)]
    \item Compute $\vecu \times \vecv$.
    \item Given any $\vecw \in \R^3$ with $\vecw = w_1 \xhat + w_2 \yhat + w_3\zhat$ we can create the matrix
    \[
    [\vecw] = \begin{pmatrix} 0 & -w_3 & w_2 \\ w_3 & 0 & -w_1 \\ -w_2 & w_1 & 0 \end{pmatrix}.
    \]
    Show that 
    \[
    [\vecu \times \vecv] = [\vecu][\vecv]-[\vecv][\vecu].
    \]
\end{enumerate}

\end{problem}


\newpage
\begin{problem}
Consider the matrices
\[
[A] = \begin{pmatrix} 1 & 0 & 0 \\ 0 & 0 & 1 \\ 0 & 1 & 0 \end{pmatrix} \quad [B] = \begin{pmatrix} 1 & 0 & 0 \\ 0 & 1 & 0 \\ 0 & 0 & 0 \end{pmatrix} \quad [C] = \begin{pmatrix} 1 & 2 & 0 \\ 0 & 2 & 1 & \\ 1 & 0 & 1 \end{pmatrix}.
\]
\begin{enumerate}[(a)]
    \item Explain what the transformation $[A]$ does to the basis vectors $\xhat$, $\yhat$, and $\zhat$.
    \item Find $[A]^{-1}$.
    \item Find the volume of the parallapiped generated by the columns of $[C]$.
    \item Find $\operatorname{Null}([B])$.
\end{enumerate}
\end{problem}


\newpage
\begin{problem}
Consider the same matrix
\[
[A] = \begin{pmatrix} 1 & 0 & 0 \\ 0 & 0 & 1 \\ 0 & 1 & 0 \end{pmatrix}.
\]
\begin{enumerate}[(a)]
    \item Is $[A]$ hermitian? Explain.
    \item Show that $\xhat$ is an eigenvector with eigenvalue $\lambda_1=1$.
    \item Compute $\det([A])$ and $\tr([A])$ and using these quantities plus your knowledge from (b), show that the other two eigenvalues are $\lambda_2 = 1$ and $\lambda_3 = -1$. (DO NOT USE THE CHARACTERISTIC POLYNOMIAL!)
    \item Find the remaining eigenvectors for $[A]$ and diagonalize $[A]$.
\end{enumerate}
\end{problem}





\end{document}  