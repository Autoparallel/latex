\documentclass[12pt]{amsbook}
\usepackage{preamble}


\begin{document}
\pagenumbering{gobble}       % This kills the page numbering

\begin{center}
   \textsc{\large MATH 271, Quiz 2}\\
   \textsc{Due October 9$^\textrm{th}$ at the end of class}
\end{center}

\vspace{1cm}

\noindent\textbf{Instructions} \; You are allowed a textbook, homework, notes, worksheets, material on our Canvas page, but no other online resources (including calculators or WolframAlpha) for this quiz.  \textbf{Do not discuss any problem any other person.} All of your solutions should be easily identifiable and supporting work must be shown.  Ambiguous or illegible answers will not be counted as correct.


\vspace*{.5cm}
\hrule
\vspace*{.5cm}

\begin{center}\textbf{\large THERE ARE 6 TOTAL PROBLEMS.}\normalsize \end{center}

\begin{problem} For the following, say whether the statement is true or false. For full credit, justify your answer with an explanation.
\begin{enumerate}[(a)]
    \item \textbf{(2 pts.)} If the sequence $a_n \to 0$, then $\displaystyle{\sum_{n=1}^\infty a_n}$ converges.
    \item \textbf{(2 pts.)} The power series $\displaystyle{\sum_{n=1}^\infty x^n}$ has an infinite radius of convergence.
\end{enumerate}
\end{problem}

\begin{problem}
\textbf{(3 pts.)} What is the derivative of the power series
\[
f(x) = \sum_{n=1}^\infty \frac{x^n}{n^n}
\]
\end{problem}



\begin{problem}
\textbf{(3 pts.)} What is the antiderivative of the power series
\[
g(x) = \sum_{n=0}^\infty (n+1)x^n
\]
\end{problem}


\begin{problem}
\textbf{(3 pts.)} For what values of $x$ does the above series $g(x)$ converge? In other words, what is the radius of convergence?
\end{problem}
\vspace*{.5cm}

\begin{problem}~
\begin{enumerate}[(a)]
    \item \textbf{(3 pts.)} Write down the first order approximation to $\tan(x)$ about the point $x=0$.
    \item \textbf{(2 pts.)} Explain how you would determine a higher order approximation.
\end{enumerate}
\end{problem}

\newpage

\begin{problem}
\textbf{(2 pts.)} Consider the differential equation
\[
x' = t^2 + \tan(x).
\]
Explain how you can use a first order approximation to $\tan(x)$ to approximate the above (nonlinear) equation as a first order linear equation. \textcolor{red}{Should say something about initial conditions}
\end{problem}













\end{document}  