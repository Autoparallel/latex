%%%%%%%%%%%%%%%%%%%%%%%%%%%%%%%%%%%%%%%%%%%%%%%%%%%%%%%%%%%%%%%%%%%%%%%%%%%%%%%%%%%%
% Document data
%%%%%%%%%%%%%%%%%%%%%%%%%%%%%%%%%%%%%%%%%%%%%%%%%%%%%%%%%%%%%%%%%%%%%%%%%%%%%%%%%%%%
\documentclass[12pt]{article} %report allows for chapters
%%%%%%%%%%%%%%%%%%%%%%%%%%%%%%%%%%%%%%%%%%%%%%%%%%%%%%%%%%%%%%%%%%%%%%%%%%%%%%%%%%%%
\usepackage{preamble}

\begin{document}

\begin{center}
   \textsc{\large MATH 271, Worksheet 10}\\
   \textsc{Inverse and similar matrices. Eigenvalue problem and diagonalization. Hermitian matrices.}
\end{center}
\vspace{.5cm}


\begin{problem}
Consider the two matrices 
\[
[A] = \begin{pmatrix} 3 & 1 \\ 6 & 2 \end{pmatrix} \qquad \textrm{and} \qquad [B] = \begin{pmatrix} 1 & 2\\ 2 & 1 \end{pmatrix}.
\]
\begin{enumerate}[(a)]
    \item Argue why the matrix $[A]$ cannot be invertible. 
    \item Compute the inverse matrix $[B]^{-1}$ for $[B]$.  
    \item Solve the system of equations $[B]\vec{\boldsymbol{x}} = \vec{\boldsymbol{y}}$ for the following vectors.
    \begin{enumerate}[i.]
        \item $\vec{\boldsymbol{y}} = \begin{pmatrix} 2 \\ 2 \end{pmatrix}$.
        \item $\vec{\boldsymbol{y}} = \begin{pmatrix} 0 \\ 1 \end{pmatrix}$.
        \item $\vec{\boldsymbol{y}} = \begin{pmatrix} 0 \\ 0 \end{pmatrix}$.
    \end{enumerate}
\end{enumerate}
\end{problem}

\begin{problem}
Consider the matrices 
\[
[A] = \begin{pmatrix} 1 & 1 \\ 0 & 1 \end{pmatrix} \qquad \textrm{and} \qquad [B] = \begin{pmatrix} 2 & 1 \\ 1 & 2 \end{pmatrix}.
\]
\begin{enumerate}[(a)]
    \item Show that $[A]$ and $[B]$ are both invertible.
    \item Find $[A]^{-1}$ and $[B]^{-1}$.
    \item Show that $([A][B])^{-1} = [B]^{-1} [A]^{-1}$.
\end{enumerate}
\end{problem}

\begin{problem}
Simplify the following expressions.
\begin{enumerate}[(a)]
    \item $([A][B])^{-1} [A] [B]$.
    \item $[A]^2[B]^3[A] ([A][B])^{-1}$.
    \item $([A][B][C]^{-1})^{-1} [A][B] [C]^{-1}$.
\end{enumerate}
\end{problem}

\begin{problem}
Show that for any invertible matrix $[A]$ that $\det([A]^{-1})=\frac{1}{\det([A])}$.
\end{problem}

\begin{problem}
Let $[B]$ be similar to $[A]$ by the relationship $[B]=[P]^{-1} [A] [P]$.
\begin{enumerate}[(a)]
    \item Given that $[P]$ is invertible, show that $[P]$ transforms the standard basis $\xhat_1,\xhat_2,\dots,\xhat_n$ into a new basis given by the columns of $[P]$.
    \item Show that $[P]^{-1}$ transforms the basis given by the columns of $[P]$ into the standard basis.
    \item Explain why $[B]$ performs the same transformation as $[A]$ but just on a different basis (e.g., different choices of coordinates).
\end{enumerate}
\end{problem}

\begin{problem}
Let $[B]$ be similar to $[A]$ by the relationship $[B]=[P]^{-1} [A] [P]$.  
\begin{enumerate}[(a)]
    \item Show that the trace is invariant under similarity. That is, show $\tr ([A]) = \tr([B])$.
    \item Show that the determinant is invariant under similarity. \emph{Hint: you will need to use the result from Problem 4.}
    \item Show that $[A]$ and $[B]$ have the same eigenvalues. It may help to think that if we have $\vecv$ as an eigenvector for $[A]$, then what is the corresponding eigenvector for $[B]$?
\end{enumerate}
\end{problem}

\begin{problem}
    Compute the eigenvalues and eigenvectors for the following matrices.
\begin{enumerate}[(a)]
    \item $[A] = \begin{pmatrix} 2 & 1 \\ 0 & 3 \end{pmatrix}$.
    \item $[B] = \begin{pmatrix} 1 & 3 \\ 5 & 1 \end{pmatrix}$.
    \item $[C] = \begin{pmatrix} 1 & 1 & 0 \\ 1 & 0 & 1 \\ 0 & 1 & 1 \end{pmatrix}$.
\end{enumerate}
\end{problem}

\begin{problem}
Diagonalize the above matrices (if possible).
\end{problem}

\begin{problem}
Argue why the eigenvectors corresponding to a zero eigenvalue are elements of the nullspace.
\end{problem}

\begin{problem}
Show that there must be at least one zero eigenvalue if the determinant of a matrix is zero. Explain what this means geometrically and relate it beck to the geometric interpretation of the determinant.
\end{problem}

\begin{problem}
Given the matrix
\[
[A] = \begin{pmatrix} 2 & 1 \\ 0 & 2 \end{pmatrix}.
\]
\begin{enumerate}[(a)]
    \item Using the definition of the adjoint and hermitian (self-adjoint), show that $[A]$ is not hermitian.
    \item Show that there exists only one eigenvector for $[A]$ (e.g., one linearly independent vector in $\operatorname{Null}([A]-\lambda[I])$.
    \item Show that there exists two linearly independent vectors in $\operatorname{Null}(([A]-\lambda [I])^2)$. 
\end{enumerate}
\end{problem}

\end{document}
