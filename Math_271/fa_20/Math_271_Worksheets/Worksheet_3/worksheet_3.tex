%%%%%%%%%%%%%%%%%%%%%%%%%%%%%%%%%%%%%%%%%%%%%%%%%%%%%%%%%%%%%%%%%%%%%%%%%%%%%%%%%%%%
% Document data
%%%%%%%%%%%%%%%%%%%%%%%%%%%%%%%%%%%%%%%%%%%%%%%%%%%%%%%%%%%%%%%%%%%%%%%%%%%%%%%%%%%%
\documentclass[12pt]{article} %report allows for chapters
%%%%%%%%%%%%%%%%%%%%%%%%%%%%%%%%%%%%%%%%%%%%%%%%%%%%%%%%%%%%%%%%%%%%%%%%%%%%%%%%%%%%
\usepackage{preamble}

\begin{document}

\begin{center}
   \textsc{\large MATH 271, Worksheet 3}\\
   \textsc{First order autonomous and linear equations. Chemical kinetics.}
\end{center}
\vspace{.5cm}

\begin{problem}
    Let's revisit Newton's law of cooling and describe the equilibria of the system. 
\begin{enumerate}[(a)]
    \item Is the system autonomous? Explain.
    \item Draw a phase line for the system.
    \item Can you explain why the value you find for the equilibrium makes sense? Is this equilibrium be stable? Does this make sense? Explain.
\end{enumerate}
\end{problem}

\begin{problem}
Are the following ODEs separable, autonomous, linear, nonlinear, or none of the above? Keep in mind that some may satisfy more than one property!
    \begin{enumerate}[(a)]
        \item $x' = \sin(tx)$.
        \item $x' = \sin(x)$.
        \item $x' = t^2 x$.
        \item $e^t x' + tx = e^{-k t} \cos(\omega t).$
    \end{enumerate}
\end{problem}

\begin{problem}
For the following problems, show that the equation is linear by writing the equation in a recognizable form.
    \begin{enumerate}[(a)]
        \item $\frac{x'}{x} = t$.
        \item $2xx'+x^2=xt$.
        \item $\tan(t) x' + \sin(t) x = \ln(t).$
    \end{enumerate}
\end{problem}

\begin{problem}
    For the above linear equations, determine the integrating factor (even if you cannot compute the integral) and determine the solution $x(t)$ (again, even if you cannot compute the integral).
\end{problem}

\begin{problem}
    Find the solution to the equation $x'+x=t^2$.
\end{problem}

\begin{problem}
    Consider the dissociation chemical reaction
    \[
        AB \to A + B.
    \]
    Write down an equation for each species $A$, $B$, and $AB$.  Find a solution.
\end{problem}

\begin{problem}
    Compare and contrast the equations for the above reaction and the synthesis reaction
    \[
        A+B \to AB.
    \]
    \emph{Hint: maybe this is a bit of an odd way to think, but is one just the reversal in time of the other?}
\end{problem}

\begin{problem}
    Photosynthesis is an extremely important chemical reaction where plants convert carbon dioxide and water into glucose and oxygen. That is,
    \[
        6CO_2 + 6H_2O \to 6C_6 H_{12} O_6 + 6 O_2.
    \]
    \begin{enumerate}[(a)]
        \item Write down the equations that describe the above equation.
        \item Do we have any techniques to solve this type of equation (yet)?
        \item The reaction also depends on the intensity of light.  How can we implement this into the equation?
    \end{enumerate}   
\end{problem}

\begin{problem}
Everybody loves combustion. It keeps us warm and it looks cool on the 4$^\textrm{th}$ of July!  Anyhow, combustion of propane in your grill at home is given by the equation
    \[
        C_3 H_8 + 5O_2 \to 4H_2O + 3CO_2 + \textrm{Energy}.
    \]
    \begin{enumerate}[(a)]
        \item Write down the equations describing the above reaction (not including the energy term).
        \item If a specific amount of energy is given off by each reaction (in this case, $2043\textrm{kJ}$ of energy per mole), include an equation for the total energy created if we begin with $1\textrm{kg}$ of propane in an abundance of oxygen.
        \item Can you give some analogy to how much energy this is so that we may understand the usefulness a bit more? E.g., how many hot dogs could I cook on the 4$^\textrm{th}$ of July?
    \end{enumerate}

\end{problem}






\end{document}
