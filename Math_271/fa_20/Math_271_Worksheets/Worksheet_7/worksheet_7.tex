%%%%%%%%%%%%%%%%%%%%%%%%%%%%%%%%%%%%%%%%%%%%%%%%%%%%%%%%%%%%%%%%%%%%%%%%%%%%%%%%%%%%
% Document data
%%%%%%%%%%%%%%%%%%%%%%%%%%%%%%%%%%%%%%%%%%%%%%%%%%%%%%%%%%%%%%%%%%%%%%%%%%%%%%%%%%%%
\documentclass[12pt]{article} %report allows for chapters
%%%%%%%%%%%%%%%%%%%%%%%%%%%%%%%%%%%%%%%%%%%%%%%%%%%%%%%%%%%%%%%%%%%%%%%%%%%%%%%%%%%%
\usepackage{preamble}

\begin{document}

\begin{center}
   \textsc{\large MATH 271, Worksheet 7}\\
   \textsc{Linear Transformations and Matrices}
\end{center}
\vspace{.5cm}

\begin{problem}
Compare and contrast the structure of the complex numbers $\C$ with the vector space $\R^2$.  Note any differences and similarities.
\end{problem}

\begin{problem}
Let $\vecu,\vecv\in \R^2$ be given by
\[
\vecu = 2\xhat + 3 \yhat \qquad \textrm{and} \qquad \vecv = -\xhat +\yhat.
\]
\begin{enumerate}[(a)]
    \item Draw $\vecu$, $\vecv$, and $\vecu + \vecv$ in the plane.
    \item Compute $\|\vecu\|$ and $\|\vecv\|$.
    \item Compute $\vecu \cdot \vecv$.
    \item Find a vector orthogonal to $\vecu$.
\end{enumerate}
\end{problem}

\begin{problem}
Let $\vecu,\vecv\in \R^3$ be given by
\[
\vecu = \xhat - \yhat + \zhat \qquad \textrm{and} \qquad \vecv = -\xhat + \yhat - \zhat.
\]
\begin{enumerate}[(a)]
    \item Are $\vecu$ and $\vecv$ orthogonal?
    \item Normalize $\vecu$ and $\vecv$ to get $\uhat$ and $\vhat$. 
    \item Compute the projection of $\vecv$ onto the direction defined by $\vecu$.
\end{enumerate}
\end{problem}

\begin{problem}
Let $\vecu,\vecv \in \R^3$ be given by
\[
\vecu = -3\xhat -2\yhat + \zhat \qquad \textrm{and} \qquad \vecv = \xhat -2\yhat +\zhat.
\]
\begin{enumerate}[(a)]
    \item Compute the angle between $\vecu$ and $\vecv$.
    \item Without computing the cross product, compute the area of the parallelogram generated by $\vecu$ and $\vecv$. \emph{Hint: you know the angle between the vectors, use this fact.}
    \item Without computing the cross product, what component of the product $\vecu\times\xhat$ must be zero? 
    \item Compute $\vecu\times \vecv$.
    \item Give a geometrical interpretation of the cross product $\vecu\times \vecv$. Explain why $\vecu\times \vecv = -\vecv \times \vecu$.
\end{enumerate}
\end{problem}

\begin{problem}
Recall that the states found in the solution to the free particle in a 1-dimensional box of length $L$ were $\psi_n = \sqrt{\frac{2}{L}} \sin \left( \frac{n\pi x}{L}\right)$. Let $S$ denote the set of all solutions to the free particle in a 1-dimensional box boundary value problem. Show that a superposition of states (with coefficients in $\C$) is also a solution. That is, if we let $\Psi(x) = \alpha_{j}(x) \psi_j + \alpha_k \psi_k(x)$, then $\Psi(x)$ is also a solution to the boundary value problem
\[
-\frac{\hbar^2}{2m}\frac{d^2 \Psi}{dx^2}=E\Psi
\]
with boundary values $\Psi(0)=0$ and $\Psi(L)=0$.

\noindent \emph{Note that this shows that the set $S$ is a vector space over the complex numbers $\C$.}
\end{problem}

\begin{problem}
Consider the transformation $T\colon \R^2 \to \R^3$ given by
\[
T \begin{pmatrix} x \\ y \end{pmatrix} = \begin{pmatrix} x \\ y \\ x+y \end{pmatrix}.
\]
\begin{enumerate}[(a)]
    \item Show that this transformation is linear.
    \item Write down a matrix for this linear transformation.
    \item Can you draw a picture of the output of this transformation? What kind of object is it?
\end{enumerate}
\end{problem}

\begin{problem}
Consider the system of linear equations:
\begin{align*}
    3x+2y+0z&=5\\
    1x+1y+1z&=3\\
    0x+2y+2z&=4.
\end{align*}
\begin{enumerate}[(a)]
    \item Write the augmented matrix $M$ for this system of equations.
    \item Use row reduction to get the augmented matrix in row-echelon form.
    \item Determine the solution to the system of equations.
\end{enumerate}
\end{problem}

\begin{problem}
Consider the equation
\[
\begin{pmatrix} 1 & 3 & 4 \\ 2 & 9 & 9 \\ 1 & 5 & 5 \end{pmatrix} \begin{pmatrix} x \\ y \\ z \end{pmatrix} = \begin{pmatrix} 8 \\ 20 \\ 11 \end{pmatrix}.
\]
Does this equation have a solution or not? If so, determine the solution.
\end{problem}

\begin{problem}
Consider the linear transformations on $\R^3$ to $\R^3$ given by
\begin{align*}
    R_x(\theta) &= \begin{pmatrix} 1 & 0 & 0 \\ 0 & \cos\theta & -\sin \theta \\ 0 & \sin\theta & \cos \theta \end{pmatrix}\\
    R_y(\theta) &= \begin{pmatrix} \cos \theta & 0 & \sin \theta \\ 0 & 1 & 0 \\ -\sin \theta & 0 & \cos \theta \end{pmatrix}\\
    R_z(\theta) &= \begin{pmatrix} \cos \theta & -\sin \theta & 0 \\ \sin \theta & \cos \theta  & 0 \\ 0 & 0 & 1 \end{pmatrix}.
\end{align*}
\textbf{Fact:} These matrices are generators for the \emph{group of rotations} $\SO(3)$ of $\R^3$.
\begin{enumerate}[(a)]
    \item Let $\theta = \pi/2$. Show that $R_x(\pi/2)$ rotates a vector by $\pi/2$ radians in the $xy$-plane.
    \item Show that the determinant of each of these matrices is 1 for any value of $\theta$.
    \item Using properties of determinants, show that the determinant of a product of rotation matrices is also 1.
    \item Explain geometrically why a rotation matrix must have a determinant of 1.
    \item Show that $R_x(\theta)R_x(\theta)^\dagger = I$. This in fact true for any rotation matrix.
\end{enumerate}
\end{problem}

\begin{problem}
Consider the matrix 
\[
M = \begin{pmatrix} 1 & 2 & 3 \\ 4 & 5 & 6 \\ 7 & 8 & 9 \end{pmatrix}.
\]
\begin{enumerate}[(a)]
    \item Compute $\tr(M)$. 
    \item Compute $M^{R_x}=R_x(\pi/2)MR_x(\pi/2)^\dagger$.
    \item What is the trace of $M^{R_x}$?
    \item Can you see why you have the answer in (c) from properties of the trace?
\end{enumerate}
\end{problem}


\begin{problem}
Consider the following three vectors $\vecu,\vecv,\vecw\in \R^3$ given by
\[
\vecu = \xhat + \yhat +\zhat, \qquad \vecv = 2\xhat + \yhat + 2\zhat, \qquad \vecw = -2\xhat + \yhat +\zhat. 
\]
\begin{enumerate}[(a)]
    \item We can write a linear combination of these vectors by taking
    \[
    \alpha \vecu + \beta \vecv + \gamma \vecw,
    \]
    where $\alpha,\beta,\gamma \in \R$.  Write this linear combination as a matrix times a vector.
    \item Does this list of vectors form a basis for $\R^3$? \emph{Hint: use the above work. Can any vector in $\R^3$ be written as a linear combination of these vectors?}
\end{enumerate}
\end{problem}


\end{document}
