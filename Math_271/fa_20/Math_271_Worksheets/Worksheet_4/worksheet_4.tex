%%%%%%%%%%%%%%%%%%%%%%%%%%%%%%%%%%%%%%%%%%%%%%%%%%%%%%%%%%%%%%%%%%%%%%%%%%%%%%%%%%%%
% Document data
%%%%%%%%%%%%%%%%%%%%%%%%%%%%%%%%%%%%%%%%%%%%%%%%%%%%%%%%%%%%%%%%%%%%%%%%%%%%%%%%%%%%
\documentclass[12pt]{article} %report allows for chapters
%%%%%%%%%%%%%%%%%%%%%%%%%%%%%%%%%%%%%%%%%%%%%%%%%%%%%%%%%%%%%%%%%%%%%%%%%%%%%%%%%%%%
\usepackage{preamble}

\begin{document}

\begin{center}
   \textsc{\large MATH 271, Worksheet 4}\\
   \textsc{Sequences and Series}
\end{center}
\vspace{.5cm}

\begin{problem}
Write down the first few terms in the sequence for the following:
\begin{enumerate}[(a)]
    \item $a_n = n$;
    \item $b_n = \frac{1}{n^2}$;
    \item $c_n = 2^{-n}$.
\end{enumerate}
\end{problem}

\begin{problem}
For the above sequences, state whether each converges or diverges.  If they converge, state the limit.
\end{problem}

\begin{problem}
Consider the recursive sequence
\[
a_n = \frac{1}{2} a_{n-1} + 1
\]
with $a_1 = 1$.  
\begin{enumerate}[(a)]
    \item Write the first few terms in the sequence.
    \item Can you write $a_n$ as a function $f(n)$? If so, what is $f(n)$?
    \item Does this sequence converge or diverge? Can you show why with a limit $\lim_{n\to \infty} f(n)$?
    \item Can you show that this is a Cauchy sequence?
\end{enumerate}
\end{problem}

\begin{problem}
Consider the sequence
\[
a_n = a r^n.
\]
\begin{enumerate}[(a)]
    \item If $|r|<1$, show that this sequence $\{a_n\}_{n=0}^\infty$ converges to zero.
    \item Consider now the \emph{geometric series}
    \[
    \sum_{n=0}^\infty a_n = \sum_{n=0}^\infty ar^n.
    \]
    Show that the $N^\textrm{th}$ partial sum for this series satisfies
    \[
    \sum_{n=0}^N ar^n = a\left( \frac{1-r^{N+1}}{1-r}\right).
    \]
    \item Does the geometric series converge for all $r$? For $|r|<1$? When it converges, what does it converge to?
\end{enumerate}
\end{problem}

\begin{problem}
Often we wish to think about functions being represented by series.  For example, we can consider the function
\[
f(x)=\sum_{n=0}^\infty \frac{x^n}{n!}
\]
where $n!$ is read as ``$n$-factorial" and 
\[
n! = n\cdot (n-1)\cdot (n-2) \cdots 2 \cdot 1.
\]
Then $1!=1$ and we define $0!=1$ as well.
\begin{enumerate}[(a)]
    \item Consider $f(1)$.  Use a tool like WolframAlpha to compute the series
    \[
    f(1)=\sum_{n=0}^\infty \frac{1}{n!}
    \]
    \item For any value of $x$, this series converges. So this defines a function on all real numbers. In fact, the series converges even for complex numbers. Simplify the series into its real and imaginary parts. Note,
    \[
    f(ix) = \sum_{n=0}^\infty \frac{(ix)^n}{n!}.
    \]
    \item We can take derivatives of the function $f(x)$ by differentiating the series \emph{term by term}. That is,
    \[
    \frac{d}{dx} f(x) = \frac{d}{dx} \sum_{n=0}^\infty \frac{x^n}{n!} = \sum_{n=0}^\infty \frac{d}{dx} \left( \frac{ x^n}{n!}\right).
    \]
    \item Show that $\frac{d}{dx}f(x)=f(x)$.
    \item What is your guess for what function $f(x)$ is?
\end{enumerate}
\end{problem}




\end{document}
