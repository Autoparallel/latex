\documentclass[12pt]{amsbook}
\usepackage{preamble}
\newcommand{\R}{\mathbb{R}}


\begin{document}
\pagenumbering{gobble}       % This kills the page numbering

\begin{center}
   \textsc{\large MATH 271, Quiz 4}\\
   \textsc{Due November 13$^\textrm{th}$ at the end of class}
\end{center}

\vspace{1cm}

\noindent\textbf{Instructions} \; You are allowed a textbook, homework, notes, worksheets, material on our Canvas page, but no other online resources (including calculators or WolframAlpha) for this quiz.  \textbf{Do not discuss any problem any other person.} All of your solutions should be easily identifiable and supporting work must be shown.  Ambiguous or illegible answers will not be counted as correct.


\vspace*{.5cm}
\hrule
\vspace*{.5cm}

\begin{center}\textbf{\large THERE ARE 5 TOTAL PROBLEMS.}\normalsize \end{center}

\begin{problem} Consider the vectors $\vecu = 3\xhat -\yhat$, $\vecv = -\yhat$, and $\vecw = -\xhat + 2\yhat$.
\begin{enumerate}[(a)]
    \item \textbf{(2 pts.)} Draw the vectors $\vecu$, $\vecv$, $\vecw$, and $\vecu+\vecv$ in the plane.
    \item \textbf{(3 pts.)} Compute $\vecu \cdot \vecv$, $\|\vecu\|$, and $\|\vecv\|$.
    \item \textbf{(2 pts.)} What is the angle betwen $\vecu$ and $\vecv$? (Do not worry about getting a numerical answer, just show the work to explain what the angle is.)
\end{enumerate}
\end{problem}

\begin{problem}
Consider the following functions.
\begin{enumerate}[i.]
    \item $f\colon \R \to \R$ given by $f(x)=2x+1$.
    \item $g\colon \R^2 \to \R$ given by $g\begin{pmatrix} x \\ y \end{pmatrix} = x+y$.
\end{enumerate}
\begin{enumerate}[(a)]
    \item \textbf{(2 pts.)} Is $f$ linear or nonlinear? Explain.
    \item \textbf{(2 pts.)} Show that $g$ is linear.
\end{enumerate}
\end{problem}



\begin{problem}
Consider the matrices
\[
[A] = \begin{pmatrix} 1 & 2 \\ 3 & 4 \end{pmatrix}, \quad [B] = \begin{pmatrix} 3 & 1 & 1 \\ 0 & 2 & -1 \end{pmatrix}, \quad [C] = \begin{pmatrix} -2 & -1 \\ 1 & 0 \\ 0 & 3 \end{pmatrix}.
\]
\begin{enumerate}[(a)]
    \item \textbf{(2 pts.)} Say which of the following products can you compute
    \[
    [A][B],\quad [B][A], \quad [A][C], \quad [C][A], \quad [B][C], \quad [C][B].
    \]
    \item \textbf{(2 pts.)} What are $n$ and $m$ if we think of $[C] \colon \R^n \to \R^m$?
\end{enumerate}
\end{problem}


\begin{problem}
Consider the two matrices 
\[
[A] = \begin{pmatrix} 1 & 1 \\ 0 & 1 \end{pmatrix}, \quad [B] = \begin{pmatrix} 1 & 1 \\ 2 & 2 \end{pmatrix}.
\]
\begin{enumerate}[(a)]
    \item \textbf{(2 pts.)} Find the solution(s) to the homogeneous equation $[A]\vecx = \zerovec$.
    \item \textbf{(2 pts.)} What is the nullspace of $[A]$?
    \item \textbf{(2 pts.)} Can you find a solution to $[B]\vecx = \vecy$ with $\vecy = 2\xhat +5\yhat$? Explain.
    \item \textbf{(2 pts.)} Explain why $\det([B])=0$ without computing it.
\end{enumerate}
\end{problem}


\begin{problem}
Take the same two two matrices from Problem 4
\[
[A] = \begin{pmatrix} 1 & 1 \\ 0 & 1 \end{pmatrix}, \quad [B] = \begin{pmatrix} 1 & 1 \\ 2 & 2 \end{pmatrix}
\]
\begin{enumerate}[(a)]
    \item \textbf{(2 pts.)} Compute $\det([A])$ and $\det([B])$.
    \item \textbf{(1 pts.)} Compute $\det([A][B])$ without computing $[A][B]$.
    \item \textbf{(2 pts.)} Are the columns of $[A]$ linearly independent? Explain.
\end{enumerate}
\end{problem}

\end{document}  