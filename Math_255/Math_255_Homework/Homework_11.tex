%%%%%%%%%%%%%%%%%%%%%%%%%%%%%%%%%%%%%%%%%%%%%%%%%%%%%%%%%%%%%%%%%%%%%%%%%%%%%%%%%%%%
% Document data
%%%%%%%%%%%%%%%%%%%%%%%%%%%%%%%%%%%%%%%%%%%%%%%%%%%%%%%%%%%%%%%%%%%%%%%%%%%%%%%%%%%%
\documentclass[12pt]{report} %report allows for chapters
\renewcommand\thesection{\arabic{section}} % ignore the title number for sections
%%%%%%%%%%%%%%%%%%%%%%%%%%%%%%%%%%%%%%%%%%%%%%%%%%%%%%%%%%%%%%%%%%%%%%%%%%%%%%%%%%%%




%%%%%%%%%%%%%%%%%%%%%%%%%%%%%%%%%%%%%%%%%%%%%%%%%%%%%%%%%%%%%%%%%%%%%%%%%%%%%%%%%%%%
% Packages
%%%%%%%%%%%%%%%%%%%%%%%%%%%%%%%%%%%%%%%%%%%%%%%%%%%%%%%%%%%%%%%%%%%%%%%%%%%%%%%%%%%%
\usepackage{color, soul, xcolor} % Colored text and highlighting, respectively

%Tikz
\usepackage{tikz-cd} % For commutative diagrams
\usepackage{tikz-3dplot}
\RequirePackage{pgfplots}
\usetikzlibrary{shadows}
\usetikzlibrary{shapes}
\usetikzlibrary{decorations}
\usetikzlibrary{arrows,decorations.markings} 
\usetikzlibrary{quotes,angles}

\usepackage{mathtools}
\usepackage{answers}
\usepackage{setspace}
\usepackage{graphicx}
\usepackage{enumerate}
\usepackage{multicol}
\usepackage{mathrsfs}
\usepackage[margin=1in]{geometry} 
\usepackage{amsmath,amsthm,amssymb}
\usepackage{marvosym,wasysym} %fucking smileys
%%%%%%%%%%%%%%%%%%%%%%%%%%%%%%%%%%%%%%%%%%%%%%%%%%%%%%%%%%%%%%%%%%%%%%%%%%%%%%%%%%%%




%%%%%%%%%%%%%%%%%%%%%%%%%%%%%%%%%%%%%%%%%%%%%%%%%%%%%%%%%%%%%%%%%%%%%%%%%%%%%%%%%%%%
% Shortcuts
%%%%%%%%%%%%%%%%%%%%%%%%%%%%%%%%%%%%%%%%%%%%%%%%%%%%%%%%%%%%%%%%%%%%%%%%%%%%%%%%%%%%
% Number systems
\newcommand{\N}{\mathbb{N}}
\newcommand{\Z}{\mathbb{Z}}
\newcommand{\C}{\mathbb{C}}
\newcommand{\R}{\mathbb{R}}
\newcommand{\Q}{\mathbb{Q}}

% Operators/functions
\newcommand{\id}{\mathrm{Id}}
\DeclareMathOperator{\sech}{sech}
\DeclareMathOperator{\csch}{csch}
%%%%%%%%%%%%%%%%%%%%%%%%%%%%%%%%%%%%%%%%%%%%%%%%%%%%%%%%%%%%%%%%%%%%%%%%%%%%%%%%%%%%




%%%%%%%%%%%%%%%%%%%%%%%%%%%%%%%%%%%%%%%%%%%%%%%%%%%%%%%%%%%%%%%%%%%%%%%%%%%%%%%%%%%%
% Environments
%%%%%%%%%%%%%%%%%%%%%%%%%%%%%%%%%%%%%%%%%%%%%%%%%%%%%%%%%%%%%%%%%%%%%%%%%%%%%%%%%%%%
% Italic font
\newtheorem{theorem}{Theorem}[section]
\newtheorem{lemma}{Lemma}[section]
\newtheorem{corollary}{Corollary}[section]
\newtheorem{axiom}{Axiom}

% Plain font
\theoremstyle{definition}
\newtheorem{definition}{Definition}[section]
\newtheorem{example}{Example}[section]
\newtheorem{remark}{Remark}[section]
\newtheorem{solution}{Solution}[section]
\newtheorem{problem}{Problem}[section]
\newtheorem{question}{Question}[section]
\newtheorem{answer}{Answer}[section]
\newtheorem{exercise}{Exercise}[section]
%%%%%%%%%%%%%%%%%%%%%%%%%%%%%%%%%%%%%%%%%%%%%%%%%%%%%%%%%%%%%%%%%%%%%%%%%%%%%%%%%%%%

\begin{document}


\begin{center}
   \textsc{\large MATH 255, Homework 11}\\
\end{center}
\vspace{.5cm}

\begin{center}
\underline{Problem 1 and 2 are related.}
\end{center}
\noindent\textbf{Problem 1.} Consider the following linear system 
\begin{align*}
    x'(t) &= x-y \\
    y'(t) &= -x-y.
\end{align*}
\begin{enumerate}[(a)]
    \item Rewrite this as a matrix equation
    \[
    \mathbf{v}' = M\mathbf{v}.
    \]
    Here the vector $\mathbf{v}$ denotes the $xy$-position of a particle at time $t$.
    \item Plot the vector field $\mathbf{v}'$.
    \item Describe what happens if your initial data is
    \begin{enumerate}[i.]
        \item $(x_0,y_0)=(0,0)$,
        \item $(x_0,y_0)=(1,1)$,
        \item $(x_0,y_0)=(-1,-1)$.
    \end{enumerate}
    
\end{enumerate}

\noindent\textbf{*Problem 2.} With the same linear system as in 1, do the following.
\begin{enumerate}[(a)]
    \item Compute the eigenvalues of the matrix $M$.
    \item Compute the eigenvectors of the matrix $M$.
    \item Write the general solution for this system.
    \item Find the particular solution corresponding to the initial data $(x_0,y_0)=(1,1)$.
\end{enumerate}
\emph{Feel free to use Wolfram Alpha to do parts of this.  However, you should be able to find eigenvalues on your own!}
\vspace*{.5cm}

\noindent\textbf{Problem 3.} Solving the one dimensional Laplace equation is much like an ODE.  However, the data given looks a bit different.  So consider the following set up.

Consider the Laplace equation
\[
\Delta u(x) = \frac{d^2u}{dx^2} = 0
\]
on the interval $\Omega = (0,1)$ with boundary conditions $u(0)=0$ and $u(1)=1$.  
\begin{enumerate}[(a)]
    \item This equation is separable. To find $u$, take two antiderivatives of
    \[
    \frac{d^2u}{dx^2}=0.
    \]
    \item To verify you did this correctly, take two derivatives of your function to see that you get $0$.
    \item Your function should have two undetermined constants.  Solve for these constants using the boundary conditions provided.
\end{enumerate}
\vspace*{.5cm}

\noindent\textbf{Problem 4.} The methods for solving many PDEs are beyond the scope of this class, but we can still see what solutions behave like and a bit of how to find these. What we'll do below are a few steps of the method of \emph{separation of variables} (not to be confused with separable ODE!) 

Consider the heat equation in one dimension on the region $\Omega = (0,1)$
\[
\frac{\partial u}{\partial t}(x,t) - \frac{\partial^2 u}{\partial x^2}(x,t) = 0
\]
with boundary conditions $u(0,t)=0$ and $u(1,t)=0$, and initial condition $u(x,0)=\sin(\pi x)$.
\begin{enumerate}[(a)]
    \item Show that $f(x)=\sin(\pi x)$ is a solution to
    \[
    f''(x)=-\pi^2 f(x)
    \]
    with $f(0)=0$ and $f(1)=0$.
    \item Show that $g(t) = e^{-\pi^2 t}$ is a solution to 
    \[
    g'(t) = -\pi^2 g(t)
    \]
    with $g(0)=1$.
    \item Show that $u(x,t)=f(x)g(t)$ solves the heat equation with these boundary and initial conditions.
\end{enumerate}
\vspace*{.5cm}

\noindent\textbf{Problem 5.} With our solution from 4, we can analyze the behavior of the system.  The physical phenomenon that Problem 4 modelled was a thin rod (the segment $(0,1)$) that had an initial temperature distribution $\sin(\pi x)$, i.e. it was warmer in the middle and coldest on the ends.  The boundary conditions $u(0)=0$ and $u(1)=0$ can be thought of as attaching a thermocouple at each end that holds the end temperature at $0$ degrees.  
\begin{enumerate}[(a)]
    \item Plot the function on CalcPlot3D by plotting
    \[
    z=e^{-\pi^2 y} \sin(\pi x)= u(x,y),
    \]
    where we just let the $t$ variable be denoted by $y$ to plot this function.
    \item Can you explain what happens as time $t$ moves forward based on your intuition, plot, or by the equation we found in 4?
\end{enumerate}



\end{document}  