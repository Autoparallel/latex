%%%%%%%%%%%%%%%%%%%%%%%%%%%%%%%%%%%%%%%%%%%%%%%%%%%%%%%%%%%%%%%%%%%%%%%%%%%%%%%%%%%%
% Document data
%%%%%%%%%%%%%%%%%%%%%%%%%%%%%%%%%%%%%%%%%%%%%%%%%%%%%%%%%%%%%%%%%%%%%%%%%%%%%%%%%%%%
\documentclass[12pt]{report} %report allows for chapters
\renewcommand\thesection{\arabic{section}} % ignore the title number for sections
%%%%%%%%%%%%%%%%%%%%%%%%%%%%%%%%%%%%%%%%%%%%%%%%%%%%%%%%%%%%%%%%%%%%%%%%%%%%%%%%%%%%




%%%%%%%%%%%%%%%%%%%%%%%%%%%%%%%%%%%%%%%%%%%%%%%%%%%%%%%%%%%%%%%%%%%%%%%%%%%%%%%%%%%%
% Packages
%%%%%%%%%%%%%%%%%%%%%%%%%%%%%%%%%%%%%%%%%%%%%%%%%%%%%%%%%%%%%%%%%%%%%%%%%%%%%%%%%%%%
\usepackage{color, soul, xcolor} % Colored text and highlighting, respectively

%Tikz
\usepackage{tikz-cd} % For commutative diagrams
\usepackage{tikz-3dplot}
\RequirePackage{pgfplots}
\usetikzlibrary{shadows}
\usetikzlibrary{shapes}
\usetikzlibrary{decorations}
\usetikzlibrary{arrows,decorations.markings} 
\usetikzlibrary{quotes,angles}

\usepackage{mathtools}
\usepackage{answers}
\usepackage{setspace}
\usepackage{graphicx}
\usepackage{enumerate}
\usepackage{multicol}
\usepackage{mathrsfs}
\usepackage[margin=1in]{geometry} 
\usepackage{amsmath,amsthm,amssymb}
\usepackage{marvosym,wasysym} %fucking smileys
%%%%%%%%%%%%%%%%%%%%%%%%%%%%%%%%%%%%%%%%%%%%%%%%%%%%%%%%%%%%%%%%%%%%%%%%%%%%%%%%%%%%




%%%%%%%%%%%%%%%%%%%%%%%%%%%%%%%%%%%%%%%%%%%%%%%%%%%%%%%%%%%%%%%%%%%%%%%%%%%%%%%%%%%%
% Shortcuts
%%%%%%%%%%%%%%%%%%%%%%%%%%%%%%%%%%%%%%%%%%%%%%%%%%%%%%%%%%%%%%%%%%%%%%%%%%%%%%%%%%%%
% Number systems
\newcommand{\N}{\mathbb{N}}
\newcommand{\Z}{\mathbb{Z}}
\newcommand{\C}{\mathbb{C}}
\newcommand{\R}{\mathbb{R}}
\newcommand{\Q}{\mathbb{Q}}

% Operators/functions
\newcommand{\id}{\mathrm{Id}}
\DeclareMathOperator{\sech}{sech}
\DeclareMathOperator{\csch}{csch}
%%%%%%%%%%%%%%%%%%%%%%%%%%%%%%%%%%%%%%%%%%%%%%%%%%%%%%%%%%%%%%%%%%%%%%%%%%%%%%%%%%%%




%%%%%%%%%%%%%%%%%%%%%%%%%%%%%%%%%%%%%%%%%%%%%%%%%%%%%%%%%%%%%%%%%%%%%%%%%%%%%%%%%%%%
% Environments
%%%%%%%%%%%%%%%%%%%%%%%%%%%%%%%%%%%%%%%%%%%%%%%%%%%%%%%%%%%%%%%%%%%%%%%%%%%%%%%%%%%%
% Italic font
\newtheorem{theorem}{Theorem}[section]
\newtheorem{lemma}{Lemma}[section]
\newtheorem{corollary}{Corollary}[section]
\newtheorem{axiom}{Axiom}

% Plain font
\theoremstyle{definition}
\newtheorem{definition}{Definition}[section]
\newtheorem{example}{Example}[section]
\newtheorem{remark}{Remark}[section]
\newtheorem{solution}{Solution}[section]
\newtheorem{problem}{Problem}[section]
\newtheorem{question}{Question}[section]
\newtheorem{answer}{Answer}[section]
\newtheorem{exercise}{Exercise}[section]
%%%%%%%%%%%%%%%%%%%%%%%%%%%%%%%%%%%%%%%%%%%%%%%%%%%%%%%%%%%%%%%%%%%%%%%%%%%%%%%%%%%%

\begin{document}


\begin{center}
   \textsc{\large MATH 255, Homework 2}\\
\end{center}
\vspace{.5cm}

\noindent\textbf{Relevant Sections:} 18.1, 18.3, 17.2, 17.2, 17.3.\\

\noindent\textbf{Problem 1.} Which of the following are linear transformations? For those that are not, which properties of \emph{linearity} (the properties (i), (ii), and (iii) in our notes) fail? Show your work.
\begin{enumerate}[(a)]
    \item $T_a \colon \R \to \R$ given by $T_a(x)=\frac{1}{x}$.
    \item $T_b \colon \R^3 \to \R^2$ given by
    \[
    T_b \left( \begin{bmatrix} x\\ y\\ z \end{bmatrix}\right)
    = \begin{bmatrix} x\\ y \end{bmatrix}.
    \]
    \item $T_c \colon \R \to \R^3$ given by
    \[
    T_c(t)=\begin{bmatrix} t\\ t^2\\ t^3 \end{bmatrix}.
    \]
    \item $T_d \colon \R^2 \to \R^3$ given by
    \[
    T_d\left( \begin{bmatrix} x\\ y \end{bmatrix}\right)
    = \begin{bmatrix} x+y\\ x+y\\ x+y \end{bmatrix}.
    \]
\end{enumerate}

\noindent\textbf{Problem 2.} Write down the matrix for the following linear transformation $T\colon \R^3 \to \R^3$:
\[
T\left( \begin{bmatrix} x\\ y\\ z \end{bmatrix}\right)
= \begin{bmatrix} x+y+z\\ 2x\\ 3y + z \end{bmatrix}.
\]


\noindent\textbf{Problem 3.} Compute the following:
\begin{enumerate}[(a)]
    \item 
    \[
    \mathbf{A}=\begin{bmatrix} 1& 1& 1 \end{bmatrix}
    \begin{bmatrix} 2\\ 1\\ 3 \end{bmatrix}.
    \]
    \item 
    \[
    \mathbf{B}=\begin{bmatrix} 5& 0& 0\\ 2& 2& 2\end{bmatrix}
    \begin{bmatrix} 3\\ 2\\ 1 \end{bmatrix}
    \]
    \item
    \[
    \mathbf{C}=\begin{bmatrix} 1& 2& 3& 4\\ 5& 6& 7& 8\\ 9& 10& 11& 12\end{bmatrix}
    \begin{bmatrix} 3& 2\\ 2& 3\\ 3& 2\\ 2& 3\end{bmatrix}
    \]
    \item Take
    \[
    \mathbf{M}=\begin{bmatrix} 10& 15\\ 20& 10 \end{bmatrix}
    \]
    and
    \[
    \mathbf{N}=\begin{bmatrix} 1 & 2\\ 2& 1\end{bmatrix}.
    \]
    Find $3\mathbf{MN}-3\mathbf{NM}$.
\end{enumerate}

\noindent\textbf{Problem 4.} Compute the following determinants:
\begin{enumerate}[(a)]
    \item
    \[
    \det(\mathbf{A})=\left| \begin{array}{cc}
    -3& 6\\
    -3& 6
    \end{array}\right|
    \]
    \item 
    \[
    \det(\mathbf{B})=\left| \begin{array}{ccc}
    1& 2& 3\\
    4& 5& 6\\
    7& 8& 9
    \end{array}\right|
    \]
    \item    
    \[
    \det(\mathbf{C})=\left| \begin{array}{ccc}
    \lambda& 2& 0\\
    0& \lambda -1& 5\\
    0& 0& \lambda
    \end{array}\right|
    \]
\end{enumerate}

\noindent\textbf{Problem 5.} A linear transformation $T\colon \R^3 \to \R^3$ is given by the matrix
\[
\mathbf{T}= \begin{bmatrix}
1& 2& 0\\
2& 1& 2\\
0& 2& 1
\end{bmatrix}.
\]
\begin{enumerate}[(a)]
    \item Compute how $T$ transforms the standard basis elements for $\R^3$. That is, find
    \[
    T\left(\begin{bmatrix} 1\\ 0\\ 0\end{bmatrix}\right), \quad
    T\left(\begin{bmatrix} 0\\ 1\\ 0\end{bmatrix}\right), \quad 
    T\left(\begin{bmatrix} 0\\ 0\\ 1\end{bmatrix}\right).
    \]
    \item If we apply this linear transformation to the unit cube (that is, all points who have $(x,y,z)$ coordinates with $0\leq x \leq 1$, $0\leq y \leq 1$, and $0\leq z \leq 1$), what will the volume of the transformed cube be? (\emph{Hint: the determinant of this matrix $\mathbf{T}$ provides us this information.})
\end{enumerate}







\end{document}  