\documentclass[leqno]{article}
\usepackage[utf8]{inputenc}
\usepackage[T1]{fontenc}
\usepackage{amsfonts}
%\usepackage{fourier}
%\usepackage{heuristica}
\usepackage{enumerate}
\author{Colin Roberts}
\title{MATH 571, Homework 6}
\usepackage[left=3cm,right=3cm,top=3cm,bottom=3cm]{geometry}
\usepackage{amsmath}
\usepackage[thmmarks, amsmath, thref]{ntheorem}
%\usepackage{kbordermatrix}
\usepackage{mathtools}
\usepackage{color}
\usepackage{hyperref}
\usepackage{tikz-cd}

\theoremstyle{nonumberplain}
\theoremheaderfont{\itshape}
\theorembodyfont{\upshape:}
\theoremseparator{.}
\theoremsymbol{\ensuremath{\square}}
\newtheorem{proof}{Proof}
\theoremsymbol{\ensuremath{\square}}
\newtheorem{lemma}{Lemma}
\theoremsymbol{\ensuremath{\blacksquare}}
\newtheorem{solution}{Solution}
\theoremseparator{. ---}
\theoremsymbol{\mbox{\texttt{;o)}}}
\newtheorem{varsol}{Solution (variant)}

\newcommand{\id}{\mathrm{Id}}
\newcommand{\im}{\mathrm{im}}
\newcommand{\R}{\mathbb{R}}
\newcommand{\N}{\mathbb{N}}
\newcommand{\Z}{\mathbb{Z}}

\begin{document}
\maketitle
\begin{large}
\begin{center}
Solutions
\end{center}
\end{large}

%%%%%%%%%%%%%%%%%%%%%%%%%%%%%%%%%%%%%%%%%%%%%%%%%%%%%%%%%%%%%%%%%%%%%%%%%%%%%%%%%%%%%%%%%%%%%%%%%%%%%%%%%%%%%%%%%%%%%
%%%%%%%%%%%%%%%%%%%%%%%%%PROBLEM%%%%%%%%%%%%%%%%%%%%%%%%%%%%%%%%%%%%%%%%%%%%%%%%%%%%%%%%%%%%%%%%%%%%%%%%%%%%%%%%%%%%%%%%%%%%%%%%%%%%%%%%%%%%%%%%%%%%%%%%%%%%%%%%%%%%%%%%%%%%%%%%%%%%%%%%%%%%%%%%%%%%%%%%%%%%%%%%%%%%%%%%%%%%%%%%%%%%%%%%%%

\noindent\textbf{Problem 1.} 
Hatcher Exercise 17 on page 80: Given a group $G$ and a normal subgroup $N$, show that there exists a normal covering space $\tilde{X}\to X$ with $\pi_1(X)\cong G$, $\pi_1(\tilde{X})\cong N$, and deck transformation group $G(\tilde{X})\cong G/N$.


\begin{proof}
Note that we can construct a space $X$ as a CW-complex so that $\pi_1(X)\cong G$. We construct $X$ with a single 0-cell, a 1-cell for each generator with endpoints identified with the 0-cell, and finally by gluing in a 2-cell along the proper 1-cells for each relation in the group presentation. Then the classification theorem states that there exists a covering space $\tilde{X}$ for each subgroup of $\pi_1(X)$, and hence we specifically have an $\tilde{X}$ so that $\pi_1(\tilde{X})\cong N \subseteq G$. We also know that $\tilde{X}$ is a normal covering space since $\pi_1(\tilde{X})$ is a normal subgroup of $\pi_1(X)$. This then yields the desired deck transformation group $G(\tilde{X}) \cong G/N \cong \pi_1(X)/\pi_1(\tilde{X})$ since $\pi_1(\tilde{X})$ is normal.  
\end{proof}

\vspace*{1cm}


\noindent\textbf{Problem 2.} Hatcher Exercise 4 on page 131: Compute the 0-, 1-, 2-, and 3-dimensional simplicial homology groups of the ``triangular parachute" obtained from $\Delta^2$ by identifying its three vertices to a single point.

\begin{proof}
Denote this triangular parachute space as $X$ and label our single vertex as $v$, the three edges as $a,b$ and $c$, and the single face as $f$.  See the drawing below.
\vspace*{4cm}\\

Now we have that $\Delta_0(X) \cong \Z$, $\Delta_1(X) \cong \Z^2$, $\Delta_2(X) \cong \Z$, $\Delta_3(X)\cong 0$, and $\Delta_4(X) \cong 0$.  We then have
\begin{align*}
\partial_0 (\Delta_0(X)) \cong 0 &\implies \ker \partial_0 \cong \Z ~~~ \im \partial_0 \cong 0.
\end{align*}
Now we also get
\begin{align*}
\partial_1 (\Delta_1(X)) \cong 0,
\end{align*}
since each edge is mapped to the identity since, for example, $a\mapsto v-v = 0$. This then implies
\begin{align*}
\ker \partial_1 \cong \Z^3 ~~~ \im \partial_1 \cong 0.
\end{align*}
Next, we have
\begin{align*}
\partial_2 (\Delta_2(X))=a-b+c, 
\end{align*}
which tells us specifically that
\begin{align*}
\ker \partial_2 \cong 0 ~~~ \im \partial_2 \cong \Z.
\end{align*}
More easily, we finally see that
\begin{align*}
\ker \partial_3 \cong 0 ~&~~ \im \partial_3 \cong 0 ~~\textrm{and}\\
\ker \partial_4 \cong 0 ~&~~ \im \partial_4 \cong 0.
\end{align*}
We then compute the homology groups per usual.  So we have
\begin{align*}
H_0^\Delta (X) &\cong Z\\
H_1^\Delta(X) &\cong \Z^2\\
H_2^\Delta(X) &\cong 0\\
H_3^\Delta(X) &\cong 0.
\end{align*}
This shows we have a single connected components, two 1-dimensional holes, and no 2- (or higher) dimensional holes.
\end{proof}

\vspace*{1cm}


\noindent\textbf{Problem 3.} Hatcher Exercise 5 on page 131: Compute the 0-, 1-, and 2-dimensional simplicial homology groups of the Klein bottle using the $\Delta$-complex structure on page 102.

\begin{proof}
We first write down the chain groups:
\begin{align*}
\Delta_0(X) &\cong \Z\\
\Delta_1(X) &\cong \Z^3\\
\Delta_2(X) &\cong \Z^2\\
\Delta_3(X) &\cong 0.
\end{align*}
Investigating each boundary map, we find for $\partial_0$ that
\begin{align*}
v \mapsto 0.
\end{align*}
For $\partial_1$,
\begin{align*}
a\mapsto v-v &= 0\\
b\mapsto v-v &= 0\\
c\mapsto v-v &= 0.
\end{align*}
For $\partial_2$,
\begin{align*}
U&\mapsto b-c+a\\
L&\mapsto a-b+c.
\end{align*}
Finally, for $\partial_3$, there are no 3-simplicies so $\im \partial_3 \cong \ker \partial_3 \cong 0$. We then compute homology,
\begin{align*}
H_0^\Delta(X) &\cong \Z,\\
H_1^\Delta(X) &\cong Z\oplus \Z/2\Z,
\end{align*}
since for $\im \partial_1$ we have a basis $\{a,b,c\}$ and for $\ker \partial_2$ we have a basis $\{b-c+a,a-b+c\}$ which can be rewritten as $\{b-c+a,2c\}$. When we take $H_1^\Delta(X) \cong \im \partial_1 / \ker \partial_2$ we mod out a copy of $\Z$ for the element $b-c+a$ and we get a single $\Z/2\Z$ term since $2c$ is an identity in $H_1^\Delta(X)$.  The other copy of $\Z$ left was from $\ker \partial_1$. Finally
\begin{align*}
H_2^\Delta(X) \cong 0.
\end{align*}
\end{proof}





\end{document}



