\documentclass[leqno]{article}
\usepackage[utf8]{inputenc}
\usepackage[T1]{fontenc}
\usepackage{amsfonts}
%\usepackage{fourier}
%\usepackage{heuristica}
\usepackage{enumerate}
\author{Colin Roberts}
\title{MATH 571, Homework 3}
\usepackage[left=3cm,right=3cm,top=3cm,bottom=3cm]{geometry}
\usepackage{amsmath}
\usepackage[thmmarks, amsmath, thref]{ntheorem}
%\usepackage{kbordermatrix}
\usepackage{mathtools}
\usepackage{color}
\usepackage{hyperref}

\theoremstyle{nonumberplain}
\theoremheaderfont{\itshape}
\theorembodyfont{\upshape:}
\theoremseparator{.}
\theoremsymbol{\ensuremath{\square}}
\newtheorem{proof}{Proof}
\theoremsymbol{\ensuremath{\square}}
\newtheorem{lemma}{Lemma}
\theoremsymbol{\ensuremath{\blacksquare}}
\newtheorem{solution}{Solution}
\theoremseparator{. ---}
\theoremsymbol{\mbox{\texttt{;o)}}}
\newtheorem{varsol}{Solution (variant)}

\newcommand{\id}{\mathrm{Id}}
\newcommand{\R}{\mathbb{R}}
\newcommand{\N}{\mathbb{N}}
\newcommand{\Z}{\mathbb{Z}}

\begin{document}
\maketitle
\begin{large}
\begin{center}
Solutions
\end{center}
\end{large}

%%%%%%%%%%%%%%%%%%%%%%%%%%%%%%%%%%%%%%%%%%%%%%%%%%%%%%%%%%%%%%%%%%%%%%%%%%%%%%%%%%%%%%%%%%%%%%%%%%%%%%%%%%%%%%%%%%%%%
%%%%%%%%%%%%%%%%%%%%%%%%%PROBLEM%%%%%%%%%%%%%%%%%%%%%%%%%%%%%%%%%%%%%%%%%%%%%%%%%%%%%%%%%%%%%%%%%%%%%%%%%%%%%%%%%%%%%%%%%%%%%%%%%%%%%%%%%%%%%%%%%%%%%%%%%%%%%%%%%%%%%%%%%%%%%%%%%%%%%%%%%%%%%%%%%%%%%%%%%%%%%%%%%%%%%%%%%%%%%%%%%%%%%%%%%%

\noindent\textbf{Problem 1.} Use van Kampen's theorem to prove that the $n$-sphere $S^n$ has trivial fundamental group for $n\ge 2$

\begin{proof}
Choose a base point $x_0$ on $S^n$ other than $\{(0,0,\dots,0,1)\}$ or $\{(0,0,\dots,0-1\}$. Then let $\mathcal{N} = S^n \setminus \{(0,0,\dots,0,-1)\}$ and $\mathcal{S}=S^n \setminus \{(0,0,\dots,0,1)\}$ be two open subsets of $S^n$ such that $\mathcal{N}\cup \mathcal{S}=S^n$ and note $\mathcal{N}\cup \mathcal{S} = S^n \setminus \{(0,0,\dots,0,\pm 1 )\} \simeq S^{n-1}$ is path connected. For the base case, $n=2$, we have that $\pi_1 (\mathcal{N}) \cong \pi_1 (\mathcal{S}) \cong \{e\}$ are trivial groups since $\mathcal{N}\simeq \mathcal{S} \simeq B^2$.  Then for any $w\in \pi_1(\mathcal{N}\cap \mathcal{S})$, note that $i_{\mathcal{NS}}(w)i_{\mathcal{SN}}(w)^{-1} \simeq C_{x_0}$ is the constant path since any loops are contractible in both $\mathcal{N}$ and $\mathcal{S}$. So $\pi_1 ( \mathcal{N}\cap.\mathcal{S} ) = \{e\}$.  Finally, by Van Kampen's theorem, we have $\pi_1 (S^2) \cong \pi_1(\mathcal{N})\ast \pi_1(\mathcal{S}/\{e\}\cong \{e\}$ and so the fundamental group of $S^2$ is trivial.

Now, suppose this is true for $n-1$, and consider the case for $S^n$.  Now $\mathcal{N}\simeq \mathcal{S} \simeq B^{n}$ and so $\pi_1(\mathcal{N})\cong \pi_1(\mathcal{S}) \cong \{e\}$.  Also, $\mathcal{N}\cap \mathcal{S}\simeq S^{n-1}$, and so $\pi_1(S^{n-1})$ is trivial by our induction hypothesis and hence $\pi_1(\mathcal{N}\cap \mathcal{S})\cong \{e\}$.  Finally, Van Kampen's theorem then gives us $\pi_1(S^n)\cong (\{e\}\ast \{e\})/\{e\} \cong \{e\}$, showing that $\pi_1(S^n)$ is trivial.
\end{proof}

\vspace*{1cm}


\noindent\textbf{Problem 2.} Let $M$ be an $n$-dimensional manifold, with $n\ge 3$. Let $p\in M$ be any point in the manifold $M$. There is a nice relationship between the fundamental groups $\pi_1(M)$ and $\pi_1(M\setminus\{p\})$ --- how are they related? Prove your answer is correct.

\begin{proof}
We claim that $\pi_1(M) \cong \pi_1(M\setminus \{p\})$. To see this, let $A_1 = M\setminus \{p\}$ and choose $\epsilon>0$ so that $A_2 = B_{\epsilon}^n(p)\subseteq M$.  Then we have
\begin{align*}
\pi_1(M)\cong(\pi_1(A_1)\ast \pi_1(A_2))/N,
\end{align*}
where $N$ is the normal subgroup generated by elements of the form $i_{21}(w)i_{12}(w)^{-1}$ with $w\in \pi_1(A_1\cap A_2)$. Note that $\pi_1(A_2)\cong \{e\}$ is trivial, and that $N$ is also trivial since any loop in $A_1\cap A_2$ is homotopy equivalent to a trivial loop.  This gives
\begin{align*}
\pi_1(M)&\cong(\pi_1(A_1) \ast \pi_1(A_2))/N\\
&\cong \pi_1(A_1) \cong \pi_1(M\setminus \{p\}).
\end{align*}
\end{proof}

\vspace*{1cm}

\pagebreak


\noindent\textbf{Problem 3.} 
\begin{enumerate}[(a)]
\item Problem 8 on page 53 of Hatcher: ``Compute the fundamental group of the space obtained from two tori $S^1\times S^1$ by identifying a circle $S^1\times\{x_0\}$ in one torus with the corresponding circle $S^1\times\{x_0\}$ in the other torus." Let's call this identification space $X$. I want you to use van Kampen's theorem to compute $\pi_1(X)$.
\item Write this identification space $X$ as a product $X=Y\times Z$ (where neither $Y$ nor $Z$ are just a single point), and use this to give an alternate computation of $\pi_1(X)$.
\end{enumerate}


\begin{proof}~
\begin{enumerate}[(a)]
\item Let $x_0$ be our base point and let $U_1=S^1\times S^1$ be the first two tori and $U_2=S^1 \times S^1$ be the second. Then let $O=\{N_x \colon x\in U_1 \cap U_2\}$ be a collection of neighborhoods of points in the intersection of the two tori and let $A_1=U_1\cup O$ and $A_2=U_2 \cup O$ so that $A_1$ and $A_2$ are open in $X$. Then  $\cap A_2 = S^1 \times \{x_0\}$ and so $A_1\cap A_2$ is path connected.  Then $\pi_1 (A_1)\cong \langle a,b_1 \vert aba^{-1}b^{-1} \rangle$ and $\pi_1 (A_2) \cong \langle b_2,c \rangle$ and $\pi_1(A_1\cap A_2)\cong \langle b \rangle \cong \Z$.  Then for any $w\in \pi_1(A_1 \cap A_2)$ we have that $i_{21}(w)=b_1$ and $i_{12}(w)^{-1}=b_2^{-1}$ so that $b_1b_2^{-1}=1$.  So now, letting $N$ be the group generated by all elements of the form $I_{21}(w)I_{12}^{-1}$ with $w\in \pi_1(A_1\cap A_2)$ we have that 
\begin{align*}
\pi_1(X) &\cong (\pi_1(A_1)\ast \pi_1(A_2))/N \\
&\cong \langle a,b_1,b_2,c \vert ab_1a^{-1}b_1^{-1}, b_2cb_2^{-1}c^{-1}, b_1b_2^{-1} \rangle \\
& \cong \langle a,b,c \vert aba^{-1}b^{-1}, bcb^{-1}c^{-1} \rangle ~~~~ \textrm{letting $b_1=b_2=b$}\\
&\cong (\Z \ast \Z) \oplus \Z.
\end{align*}
To see this visually, a loop (denoted as $b$ above) in the intersection $A_1 \cap A_2$ is a loop that commutes with all other loops, but the other two loops (denoted as $a$ and $c$ above) come from a wedge of two circles and thus will not commute with each other.

\item We let $Y=S^1 \wedge S^1$ and $Z=S^1$ and we have that $X=Y\times Z$.  We then have that $\pi_1(X)=\pi_1(Y)\times \pi_1(Z)$ and so $\pi_1(Y)=\Z\ast \Z$ and $\pi_1(Z)=\Z$, hence $\pi_1(X)=(\Z\ast \Z) \oplus \Z$.
\end{enumerate}
\end{proof}








\end{document}



