\documentclass[leqno]{article}
\usepackage[utf8]{inputenc}
\usepackage[T1]{fontenc}
\usepackage{amsfonts}
%\usepackage{fourier}
%\usepackage{heuristica}
\usepackage{enumerate}
\author{Colin Roberts}
\title{MATH 571, Homework 4}
\usepackage[left=3cm,right=3cm,top=3cm,bottom=3cm]{geometry}
\usepackage{amsmath}
\usepackage[thmmarks, amsmath, thref]{ntheorem}
%\usepackage{kbordermatrix}
\usepackage{mathtools}
\usepackage{color}
\usepackage{hyperref}
\usepackage{tikz-cd}

\theoremstyle{nonumberplain}
\theoremheaderfont{\itshape}
\theorembodyfont{\upshape:}
\theoremseparator{.}
\theoremsymbol{\ensuremath{\square}}
\newtheorem{proof}{Proof}
\theoremsymbol{\ensuremath{\square}}
\newtheorem{lemma}{Lemma}
\theoremsymbol{\ensuremath{\blacksquare}}
\newtheorem{solution}{Solution}
\theoremseparator{. ---}
\theoremsymbol{\mbox{\texttt{;o)}}}
\newtheorem{varsol}{Solution (variant)}

\newcommand{\id}{\mathrm{Id}}
\newcommand{\R}{\mathbb{R}}
\newcommand{\N}{\mathbb{N}}
\newcommand{\Z}{\mathbb{Z}}

\begin{document}
\maketitle
\begin{large}
\begin{center}
Solutions
\end{center}
\end{large}

%%%%%%%%%%%%%%%%%%%%%%%%%%%%%%%%%%%%%%%%%%%%%%%%%%%%%%%%%%%%%%%%%%%%%%%%%%%%%%%%%%%%%%%%%%%%%%%%%%%%%%%%%%%%%%%%%%%%%
%%%%%%%%%%%%%%%%%%%%%%%%%PROBLEM%%%%%%%%%%%%%%%%%%%%%%%%%%%%%%%%%%%%%%%%%%%%%%%%%%%%%%%%%%%%%%%%%%%%%%%%%%%%%%%%%%%%%%%%%%%%%%%%%%%%%%%%%%%%%%%%%%%%%%%%%%%%%%%%%%%%%%%%%%%%%%%%%%%%%%%%%%%%%%%%%%%%%%%%%%%%%%%%%%%%%%%%%%%%%%%%%%%%%%%%%%

\noindent\textbf{Problem 1.} 
The classification of closed surfaces says that a connected closed surface (where ``closed" here means compact with no boundary) is homeomorphic to exactly one of the following:
\begin{itemize}
\item the sphere $M_0:=S^2$, 
\item the connected sum of $g$ tori for $g\ge1$, denoted $M_g$, and also called the torus of genus $g$, or
\item the connected sum of $g\ge 1$ projective planes for $g\ge 1$, denoted $N_g$.
\end{itemize}
The connected sum of two surfaces is obtained by deleting a disk from each, and then gluing the two surfaces together along their two boundary circles. It turns out that the $M_g$ surfaces are orientable, whereas the $N_g$ surfaces are not.
\begin{enumerate}
\item Show that $M_g$ has a CW complex structure with one 0-cell, $2g$ 1-cells, and one 2-cell, and deduce from this CW structure that the fundamental group of $M_g$ is \[\pi_1(M_g)\cong\langle a_1,b_1,\ldots,a_g,b_g~|~a_1b_1a_1^{-1}b_1^{-1}\cdots a_gb_ga_g^{-1}b_g^{-1}\rangle\]. 
\item Show that $N_g$ has a CW complex structure with one 0-cell, $g$ 1-cells, and one 2-cell, and deduce from this CW structure that the fundamental group of $N_g$ is \[\pi_1(N_g)\cong\langle a_1,\ldots,a_g~|~a_1^2\cdots a_g^2\rangle\].
\item ``Abelianization" is a way to turn any group into an ableian group. Indeed, there is a functor $\mathrm{Ab}\colon \mathrm{Grp}\to\mathrm{AbGrp}$ (called abelianization)  from the category of groups to the category of abelian groups. Compute the abelianizations of $\pi_1(M_g)$ and $\pi_1(N_g)$.
\item Conclude that none of the connected closed surfaces $M_g$ for $g\ge0$ or $N_g$ for $g\ge1$ are homeomorphic (or even homotopy equivalent) to each other. 
\end{enumerate}

\emph{Remark: Hatcher talks about this on pages 51-52; the point here is to learn all of the details.}

\begin{proof}~
\begin{enumerate}[(a)]
\item For $g=1$ we have the fundamental group of the torus which is given by $\langle a_1,b_1 ~\vert~ a_1b_1a_1^{-1}b_1^{-1}\rangle$, which confirms our base case.  Then, we assume this is true up to $g-1$, and we show it is true for $g$.  We then construct $M_g$ from $M_{g-1}$ by taking $M_{g-1}\# M_1$.  See the diagram below.
\vspace*{6cm}\\
Hence, $\pi_1(M_g)\cong \langle a_1,b_1,\dots,a_g,b_g ~\vert~ a_1b_1a_1^{-1}b_1^{-1}\cdots a_gb_ga_g^{-1}b_g^{-1} \rangle$.
\item For $g=1$ we have the fundamental group of the projective plane which is given by $\langle a_1 ~\vert ~ a_1^2\rangle$.  We then assume this is true up to $g-1$ and show it is true for $g$ as well.  We construct $N_g$ by taking $N_{g-1}\# N_1$. See the diagram below.
\vspace*{6cm}\\
Hence, $\pi_1(N_g)\cong \langle a_1,\dots,a_g ~\vert~ a_1^2a_2^2\cdots a_g^2 \rangle$.
\item The abelianization of a group $G$ is the group $G/[G,G]$, where $[G,G]$ is the commutator subgroup.  For $\pi_1(M_g)$, we find that $[\pi_1(M_g),\pi_1(M_g)]$ is generated by $aba^{-1}b^{-1}$ for $a,b\in \pi_1(M_g)$.  Then we have
\begin{align*}
\pi_1(M_g)/[\pi_1(M_g),\pi_1(M_g)]&\cong \langle a_1,b_1,\dots,a_g,b_g ~\vert~ a_1b_1a_1^{-1}b_1^{-1}\cdots a_gb_ga_g^{-1}b_g^{-1}\rangle/[\pi_1(M_g),\pi_1(M_g)] \\
&\cong \langle a_1,b_1,\dots,a_g,b_g ~\vert ~ a_1b_1a_1^{-1}b_1^{-1}\cdots a_gb_ga_g^{-1}b_g^{-1},a_1b_1a_1^{-1}b_1^{-1}, a_1b_2a_1^{-1}b_2^{-1},\\
&\dots,a_1b_ga_1^{-1}b_g^{-1},a_2b_1a_2^{-1}b_1^{-1},\dots,a_2b_ga_2^{-1}b_g^{-1},\dots,a_gb_ga_g^{-1}b_g^{-1}\rangle\\
&\cong \Z^{2g}.
\end{align*} 

Now we do the same process for $\pi_1(N_g)$ and we find that
\begin{align*}
\pi_1(N_g)/[\pi_1(N_g),\pi_1(N_g)]&\cong \langle a_1,\dots,a_g ~\vert~ a_1^2\cdots a_g^2 \rangle/[\pi_1(N_g),\pi_1(N_g)]\\
&\cong \langle a_1,dots,a_g ~\vert~ a_1^2\cdots a_g^2, a_1a_2a_1^{-1}a_2^{-1},\dots,a_1a_ga_1^{-1}a_g^{-1}\rangle\\
&\cong \Z_2 \oplus \Z^{g-1}.
\end{align*}

\item To see that no $M_g$ is homotopy equivalent to any other $M_{g'}$ with $g \neq g'$, note that the abelianization of $\pi_1(M_g)$ is the direct sum of $2g$ copies of $\Z$ and hence $2g'=2g$.  But this contradicts $g\neq g'$ and thus no $M_g$ is homotopy equivalent to any other $M_{g'}$.  

To see that no $M_g$ is homotopy equivalent to any $N_{g'}$, just note that the abelianization of $\pi_1(N_{g'}$ is $\Z_2$ direct sum with $g'-1$ copies of $\Z$.  This is not isomorphic to any abelianization of $\pi_1(M_g)$ and thus no $M_g$ is homotopy equivalent to any $N_{g'}$ even if $g=g'$.
\end{enumerate}
\end{proof}

\vspace*{1cm}


\noindent\textbf{Problem 2.} Hatcher Exercise 9 on page 79: Show that if a path-connected, locally path-connected space $X$ has $\pi_1(X)$ finite, then every map $X\to S^1$ is nullhomotopic.

\begin{proof}
Suppose $X$ is path-connected and locally path-connected and then let $f \colon X \to S^1$ be an arbitrary map. These suppositions allow us to consider Proposition 1.33 and use this to prove our propoistion in this problem.  Since $\pi_1(X)$ is finite, and $f_*\colon \pi_1(X)\to \pi_1(S^1)$ is an induced homomorphism, we know that $\ker f_* = \pi_1(X)$ since there can only be a trivial homomorphism between a group of finite order and a group with infinite order. Now, letting $\R=\tilde{X}$ be the universal cover of $S^1$, we get that $f_*(\pi_1(X))\subset p_*(\pi_1(\R))$.  This implies that $\exists \tilde{f}$ that lifts $f$ to $\tilde{f}$ given by the following commutative diagram:
\begin{center}
\[
\begin{tikzcd}
 & \mathbb{R} \arrow[d, "p"] \\
X \arrow[ru, "\tilde{f}", dashed] \arrow[r, "f"'] & S^1
\end{tikzcd}
\]
\end{center}
Then we know $f=p\circ \tilde{f}$.  Note that any path $\gamma$ in $X$ lifts to a path in $\R$, and since $\R$ is contractible, we have that $\tilde{f}(\gamma)$ is contractible.  Then $p(\tilde{f}(\gamma))$ is then contractible in $S^1$, which means that our arbitrary $f$ is a nullhomotopic map, and hence we are done.
\end{proof}

\vspace*{1cm}



\noindent\textbf{Problem 3.} Hatcher Exercise 10 on page 79: Find all the connected 2-sheeted and 3-sheeted covering spaces of $S^1\vee S^1$ up to isomorphism of covering spaces without basepoints (defined on page 67). You do not need to prove your answer is correct.


\begin{proof}
Here are the drawings of the different possibilities with the respective group presentation.
\end{proof}








\end{document}



