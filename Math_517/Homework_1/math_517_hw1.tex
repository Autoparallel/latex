\documentclass[leqno]{article}
\usepackage[utf8]{inputenc}
\usepackage[T1]{fontenc}
\usepackage{amsfonts}
\usepackage{fourier}
\usepackage{heuristica}
\usepackage{enumerate}
\author{Colin Roberts}
\title{MATH 517, Homework 1}
\usepackage[left=3cm,right=3cm,top=3cm,bottom=3cm]{geometry}
\usepackage{amsmath}
\usepackage[thmmarks, amsmath, thref]{ntheorem}
%\usepackage{kbordermatrix}
\usepackage{mathtools}

\theoremstyle{nonumberplain}
\theoremheaderfont{\itshape}
\theorembodyfont{\upshape:}
\theoremseparator{.}
\theoremsymbol{\ensuremath{\square}}
\newtheorem{proof}{Proof}
\theoremsymbol{\ensuremath{\square}}
\newtheorem{lemma}{Lemma}
\theoremsymbol{\ensuremath{\blacksquare}}
\newtheorem{solution}{Solution}
\theoremseparator{. ---}
\theoremsymbol{\mbox{\texttt{;o)}}}
\newtheorem{varsol}{Solution (variant)}

\newcommand{\tr}{\mathrm{tr}}

\begin{document}
\maketitle
\begin{large}
\begin{center}
Solutions
\end{center}
\end{large}
\pagebreak

%%%%%%%%%%%%%%%%%%%%%%%%%%%%%%%%%%%%%%%%%%%%%%%%%%%%%%%%%%%%%%%%%%%%%%%%%%%%%%%%%%%%%%%%%%%%%%%%%%%%%%%%%%%%%%%%%%%%%
%%%%%%%%%%%%%%%%%%%%%%%%%PROBLEM 1%%%%%%%%%%%%%%%%%%%%%%%%%%%%%%%%%%%%%%%%%%%%%%%%%%%%%%%%%%%%%%%%%%%%%%%%%%%%%%%%%%%%%%%%%%%%%%%%%%%%%%%%%%%%%%%%%%%%%%%%%%%%%%%%%%%%%%%%%%%%%%%%%%%%%%%%%%%%%%%%%%%%%%%%%%%%%%%%%%%%%%%%%%%%%%%%%%%%%%%%

\noindent\textbf{Problem 1.} Let $E$ be a nonempty subset of an ordered set. Assume $\alpha$ is a lower bound for $E$ and $\beta$ is an upper bound. Show that $\alpha\leq \beta$.

\noindent\rule[0.5ex]{\linewidth}{1pt}

\begin{proof}
Since $\alpha$ is a lower bound, necessarily $\alpha \leq x$ $\forall x \in E$ and similarly $\beta\geq x$ $\forall x\in E$ since $E$ is non-empty and ordered. Thus we have 
\begin{align*}
\alpha &\leq x \leq \beta\\
\implies \alpha &\leq \beta
\end{align*}
\end{proof}

\pagebreak

%%%%%%%%%%%%%%%%%%%%%%%%%%%%%%%%%%%%%%%%%%%%%%%%%%%%%%%%%%%%%%%%%%%%%%%%%%%%%%%%%%%%%%%%%%%%%%%%%%%%%%%%%%%%%%%%%%%%%
%%%%%%%%%%%%%%%%%%%%%%%%%PROBLEM 2%%%%%%%%%%%%%%%%%%%%%%%%%%%%%%%%%%%%%%%%%%%%%%%%%%%%%%%%%%%%%%%%%%%%%%%%%%%%%%%%%%%%%%%%%%%%%%%%%%%%%%%%%%%%%%%%%%%%%%%%%%%%%%%%%%%%%%%%%%%%%%%%%%%%%%%%%%%%%%%%%%%%%%%%%%%%%%%%%%%%%%%%%%%%%%%%%%%%%%%%


\noindent\textbf{Problem 2.} If $x\in \mathbb{C}$, show that there exists $r\geq 0$ and $w\in \mathbb{C}$ with $\|w\|=1$ so that $z=rw$. Are $r$ and $w$ uniquely determined by $z$.

\noindent\rule[0.5ex]{\linewidth}{1pt}

\begin{proof}
Let $r=|z|$ and then we have that $w=\frac{z}{r}=\frac{z}{|z|}$ so that $|w|=1$. Then we have $wr=\frac{z}{r}r=z$.  But note that $z$ does not uniquely define $r$ and $w$ since $|-z|=|z|$.
\end{proof}


\pagebreak


%%%%%%%%%%%%%%%%%%%%%%%%%%%%%%%%%%%%%%%%%%%%%%%%%%%%%%%%%%%%%%%%%%%%%%%%%%%%%%%%%%%%%%%%%%%%%%%%%%%%%%%%%%%%%%%%%%%%%
%%%%%%%%%%%%%%%%%%%%%%%%%PROBLEM 3%%%%%%%%%%%%%%%%%%%%%%%%%%%%%%%%%%%%%%%%%%%%%%%%%%%%%%%%%%%%%%%%%%%%%%%%%%%%%%%%%%%%%%%%%%%%%%%%%%%%%%%%%%%%%%%%%%%%%%%%%%%%%%%%%%%%%%%%%%%%%%%%%%%%%%%%%%%%%%%%%%%%%%%%%%%%%%%%%%%%%%%%%%%%%%%%%%%%%%%%


\noindent\textbf{Problem 3.} Let $x,y\in \mathbb{R}^n$. Show that
\[
\|x+y\|^2+\|x-y\|^2=2\|x\|^2+2\|y\|^2.
\]
What does this mean geometrically as a statement about parallelograms?

\noindent\rule[0.5ex]{\linewidth}{1pt}

\begin{solution}
\begin{align*}
|x+y|^2+|x-y|^2&=(x+y)\cdot(x+y)+(x-y)\cdot (x-y)\\
&=x\cdot x + 2x\cdot y + y\cdot y + x\cdot x -2x\cdot y + y\cdot y\\
&=2x\cdot x +2y\cdot y\\
&=2|x|^2+2|y|^2
\end{align*}
This statement relates the length of the diagonals of the parallelograms to the side lengths.
\end{solution}

\pagebreak



%%%%%%%%%%%%%%%%%%%%%%%%%%%%%%%%%%%%%%%%%%%%%%%%%%%%%%%%%%%%%%%%%%%%%%%%%%%%%%%%%%%%%%%%%%%%%%%%%%%%%%%%%%%%%%%%%%%%%
%%%%%%%%%%%%%%%%%%%%%%%%%PROBLEM 4%%%%%%%%%%%%%%%%%%%%%%%%%%%%%%%%%%%%%%%%%%%%%%%%%%%%%%%%%%%%%%%%%%%%%%%%%%%%%%%%%%%%%%%%%%%%%%%%%%%%%%%%%%%%%%%%%%%%%%%%%%%%%%%%%%%%%%%%%%%%%%%%%%%%%%%%%%%%%%%%%%%%%%%%%%%%%%%%%%%%%%%%%%%%%%%%%%%%%%%%


\noindent\textbf{Problem 4.} Let $X$ be an infinite set with the trivial metric.
\begin{enumerate}[(a)]
\item Prove that $d$ is a metric $X$.
\item What are the open sets of $X$?
\item What are the closed sets?
\item What are the compact sets?
\end{enumerate}

\noindent\rule[0.5ex]{\linewidth}{1pt}

\begin{proof}[a] 
First note that $\d(p,q)=0$ if and only if $p=q$ for $p,q\in X$. With $p\neq q \in X$ we have $\d(p,q)=1>0$. Second, if $p=q$ then $\d(p,q)=0=\d(q,p)$. With $p\neq q$, $\d(p,q)=1$ and by the trivial metric $\d(q,p)=1$ since $p\neq q$.  Finally let $p=q$ then for $p,q \in X$ we have $\d(p,q)=0\leq \d(p,r)+\d(r,q)\leq 2$ with $0\leq \d(p,r)+d(r,q)$ if $r=q=p$ else $\d(p,r)+\d(r,q)=2$ if $r\neq q = p$.  If $p\neq q$ and $p,q,r\in X$ then $\d(p,q)=1\leq \d(p,r)+\d(r,q)\leq 2$ with $1=\d(p,r)+\d(r,q)$ if $r=q$ or $r=p$ and $\d(p,r)+\d(r,q)=2$ if $r\neq q$ and $r\neq p$.
\end{proof}


\begin{solution}[b]
\item The open sets of $X$ are any subset.  We can separate any singleton from the others with an open ball of radius $r<1$ due to the trivial metric and we can make any subset from a union of the singletons. 
\end{solution}
\begin{solution}[c]
The closed sets are all sets by the properties of compliments of open sets.
\end{solution}
\begin{solution}[d]
The compact sets are any finite set.  Since any finite set will have a finite open subcover and an infinite set won't if we choose the open sets that form an open cover to be singletons.
\end{solution}

\pagebreak


%%%%%%%%%%%%%%%%%%%%%%%%%%%%%%%%%%%%%%%%%%%%%%%%%%%%%%%%%%%%%%%%%%%%%%%%%%%%%%%%%%%%%%%%%%%%%%%%%%%%%%%%%%%%%%%%%%%%%
%%%%%%%%%%%%%%%%%%%%%%%%%PROBLEM 5%%%%%%%%%%%%%%%%%%%%%%%%%%%%%%%%%%%%%%%%%%%%%%%%%%%%%%%%%%%%%%%%%%%%%%%%%%%%%%%%%%%%%%%%%%%%%%%%%%%%%%%%%%%%%%%%%%%%%%%%%%%%%%%%%%%%%%%%%%%%%%%%%%%%%%%%%%%%%%%%%%%%%%%%%%%%%%%%%%%%%%%%%%%%%%%%%%%%%%%%


\noindent\textbf{Problem 5.} Consider $\mathbb{Q}$ as a metric space with $\d(x,y)=|x-y|$. Let
\[
E=\{x\in \mathbb{Q} \vert 2 < x^2 <3\}
\]
Show that $E$ is both open and closed in $\mathbb{Q}$.

\noindent\rule[0.5ex]{\linewidth}{1pt}

\begin{proof}[E is open]
Let $x\in E$ so that $2.5\leq x^2 <3$ and fix $r>0$.  Then let $\frac{\delta}{2}=x^2+3$ with $\delta\in \mathbb{Q}$ and $\delta<r$. Then $|x^2-3|=|3-\frac{\delta}{2}-3|=\frac{\delta}{2}<\frac{r}{2}<r$. Then note that $N_{3-\frac{\delta}{2}}(x)\subseteq E$ and is open in $E$.  Again let $x\in E$ so that $2<x^2\leq 2.5\in E$ and fix $r>0$.  Then let $\frac{\delta}{2}=x^2+2$ with $\delta\in \mathbb{Q}$ and $\delta<r$. Then $|x^2-2|=|2-\frac{\delta}{2}-2|=\frac{\delta}{2}<\frac{r}{2}<r$. Then note that $N_{2-\frac{\delta}{2}}(X)\subseteq E$ and is open in $E$. So every point in $E$ has an open neighborhood about that point contained in $E$.
\end{proof}

\begin{proof}[E is closed]
To show $E$ is closed, suppose there exists a limit point $x$ of $E$ such that $x^2<2$ or $x^2>3$.  Then $\forall r>0$ we have that $N_r(x^2)\cap E\neq\emptyset$.  Suppose $x^2>3$ then let $|x^2-3|=\delta$, then let $r<\frac{\delta}{2}$ and note that $N_r(x^2)\cap E = \emptyset$.  Finally, suppose that $x^2<2$ then let $|x^2-2|=\delta$, then let $r<\frac{\delta}{2}$ and note that $N_r(x^2)\cap E = \emptyset$.
\end{proof}

\pagebreak



\end{document}

