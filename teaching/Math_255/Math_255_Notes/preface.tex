\section*{Preface}

This text was created to be a companion to the Math 255 - \emph{Calculus for Biological Students II} course at Colorado State University.  The reader is expected to have completed the Math 155 course with a solid grasp of the material in order for one to best absorb new topics in this course.

The aim of this text is to present relevant topics in \emph{multivariate calculus} and \emph{differential equations}.  To begin, one must learn some \emph{linear algebra}. Linear algebra provides a foundation that is required for studying functions of more than one variable.  It is also an extremely beautiful and applicable field of mathematics itself.  Vectors will serve as a generalization of the system of numbers one is already comfortable with.  With vectors, one can study systems that are naturally living in space.  We are already familiar with how functions can be applied to the real numbers, but they can also be applied to vectors living in space.  The simplest types of vector-functions are those that are \emph{linear}.  Given the simplicity, we often wish to understand more complicated functions by investigating their linear approximations.

Calculus studies the rate of change of functions, but it can be rephrased slightly.  Rather, we take the approach that calculus investigates the best linear approximations to functions.  With one variable, a function can be approximated by a tangent line with a slope found by computing the derivative. This is the best linear approximation to a function of one variable.  When functions input and output more than one variable, the derivative is thus a linear function. This slight rewording of the derivative definition allows for the analysis of functions defined in space and time much more naturally.  Of course, we also care about integration. Integration of functions with more than one variable allows for more types of integration.  That is, we can consider integration as a tool to find lengths of curves, areas, volumes, and other physically meaningful quantities.  

Many systems are easier to understand by studying how they change.  For example, if we push an object it accelerates; acceleration is the second derivative of the position function of an object.  In this case, the system is determined by rates of change and so the equation one writes down for this system is a differential equation.  To model systems that change over space and time, one must be able to write down and solve these differential equations.  Physical intuition can help guide one to a solution, but there are more surefire methods.  The main technique proposed is to simplify the problem just enough so that we can solve it more easily but not lose too much information.  Techniques from linear algebra continue to reemerge and allow us to solve a large class of (linear/first order) problems.  Other (nonlinear/higher order) problems can be simplified down to the case we work to solve.  Thus, we create quite a toolbox to investigate differential systems.

