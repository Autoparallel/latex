%%%%%%%%%%%%%%%%%%%%%%%%%%%%%%%%%%%%%%%%%%%%%%%%%%%%%%%%%%%%%%%%%%%%%%%%%%%%%%%%%%%%
% Document data
%%%%%%%%%%%%%%%%%%%%%%%%%%%%%%%%%%%%%%%%%%%%%%%%%%%%%%%%%%%%%%%%%%%%%%%%%%%%%%%%%%%%
\documentclass[12pt]{report} %report allows for chapters
\renewcommand\thesection{\arabic{section}} % ignore the title number for sections
%%%%%%%%%%%%%%%%%%%%%%%%%%%%%%%%%%%%%%%%%%%%%%%%%%%%%%%%%%%%%%%%%%%%%%%%%%%%%%%%%%%%




%%%%%%%%%%%%%%%%%%%%%%%%%%%%%%%%%%%%%%%%%%%%%%%%%%%%%%%%%%%%%%%%%%%%%%%%%%%%%%%%%%%%
% Packages
%%%%%%%%%%%%%%%%%%%%%%%%%%%%%%%%%%%%%%%%%%%%%%%%%%%%%%%%%%%%%%%%%%%%%%%%%%%%%%%%%%%%
\usepackage{color, soul, xcolor} % Colored text and highlighting, respectively

%Tikz
\usepackage{tikz-cd} % For commutative diagrams
\usepackage{tikz-3dplot}
\RequirePackage{pgfplots}
\usetikzlibrary{shadows}
\usetikzlibrary{shapes}
\usetikzlibrary{decorations}
\usetikzlibrary{arrows,decorations.markings} 
\usetikzlibrary{quotes,angles}

\usepackage{mathtools}
\usepackage{answers}
\usepackage{setspace}
\usepackage{graphicx}
\usepackage{enumerate}
\usepackage{multicol}
\usepackage{mathrsfs}
\usepackage[margin=1in]{geometry} 
\usepackage{amsmath,amsthm,amssymb}
\usepackage{marvosym,wasysym} %fucking smileys
%%%%%%%%%%%%%%%%%%%%%%%%%%%%%%%%%%%%%%%%%%%%%%%%%%%%%%%%%%%%%%%%%%%%%%%%%%%%%%%%%%%%




%%%%%%%%%%%%%%%%%%%%%%%%%%%%%%%%%%%%%%%%%%%%%%%%%%%%%%%%%%%%%%%%%%%%%%%%%%%%%%%%%%%%
% Shortcuts
%%%%%%%%%%%%%%%%%%%%%%%%%%%%%%%%%%%%%%%%%%%%%%%%%%%%%%%%%%%%%%%%%%%%%%%%%%%%%%%%%%%%
% Number systems
\newcommand{\N}{\mathbb{N}}
\newcommand{\Z}{\mathbb{Z}}
\newcommand{\C}{\mathbb{C}}
\newcommand{\R}{\mathbb{R}}
\newcommand{\Q}{\mathbb{Q}}

% Operators/functions
\newcommand{\id}{\mathrm{Id}}
\DeclareMathOperator{\sech}{sech}
\DeclareMathOperator{\csch}{csch}
%%%%%%%%%%%%%%%%%%%%%%%%%%%%%%%%%%%%%%%%%%%%%%%%%%%%%%%%%%%%%%%%%%%%%%%%%%%%%%%%%%%%




%%%%%%%%%%%%%%%%%%%%%%%%%%%%%%%%%%%%%%%%%%%%%%%%%%%%%%%%%%%%%%%%%%%%%%%%%%%%%%%%%%%%
% Environments
%%%%%%%%%%%%%%%%%%%%%%%%%%%%%%%%%%%%%%%%%%%%%%%%%%%%%%%%%%%%%%%%%%%%%%%%%%%%%%%%%%%%
% Italic font
\newtheorem{theorem}{Theorem}[section]
\newtheorem{lemma}{Lemma}[section]
\newtheorem{corollary}{Corollary}[section]
\newtheorem{axiom}{Axiom}

% Plain font
\theoremstyle{definition}
\newtheorem{definition}{Definition}[section]
\newtheorem{example}{Example}[section]
\newtheorem{remark}{Remark}[section]
\newtheorem{solution}{Solution}[section]
\newtheorem{problem}{Problem}[section]
\newtheorem{question}{Question}[section]
\newtheorem{answer}{Answer}[section]
\newtheorem{exercise}{Exercise}[section]
%%%%%%%%%%%%%%%%%%%%%%%%%%%%%%%%%%%%%%%%%%%%%%%%%%%%%%%%%%%%%%%%%%%%%%%%%%%%%%%%%%%%

\begin{document}


\begin{center}
   \textsc{\large MATH 255, Homework 6}\\
\end{center}
\vspace{.5cm}

\noindent\textbf{Relevant Sections:} 16.7, 10.3, 9.3, 9.5, 16.8 \\

\noindent\textbf{Problem 1.} 
\begin{enumerate}[(a)]
    \item Write the equation for a curve 
    \[
    \gamma \colon \R \to \R^3
    \] 
    satisfying:
    \begin{itemize}
    \item Starts with $\gamma(0)=(0,0,0)$.
    \item Ignoring the $z$-component, makes a spiral. (Imagine looking from above the $xy$-plane for this).
    \item Moves upward at a constant rate in the $z$-direction.
\end{itemize}
In other words, write the equation for a curve that makes a helix with increasing radius and constant ``pitch." (Note that there are many different ways you can write such a curve!)

Plot this curve that you made to verify it is correct.
    \item Find the tangent vector $\gamma'(t)$ to this curve.
    \item Find the acceleration vector $\gamma''(t)$ to this curve.
\end{enumerate}


\vspace*{.5cm}

\noindent\textbf{Problem 2.} 
\begin{enumerate}[(a)]
    \item Write the equation for a scalar function
    \[
    f\colon \R^2 \to \R
    \]
    \begin{itemize}
        \item Has positive $\frac{\partial f}{\partial x}$ everywhere.
        \item Has negative $\frac{\partial f}{\partial y}$ everywhere.
    \end{itemize}
    (\emph{Hint: it may help to try adding single variable functions together. That is, let $f(x,y)=u(x)+v(y)$.}
    )
    \item Find the gradient of the function you chose.
\end{enumerate}
\vspace*{.5cm}

\noindent\textbf{Problem 3.} A rough model of a molecular crystal can be described in the following way.  Take the scalar function
\[
u(x,y)=\cos^2(x)+\cos^2(y).
\]
This function $u(x,y)$ describes the \emph{potential energy} for electrons in the crystal. Electrons are attracted to the areas with the smallest potential energy and move away from areas of high potential energy. 
\begin{enumerate}[(a)]
    \item Plot this function and include a printout.  Notice what this looks like.  You can imagine that each of the low points (well) is where a nucleus is located in the crystal.
    \item Plot the level curves where $u(x,y)=0$, $u(x,y)=\frac{1}{4}$, $u(x,y)=\frac{1}{2}$, and $u(x,y)=1$ for the range of values $-\frac{3\pi}{2}\leq x \leq \frac{3\pi}{2}$ and $-\frac{3\pi}{2}\leq y \leq \frac{3\pi}{2}$. 
    
    Picking the constant for the level curve tells you the \emph{kinetic energy} of the electron you are looking at.  It turns out that electrons (roughly) will orbit along these level curves.  Notice, some level curves bleed into the different troughs of neighboring molecules which means that electrons of sufficient energy happily flow through the crystal. However, electrons like to behave a bit differently thanks to their quantum nature!
    \item Find the gradient of this function $\nabla u(x,y)$.
    \item At what point(s) is the gradient zero? 
\end{enumerate}   
\vspace*{.5cm}


\noindent\textbf{Problem 4.} This function
\[
f(x,y)=\sin\left(\frac{2\pi x}{5}\right)\sin\left(\frac{2\pi y}{5}\right).
\]
comes up when you want to find out how a square shaped drum head will vibrate when hit. We'll see why later in the class, but for now do the following.
\begin{enumerate}[(a)]
    \item Plot this function on $0\leq x \leq 5$ and $0\leq y \leq 5$.  
    \item What can we say about the function $f(x,y)$ when $x=0$, $x=5$, $y=0$, and $y=5$?
    \item Compute $\frac{\partial^2 f}{\partial x^2}$, $\frac{\partial^2 f}{\partial y^2}$ and find $\frac{\partial^2 f}{\partial x^2}+\frac{\partial^2 f}{\partial y^2}$.
\end{enumerate}

\vspace*{.5cm}

\noindent\textbf{Problem 5.} We have briefly discussed the idea of \emph{work} (think change in energy) before and wrote
\[
W=\mathbf{F}\cdot \mathbf{r}
\]
where $\mathbf{F}$ was a constant force and $\mathbf{r}$ was a straight line displacement.

Now we can write the real version of this.  The work done on a particle moving along a curve $\gamma(t)$ that starts at $t=t_0$ and ends at $t=t_1$ and experiencing a force $\mathbf{F}(x,y,z)$ is
\[
W=\int_\gamma \mathbf{F}(\gamma)\cdot d\gamma \coloneqq \int_{t_0}^{t_1} \mathbf{F}(\gamma(t))\cdot \gamma'(t) dt.
\]
Compute the work done given
\[
\mathbf{F}(x,y,z)=\begin{bmatrix} x^2\\ y\\ \sqrt{z}\end{bmatrix} \quad \gamma(t)=\begin{bmatrix} t\\ t^2\\t^3 \end{bmatrix} \quad t_0=0, ~ t_1=1.
\]




\end{document}  