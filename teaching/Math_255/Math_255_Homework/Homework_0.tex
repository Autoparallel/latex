\documentclass[12pt]{amsbook}
\usepackage{geometry}                % See geometry.pdf to learn the layout options. There are lots.
%\geometry{letterpaper}                   % ... or a4paper or a5paper or ... 
\geometry{a4paper, top=25mm, right=25mm, bottom=25mm}
%\geometry{landscape}                % Activate for rotated page geometry
\usepackage[parfill]{parskip}    % Activate to begin paragraphs with an empty line rather than an indent
\usepackage{relsize}             % Allows us to define \bigast
\usepackage{graphicx}
\usepackage{amssymb}
\usepackage{epstopdf}
%\usepackage{pause}
\usepackage{wasysym}            % Provides \checkmark
\usepackage[firstpage]{draft watermark}             % Allows the watermark stuff
\usepackage{wrapfig}
\DeclareGraphicsRule{.tif}{png}{.png}{`convert #1 `dirname #1`/`basename #1 .tif`.png}

\newcommand{\DD}{\displaystyle}

\begin{document}
\pagenumbering{gobble}       % This kills the page numbering

\SetWatermarkText{
\begin{minipage}[c][8cm]{8cm}
\begin{center}
 
\end{center}
\end{minipage}
}
\SetWatermarkScale{1.5}
\SetWatermarkColor[gray]{0.75}



\begin{center}
   \textsc{\large %MATH 255\\ 
    Reading Assignments and Homework Expectations}\\
\end{center}
\vspace{.5cm}


Reading ahead of time and being prepared for lecture will be fundamental expectations in this course. As such, reading assignments from your textbook will be given regularly. Here are some tips to help when reading math:
\begin{itemize}
\item Go through the material slowly, taking notes as you go. There may only be nine pages to read for a given lecture, but it should take you significantly longer to read those nine pages than it would take to read nine pages of a novel, or even a history book. In math texts, every word has meaning and is intentional.
\item That in mind, consider reading in chunks instead of doing it all in one marathon sitting.
\item \textbf{Do} draw pictures.
\item \textbf{Don't} expect to understand everything perfectly. We'll tidy up in lecture.
\item \textbf{Do} use the recommended exercises. They are there to clarify the reading. You don't need to do all of them (unless a homework assignment explicitly says so), but taking a look at them/giving them a try will give you an idea of how well you understood what you just read.
\item \textbf{Do} write down questions you have, as well as a list of things you don't understand or want additional clarification on.
\item \textbf{Do} take breaks to think about what you've read, even if it's just in the back of your mind as you do something else.
\item \textbf{Do} use Google and Wikipedia when appropriate.
\end{itemize}

When we start a class, I will expect that you have read and at least attempted to understand the specified material. If the topic of the day is ``divergence and curl of a vector field," you should have at least a foggy notion of what ``divergence" means, gleaned from the reading.

Homework is there to help you cement the things learned either in reading or in lecture. You will be expected (within reason) to be able to do problems without having seen identical examples in class. In addition, it bears repeating from the syllabus: homework that is not written legibly or that is a loose collection of papers with no staple will not be accepted (if your handwriting is atrocious, practice or type up your work).

Reading assignments, homework, and lectures will work together to cover the course material. There will be plenty of overlap, but there will be topics that won't be covered by all three. This is not an excuse to not know the material. In general, we'll be subscribing to a two-thirds rule: if it's been covered on any two of [reading, homework, lecture], it's fair game for an exam.

\pagebreak
\begin{center}
   \textsc{\large MATH 255, Homework 0}\\
   Due January 25
\end{center}
\vspace{.5cm}

\textbf{New Reading:} Read sections 16.1, 16.2, 16.3, 16.10.

\textbf{Problem 1.} This class is not intended to be a random elective. For whichever major(s) you've decided to pursue, the powers-that-be decided that the content of this course was important for you to know. Your first Problem is to figure out why you are taking this class and then to write a brief essay detailing the reason(s). Here are a few tips:
\begin{itemize}
\item Consider consulting with your advisor. They should be able to tell you why this is a useful class.
\item The internet is your friend, as usual. Wikipedia \textbf{is} an acceptable source in this case, as it can often provide a good overview of your particular focus.
\item Speculation is both allowed and encouraged, as long as it is based in fact.
\item CSU has a Writing Center, which is a valuable asset in times of doubt.
\end{itemize}
Specifically, your essay should
\begin{itemize}
\item answer the question: ``Why am I taking this class?" with more depth than ``because my major requires it" -- tell me \emph{why} your major requires it;
\item be typed;
\item be at least one page, double-spaced;
\item include at least one source, cited according to APA style;
\item have margins $\leq$ 1 inch;
\item use 12 point Times New Roman font;
\item avoid first and second person pronouns (e.g. I, me, my, you, we, us, etc.);
\item avoid contractions;
\item and use correct grammar and spelling.
\end{itemize}





























\end{document}  