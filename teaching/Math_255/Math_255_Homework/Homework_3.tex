%%%%%%%%%%%%%%%%%%%%%%%%%%%%%%%%%%%%%%%%%%%%%%%%%%%%%%%%%%%%%%%%%%%%%%%%%%%%%%%%%%%%
% Document data
%%%%%%%%%%%%%%%%%%%%%%%%%%%%%%%%%%%%%%%%%%%%%%%%%%%%%%%%%%%%%%%%%%%%%%%%%%%%%%%%%%%%
\documentclass[12pt]{report} %report allows for chapters
\renewcommand\thesection{\arabic{section}} % ignore the title number for sections
%%%%%%%%%%%%%%%%%%%%%%%%%%%%%%%%%%%%%%%%%%%%%%%%%%%%%%%%%%%%%%%%%%%%%%%%%%%%%%%%%%%%




%%%%%%%%%%%%%%%%%%%%%%%%%%%%%%%%%%%%%%%%%%%%%%%%%%%%%%%%%%%%%%%%%%%%%%%%%%%%%%%%%%%%
% Packages
%%%%%%%%%%%%%%%%%%%%%%%%%%%%%%%%%%%%%%%%%%%%%%%%%%%%%%%%%%%%%%%%%%%%%%%%%%%%%%%%%%%%
\usepackage{color, soul, xcolor} % Colored text and highlighting, respectively

%Tikz
\usepackage{tikz-cd} % For commutative diagrams
\usepackage{tikz-3dplot}
\RequirePackage{pgfplots}
\usetikzlibrary{shadows}
\usetikzlibrary{shapes}
\usetikzlibrary{decorations}
\usetikzlibrary{arrows,decorations.markings} 
\usetikzlibrary{quotes,angles}

\usepackage{mathtools}
\usepackage{answers}
\usepackage{setspace}
\usepackage{graphicx}
\usepackage{enumerate}
\usepackage{multicol}
\usepackage{mathrsfs}
\usepackage[margin=1in]{geometry} 
\usepackage{amsmath,amsthm,amssymb}
\usepackage{marvosym,wasysym} %fucking smileys
%%%%%%%%%%%%%%%%%%%%%%%%%%%%%%%%%%%%%%%%%%%%%%%%%%%%%%%%%%%%%%%%%%%%%%%%%%%%%%%%%%%%




%%%%%%%%%%%%%%%%%%%%%%%%%%%%%%%%%%%%%%%%%%%%%%%%%%%%%%%%%%%%%%%%%%%%%%%%%%%%%%%%%%%%
% Shortcuts
%%%%%%%%%%%%%%%%%%%%%%%%%%%%%%%%%%%%%%%%%%%%%%%%%%%%%%%%%%%%%%%%%%%%%%%%%%%%%%%%%%%%
% Number systems
\newcommand{\N}{\mathbb{N}}
\newcommand{\Z}{\mathbb{Z}}
\newcommand{\C}{\mathbb{C}}
\newcommand{\R}{\mathbb{R}}
\newcommand{\Q}{\mathbb{Q}}

% Operators/functions
\newcommand{\id}{\mathrm{Id}}
\DeclareMathOperator{\sech}{sech}
\DeclareMathOperator{\csch}{csch}
%%%%%%%%%%%%%%%%%%%%%%%%%%%%%%%%%%%%%%%%%%%%%%%%%%%%%%%%%%%%%%%%%%%%%%%%%%%%%%%%%%%%




%%%%%%%%%%%%%%%%%%%%%%%%%%%%%%%%%%%%%%%%%%%%%%%%%%%%%%%%%%%%%%%%%%%%%%%%%%%%%%%%%%%%
% Environments
%%%%%%%%%%%%%%%%%%%%%%%%%%%%%%%%%%%%%%%%%%%%%%%%%%%%%%%%%%%%%%%%%%%%%%%%%%%%%%%%%%%%
% Italic font
\newtheorem{theorem}{Theorem}[section]
\newtheorem{lemma}{Lemma}[section]
\newtheorem{corollary}{Corollary}[section]
\newtheorem{axiom}{Axiom}

% Plain font
\theoremstyle{definition}
\newtheorem{definition}{Definition}[section]
\newtheorem{example}{Example}[section]
\newtheorem{remark}{Remark}[section]
\newtheorem{solution}{Solution}[section]
\newtheorem{problem}{Problem}[section]
\newtheorem{question}{Question}[section]
\newtheorem{answer}{Answer}[section]
\newtheorem{exercise}{Exercise}[section]
%%%%%%%%%%%%%%%%%%%%%%%%%%%%%%%%%%%%%%%%%%%%%%%%%%%%%%%%%%%%%%%%%%%%%%%%%%%%%%%%%%%%

\begin{document}


\begin{center}
   \textsc{\large MATH 255, Homework 3}\\
\end{center}
\vspace{.5cm}

\noindent\textbf{Relevant Sections:} 17.4, 17.5, 17.6, 18.4, 18.2, 18.6 \\

\noindent\textbf{Problem 1.} Consider the system of linear equations:
\begin{align*}
    3x+2y+0z&=5\\
    1x+1y+1z&=3\\
    0x+2y+2z&=4.
\end{align*}
\begin{enumerate}[(a)]
    \item Write the augmented matrix $M$ for this system of equations.
    \item Use row reduction to get the augmented matrix in row-echelon form.
    \item Determine the solution to the system of equations.
\end{enumerate}

\noindent\textbf{Problem 2.} Let
\[
A=\begin{bmatrix} 1 & 3 & 2 \\ 0 & 2 & 1\\ 2 & 1 & 2\\ \end{bmatrix} \quad \mathbf{b}= \begin{bmatrix} 1 \\ -3 \\ 13 \end{bmatrix}.
\]
\begin{enumerate}[(a)]
    \item Compute $\det(A)$ and determine whether the equation $A\mathbf{x}=\mathbf{b}$ has a solution.
    \item Create an augmented matrix $M$ for this system of equations.
    \item Determine the solution to the system of equations.
\end{enumerate}

\noindent\textbf{Problem 3.} Find the inverse matrix for each of the following:
\begin{enumerate}[(a)]
    \item \[
    A=\begin{bmatrix} 0 & 1 \\ -1 & 0 \end{bmatrix}.
    \]
    \item \[
    B=\begin{bmatrix} 1 & 1 & 0\\ 1 & 0 & 1 \\ 0 & 1 & 1 \end{bmatrix}.
    \]
\end{enumerate}

\noindent\textbf{Problem 4.} Construct transformation matrices that represent the following rotations about the $z$-axis:
\begin{enumerate}[(a)]
    \item Counterclockwise through $45^\circ = \frac{\pi}{4}$.
    \item Counterclockwise through $90^\circ = \frac{\pi}{2}$.
    \item Clockwise through $90^\circ = \frac{\pi}{2}$.
\end{enumerate}
(\emph{Hint: This necessary matrix is given to you in the notes and in the book, chapter 18).}
\pagebreak

\noindent\textbf{Problem 5.} Find the eigenvalues and eigenvectors for the following matrices.
\begin{enumerate}[(a)]
    \item \[
    A= \begin{bmatrix} 5/2 & 1/2 \\ 1/2 & 5/2 \end{bmatrix}.
    \]
    \item \[
    B= \begin{bmatrix} -1/2 & 1/2 & -1/2 \\ -1/2 & 1/2 & 1/2 \\ -1 & 1 & 0\end{bmatrix}.
    \]
\end{enumerate}







\end{document}  