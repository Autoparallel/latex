\documentclass[pdf]{beamer}
\mode<presentation>{\usetheme{Warsaw}}
\usepackage{textpos} %package for text positioning

%%%%%%%%%%%%%%%%%%%%%%%%%%%%%%%%%%%%%%%%%%%%%%%%%%%%%%%%%%%%%%%%%%%%%%%%%%%%%%%%%%%%
%Normal Math Packages
\usepackage{amsmath}
\usepackage{amsfonts}
\usepackage{enumerate}
\usepackage{amsmath}
\usepackage{mathtools}
\usepackage{tikz-cd}
\usepackage{ragged2e}
\usepackage{mathrsfs}
%%%%%%%%%%%%%%%%%%%%%%%%%%%%%%%%%%%%%%%%%%%%%%%%%%%%%%%%%%%%%%%%%%%%%%%%%%%%%%%%%%%%

%%%%%%%%%%%%%%%%%%%%%%%%%%%%%%%%%%%%%%%%%%%%%%%%%%%%%%%%%%%%%%%%%%%%%%%%%%%%%%%%%%%%
%Created Commands
\theoremstyle{definition}
\newtheorem*{remark}{Remark}
\newtheorem*{question}{Question}

\theoremstyle{theorem}
\newtheorem*{proposition}{Proposition}
\newtheorem*{axiom}{Axiom}

\newcommand{\R}{\mathbb{R}}
\newcommand{\Q}{\mathbb{Q}}
\newcommand{\F}{\mathbb{F}}
\newcommand{\A}{\mathcal{A}}
\newcommand{\N}{\mathbb{N}}
\newcommand{\C}{\mathbb{C}}
\newcommand{\opO}{\mathcal{O}}
\newcommand{\states}{\mathcal{S}}
%%%%%%%%%%%%%%%%%%%%%%%%%%%%%%%%%%%%%%%%%%%%%%%%%%%%%%%%%%%%%%%%%%%%%%%%%%%%%%%%%%%%


%%%%%%%%%%%%%%%%%%%%%%%%%%%%%%%%%%%%%%%%%%%%%%%%%%%%%%%%%%%%%%%%%%%%%%%%%%%%%%%%%%%%
%Stuff to make things look good
% Color modification
\setbeamercolor{structure}{fg=green!30!black}% to modify  immediately all palettes
\setbeamercolor{title}{fg=white}
\setbeamercolor{title in head/foot}{fg=yellow}

%Size Modification
\setbeamerfont{frametitle}{size=\small}

% position the logo
\addtobeamertemplate{frametitle}{}{%
\begin{textblock*}{1cm}(\textwidth,-1.1cm)
\includegraphics[height=.9cm,width=.9cm,keepaspectratio]{csu.png}
\end{textblock*}}
%%%%%%%%%%%%%%%%%%%%%%%%%%%%%%%%%%%%%%%%%%%%%%%%%%%%%%%%%%%%%%%%%%%%%%%%%%%%%%%%%%%%


%% preamble
\title{The Framework of Quantum Mechanics}
\subtitle{A $C^*$-Algebraic Approach}
\author{Colin Roberts}

\begin{document}

%%Title Frame
{
\setbeamertemplate{headline}{}
\addtobeamertemplate{frametitle}{\vspace*{-0.9\baselineskip}}{}
\begin{frame}
\titlepage
\end{frame}
}


\AtBeginSection[]
{
	\begin{frame}{Table of Contents}
		\tableofcontents[currentsection]
	\end{frame}
}
%%%%%%%%%%%%%%%%%%%%%%%%%%%%%%%%%%%%%%%%%%%%%%%%%%%%%%%%%%%%%%%%%%%%%%%%%%%%%%%
\section{Introduction to Algebras}
\begin{frame}{Algebras}
	\begin{definition}
	An \emph{algebra} $\A$ over a field $\F$ is a vector space $\A$ over $\F$ with an binary operation such that if $\alpha \in \F$ and $A,B\in \A$ then $\alpha(AB)=(\alpha A)B=A(\alpha A)$.
	\end{definition}
	\begin{example}
	Consider an $n$-dimensional vector space $V$ over any $\F$.  Then the elements in $\mathcal{L}(V)$ form an algebra.\\
	Then $\C$ is a matrix algebra over $\R$, and the quaternions $\mathcal{Q}$ are a matrix algebra over $\C$.
	\end{example}
\end{frame}

\subsection{Banach Algebras}
\begin{frame}{Banach Algebras}
	\begin{definition}
	A \emph{Banach algebra} is an algebra $\A$ over $\F$ that has a norm $\|\cdot \|$ relative to which $\A$ is a Banach space and such that for all $A,B\in \A$ we have
	\[
	\|AB\|\leq \|A\|\|B\|.
	\]
	\end{definition}
	
	\begin{example}
	If $X$ is a compact Hausdorff space, then $\A=\mathcal{C}_0(X,\C)$ form a Banach algebra if we define multiplication for $f,g\in \A$ by $(fg)(x)=f(x)g(x)$ for $x\in X$.  This is a commutative unital algebra.
	\end{example}
\end{frame}

\begin{frame}{Banach Algebras}
We can safely assume that the algebras we look at are unital by the following.
	\begin{proposition}
		If $\A$ is a Banach algebra without identity, then $\tilde{\A} = \A\times \F$ with operations
		\begin{itemize}
		\item $(A,\alpha)+(B+\beta)=(A+B,\alpha+\beta)$;
		\item $\beta(A,\alpha)=(\beta A,\beta \alpha)$;
		\item $(A,\alpha)(B,\beta)=(AB+\alpha B+\beta A,\alpha \beta)$
		\item $\|(A,\alpha)\|=\|A\|+|\alpha|$
		\end{itemize}
	give us that $\tilde{\A}$ is a Banach algebra with identity $(0,1)$ and $A\mapsto 		(A,0)$ is an \emph{isometric isomorphism} (linear bijective isometry) of $\A$ into $\tilde{\A}$.
	\end{proposition}
\end{frame}

\subsection{$C^*$-Algebras}
\begin{frame}{$*$-algebras}
	\begin{definition}
	A $*$-algebra $\A$ is an algebra together with an \emph{involution} that for $A,B\in \A$ and $\lambda \in \C$, satisfies
	\begin{itemize}
	\item $A^{**}=A,$
	\item $(A+B)^*=A^*+B^*$ and $(AB)^*=B^*A^*,$
	\item $(\lambda A)^*=\overline{\lambda}A^*.$
	\end{itemize}
	\end{definition}
\end{frame}

\begin{frame}{$C^*$-Algebras}
	\begin{definition}
A $C^*$-algebra $\A$ is a Banach algebra over $\C$ with an involution $*$ so that $\A$ is also a $*$-algebra and $\forall a \in \A$
\[
\|A^* A\|=\|A\|~\|A^*\|.
\]
The extra requirement is called the \emph{$C^*$-condition}, and shows $\|A A^*\|=\|A\|^2.$
	\end{definition}
\end{frame}

\begin{frame}{Examples}
	\begin{example}
	$\C$ itself is a $C^*$-algebra with $*$ being the complex conjugate.
	\end{example}
	\begin{example}
	If $H$ is a Hilbert space, $\A=\mathscr{B}(H)$ is a $C^*$-algebra where $*$ denotes the adjoint.
	\end{example}
\end{frame}


%%%%%%%%%%%%%%%%%%%%%%%%%%%%%%%%%%%%%%%%%%%%%%%%%%%%%%%%%%%%%%%%%%%%%%%%%%%%%%%
\section{Some Necessary $C^*$-Algebra Theory}
\subsection{$*$-Isomorphisms}
	\begin{frame}{$*$-Isomorphisms}
	\begin{definition}
	If $\A$ and $\mathcal{B}$ are $C^*$-algebras, then a bounded linear map $\pi\colon \A \to \mathcal{B}$ is a \emph{$*$-homomorphism} if
	\begin{itemize}
	\item For $A,B\in \A$, we have $\pi(AB)=\pi(A)\pi(B)$.
	\item For $A\in \A$, we have $\pi(A^*)=\pi(A)^*$.
	\end{itemize}
	\end{definition}
	For $C^*$-algebras, any $*$-homomorphism is bounded with norm $\leq 1$. We also have that injective $*$-homomorphisms are isometries.  A bijective $*$-homomorphism is a \emph{$C^*$-isomorphism} and we say $\A$ and $\mathcal{B}$ are \emph{isomorphic}.
	\end{frame}


\subsection{States}
%possibly states and representations

	\begin{frame}{States}
	\begin{definition}
	Let $\A$ be a $C^*$-algebra. Then a \emph{state} is a positive linear functional $S\colon \A \to \R$ with norm 1. 
	\end{definition}
	Specifically, we will care about the following.
	\begin{definition}
	Let $\A$ be a $C^*$-algebra of bounded operators on a corresponding Hilbert space $H$, then the linear functional $S_x \colon \A \to \R$ is given by 
	\[
	S_x(A)\coloneqq \langle Ax,x \rangle
	\]
	for $A\in A$. Note $S_x(1)=\|x\|^2$ thus $S_x$ is a state if $\|x\|=1$. Specifically $S_x$ is a \emph{vector state}.
	\end{definition}
	\end{frame}
	
\subsection{Gelfand-Naimark (GN) Theorems}

	\begin{frame}{GN Theorems}
	\begin{theorem}[GN Theorem for Commutative $C^*$-Algebras]
	A commutative (unital) $C^*$-algebra $\A$ is isomorphic to the $C^*$-algebra of bounded continuous functions on a compact Hausdorff space $X$.
	\end{theorem}
	\begin{theorem}[GN Theorem for Non-Commutative $C^*$-Algebras]
	An arbitrary $C^*$-Algebra $\A$ is isomorphic to a $C^*$-algebra of bounded operators on a Hilbert space $H$.
	\end{theorem}
	\end{frame}		
	
%Maybe just have the results of the GN theorems and stuff	
	
%	\begin{frame}{$*$-Representations}
%	\begin{definition}
%	A $*$-representation of a $C^*$
%	\end{definition}
%	\end{frame}

%\subsection{Gelfand-Nemark-Segal (GNS) Construction}
%	\begin{frame}
%	
%	\end{frame}
%\subsection{Gelfand-Nemark (GN) Theorem}
%	\begin{frame}
%	blah blah blah
%	\end{frame} 
%\subsection{Implications of GNS}
%	\begin{frame}
%	
%	\end{frame}

%%%%%%%%%%%%%%%%%%%%%%%%%%%%%%%%%%%%%%%%%%%%%%%%%%%%%%%%%%%%%%%%%%%%%%%%%%%%%%%
\section{Applications to Quantum Mechanics}
\begin{frame}{Goal}
Now, with some machinery defined (and more to come), we want to relate operator algebras in classical mechanics to those in quantum mechanics. \\
\textbf{Goal:} Provide a way to realize the axioms for quantum mechanics.
\end{frame}
\subsection{The Algebras of Classical Observables}
	\begin{frame}
	Classically, we think of \emph{phase space} (think position and momentum as coordinates) of a system to be a compact (symplectic) manifold $\Gamma$. Then:
	\begin{definition}[Classical Observables]
	The \emph{classical observables} are the continuous real-valued functions acting on the phase space. Namely, $\mathcal{C}^0(\Gamma,\R)$. We will denote all classical observables on $\Gamma$ as $\mathcal{O}=\mathcal{C}^0(\Gamma,\R)$.
	\end{definition}
	Why is it the case that $\Gamma$ is compact and that the observables are continuous?
	\end{frame}
	
	\begin{frame}{First Result}
	If we let $S=(p,q)\in \Gamma$, $A,B\in \opO$, and $\lambda \in \R$, then we define
	\begin{itemize}
	\item $(A+B)(S) \coloneqq A(S)+B(S)$,
	\item $(\lambda A)(S) \coloneqq \lambda A(S)$,
	\item $(AB)(S)\coloneqq A(S)B(S)$,
	\item $\|A\|\coloneqq \sup \{|A(S)| ~\colon~ S\in \Gamma\}$,
	\item $(A^*)(S)=\overline{A(S)}$.
	\end{itemize}
	Notice that $A$ is a real valued function, and thus $\overline{A(S)}=A(S)$.  This leads us to believe that elements of $\opO$ are self-adjoint. 
	\end{frame}
	
	\begin{frame}{First Result}
	What can be stated in the classical case as a theorem will eventually become an axiom for the quantum case. Namely:
	\begin{theorem}[Properties of Classical Observables]
	The set of observables $\opO$ of a classical system are the self-adjoint elements of a separable commutative $C^*$-algebra $\A$.
	\end{theorem}
	\end{frame}
	
	\begin{frame}{Classical States}
	\begin{itemize}
	\item It is a result of the Reisz-Markov-Kakutani Representation Theorem that we can write a classical state as a linear functional $S\colon \A \to \C$ by
	\[
	S(A)=\int_\Gamma A d\mu_S
	\]
	where $\mu_S$ is a uniquely defined Borel probability measure. 
	\item We can then think of $S(A)$ as the \emph{expected value of the observable $A$ with the particle in the state $S$}.  
	\end{itemize}
	\end{frame}

	\begin{frame}
	\begin{itemize}
	\item We can then define variance from this by
	\[
	\Delta_S(A)^2\coloneqq S[(A-S(A))^2].
	\]
	\item Yet we find that for classical states and observables that $\Delta_S(A)=0$.  
	\item This brings to light that the algebra of observables for quantum systems must not be commutative so that $\Delta_S(P)\Delta_S(Q)\geq \frac{\hbar}{2}$.
	\item Digression on uncertainty.
	\end{itemize}
	\end{frame}
	
\subsection{The Algebras of Quantum Observables}
	\begin{frame}
	\begin{itemize}
	\item From the earlier theorem for classical systems, we consider just removing commutivity.  
	\item We will see now that this coupled Heisenberg's commutation relation
	\[
	[Q,P]=\alpha \hbar \mathbf{1},
	\]
	with $\alpha\in \C$ and $|\alpha|=1$ forces the Heisenberg uncertainty principle \[
	\Delta_S(P)\Delta_S(Q)\geq \frac{\hbar}{2},
	\]
	to hold.
	\end{itemize}
	\end{frame}
	
	\begin{frame}{Heisenberg's Uncertainty}
	\begin{proof}[Non-Commuting Observables Imply Uncertainty Principle]\renewcommand{\qedsymbol}{}
	Let $A,B\in \opO$ and fix a state $S$.  We can assume $S(A)=S(B)=0$ since we could take the observables $A-S(A)$ and $B-S(B)$. Then 
	\[
	\Delta_S(A)^2\Delta_S(B)^2=S(A^2)S(B^2).
	\]
	Since $(A-i\lambda B)(A+i\lambda B)\geq 0$, $\forall \lambda \in \R$, positivity of $S$ implies
	\[
	S(A^2)+|\lambda|^2 S(B^2)+i\lambda S([A,B])\geq 0,
	\]
	where $[A,B]=AB-BA$. 
	\end{proof}
	\end{frame}
	
	\begin{frame}{Heisenberg's Uncertainty}
	\begin{proof}[Continued]\renewcommand{\qedsymbol}{}
	Now, define 
	\[
	M=\begin{bmatrix}
	S(A^2) & \frac{1}{2} S(i[A,B])\\
	\frac{1}{2} S(i[A,B]) & S(B^2)
	\end{bmatrix}
	~~ \textrm{and} ~~ \vec{\alpha}=\begin{bmatrix}
	\alpha\\
	\beta
	\end{bmatrix}
	\]
	and we see that the inequality
	\[
	(A-i\lambda B)(A+i\lambda B)\geq 0
	\]
	becomes 
	\[
	\vec{\alpha}^T M \vec{\alpha}\geq 0.
	\]
	This means $M$ is positive semi-definite.
	\end{proof}
	\end{frame}
	
	\begin{frame}{Heisenberg's Uncertainty}
	\begin{proof}[Continued]
	Hence
	\[
	\det M = S(A^2)S(B^2)-\frac{1}{4}S(i[A,B])^2\geq 0
	\]
	and thus
	\[
	\Delta_S(A)\Delta_S(B)\geq \frac{1}{2} |S([A,B])|.
	\]
	We then find that if $[P,Q]=\alpha \hbar \mathbf{1}$ with $\alpha \in \C$ and $|\alpha|=1$,
	\[
	\Delta_S(P)\Delta_S(Q)\geq \frac{\hbar}{2}.
	\]
	\end{proof}
	\end{frame}
	
	\begin{frame}{Conclusions}
	With what we've shown above, we can conclude the two major axioms for quantum mechanics. 
	\begin{axiom}[Quantum Observables]
	The observables of a quantum system are the self adjoint elements of a separable Hilbert space.
	\end{axiom}
	\begin{axiom}[Quantum States]
	The set of states $\mathcal{S}$ of a quantum system is the set of all positive linear functionals $\psi$ on $\A$ such that $\psi(\mathbf{1})=1$. We think of the functional $\psi(A)$ as $\langle A\psi, \psi\rangle$.
	\end{axiom}
	\end{frame}
	
	\begin{frame}{Future}
	These results start to bleed into other specific areas of research surrounding quantum mechanics.  Just to list a few,
	\begin{itemize}
	\item second quantization,
	\item (local) quantum field theory,
	\item deformation quantization,
	\item geometrical quantization.
	\end{itemize}
	\end{frame}
	
	\begin{frame}{Main Sources}
	If you're interested to read more, my main sources were
	\begin{itemize}
	\item \emph{The $C^*$-Algebraic Formalism of Quantum Mechanics}, Jonathan Gleason,
	\item \emph{An Introduction to the Mathematical Structure of Quantum Mechanics}, Franco Strocchi.
	\end{itemize}
	\end{frame}

	
	
	
\end{document}