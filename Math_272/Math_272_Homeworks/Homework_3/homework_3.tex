%%%%%%%%%%%%%%%%%%%%%%%%%%%%%%%%%%%%%%%%%%%%%%%%%%%%%%%%%%%%%%%%%%%%%%%%%%%%%%%%%%%%
% Document data
%%%%%%%%%%%%%%%%%%%%%%%%%%%%%%%%%%%%%%%%%%%%%%%%%%%%%%%%%%%%%%%%%%%%%%%%%%%%%%%%%%%%
\documentclass[12pt]{article} %report allows for chapters
%%%%%%%%%%%%%%%%%%%%%%%%%%%%%%%%%%%%%%%%%%%%%%%%%%%%%%%%%%%%%%%%%%%%%%%%%%%%%%%%%%%%
\usepackage{preamble}

\begin{document}

\begin{center}
   \textsc{\large MATH 272, Homework 3}\\
   \textsc{Due February 17$^\textrm{th}$}
\end{center}
\vspace{.5cm}


\begin{problem}
Compute the Fourier series for the following functions on the interval $[0,L]$. Then plot your result (for $N=1,50,100,500$) compared to the original function. What do you notice if you plot the Fourier series outside the range of $[0,L]$?
\begin{enumerate}[(a)]
	\item $f(x)=\frac{1}{2}\cos\left(\frac{2\pi x}{L}\right)+\sin\left(\frac{-4\pi x}{L}\right)$.
	\item $\sin\left(\frac{3\pi x}{L}\right)$.
	\item $\delta(x-L/2)$.
\end{enumerate}
\end{problem}

\begin{problem}
Consider a function $f(x)$ that describes the height of a rubber string with rest length $L$. We can attach the ends of the string at $x=0$ and $x=L$ by requiring that $f(0)=f(L)=0$.  Then, one can subject the string to an external force $g(x)$ and find the profile of the string by solving
\[
-\frac{d^2}{dx^2} f(x) = g(x).
\]
\begin{enumerate}[(a)]
	\item Let $g(x)=\delta(x-L/2)$ and let $f(x)$ be given by some Fourier series.  Using the equation above, solve for the coefficients of the Fourier series for $f(x)$.
	\item Plot the Fourier series for $f(x)$ for $N=1,5,50$.
\end{enumerate}
This is an extremely important to solve. The fact that we can determine a solution $f(x)$ where the external force is the Dirac delta function means that we have the ability to determine a the deformation of a string from a point force.
\end{problem}


\begin{problem}
Compute the following Fourier transforms (using a table or WolframAlpha if need be).  
\begin{enumerate}[(a)]
	\item $\sin(3\pi x)$.
	\item $e^{\frac{-x^2}{2}}$.
	\item $\delta(x)$.
\end{enumerate}
\end{problem}

%\begin{problem}
%The electric potential $V(r)$ is a function that, through a gradient $\nabla$ (a derivative we'll soon see), describes the electric field $\vecE=-\nabla V$ generated by charged particles. One can determine that the potential $V(r)$ created from a point charge $Q$ is
%\[
%\frac{Q}{4\pi \epsilon_0 r}.
%\]
%But why does it have that form?
%
%Consider the equation
%\[
%-\frac{1}{x}\frac{d}{dx}\left(x \frac{d}{dx} f(x)\right) = \delta(x)
%\]
%over the region $\R$.  
%\begin{enumerate}[(a)]
%	\item Simplify the left hand side of the equation.
%	\item Compute the Fourier transform of both sides of the equation
%	\[
%	\fourier\left(-\frac{1}{x}\frac{d}{dx}\left(x \frac{d}{dx} f(x)\right)\right)=\fourier(\delta(x)).
%	\]
%	\item Using the inverse Fourier transform, determine your solution $f(x)$ and compare to the function $V(r)=\frac{q}{4\pi \epsilon_0 r}$.
%\end{enumerate}
%Note that we were going about solving for the same equation as in Problem 2 except for that the region is no longer $[0,L]$ but rather all of $\R$. Also, the differential operator looks a bit different, but it is in fact the same operator but written in different coordinates!
%\end{problem}

\begin{problem}
A common application for the Fourier transform is to solve differential equations whose domain is time $t\in [0,\infty)$. We can model how a point $x$ on a rubber string oscillates over time consider the differential equation
\[
u''(t)+v^2u(t)=0,
\]
with initial conditions $u(0)=L$ and $u'(0)=0$. Here $u(t)$ is the displacement of the string at position $x$ with the initial conditions describing the string being pulled tight at time $t=0$.
\begin{figure}[H]
	\centering
	\def\svgwidth{\columnwidth}
	\input{point_on_string.pdf_tex}
\end{figure}
To solve this equation, we could use use methods we learned previously, or apply the Fourier transform to the whole equation by
\[
\fourier(u''(t)+v^2u(t))=\fourier(0).
\]
\begin{enumerate}[(a)]
	\item Compute the Fourier transform above.
	\item One should then have a new equation
	\[
	-4\pi^2k^2\hat{u}(k)+v^2\hat{u}(k)=0.
	\]
	Solve this new equation for $k$.
	\item One should have two values $k_1$ and $k_2$ from the work in (b).  This corresponds to the solution
	\[
	\hat{u}(k)=\delta(k-k_1) \qquad \textrm{and} \qquad \hat{u}(k)=\delta(k-k_2).
	\]
	Compute the inverse Fourier transform of the two delta functions. A linear combination of these correspond to your solution $u(t)$.
\end{enumerate}
\end{problem}




\end{document}