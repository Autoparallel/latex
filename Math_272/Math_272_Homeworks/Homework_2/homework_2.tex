%%%%%%%%%%%%%%%%%%%%%%%%%%%%%%%%%%%%%%%%%%%%%%%%%%%%%%%%%%%%%%%%%%%%%%%%%%%%%%%%%%%%
% Document data
%%%%%%%%%%%%%%%%%%%%%%%%%%%%%%%%%%%%%%%%%%%%%%%%%%%%%%%%%%%%%%%%%%%%%%%%%%%%%%%%%%%%
\documentclass[12pt]{article} %report allows for chapters
%%%%%%%%%%%%%%%%%%%%%%%%%%%%%%%%%%%%%%%%%%%%%%%%%%%%%%%%%%%%%%%%%%%%%%%%%%%%%%%%%%%%
\usepackage{preamble}

\begin{document}

\begin{center}
   \textsc{\large MATH 272, Homework 2}\\
   \textsc{Due February 11$^\textrm{th}$}
\end{center}
\vspace{.5cm}

\begin{problem}
   
\end{problem}

\begin{problem}

\end{problem}

\begin{problem}

\end{problem}

\begin{problem}

\end{problem}

\begin{problem}
\textcolor{red}{Consider maybe the momentum operator $i\frac{d}{dx}$ instead and show how you can build the Hamiltonian with this}
Consider the differential operator $\frac{d}{dx}$ acting on (sufficiently) differentiable complex valued functions with input $[0,L]$.  
\begin{enumerate}[(a)]
	\item Using the inner product for complex valued functions
	\[
	\innprod{f}{g} = \int_0^L f(x)g^*(x)dx,
	\]
	show that $\frac{d}{dx}$ is \underline{not} Hermitian. \emph{Hint: you will want to use integration by parts.}
	\item Indeed, this means the spectrum of $\frac{d}{dx}$ is not necessarily real.  To see the spectrum for this operator, we can look at the eigenvalues $\lambda$ that solve the equation
	\[
	\frac{d}{dx} f(x) = \lambda f(x).
	\]
	\begin{itemize}
		\item What (complex) values can $\lambda$ take on?
		\item What are the corresponding eigenfunctions for the derivative operator?
	\end{itemize}
	\item Compare your result in (b) with the spectrum of the operator $\frac{d^2}{dx^2}$ that we see in the free Schr\"odinger equation (for either the particle in the 1-dimensional box or the particle on the ring)
	\[
	-\frac{\hbar^2}{2m}\frac{d^2}{dx^2} \Psi(x) = E \Psi(x).
	\]
\end{enumerate}
\end{problem}


\end{document}