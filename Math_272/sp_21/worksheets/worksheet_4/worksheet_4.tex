%%%%%%%%%%%%%%%%%%%%%%%%%%%%%%%%%%%%%%%%%%%%%%%%%%%%%%%%%%%%%%%%%%%%%%%%%%%%%%%%%%%%
% Document data
%%%%%%%%%%%%%%%%%%%%%%%%%%%%%%%%%%%%%%%%%%%%%%%%%%%%%%%%%%%%%%%%%%%%%%%%%%%%%%%%%%%%
\documentclass[12pt]{article} %report allows for chapters
%%%%%%%%%%%%%%%%%%%%%%%%%%%%%%%%%%%%%%%%%%%%%%%%%%%%%%%%%%%%%%%%%%%%%%%%%%%%%%%%%%%%
\usepackage{preamble}
\usepackage{hyperref}

\newcommand{\vecfieldV}{\boldsymbol{\vec{V}}}
\newcommand{\vecfieldU}{\boldsymbol{\vec{U}}}
\newcommand{\vecfieldW}{\boldsymbol{\vec{W}}}
%\newcommand{\curvegamma}{\boldsymbol{\vec{\gamma}}}


\begin{document}

\begin{center}
   \textsc{\large MATH 272, Worksheet 4}\\
   \textsc{Surfaces, volumes, parameterization, and integration}
\end{center}
\vspace{.5cm}

\begin{problem}
Consider the function $f(x,y)=-(x^2+y^2)+8$ defined on the domain $x,y \in [-2,2]$. 
\begin{enumerate}[(a)]
    \item Plot a graph of this function over the given domain.
    \item If we were to integrate over the domain, would the integral be positive, negative, or zero? Explain.
    \item Compute the volume under the graph of $f(x,y)$ over the given domain via a double integral.
    \item Instead, compute the double integral over the domain $x\in [-2,2]$ and for $0<y<x$. 
\end{enumerate}
\end{problem}



\begin{problem}
    Compute the integral of the following scalar fields over the following regions.
\begin{itemize}
    \item $E(x,y,z) = \frac{x^2}{25} + \frac{y^2}{16} + \frac{z^2}{9}$.
    \item $f(x,y,z) = xyz$.
    \item $g(x,y,z) = e^x-y^2-z^2$.
    \item $q(x,y,z) = x^2+xy+y^2+\sin(yz)$.
\end{itemize}
  \begin{multicols}{2}
  \begin{enumerate}[(a)]
      \item $\Omega_1$ is the unit cube.
      \item $\Omega_2$ is the unit cube along with the rectangular prism given by $2<x<3$, $2<y<4$, and $-1<z<1$.
      \item $\Omega_3$ is the portion of the unit cube left over after slicing diagonally by the plane given by the equation $x+y+z=\frac{1}{2}$. Specifically, take the region left \underline{under} the plane and in the unit cube.
  \end{enumerate}
  \end{multicols}  
\end{problem}

\begin{problem}
    Plot the following surfaces and determine the surface normal $\unitvec$.
\begin{enumerate}[(a)]
    \item The graph of $f(x,y)=\sin(\sin(x^3)\sin(y^3))$.
    \item The level (implicit) surface $\left(\sqrt{x^2+y^2}-5\right)^2 + z^2 - 3^2 = 0$.
\end{enumerate}
\end{problem}

\begin{problem}
    Can the implicit surface defined by
    \[
        2y(y^2-3x^2)(1-z^2)+(x^2+y^2)^2-(9z^2-1)(1-z^2)=0,
    \]
    be described as the graph of a single $f(x,y)$? Explain.
\end{problem}

\begin{problem}
    Find either an explicit or implicit description for the following surfaces. 
\begin{enumerate}[(a)]
    \item A plane passing through the point $(1,2,3)$ perpendicular to the vector $\begin{pmatrix} -3 \\ 0 \\ 1 \end{pmatrix}$.
    \item A sphere of radius 3 centered at $(0,1,1)$.
    \item A hyperboloid of one sheet with a ``neck" radius of $1$.
\end{enumerate}
\end{problem}

\begin{problem}
    For the following scalar fields $f$, decide whether the level set is a surface or not. If the level set is a surface, what portions (if any) can be described as a graph $z=g(x,y)$?  What about as graphs $y=h(x,z)$ or $x=p(y,z)$? Plot after your analysis to see if your predictions are correct.
    \begin{enumerate}[(a)]
        \item $f(x,y,z)=xyz$ with $C=1$.
        \item $f(x,y,z)=xyz$ with $C=0$.
        \item $f(x,y,z) = z(x-y)+z(e^x-e^y)$ with $C=1$.
        \item $f(x,y,z) = z + \cos(xy)$ with $C=0$.
        \item $f(x,y,z) = z\sin(x)\sin(y)\sin(z)$ with $C=1$.
    \end{enumerate}
\end{problem}

\begin{problem}
    For the sets above that do indeed describe surfaces, find the surface normal $\unitvec$ and the area form $d\Sigma$.
\end{problem}

\begin{problem}
    Consider the surface of the unit cube in $\R^3$ which we'll call $\Sigma$.  Does the normal $\unitvec$ and the area form $d\Sigma$ change in smooth way as we move along the cube?  What is the (right handed) area form on each portion of the cube? Is there a well defined $\unitvec$ along edges or corners? Why does this not matter when we consider the total flux of some vector field $\vecfieldV$,
    \[
        \iint_\Sigma \vecfieldV \cdot \unitvec d\Sigma?
    \]
\end{problem}

\begin{problem}
    Consider the surface defined by the graph of $z = \sqrt{1-x^2-y^2}$ for $x^2+y^2\leq 1$. Become familiar with this example!
    \begin{enumerate}[(a)]
        \item Plot this surface.
        \item Plot the image of following curves \underline{on the surface $\Sigma$}. 
        \begin{itemize}
            \item $\curvegamma_1(t) = \begin{pmatrix} t\\ 0 \end{pmatrix}$.
            \item $\curvegamma_2(t) = \begin{pmatrix} 0 \\ t \end{pmatrix}$. 
            \item $\curvegamma_3(t) = \begin{pmatrix} \frac{1}{2} \cos(t) \\ \frac{1}{2} \sin(t) \end{pmatrix}$.
        \end{itemize}
        Which (if any) correspond to a line of longitude or a line of latitude?
        \item Find an equation for the tangent plane at the point $(0,0,1)$.
        \item Note that the point $(0.1,0.1,1)$ is on this tangent plane.  How close is this point to corresponding point on the sphere? That is, how close is $(0.1,0.1,1)$ to the point $(0.1,0.1,\sqrt{1-(0.1)^2-(0.1)^2})$? Is the tangent plane a reasonable approximation? What if instead we take $(0.01,0.01,1)$ instead?
        \item Find the surface normal $\unitvec$.
        \item What is the area form $d\Sigma$?
        \item Set up an integral that computes the surface area of $\Sigma$.
        \item Set up an integral that computes the total flux of the vector field $\vecfieldV(x,y,z) = \begin{pmatrix} yz \\ xz \\ xy \end{pmatrix}$ through the surface.
    \end{enumerate}
\end{problem}

\begin{problem}
    Consider the surface defined by the graph of $z=\sin(x)\sin(y)$ for $0\leq x \leq \pi$ and $0\leq y\leq \pi$. 
    \begin{enumerate}[(a)]
        \item Consider as well the curve $\curvegamma(t) = \begin{pmatrix} t \\ t \end{pmatrix}$.  What is the area under the curve if we take its image on the surface?
        \item Find the area form $d\Sigma$.
        \item If we have as well a scalar function $f(x,y,z)=x^2+y^2+z^2$, compute the integral
        \[
        \iint_\Sigma f d\Sigma.
        \]
    \end{enumerate}
\end{problem}

\begin{problem}
    Our surfaces have been frozen in time.  However, essentially every physical phenomenon evolves over time.  There are a few ways surfaces arise when time is involved.  Let us consider two examples.
    \begin{enumerate}[(a)]
        \item Consider the two variable scalar field $T(x,t) = \sin(x)e^{-t}$ with $0\leq x \leq L$ and $t\geq 0$. 
        \begin{itemize}
            \item Show that this function satisfies $\left(-\frac{\partial^2}{\partial x^2} + \frac{\partial}{\partial t}\right) T(x,t)=0$. This is known as the 1-dimensional heat equation. Here $T(x,t)$ models the temperature of point $x$ at time $t$ on a rod of length $L$.
            \item Plot the graph $z=T(x,t)$ for $0\leq x \leq L$ and $t\geq 0$.  What can we say about the temperature of the rod as $t\to \infty$?
        \end{itemize}
        \item Consider the three variable scalar field $u(x,y,t) = \sin(mx)\sin(ny)\sin(t)$ with $0\leq x \leq \pi$, $0\leq y \leq \pi$, $t\geq 0$, and $m$ and $n$ are positive integers.
        \begin{itemize}
            \item Show that this function satisfies $\left(-\frac{\partial^2}{\partial x^2}-\frac{\partial^2}{\partial y^2} + \frac{\partial^2}{\partial t^2}\right)u(x,y,t) = 0.$ This is known as the 2-dimensional wave equation. Here, $u(x,y,t)$ models the height of a membrane at the point $(x,y)$ and time $t$.
            \item Plot the graph of the surface $u(x,y,t_0)$ for various values of $t_0$, $m$ and $n$.  Or, visit \url{https://www.geogebra.org/3d/y55rd83m} to have full freedom with this surface (and watch it move over time).
        \end{itemize}
    \end{enumerate}
\end{problem}


\begin{problem}
Compute the flux of the following vector fields through the following surfaces.
\begin{itemize}
    \item $\vecfieldU = \begin{pmatrix} xyz \\ xyz \\ xyz \end{pmatrix}$.
    \item $\vecfieldV = \begin{pmatrix} e^{x+y+z} \\ e^{x+y+z} \\ e^{x+y+z} \end{pmatrix}$.
    \item $\vecfieldW = \begin{pmatrix} x \sin(y) \\ x \cos(y) \\ xz \end{pmatrix}$.
    \item $\forcevec = \begin{pmatrix} 5 + yz \\ -5 -xz \\ xy \end{pmatrix}$.
\end{itemize}
\begin{multicols}{2}
\begin{enumerate}[(a)]
    \item $\Sigma_1$ is the unit square in the $xy$-plane.  
    \item $\Sigma_2$ is the unit square in the $xz$-plane.
    \item $\Sigma_3$ is the unit square in the $yz$-plane.
    \item $\Sigma_4$ is the surface of the unit cube.
\end{enumerate}
\end{multicols}
\end{problem}

\begin{problem}
Compare the result for integrating $\vecfieldU$ from Problem 1 around $\curvegamma_1$ from Problem 8 with computing the flux of the curl of $\vecfieldU$ through $\Sigma_1$ from Problem. That is, check to see
\[
\int_{\curvegamma_1} \vecfieldU \cdot d\curvegamma_1 \stackrel{?}{=} \iint_{\Sigma_1} \left(\grad \times \vecfieldU \right)\cdot \unitvec d\Sigma_1.
\]
\emph{This result is typically referred to as Stokes' theorem. It is another way of relating an integral in a region to an integral along its boundary.}
\end{problem}



\end{document}
