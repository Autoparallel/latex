%%%%%%%%%%%%%%%%%%%%%%%%%%%%%%%%%%%%%%%%%%%%%%%%%%%%%%%%%%%%%%%%%%%%%%%%%%%%%%%%%%%%
% Document data
%%%%%%%%%%%%%%%%%%%%%%%%%%%%%%%%%%%%%%%%%%%%%%%%%%%%%%%%%%%%%%%%%%%%%%%%%%%%%%%%%%%%
\documentclass[12pt]{article} %report allows for chapters
%%%%%%%%%%%%%%%%%%%%%%%%%%%%%%%%%%%%%%%%%%%%%%%%%%%%%%%%%%%%%%%%%%%%%%%%%%%%%%%%%%%%
\usepackage{preamble}
\newcommand{\curvegamma}{\boldsymbol{\vec{\gamma}}}
\newcommand{\tangentgamma}{\boldsymbol{\dot{\vec{\gamma}}}}
\newcommand{\normalgamma}{\boldsymbol{\ddot{\vec{\gamma}}}}
%\newcommand{\forcevec}{\boldsymbol{\vec{F}}}
\newcommand{\rvec}{\boldsymbol{\vec{r}}}
\newcommand{\grad}{\boldsymbol{\vec{\nabla}}}
\newcommand{\vecfieldV}{\boldsymbol{\vec{V}}}
\newcommand{\vecfieldU}{\boldsymbol{\vec{U}}}
\newcommand{\vecfieldW}{\boldsymbol{\vec{W}}}
\newcommand{\veclaplace}{\boldsymbol{\vec{\Delta}}}
\newcommand{\vecfieldB}{\boldsymbol{\vec{B}}}
\usepackage{multicol}

\begin{document}

\begin{center}
   \textsc{\large MATH 272, Worksheet 2}\\
   \textsc{Vector fields and differential calculus of fields.}
\end{center}
\vspace{.5cm}


\begin{problem}
Plot the following vector fields.
\begin{multicols}{2}
\begin{enumerate}[(a)]
    \item $\vecfieldU(x,y,z) = \begin{pmatrix} xyz \\ xyz \\ xyz \end{pmatrix}$.
    \item $\vecfieldV(x,y,z) = \begin{pmatrix} e^{x+y+z} \\ e^{x+y+z} \\ e^{x+y+z} \end{pmatrix}$.
    \item $\vecfieldW(x,y,z) = \begin{pmatrix} x \sin(y) \\ x \cos(y) \\ xz \end{pmatrix}$.
    \item $\forcevec(x,y,z) = \begin{pmatrix} 5 + yz \\ -5 -xz \\ xy \end{pmatrix}$.
\end{enumerate}
\end{multicols}
\end{problem}

\begin{problem}
Compute the divergence, curl, and Jacobian matrix of the vector fields in Problem 1.
\end{problem}

\begin{problem}
For the following functions, compute and plot the gradient vector fields. 
\begin{multicols}{2}
\begin{enumerate}[(a)]
    \item For just $c=1$, plot the level set for $E(x,y,z) = \frac{x^2}{25} + \frac{y^2}{16} + \frac{z^2}{9}$.
    \item $f(x,y,z) = xyz$.
    \item $g(x,y,z) = e^x-y^2-z^2$.
    \item $h(x,y,z) = \sin(x)+\cos(y)-\tanh(z)$.
    \item $p(x,y,z) = \sin^2(x)+\sin^2(y)-\frac{1}{2}\sin(z)$.
    \item $q(x,y,z) = x^2+xy+y^2+sin(yz)$.
    \item One of your own choosing.
\end{enumerate}
\end{multicols}
\end{problem}

\begin{problem}
Given a function, a vector, and a point, compute the directional derivatives at that point.
\begin{enumerate}[(a)]
    \item $f(x,y)=x^2/y$, $\vecu = 3\xhat -\yhat$, and $(x,y)=(2,2)$.
    \item $g(x,y,z)=x\cos(yz^2)$, $\vecv = \frac{1}{2}\xhat + \frac{1}{2} \yhat + \zhat$, and $(x,y,z)=(0,-1,2)$.
\end{enumerate}
\end{problem}

\begin{problem}
    Give a physical argument for why the field $\vecfieldU = \begin{pmatrix} 0 \\ x \\ 0 \end{pmatrix}$ has nonzero curl everywhere. Reason why the direction of the curl is solely along the $z$-axis.  Do this \underline{without} computing the curl.
\end{problem}

\begin{problem}
    Give a physical argument why the field $\vecfieldV = \begin{pmatrix} x \\ 0 \\ 0 \end{pmatrix}$ has nonzero divergence everywhere. Reason why the divergence is a scalar quantity as opposed to a vector quantity. Do this \underline{without} computing the curl.
\end{problem}

\begin{problem}
Compute the Laplacian of the following fields.
\begin{multicols}{2}
\begin{enumerate}[(a)]
    \item $f(x,y) = (x+y)^2$
    \item $g(x,y) = (x+y)e^{x^2+y^2}$.
    \item $\vecfieldU = \begin{pmatrix} \cos(x)\cos(y)\cos(z) \\ \sin(x)\sin(y)\sin(z) \\ xyz \end{pmatrix}$.
    \item $\vecfieldV = \begin{pmatrix} \frac{y}{z} \\ \frac{x}{z} \\ \frac{x}{y} \end{pmatrix}$.
\end{enumerate}
\end{multicols}
\end{problem}

\begin{problem}
    Suppose that $\vecfieldV = \grad \phi$ for some scalar field $\phi$.  Explain why if $\vecfieldV$ is divergence free (i.e., if $\grad \cdot \vecfieldV=0$) that $\veclaplace \vecfieldV = \zerovec$.
\end{problem}

\begin{problem}
    One of Maxwell's equations states for a magnetic field $\vecfieldB$ that
    \[
    \grad \cdot \vecfieldB = 0,
    \]
    is an identity.  Does this mean that $\veclaplace \vecfieldB = 0$? 
\end{problem}


\end{document}
