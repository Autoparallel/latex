%%%%%%%%%%%%%%%%%%%%%%%%%%%%%%%%%%%%%%%%%%%%%%%%%%%%%%%%%%%%%%%%%%%%%%%%%%%%%%%%%%%%
% Document data
%%%%%%%%%%%%%%%%%%%%%%%%%%%%%%%%%%%%%%%%%%%%%%%%%%%%%%%%%%%%%%%%%%%%%%%%%%%%%%%%%%%%
\documentclass[12pt]{article} %report allows for chapters
%%%%%%%%%%%%%%%%%%%%%%%%%%%%%%%%%%%%%%%%%%%%%%%%%%%%%%%%%%%%%%%%%%%%%%%%%%%%%%%%%%%%
\usepackage{preamble}

\newcommand{\vecfieldV}{\boldsymbol{\vec{V}}}
\newcommand{\vecfieldW}{\boldsymbol{\vec{W}}}
\newcommand{\vecfieldU}{\boldsymbol{\vec{U}}}
\newcommand{\curvegamma}{\boldsymbol{\vec{\gamma}}}
\newcommand{\grad}{\boldsymbol{\vec{\nabla}}}
\newcommand{\curveeta}{\boldsymbol{\vec{\eta}}}

\usepackage{hyperref}

\begin{document}

\begin{center}
   \textsc{\large MATH 272, Worksheet 3}\\
   \textsc{Integration over curves and potential functions.}
\end{center}
\vspace{.5cm}

\begin{problem}
Compute the line integral of the vector fields given in Problem 1 along the following curves.
\begin{multicols}{2}
\begin{enumerate}[(a)]
    \item $\curvegamma_1$ is the boundary unit square in the $xy$-plane.  
    \item $\curvegamma_2$ is the unit circle in the $xy$-plane.
    \item $\curvegamma_3$ is the curve $\curvegamma_3(t)=\begin{pmatrix} t \\ t^2 \\ t^3 \end{pmatrix}$ from time $t=0$ to $t=1$.
\end{enumerate}
\end{multicols}
\end{problem}

\begin{problem}
    Using your curve $\curvegamma$ from Problem 2 with time starting at $t_0=0$ and $t_1=2\pi$ and your scalar field $f$ from Problem 4, compute the following. 
    \[
    \int_{\curvegamma} f(\gamma)d\curvegamma = \int_{t_0}^{t_1} f(\gamma(t))\left|\tangentgamma(t)\right|dt.
    \]
\end{problem}

\begin{problem} 
We have briefly discussed the idea of \emph{work} (change in energy) before and wrote
\[
W = \forcevec \cdot \rvec,
\]
where $\forcevec$ was a constant force and $\rvec$ was a straight line displacement.

Now, we can write the real version of this. The work done on a particle moving along a curve $\curvegamma(t)$ that starts at time $t_0$ and ends at time $t_1$ experiencing a (spatially dependent) force field $\forcevec(x,y,z)$ is
\[
W = \int_{\curvegamma} \forcevec(\curvegamma) \cdot d\curvegamma = \int_{t_0}^{t_1} \forcevec(\curvegamma(t))\cdot \tangentgamma(t) dt.
\]
Compute the work given the following
\[
\forcevec(x,y,z) = \begin{pmatrix} x^2 \\ y \\ \sqrt{z} \end{pmatrix} \qquad \textrm{and} \qquad \curvegamma(t) = \begin{pmatrix} t \\ t^2 \\ t^3 \end{pmatrix}.
\]
\end{problem}

\begin{problem}
    Compute the integral of the scalar fields given in Problem 7 over the following regions.
  \begin{multicols}{2}
  \begin{enumerate}[(a)]
      \item $\Omega_1$ is the unit cube.
      \item $\Omega_2$ is the unit cube along with the rectangular prism given by $2<x<3$, $2<y<4$, and $-1<z<1$.
      \item $\Omega_3$ is the portion of the unit cube left over after slicing diagonally by the plane given by the equation $x+y+z=\frac{1}{2}$. Specifically, take the region left \underline{under} the plane and in the unit cube.
  \end{enumerate}
  \end{multicols}  
\end{problem}

\begin{problem}
    Note that the identity $\grad \times (\grad f)=\zerovec$ always holds for any smooth scalar field $f$. 
    \begin{itemize}
        \item Pick a few functions $f(x,y)$ of your own and plot the graphs $z=f(x,y)$ and plot the vector field $\grad f$ as well.  Can you reason why the identity must be true from these plots? 
        \item If you plot the vector field $\vecfieldV = \begin{pmatrix} 0 \\ x \end{pmatrix}$ (which has nonzero curl), could this have come from the gradient of some function? What would the surface have to look like in order to have this as a gradient? Could it even be a valid function/surface?
    \end{itemize}
\end{problem}

\begin{problem}
    Decide whether the following fields have potentials.  If so, determine what they are. Plot the fields as well.
    \begin{enumerate}[(a)]
        \item $\vecfieldU(x,y,z) = \begin{pmatrix} 2x + 2y + 2z \\ 2x + 2y + 2z \\ 2x + 2y + 2z \end{pmatrix}$.
        \item $\vecfieldV(x,y,z) = \begin{pmatrix} yz \\ xz \\ xy \end{pmatrix}$.
        \item $\vecfieldW(x,y,z) = \begin{pmatrix} e^y \\ e^x \\ \sin(x)\sin(y) \end{pmatrix}$.
    \end{enumerate}
\end{problem}

\begin{problem}
    For the fields in Problem 2, take the curves $\curvegamma_1(t) =\begin{pmatrix} t \\ t \\ t\end{pmatrix}$ and $\curvegamma_2(t)=\begin{pmatrix} t \\ t^2 \\ t^3 \end{pmatrix}$ from time $t=0$ to time $t=1$ and integrate
    \[
    \int_{\curvegamma_i} \forcevec \cdot d\curvegamma_i.
    \]
    For which fields should this integral \underline{not} depend on the choice of curve? Recall we refer to these fields whose that are independent of the choice of path from point $a$ to point $b$ as conservative.
\end{problem}

\begin{problem} ***
    Given a conservative vector field $\vecfieldV$ and a curve $\curvegamma \colon [a,b]\to \R^3$, we know that 
    \[
    \int_{\curvegamma} \vecfieldV\cdot d\curvegamma,
    \]
    only depends on the start and end points of the curve $\curvegamma$.  That is, if we fix $\curvegamma(a)$ and $\curvegamma(b)$, the path between those two points does \underline{not} change the integral.
    
    If $\vecfieldV$ is conservative, then $\vecfieldV = \grad f$ for some scalar field $f$.  This yields the identity,
    \begin{equation}
    \int_{\curvegamma} \left(\grad f\right) \cdot d\curvegamma = f(\curvegamma(b))-f(\curvegamma(a)).
    \end{equation}
    This is, once again, some type of generalization of the Fundamental Theorem of Calculus (FTC) via the very general Stokes' theorem.
    \begin{enumerate}[(a)]
        \item Show that the above identity in Equation (1) is nothing but FTC.  \emph{Hint: take your (1-dimensional) curve $\curvegamma(x)=x$ so that $\curvegamma(a)=a$ and $\curvegamma(b)=b$. Finally, note $\grad = \frac{d}{dx}$.}
        \item Consider now a different curve $\curveeta \colon [\tilde{a},\tilde{b}]\to \R$. So long as $\curveeta(\tilde{a})=a$ and $\curveeta(\tilde{b})=b$, our identity states that the integral should output the same value.  Realize this as the $u$-substitution (or change of variables) that you learned in Calc. 1.
        \item Now, in 3-dimensions, we can discretize any curve to $n$ small movements in the $x$-, $y$-, or $z$-direction, and in each direction FTC will hold.  Thus, summing up the $n$ integrals (one from each movement) will cancel off many contributions and leave you only with the beginning and end point of the whole curve as a contribution.  Taking the limit that these movements are $dx$, $dy$, and $dz$, one can see that a choice of path for a smooth curve will not matter.  Draw a picture of this argument and clarify the approach to a proof.
    \end{enumerate}
\end{problem}

\begin{problem}
    For the following scalar fields $f$, decide whether the level set is a surface or not. If the level set is a surface, what portions (if any) can be described as a graph $z=g(x,y)$?  What about as graphs $y=h(x,z)$ or $x=p(y,z)$? Plot after your analysis to see if your predictions are correct.
    \begin{enumerate}[(a)]
        \item $f(x,y,z)=xyz$ with $C=1$.
        \item $f(x,y,z)=xyz$ with $C=0$.
        \item $f(x,y,z) = z(x-y)+z(e^x-e^y)$ with $C=1$.
        \item $f(x,y,z) = z + \cos(xy)$ with $C=0$.
        \item $f(x,y,z) = z\sin(x)\sin(y)\sin(z)$ with $C=1$.
    \end{enumerate}
\end{problem}

\begin{problem}
    For the sets above that do indeed describe surfaces, find the surface normal $\unitvec$ and the area form $d\Sigma$.
\end{problem}

\begin{problem}
    Consider the surface of the unit cube in $\R^3$ which we'll call $\Sigma$.  Does the normal $\unitvec$ and the area form $d\Sigma$ change in smooth way as we move along the cube?  What is the (right handed) area form on each portion of the cube? Is there a well defined $\unitvec$ along edges or corners? Why does this not matter when we consider the total flux of some vector field $\vecfieldV$,
    \[
        \iint_\Sigma \vecfieldV \cdot \unitvec d\Sigma?
    \]
\end{problem}

\begin{problem}
    Consider the surface defined by the graph of $z = \sqrt{1-x^2-y^2}$ for $x^2+y^2\leq 1$. Become familiar with this example!
    \begin{enumerate}[(a)]
        \item Plot this surface.
        \item Plot the image of following curves \underline{on the surface $\Sigma$}. 
        \begin{itemize}
            \item $\curvegamma_1(t) = \begin{pmatrix} t\\ 0 \end{pmatrix}$.
            \item $\curvegamma_2(t) = \begin{pmatrix} 0 \\ t \end{pmatrix}$. 
            \item $\curvegamma_3(t) = \begin{pmatrix} \frac{1}{2} \cos(t) \\ \frac{1}{2} \sin(t) \end{pmatrix}$.
        \end{itemize}
        Which (if any) correspond to a line of longitude or a line of latitude?
        \item Find an equation for the tangent plane at the point $(0,0,1)$.
        \item Note that the point $(0.1,0.1,1)$ is on this tangent plane.  How close is this point to corresponding point on the sphere? That is, how close is $(0.1,0.1,1)$ to the point $(0.1,0.1,\sqrt{1-(0.1)^2-(0.1)^2})$? Is the tangent plane a reasonable approximation? What if instead we take $(0.01,0.01,1)$ instead?
        \item Find the surface normal $\unitvec$.
        \item What is the area form $d\Sigma$?
        \item Set up an integral that computes the surface area of $\Sigma$.
        \item Set up an integral that computes the total flux of the vector field $\vecfieldV(x,y,z) = \begin{pmatrix} yz \\ xz \\ xy \end{pmatrix}$ through the surface.
    \end{enumerate}
\end{problem}

\begin{problem}
    Consider the surface defined by the graph of $z=\sin(x)\sin(y)$ for $0\leq x \leq \pi$ and $0\leq y\leq \pi$. 
    \begin{enumerate}[(a)]
        \item Consider as well the curve $\curvegamma(t) = \begin{pmatrix} t \\ t \end{pmatrix}$.  What is the area under the curve if we take its image on the surface?
        \item Find the area form $d\Sigma$.
        \item If we have as well a scalar function $f(x,y,z)=x^2+y^2+z^2$, compute the integral
        \[
        \iint_\Sigma f d\Sigma.
        \]
    \end{enumerate}
\end{problem}

\begin{problem}
    Our surfaces have been frozen in time.  However, essentially every physical phenomenon evolves over time.  There are a few ways surfaces arise when time is involved.  Let us consider two examples.
    \begin{enumerate}[(a)]
        \item Consider the two variable scalar field $T(x,t) = \sin(x)e^{-t}$ with $0\leq x \leq L$ and $t\geq 0$. 
        \begin{itemize}
            \item Show that this function satisfies $\left(-\frac{\partial^2}{\partial x^2} + \frac{\partial}{\partial t}\right) T(x,t)=0$. This is known as the 1-dimensional heat equation. Here $T(x,t)$ models the temperature of point $x$ at time $t$ on a rod of length $L$.
            \item Plot the graph $z=T(x,t)$ for $0\leq x \leq L$ and $t\geq 0$.  What can we say about the temperature of the rod as $t\to \infty$?
        \end{itemize}
        \item Consider the three variable scalar field $u(x,y,t) = \sin(mx)\sin(ny)\sin(t)$ with $0\leq x \leq \pi$, $0\leq y \leq \pi$, $t\geq 0$, and $m$ and $n$ are positive integers.
        \begin{itemize}
            \item Show that this function satisfies $\left(-\frac{\partial^2}{\partial x^2}-\frac{\partial^2}{\partial y^2} + \frac{\partial^2}{\partial t^2}\right)u(x,y,t) = 0.$ This is known as the 2-dimensional wave equation. Here, $u(x,y,t)$ models the height of a membrane at the point $(x,y)$ and time $t$.
            \item Plot the graph of the surface $u(x,y,t_0)$ for various values of $t_0$, $m$ and $n$.  Or, visit \url{https://www.geogebra.org/3d/y55rd83m} to have full freedom with this surface (and watch it move over time).
        \end{itemize}
    \end{enumerate}
\end{problem}

\end{document}
