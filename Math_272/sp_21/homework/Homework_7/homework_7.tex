%%%%%%%%%%%%%%%%%%%%%%%%%%%%%%%%%%%%%%%%%%%%%%%%%%%%%%%%%%%%%%%%%%%%%%%%%%%%%%%%%%%%
% Document data
%%%%%%%%%%%%%%%%%%%%%%%%%%%%%%%%%%%%%%%%%%%%%%%%%%%%%%%%%%%%%%%%%%%%%%%%%%%%%%%%%%%%
\documentclass[12pt]{article} %report allows for chapters
%%%%%%%%%%%%%%%%%%%%%%%%%%%%%%%%%%%%%%%%%%%%%%%%%%%%%%%%%%%%%%%%%%%%%%%%%%%%%%%%%%%%
\usepackage{preamble}
\newcommand{\grad}{\boldsymbol{\vec{\nabla}}}
\newcommand{\curvegamma}{\boldsymbol{\vec{\gamma}}}
\newcommand{\tangentgamma}{\boldsymbol{\dot{\vec{\gamma}}}}
\newcommand{\normalgamma}{\boldsymbol{\ddot{\vec{\gamma}}}}
\newcommand{\vecfieldE}{\boldsymbol{\vec{E}}}
\newcommand{\rhat}{\boldsymbol{\hat{r}}}
\newcommand{\thetahat}{\boldsymbol{\hat{\theta}}}
\newcommand{\phihat}{\boldsymbol{\hat{\phi}}}
\newcommand{\rhohat}{\boldsymbol{\hat{\rho}}}
\newcommand{\unitvec}{\boldsymbol{\hat{n}}}
\newcommand{\vecfieldB}{\boldsymbol{\vec{B}}}
\newcommand{\vecfieldJ}{\boldsymbol{\vec{J}}}
\newcommand{\vecfieldF}{\boldsymbol{\vec{F}}}



\begin{document}

\begin{center}
   \textsc{\large MATH 272, Homework 7}\\
   \textsc{Due March 31$^\textrm{st}$}
\end{center}
\vspace{.5cm}

\begin{problem}
    Plot each of the following vector fields.
    \begin{enumerate}[(a)]
        \item $\rhohat = \frac{x}{\sqrt{x^2+y^2}}\xhat + \frac{y}{\sqrt{x^2+y^2}}\yhat$.
        \item $\thetahat = \frac{-y}{\sqrt{x^2+y^2}}\xhat + \frac{x}{\sqrt{x^2+y^2}}\yhat$.
        \item $\zhat$.
    \end{enumerate}
\end{problem}

\begin{problem}
Consider the following vector field
\[
\vecfieldB = -\frac{y}{2}\xhat + \frac{x}{2}\yhat.
\]
Here, $\vecfieldB$ denotes the magnetic field. It may be helpful to plot the fields in this problem.
\begin{enumerate}[(a)]
    \item Show that $\grad \times \vecfieldB = \vecfieldJ$ (Amp\'ere's law) where 
    \[
    \vecfieldJ = \zhat.
    \]
    This vector field $\vecfieldJ$ represents the electric current (moving charges) in space. One could argue that the current creates the magnetic field via Amp\'ere's law.
    \item Magnetic fields induce a force $\vecfieldF$ on charged particles by the Lorentz force
    \[
    \vecfieldF = \tangentgamma \times \vecfieldB = \normalgamma
    \]
    Where $\tangentgamma$ is the velocity of the particle (where we have chosen a mass $m=1$ and charge $q=1$).  Let us do the following.
    \begin{itemize}
        \item Assume that $\tangentgamma = \xhat$, what is the force on the particle?
        \item Repeat the previous step for $\tangentgamma=\yhat$ and $\tangentgamma=\zhat$. 
        \item Compare and contrast the forces you found.
    \end{itemize}
    \item Can you argue why applying a magnetic field to a molecule may cause it to heat up? Can you compare this idea with your home microwave? (Please don't put anything dangerous in your microwave due to this discovery!)
\end{enumerate}
\end{problem}

\begin{problem}
Let us see some of the usefulness of cylindrical coordinates.
\begin{enumerate}[(a)]
    \item Using the fact that $\thetahat = \frac{-y}{\sqrt{x^2+y^2}}\xhat + \frac{x}{\sqrt{x^2+y^2}}\yhat$, convert the magnetic field from Problem 2 into Cylindrical coordinates (i.e., only a function of $\rho$, $\theta$, $z$, and $\rhohat$, $\thetahat$, and $\zhat$).  
        \item Parameterize a curve $\curvegamma(t)$ that traces out the unit circle in the $xy$-plane in cylindrical coordinates.
        \item What is the tangent vector $\tangentgamma(t)$ in terms of $\rhohat$, $\thetahat$, and $\zhat$?
        \item Compute the following integral using cylindrical coordinates that we have found:
        \[
        \int_{\curvegamma} \vecfieldB \cdot d\curvegamma
        \]
        \item Using cylindrical coordinates, compare your result with
        \[
        \iint_\Sigma \left(\grad \times \vecfieldB\right) \cdot \unitvec d\Sigma,
        \]
        where $\Sigma$ is the unit disk.
\end{enumerate}
\end{problem}

\begin{problem}
    Convert the following integrals to integrals in cylindrical coordinates. Also, describe the region in which you are integrating over. Do not evaluate the integrals.
    \begin{enumerate}[(a)]
        \item $\displaystyle{\int_{-1}^{1} \int_{-1}^{1} \int_{-\sqrt{1-y^2}}^{\sqrt{1-y^2}} xyz dxdydz}$.
        \item $\displaystyle{\int_0^1 \int_{-z}^z \int_{-\sqrt{1-y^2}}^{\sqrt{1-y^2}} x^2+y^2+z^2 dxdydz}$.
    \end{enumerate}
\end{problem}

\begin{problem}
	In spherical coordinates (either implicitly or explicitly), parameterize the following objects.
	\begin{enumerate}[(a)]
		\item A solid sphere with radius 3.
		\item The surface of an infinite cone with a vertex angle of $\pi/4$.
		\item A latitudinal curve on the unit sphere at the latitude of 30$^\circ$ above the equator.
		\item A solid unit sphere with a cylinder of radius 1/2 removed from the core.
	\end{enumerate}
\end{problem}
 

\end{document}