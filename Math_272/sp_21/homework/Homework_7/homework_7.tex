%%%%%%%%%%%%%%%%%%%%%%%%%%%%%%%%%%%%%%%%%%%%%%%%%%%%%%%%%%%%%%%%%%%%%%%%%%%%%%%%%%%%
% Document data
%%%%%%%%%%%%%%%%%%%%%%%%%%%%%%%%%%%%%%%%%%%%%%%%%%%%%%%%%%%%%%%%%%%%%%%%%%%%%%%%%%%%
\documentclass[12pt]{article} %report allows for chapters
%%%%%%%%%%%%%%%%%%%%%%%%%%%%%%%%%%%%%%%%%%%%%%%%%%%%%%%%%%%%%%%%%%%%%%%%%%%%%%%%%%%%
\usepackage{preamble}
\newcommand{\grad}{\boldsymbol{\vec{\nabla}}}
\newcommand{\curvegamma}{\boldsymbol{\vec{\gamma}}}
\newcommand{\tangentgamma}{\boldsymbol{\dot{\vec{\gamma}}}}
\newcommand{\normalgamma}{\boldsymbol{\ddot{\vec{\gamma}}}}
\newcommand{\vecfieldE}{\boldsymbol{\vec{E}}}
\newcommand{\rhat}{\boldsymbol{\hat{r}}}
\newcommand{\thetahat}{\boldsymbol{\hat{\theta}}}
\newcommand{\phihat}{\boldsymbol{\hat{\phi}}}
\newcommand{\rhohat}{\boldsymbol{\hat{\rho}}}
\newcommand{\unitvec}{\boldsymbol{\hat{n}}}
\newcommand{\vecfieldB}{\boldsymbol{\vec{B}}}
\newcommand{\vecfieldJ}{\boldsymbol{\vec{J}}}
\newcommand{\vecfieldF}{\boldsymbol{\vec{F}}}
\newcommand{\vecfieldV}{\boldsymbol{\vec{V}}}
\newcommand{\vecx}{\boldsymbol{\vec{x}}}

\newcommand{\veclaplace}{\boldsymbol{\vec{\Delta}}}


\begin{document}

\begin{center}
   \textsc{\large MATH 272, Homework 7}\\
   \textsc{Due March 31$^\textrm{st}$}
\end{center}
\vspace{.5cm}

\begin{problem}
Previously we studied the time-independent Schr\"odinger equation. Now, we can take a look at the time-dependent version given by
\[
H \Psi(x,t) = i\hbar \frac{\partial}{\partial t} \Psi(x,t),
\]
where $H$ is the Hamiltonian operator.  Consider the situation for the free particle in the 1-dimensional box of length $L$ so that $V(x)=0$ and $\Psi(0,t)=0=\Psi(L,t)$.  
\begin{enumerate}[(a)]
    \item Take a separation of variables ansatz and find a set of solutions (one for every positive integer $n$) to the time-dependent equation.
    \item Show that a super position of solutions is also a solution.
    \item For a single state $\psi_n(x,t)$, show that 
    \[
    \int_0^L \left|\psi_n(x,t)\right|^2 dx,
    \]
    is independent of $t$. This shows that the states $\psi_n$ are \emph{stationary} since their total probability does not depend on time.
\end{enumerate}
\end{problem}


\begin{problem}
Maxwell's equations are given as
\begin{align*}
\grad \cdot \vecfieldB &= 0  & \grad \cdot \vecfieldE &= \frac{\rho}{\epsilon_0}\\
\grad \times \vecfieldB -\mu_0 \epsilon_0\frac{\partial \vecfieldE}{\partial t}&=\mu_0 \vecfieldJ & \grad \times \vecfieldE + \frac{\partial \vecfieldB}{\partial t} &= \zerovec
\end{align*}
\begin{enumerate}[(a)]
    \item Look up each of the terms in the equations above and describe them.
    \item Describe what each equation is saying and why these are PDEs.
    \item In the absence of all charges we will have $\vecfieldJ=\zerovec$ and $\rho=0$.  Using that and the following two facts
    \[
    \veclaplace \vecfieldV = \grad (\grad \cdot \vecfieldV) - \grad \times (\grad \times \vecfieldV) \qquad \textrm{and} \qquad \grad \times \frac{\partial \vecfieldV}{\partial t} = \frac{\partial}{\partial t} (\grad \times \vecfieldV),
    \]
    derive the vector wave equations for light
    \[
    \left( - \veclaplace + \mu_0 \epsilon_0 \frac{\partial^2 }{\partial t^2}\right) \vecfieldE = \zerovec
    \]
    and
    \[
    \left( - \veclaplace + \mu_0 \epsilon_0 \frac{\partial^2 }{\partial t^2}\right) \vecfieldB = \zerovec
    \]
    \item From the equations you derived, determine the wave speed of light in the vacuum, $c_0$.
\end{enumerate}
\end{problem}

\begin{problem}
In 3-dimensional space, we can write down the Hamiltonian operator for an electron orbiting a proton. Specifically, this is
\[
H = -\frac{\hbar^2}{2\mu} \Delta + V(x,y,z).
\]
where $\mu$ is the reduced mass and with the Coulomb potential
    \[
    V(x,y,z) = \frac{1}{4\pi \epsilon_0} \frac{e^2}{\sqrt{x^2+y^2+z^2}},
    \]
which is the electrostatic potential created by a single proton pulling on a single electron. 
\begin{enumerate}[(a)]
    \item Write down the time independent Schr\"odinger equation in spherical coordinates. \emph{That means you must also convert the laplacian as well.}
    \item Take the separation of variables ansatz $\Psi(r,\theta,\phi) = R(r)Y(\theta,\phi)$ and show that the time independent equation is separable into radial and angular components.
\end{enumerate}
\end{problem}


\begin{problem}[BONUS]
The differential equation for the static magnetic field $\vecfieldB$ is given by Ampere's law
\[
\grad \times \vecfieldB = \mu_0 \vecfieldJ.
\]
This equation is solvable by the Biot-Savart law. Let us consider a loop of wire with a constant current so that
\[
\curvegamma(t) = \begin{pmatrix} \cos(t) \\ \sin(t) \\ 0 \end{pmatrix} \qquad t\in [0,2\pi],
\]
and $\vecfieldJ= J \tangentgamma(t)$ along $\curvegamma$ and is zero elsewhere. Let $\vecx = \begin{pmatrix} x \\ y \\ z \end{pmatrix}$ denote the position in space we wish to measure the magnetic field.
\begin{enumerate}[(a)]
    \item The Biot-Savart law says
    \[
    \vecfieldB(\vecx) = \frac{\mu_0}{4\pi} \int_{\curvegamma} \frac{(J d\curvegamma) \times (\vecx - \curvegamma)}{|\vecx-\curvegamma|}.
    \]
    Find $\vecfieldB$ using this law. \emph{Note that the magnetic field is ill defined along the current loop itself.} See \url{https://pages.uncc.edu/phys2102/online-lectures/chapter-7-magnetism/7-2-magnetic-field-biot-savart-law/example-magnetic-field-of-a-current-loop/} for help.
    \item Draw a picture displaying what the above integral is computing.
    \item Compute $\grad \times \vecfieldB$. Is your answer a solution to Ampere's law?
\end{enumerate}
\end{problem}
 

\end{document}