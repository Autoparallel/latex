%%%%%%%%%%%%%%%%%%%%%%%%%%%%%%%%%%%%%%%%%%%%%%%%%%%%%%%%%%%%%%%%%%%%%%%%%%%%%%%%%%%%
% Document data
%%%%%%%%%%%%%%%%%%%%%%%%%%%%%%%%%%%%%%%%%%%%%%%%%%%%%%%%%%%%%%%%%%%%%%%%%%%%%%%%%%%%
\documentclass[12pt]{article} %report allows for chapters
%%%%%%%%%%%%%%%%%%%%%%%%%%%%%%%%%%%%%%%%%%%%%%%%%%%%%%%%%%%%%%%%%%%%%%%%%%%%%%%%%%%%
\usepackage{preamble}
\newcommand{\grad}{\boldsymbol{\vec{\nabla}}}
\newcommand{\curvegamma}{\boldsymbol{\vec{\gamma}}}
\newcommand{\tangentgamma}{\boldsymbol{\dot{\vec{\gamma}}}}
\newcommand{\vecfieldE}{\boldsymbol{\vec{E}}}
\newcommand{\rhat}{\boldsymbol{\hat{r}}}
\newcommand{\thetahat}{\boldsymbol{\hat{\theta}}}
\newcommand{\phihat}{\boldsymbol{\hat{\phi}}}
\newcommand{\unitvec}{\boldsymbol{\hat{n}}}
\begin{document}

\begin{center}
   \textsc{\large MATH 272, Homework 8}\\
   \textsc{Due April 6$^\textrm{th}$}
\end{center}
\vspace{.5cm}

\begin{problem}
    Plot each of the following vector fields.
    \begin{enumerate}[(a)]
        \item $\rhat = \frac{x}{\sqrt{x^2+y^2+z^2}}\xhat + \frac{y}{\sqrt{x^2+y^2+z^2}}\yhat+\frac{z}{\sqrt{x^2+y^2+z^2}}\zhat$.
        \item $\thetahat = \frac{-y}{\sqrt{x^2+y^2}}\xhat + \frac{x}{\sqrt{x^2+y^2}}\yhat$.
        \item $\phihat = \frac{xz}{\sqrt{x^2+y^2}\sqrt{x^2+y^2+z^2}}\xhat + \frac{yz}{\sqrt{x^2+y^2}\sqrt{x^2+y^2+z^2}}\yhat+\frac{-\sqrt{x^2+y^2}}{\sqrt{x^2+y^2+z^2}}\zhat$.
    \end{enumerate}
\end{problem}

\begin{problem}
    Consider the following vector field
    \[
    \vecfieldE = \frac{x}{\left(x^2+y^2+z^2\right)^{3/2}} \xhat + \frac{y}{\left(x^2+y^2+z^2\right)^{3/2}} \yhat + \frac{z}{\left(x^2+y^2+z^2\right)^{3/2}} \zhat,
    \]
    which you can think of as the electric field of a positive point charge.  We argued that this field $\vecfieldE$ is conservative in a previous homework problem. Specifically, $\vecfieldE = \grad \phi$, for some scalar field $\phi$. This follows from Faraday's law for static charges.
    \begin{enumerate}[(a)]
        \item Compute the integral
        \[
        T=\int_{\curvegamma} \vecfieldE \cdot d\curvegamma \qquad \textrm{where} \qquad \curvegamma(t) = \begin{pmatrix} t \\ t \\ t \end{pmatrix},
        \]
        and $a\leq t \leq b$ with $a$ and $b$ both greater than 0.  Note that this integral $T$ describes the gain in kinetic energy of a charged particle that moved along the path $\curvegamma$. 
        \item Equivalently, since $\vecfieldE$ is conservative, we have
        \[
        T=\int_{\curvegamma} \vecfieldE \cdot d\curvegamma = \phi(\curvegamma(b))-\phi(\curvegamma(a)).
        \]
        Show that this is true for the given vector field and potential. This shows that the choice of path does not matter; only the endpoints $\curvegamma(a)$ and $\curvegamma(b)$ matter.
        \item Argue why the integral around any closed curve must be zero.
    \end{enumerate}
\end{problem}

\begin{problem}
    Let us see some of the benefit of using spherical coordinates. 
    \begin{enumerate}[(a)]
        \item Using the fact that 
        \[
        \rhat = \frac{x}{\sqrt{x^2+y^2+z^2}}\xhat + \frac{y}{\sqrt{x^2+y^2+z^2}}\yhat + \frac{z}{\sqrt{x^2+y^2+z^2}}\zhat,
        \]
        convert the vector field $\vecfieldE$ into spherical coordinates (i.e., only a function of $r$, $\theta$, $\phi$, and $\rhat$, $\thetahat$, and $\phihat$).
        \item Parameterize the surface of a sphere of radius $R$ (which we'll call $\Sigma$) as well as the outward normal vector $\unitvec$ and  in spherical coordinates.
        \item Compute the following integral using spherical coordinates that we have found:
        \[
        \iint_\Sigma \vecfieldE \cdot \unitvec d\Sigma,
        \]
        where $d\Sigma$ will be the area form in spherical coordinates.
    \end{enumerate}
\end{problem}

\begin{problem}
    Note that the Laplacian $\Delta$ in cylindrical coordinates is given by
    \[
        \Delta f(\rho,\theta,z) = \frac{1}{\rho} \frac{\partial}{\partial \rho} \left(\rho \frac{\partial f}{\partial \rho}\right)+\frac{1}{\rho^2}\frac{\partial^2 f}{\partial \theta^2} + \frac{\partial^2 f}{\partial z^2}.
    \]
    Compute the Laplacian of
    \[
        f(\rho,\theta,z) = \sqrt{\rho^2+z^2} z \cos(\theta).
    \]
\end{problem}

\begin{problem}
    Note that the Laplacian $\Delta$ in spherical coordinates is given by
    \[
        \Delta f(r,\theta,\phi) = \frac{1}{r^2} \frac{\partial}{\partial r} \left(r^2 \frac{\partial f}{\partial r}\right)+\frac{1}{r^2 \sin^2 \phi} \frac{\partial^2 f}{\partial \theta^2} + \frac{1}{r^2 \sin\phi}\frac{\partial}{\partial \phi} \left(\sin \phi \frac{\partial f}{\partial \phi}\right).
    \]
    Compute the Laplacian of
    \[
       f(r,\theta,\phi) = r^2 \cos(\theta)\cos(\phi).
    \]
\end{problem}

\begin{problem} (BONUS)
The following problem is a somewhat pop-culture math paradox known as the \emph{napkin ring problem} (see Vsauce for more).  Consider the following problem.  We want to compute the volume inside a ball of radius $R$ after drilling out an inscribed cylinder of height $h$. See the following picture.
\begin{figure}[H]
	\centering
	\def\svgwidth{0.60\columnwidth}
	\input{Sphere_Bands.pdf_tex}
\end{figure}
The question is, does the left over volume (of the napkin ring) depend on the radius $R$ of the sphere. You have your choice of working in spherical or cylindrical coordinates. Use whichever helps you most.
\end{problem}


\end{document}