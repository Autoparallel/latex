%%%%%%%%%%%%%%%%%%%%%%%%%%%%%%%%%%%%%%%%%%%%%%%%%%%%%%%%%%%%%%%%%%%%%%%%%%%%%%%%%%%%
% Document data
%%%%%%%%%%%%%%%%%%%%%%%%%%%%%%%%%%%%%%%%%%%%%%%%%%%%%%%%%%%%%%%%%%%%%%%%%%%%%%%%%%%%
\documentclass[12pt]{article} %report allows for chapters
%%%%%%%%%%%%%%%%%%%%%%%%%%%%%%%%%%%%%%%%%%%%%%%%%%%%%%%%%%%%%%%%%%%%%%%%%%%%%%%%%%%%
\usepackage{preamble}
\newcommand{\grad}{\boldsymbol{\vec{\nabla}}}
\newcommand{\vecfieldB}{\boldsymbol{\vec{B}}}
\newcommand{\vecfieldE}{\boldsymbol{\vec{E}}}
\newcommand{\rhat}{\boldsymbol{\hat{r}}}
\newcommand{\thetahat}{\boldsymbol{\hat{\theta}}}
\newcommand{\phihat}{\boldsymbol{\hat{\phi}}}
\newcommand{\rhohat}{\boldsymbol{\hat{\rho}}}

\begin{document}

\begin{center}
   \textsc{\large MATH 272, Homework 4}\\
   \textsc{Due February 26$^\textrm{th}$}
\end{center}
\vspace{.5cm}

\begin{problem}
    Plot each of the following vector fields. Describe what each field represents in the relevant coordinate system. Do we see any points in space where there are issues with these vector fields?
    \begin{enumerate}[(a)]
        \item $\rhohat = \frac{x}{\sqrt{x^2+y^2}}\xhat + \frac{y}{\sqrt{x^2+y^2}}\yhat$.
        \item $\thetahat = \frac{-y}{\sqrt{x^2+y^2}}\xhat + \frac{x}{\sqrt{x^2+y^2}}\yhat$.
        \item $\phihat = \frac{xz}{\sqrt{x^2+y^2}\sqrt{x^2+y^2+z^2}}\xhat + \frac{yz}{\sqrt{x^2+y^2}\sqrt{x^2+y^2+z^2}}\yhat+\frac{-\sqrt{x^2+y^2}}{\sqrt{x^2+y^2+z^2}}\zhat$.
        \item $\rhat = \frac{x}{\sqrt{x^2+y^2+z^2}}\xhat + \frac{y}{\sqrt{x^2+y^2+z^2}}\yhat+\frac{z}{\sqrt{x^2+y^2+z^2}}\zhat$.
    \end{enumerate}
\end{problem}

\vspace*{.5cm}

\begin{problem}
Let us see some of the usefulness of cylindrical coordinates.
\begin{enumerate}[(a)]
    \item Using the fact that $\thetahat = \frac{-y}{\sqrt{x^2+y^2}}\xhat + \frac{x}{\sqrt{x^2+y^2}}\yhat$, convert the magnetic field 
    \[
    \vecfieldB = -\frac{y}{2}\xhat + \frac{x}{2}\yhat,
    \]
    into cylindrical coordinates (i.e., only a function of $\rho$, $\theta$, $z$, and $\rhohat$, $\thetahat$, and $\zhat$).  
        \item Parameterize a curve $\curvegamma(t)$ that traces out the unit circle in the $xy$-plane in cylindrical coordinates.
        \item Compute the tangent vector $\tangentgamma(t)$ in cylindrical coordinates?
        \item Compute the following integral using cylindrical coordinates
        \[
        \int_{\curvegamma} \vecfieldB \cdot d\curvegamma.
        \]
        \item Using cylindrical coordinates, compare your result with
        \[
        \iint_\Sigma \left(\grad \times \vecfieldB\right) \cdot \unitvec d\Sigma,
        \]
        where $\Sigma$ is the unit disk.
\end{enumerate}
\end{problem}

\vspace*{.5cm}

\begin{problem}
    Convert the following integrals to integrals in cylindrical coordinates. Also, describe the region in which you are integrating over. Do not evaluate the integrals.
    \begin{enumerate}[(a)]
        \item $\displaystyle{\int_{-1}^{1} \int_{-1}^{1} \int_{-\sqrt{1-y^2}}^{\sqrt{1-y^2}} xyz dxdydz}$.
        \item $\displaystyle{\int_0^1 \int_{-z}^z \int_{-\sqrt{1-y^2}}^{\sqrt{1-y^2}} x^2+y^2+z^2 dxdydz}$.
    \end{enumerate}
\end{problem}

\vspace*{.5cm}

\begin{problem}
    Note that the Laplacian $\Delta$ in cylindrical coordinates is given by
    \[
        \Delta f(\rho,\theta,z) = \frac{1}{\rho} \frac{\partial}{\partial \rho} \left(\rho \frac{\partial f}{\partial \rho}\right)+\frac{1}{\rho^2}\frac{\partial^2 f}{\partial \theta^2} + \frac{\partial^2 f}{\partial z^2}.
    \]
    Compute the Laplacian of
    \[
        f(\rho,\theta,z) = \sqrt{\rho^2+z^2} z \cos(\theta).
    \]
\end{problem}

\begin{problem}
	In spherical coordinates (either implicitly or explicitly), parameterize the following objects.
	\begin{enumerate}[(a)]
		\item A solid sphere with radius 3.
		\item The surface of an infinite cone with a vertex angle of $\pi/4$.
		\item A latitudinal curve on the unit sphere at the latitude of 30$^\circ$ above the equator.
		\item A solid unit sphere with a cylinder of radius 1/2 removed from the core.
	\end{enumerate}
\end{problem}

\vspace*{.5cm}

\begin{problem}
    Let us see some of the benefit of using spherical coordinates. 
    \begin{enumerate}[(a)]
        \item Using the fact that 
        \[
        \rhat = \frac{x}{\sqrt{x^2+y^2+z^2}}\xhat + \frac{y}{\sqrt{x^2+y^2+z^2}}\yhat + \frac{z}{\sqrt{x^2+y^2+z^2}}\zhat,
        \]
        convert the vector field 
        \[
        \vecfieldE(x,y,z) = \begin{pmatrix} \frac{x}{(x^2+y^2+z^2)^{3/2}} \\ \frac{y}{(x^2+y^2+z^2)^{3/2}} \\ \frac{z}{(x^2+y^2+z^2)^{3/2}} \end{pmatrix},
        \] into spherical coordinates (i.e., only a function of $r$, $\theta$, $\phi$, and $\rhat$, $\thetahat$, and $\phihat$).
        \item Parameterize the surface of a sphere of radius $R$ (which we'll call $\Sigma$) as well as the outward normal vector $\unitvec$ and  in spherical coordinates.
        \item Compute the following integral using spherical coordinates that we have found:
        \[
        \iint_\Sigma \vecfieldE \cdot \unitvec d\Sigma,
        \]
        where $d\Sigma$ will be the area form in spherical coordinates.
    \end{enumerate}
\end{problem}

\vspace*{.5cm}

\begin{problem}
    Note that the Laplacian $\Delta$ in spherical coordinates is given by
    \[
        \Delta f(r,\theta,\phi) = \frac{1}{r^2} \frac{\partial}{\partial r} \left(r^2 \frac{\partial f}{\partial r}\right)+\frac{1}{r^2 \sin^2 \phi} \frac{\partial^2 f}{\partial \theta^2} + \frac{1}{r^2 \sin\phi}\frac{\partial}{\partial \phi} \left(\sin \phi \frac{\partial f}{\partial \phi}\right).
    \]
    Compute the Laplacian of
    \[
       f(r,\theta,\phi) = r^2 \cos(\theta)\cos(\phi).
    \]
\end{problem}


\end{document}