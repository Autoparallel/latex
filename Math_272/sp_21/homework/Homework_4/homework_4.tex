%%%%%%%%%%%%%%%%%%%%%%%%%%%%%%%%%%%%%%%%%%%%%%%%%%%%%%%%%%%%%%%%%%%%%%%%%%%%%%%%%%%%
% Document data
%%%%%%%%%%%%%%%%%%%%%%%%%%%%%%%%%%%%%%%%%%%%%%%%%%%%%%%%%%%%%%%%%%%%%%%%%%%%%%%%%%%%
\documentclass[12pt]{article} %report allows for chapters
%%%%%%%%%%%%%%%%%%%%%%%%%%%%%%%%%%%%%%%%%%%%%%%%%%%%%%%%%%%%%%%%%%%%%%%%%%%%%%%%%%%%
\usepackage{preamble}

\begin{document}

\begin{center}
   \textsc{\large MATH 272, Homework 4}\\
   \textsc{Due March$^\textrm{nd}$}
\end{center}
\vspace{.5cm}

\begin{problem}
Plot the following curves and print pictures of each using GeoGebra.  
\begin{enumerate}[(a)]
	\item (Helix) $\curvegamma_1(t) = \begin{pmatrix} 3\cos(t) \\ 3\sin(t) \\ t\end{pmatrix}$, from $t=0$ to $t=2\pi$. Where might this show up? If you think about the Earth moving through space and Moon orbiting Earth, then the Moon follows a (locally) helical path.
	
	\item (Falling Ball) $\curvegamma_2(t) = \begin{pmatrix} t\\  0.5t \\ 9-t^2 \end{pmatrix}$ from $t=0$ to $t=3$.
	
	\item (Perturbed Orbiter) $\curvegamma_3(t) = \begin{pmatrix} 3\cos(t) \\ 3 \sin(t) \\ \sin(10t)\end{pmatrix}$ from $t=0$ to $t=2\pi$.
	
	\item Create your own curve.
\end{enumerate}
\end{problem}

\begin{problem}
The length of a curve is an important notion. In fact, the length of a curve is often related to the energy of some configuration. We can compute the length of a curve over the time $t=t_0$ to $t=t_1$ by integrating the \emph{speed} of the curve over that time.  That is,
\[
l(\curvegamma) = \int_{t_0}^{t_1} \left|\tangentgamma(t)\right| dt.
\]
We can compute the \emph{energy} of a curve by taking
\[
E(\curvegamma) = \int_{t_0}^{t_1} \frac{1}{2} \left| \tangentgamma(t)\right|^2dt.
\]
\begin{enumerate}[(a)]
	\item Find the length of the Helix from Problem 1 (a).
	\item Find the energy of the Perturbed Orbiter in Problem 1 (c).  Roughly, this corresponds to the potential energy a rubber band stretched this way would have.
\end{enumerate}
\end{problem}

\begin{problem}
Given a scalar field of two variables $F(x,y)$, we can create an object called the \emph{graph} of $F(x,y)$ by plotting the set of points $(x,y,F(x,y))$. In fact, you have done this many times in your life. For example, you have consistently plotted the graph of a function $f(x)$ by plotting $(x,f(x))$ in the plane!  

Using GeoGebra, plot the graph of the following functions. Print these off and include them.  Also, describe the what the graph of the function does as we move along the $x$-axis, the $y$-axis, and along the line $y=x$. For each, use the range $-3\leq x \leq 3$ and $-3\leq y \leq 3$.
\begin{enumerate}[(a)]
	\item $F(x,y) = \frac{4xy}{1+x^2+y^2}$.
	\item $G(x,y) = \sin(xy)$.
	\item $H(x,y) = \frac{-x^2-y^2}{5}$.
\end{enumerate}
\end{problem}

\begin{problem}
For this problem, let us consider a family of scalar fields of varying dimensionality.  We will seek out an understanding of the level sets and how to relate these to the derivatives of a scalar field. For each part, compute the set of points such that $f(\vecx)=\frac{1}{2}$, $f(\vecx)=1$, $f(\vecx)=2$, and $f(\vecx)=3$ and plot these sets. 

Compute as well the vector of partial derivatives (referred to as the \emph{gradient vector}) which we write as
\[
\nablavec f(\vecx) = \begin{pmatrix} \frac{\partial f}{\partial x_1} \\ \frac{\partial f}{\partial x_2} \\ \vdots \\ \frac{\partial f}{\partial x_n}\end{pmatrix}.
\]
in $\R^n$. Draw the gradient vector on your plots at a three different points for each part.
\begin{enumerate}[(a)]
	\item Consider the 1-dimensional scalar field 
	\[
	f(x) = \frac{1}{|x|}=\frac{1}{\sqrt{x^2}}.
	\]
	Here each level set will be made up of distinct points.
	\item Consider the 2-dimensional scalar field
	\[
	f(x,y) = \frac{1}{|\vec{x}|} = \frac{1}{\sqrt{x^2+y^2}}.
	\]
	Here each level set will be a curve.
	\item Consider the 3-dimensional scalar field
		\[
		f(x,y,z) = \frac{1}{|\vec{x}|} = \frac{1}{\sqrt{x^2+y^2+z^2}}.
		\]
		Here each level set will be a surface.
\end{enumerate}
\end{problem}

\begin{problem}
Consider the two dimensional scalar field $T(x,y)=x+y$ that describes the temperature on the square plate $\Omega$ given by the set $0\leq x,y \leq 1$.  Compare the two answers you get!
\begin{enumerate}[(a)]
	\item Compute the integral
	\[
	\int_\Omega T(x,y)d\Omega.
	\]
	\item Let $\curvegamma$ be the curve that traverses the boundary of the square plate in the counterclockwise direction.  Compute
	\[
	\int_{\curvegamma} T(\curvegamma)d\curvegamma. 
	\]
\end{enumerate}
\end{problem}

\end{document}