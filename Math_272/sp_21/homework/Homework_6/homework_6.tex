%%%%%%%%%%%%%%%%%%%%%%%%%%%%%%%%%%%%%%%%%%%%%%%%%%%%%%%%%%%%%%%%%%%%%%%%%%%%%%%%%%%%
% Document data
%%%%%%%%%%%%%%%%%%%%%%%%%%%%%%%%%%%%%%%%%%%%%%%%%%%%%%%%%%%%%%%%%%%%%%%%%%%%%%%%%%%%
\documentclass[12pt]{article} %report allows for chapters
%%%%%%%%%%%%%%%%%%%%%%%%%%%%%%%%%%%%%%%%%%%%%%%%%%%%%%%%%%%%%%%%%%%%%%%%%%%%%%%%%%%%
\usepackage{preamble}

\begin{document}

\begin{center}
   \textsc{\large MATH 272, Homework 6}\\
   \textsc{Due March 22$^\textrm{nd}$}
\end{center}
\vspace{.5cm}

\begin{problem}
    Consider the 1-dimensional wave equation given by
    \[
    \left( - \frac{\partial^2}{\partial x^2} +\frac{1}{c^2} \frac{\partial^2}{\partial t^2} \right) u(x,t) =0,
    \]
    with the domain $\Omega$ as the unit interval on the $x$-axis.  We shall fix the string at each endpoint which requires $u(0,t)=0$ and $u(1,t)=0$ for all $t$.  Take the initial condition as well to be a plucked string so that $u(x,0)=\sin(\pi x)$ and $\frac{\partial}{\partial t}u(x,0)=0$. 
    \begin{enumerate}[(a)]
        \item Use the separation of variables ansatz $u(x,t)=X(x)T(t)$ to get a new separation constant. This will give two ODEs: one will be in terms of $X(x)$ and the other will be in terms of $T(t)$.
        \item Use the boundary conditions and solve the ODE that is in terms of $X(x)$ which will simultaneously find the allowed values for the separation constant.
        \item Using these allowed values for the separation constant, find the solution for $T(t)$.
        \item Find the particular solution for $u(x,t)$ by matching the initial condition.
        \item Plot your solution for $x\in [0,1]$ and $t\in [0,\infty)$ (i.e., just plot up to a large value of $t$). In this case, compare your plots for $c=1/2$ and $c=1$.
    \end{enumerate}
\end{problem}

\end{document}