%%%%%%%%%%%%%%%%%%%%%%%%%%%%%%%%%%%%%%%%%%%%%%%%%%%%%%%%%%%%%%%%%%%%%%%%%%%%%%%%%%%%
% Document data
%%%%%%%%%%%%%%%%%%%%%%%%%%%%%%%%%%%%%%%%%%%%%%%%%%%%%%%%%%%%%%%%%%%%%%%%%%%%%%%%%%%%
\documentclass[12pt]{article} %report allows for chapters
%%%%%%%%%%%%%%%%%%%%%%%%%%%%%%%%%%%%%%%%%%%%%%%%%%%%%%%%%%%%%%%%%%%%%%%%%%%%%%%%%%%%
\usepackage{preamble}

\begin{document}

\begin{center}
   \textsc{\large MATH 272, Homework 6, \emph{Solutions}}\\
\end{center}
\vspace{.5cm}

\begin{problem}
    Consider the 1-dimensional wave equation given by
    \[
    \left( - \frac{\partial^2}{\partial x^2} +\frac{1}{c^2} \frac{\partial^2}{\partial t^2} \right) u(x,t) =0,
    \]
    with the domain $\Omega$ as the unit interval on the $x$-axis.  We shall fix the string at each endpoint which requires $u(0,t)=0$ and $u(1,t)=0$ for all $t$.  Take the initial condition as well to be a plucked string so that $u(x,0)=\sin(\pi x)$ and $\frac{\partial}{\partial t}u(x,0)=0$. 
    \begin{enumerate}[(a)]
        \item Use the separation of variables ansatz $u(x,t)=X(x)T(t)$ to get a new separation constant. This will give two ODEs: one will be in terms of $X(x)$ and the other will be in terms of $T(t)$.
        \item Use the boundary conditions and solve the ODE that is in terms of $X(x)$ which will simultaneously find the allowed values for the separation constant.
        \item Using these allowed values for the separation constant, find the solution for $T(t)$.
        \item Find the particular solution for $u(x,t)$ by matching the initial condition.
        \item Plot your solution for $x\in [0,1]$ and $t\in [0,\infty)$ (i.e., just plot up to a large value of $t$). In this case, compare your plots for $c=1/2$ and $c=1$.
    \end{enumerate}
\end{problem}
\begin{solution}~
\begin{enumerate}[(a)]
    \item If we take $u(x,t)=X(x)T(t)$, then plugging this into the PDE yields
    \[
    -X''T+\frac{1}{c^2} XT'' = 0.
    \]
    We can then isolate each variable on one side of the equal sign to get
    \[
    \frac{X''}{X} = \frac{1}{c^2}\frac{T''}{T}.
    \]
    Note that the left hand side depends only on $x$, whereas the right hand side depends solely on $t$. Thus, it must be that both sides equal the same constant $\lambda$. This gives us two ODEs
    \[
    X'' -\lambda X = 0 \qquad \textrm{and} \qquad T'' -c^2 \lambda T = 0.
    \]
    
    \item The boundary conditions are a spatial condition and thus we must satisfy them independent of time.  Hence, we must have that $X(0)=0$ and $X(1)=0$ so that $u(0,t)=0$ and $u(1,t)=0$ respectively.  This means that $\lambda<0$ so that we get
    \[
    X(x) = a\cos(\sqrt{-\lambda}x) +b \sin(\sqrt{-\lambda} x),
    \]
    since if $\lambda \geq 0$ we will only get constant solutions which are trivial and can't match the initial conditions or exponential solutions which can't match the boundary conditions.  Applying the boundary conditions yields that $A=0$ and $\sqrt{-\lambda}=n\pi$ for any positive integer $n$. Thus we have
    \[
    X_n(x) = a_n \sin(n\pi x), 
    \]
    is (nontrivial) a solution to this ODE for every positive integer $n$.
    
    \item Using this result for $\lambda$, we have that
    \[
    T_n(t) = b_n \sin(cn\pi t)+c_n \cos(cn\pi t),
    \]
    for every positive integer $n$.  
    
    \item Combining to get $u_n(x,t) =X_n(x)T_n(t)$, we can write
    \[
    u_n(x,t) = \left(a_n \sin(cn\pi t)+b_n \cos(cn\pi t)\right) \sin(n\pi x),
    \]
    is a general solution for each positive integer $n$.  Note that we have just renamed constants here to write this more simply.  Now, in general, a sum of these solutions is also a solution, but we have
    \[
    u(x,0)=\sin(\pi x),
    \]
    which means that $n=1$, $a_1=0$, and $b_1=1$.  All other constants $a_j$ and $b_j$ are all zero.  One can then check that our solution
    \[
    u(x,t) = \cos(c \pi t) \sin(\pi x),
    \]
    satisfies $\frac{\partial}{\partial t} u(x,0) = 0$.  
    
    \item Here are the plots for these functions. Note that in this case, we plotted the solution as a surface with one of the axes representing time.  This still just shows how the 1-dimensional elastic evolves over time. We see that when $c$ is increased, the elastic vibrates more quickly.
    
                \begin{figure}[H]
                	\centering
                	\def\svgwidth{0.6\columnwidth}
                	\input{wave_solution_c=1_2.pdf_tex}
                    \caption{The graph of the solution $u(x,t)$ for $c=1/2$.}
                \end{figure}
                \begin{figure}[H]
                                	\centering
                                	\def\svgwidth{0.6\columnwidth}
                                	\input{wave_solution_c=1.pdf_tex}
                                    \caption{The graph of the solution $u(x,t)$ for $c=1$.}
                                \end{figure}
\end{enumerate}
\end{solution}

\end{document}