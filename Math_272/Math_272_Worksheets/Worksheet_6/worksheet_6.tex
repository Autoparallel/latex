%%%%%%%%%%%%%%%%%%%%%%%%%%%%%%%%%%%%%%%%%%%%%%%%%%%%%%%%%%%%%%%%%%%%%%%%%%%%%%%%%%%%
% Document data
%%%%%%%%%%%%%%%%%%%%%%%%%%%%%%%%%%%%%%%%%%%%%%%%%%%%%%%%%%%%%%%%%%%%%%%%%%%%%%%%%%%%
\documentclass[12pt]{article} %report allows for chapters
%%%%%%%%%%%%%%%%%%%%%%%%%%%%%%%%%%%%%%%%%%%%%%%%%%%%%%%%%%%%%%%%%%%%%%%%%%%%%%%%%%%%
\usepackage{preamble}

\begin{document}

\begin{center}
   \textsc{\large MATH 271, Worksheet 6}\\
   \textsc{Vectors and Vector Spaces}
\end{center}
\vspace{.5cm}

\begin{problem}
Compare and contrast the structure of the complex numbers $\C$ with the vector space $\R^2$.  Note any differences and similarities.
\end{problem}

\begin{problem}
Let $\vecu,\vecv\in \R^2$ be given by
\[
\vecu = 2\xhat + 3 \yhat \qquad \textrm{and} \qquad \vecv = -\xhat +\yhat.
\]
\begin{enumerate}[(a)]
    \item Draw $\vecu$, $\vecv$, and $\vecu + \vecv$ in the plane.
    \item Compute $\|\vecu\|$ and $\|\vecv\|$.
    \item Compute $\vecu \cdot \vecv$.
    \item Find a vector orthogonal to $\vecu$.
\end{enumerate}
\end{problem}

\begin{problem}
Let $\vecu,\vecv\in \R^3$ be given by
\[
\vecu = \xhat - \yhat + \zhat \qquad \textrm{and} \qquad \vecv = -\xhat + \yhat - \zhat.
\]
\begin{enumerate}[(a)]
    \item Are $\vecu$ and $\vecv$ orthogonal?
    \item Normalize $\vecu$ and $\vecv$ to get $\uhat$ and $\vhat$. 
    \item Compute the projection of $\vecv$ onto the direction defined by $\vecu$.
\end{enumerate}
\end{problem}

\begin{problem}
Let $\vecu,\vecv \in \R^3$ be given by
\[
\vecu = -3\xhat -2\yhat + \zhat \qquad \textrm{and} \qquad \vecv = \xhat -2\yhat +\zhat.
\]
\begin{enumerate}[(a)]
    \item Compute the angle between $\vecu$ and $\vecv$.
    \item Without computing the cross product, compute the area of the parallelogram generated by $\vecu$ and $\vecv$. \emph{Hint: you know the angle between the vectors, use this fact.}
    \item Without computing the cross product, what component of the product $\vecu\times\xhat$ must be zero? 
    \item Compute $\vecu\times \vecv$.
    \item Give a geometrical interpretation of the cross product $\vecu\times \vecv$. Explain why $\vecu\times \vecv = -\vecv \times \vecu$.
\end{enumerate}
\end{problem}

\begin{problem}
Recall that the states found in the solution to the free particle in a 1-dimensional box of length $L$ were $\psi_n = \sqrt{\frac{2}{L}} \sin \left( \frac{n\pi x}{L}\right)$. Let $S$ denote the set of all solutions to the free particle in a 1-dimensional box boundary value problem. Show that a superposition of states (with coefficients in $\C$) is also a solution. That is, if we let $\Psi(x) = \alpha_{j}(x) \psi_j + \alpha_k \psi_k(x)$, then $\Psi(x)$ is also a solution to the boundary value problem
\[
-\frac{\hbar^2}{2m}\frac{d^2 \Psi}{dx^2}=E\Psi
\]
with boundary values $\Psi(0)=0$ and $\Psi(L)=0$.

\noindent \emph{Note that this shows that the set $S$ is a vector space over the complex numbers $\C$.}
\end{problem}




\end{document}
