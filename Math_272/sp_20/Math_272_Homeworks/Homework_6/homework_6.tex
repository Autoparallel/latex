%%%%%%%%%%%%%%%%%%%%%%%%%%%%%%%%%%%%%%%%%%%%%%%%%%%%%%%%%%%%%%%%%%%%%%%%%%%%%%%%%%%%
% Document data
%%%%%%%%%%%%%%%%%%%%%%%%%%%%%%%%%%%%%%%%%%%%%%%%%%%%%%%%%%%%%%%%%%%%%%%%%%%%%%%%%%%%
\documentclass[12pt]{article} %report allows for chapters
%%%%%%%%%%%%%%%%%%%%%%%%%%%%%%%%%%%%%%%%%%%%%%%%%%%%%%%%%%%%%%%%%%%%%%%%%%%%%%%%%%%%
\usepackage{preamble}

\begin{document}

\begin{center}
   \textsc{\large MATH 272, Homework 6}\\
   \textsc{Due March \textcolor{red}{24}$^\textrm{th}$}
\end{center}
\vspace{.5cm}


\begin{problem} 
Let 
\[
\curvegamma(t) = \begin{pmatrix} \cos(t) \\ \sin(t) \\ t \end{pmatrix}, \quad f(x,y,z) = x^2 + y^2 - 2z^2, \quad \vecfieldV(x,y,z) = \begin{pmatrix} x-y \\ y+x \\ z \end{pmatrix}.
\]
Compute derivatives of the following composite functions.
\begin{enumerate}[(a)]
	\item $f(\curvegamma(t))$.
	\item $\vecfieldV(\curvegamma(t))$.
	\item $f(\vecfieldV(x,y,z))$.
\end{enumerate}
\end{problem}

\begin{problem}
Show that for any smooth (more than twice differentiable) fields $f(x,y,z)$ and $\vecfieldV(x,y,z)$ that
\begin{enumerate}[(a)]
	\item $\grad \times \left(\grad f\right)=\boldsymbol{\vec{0}}$;
	\item $\grad \cdot \left(\grad \times \vecfieldV\right)=0$.
\end{enumerate}
\end{problem}

\begin{problem}
	Let 
	\[
	\vecfieldU(x,y,z) = \begin{pmatrix} -y \\ x \\ 0 \end{pmatrix} \qquad \textrm{and} \qquad \vecfieldV(x,y,z) = \begin{pmatrix} 2x \\ 2y \\ 2z \end{pmatrix},
	\] 
	be vector fields.  
	\begin{enumerate}[(a)]
		\item Explain why there exists no potential function $\phi(x,y,z)$ for the vector field $\vecfieldU$.
		\item Explain why there does exist a potential function $\phi(x,y,z)$ for the field $\vecfieldV$.
		\item Compute the potential function for $\vecfieldV$.
	\end{enumerate}
\end{problem}


\begin{problem} 
Parameterize the following either implicitly or explicitly. In Cartesian coordinates, find the parameterization of the normal vector as well.
\begin{enumerate}[(a)]
	\item The plane perpendicular to the vector $\vecv = \xhat + \yhat + \zhat$ passing through the point $(1,1,1)$.
	\item The upper half of the unit circle in $\R^2$.
	\item The surface of the unit sphere in $\R^3$.
\end{enumerate}
\end{problem}

\begin{problem}
	In cylindrical coordinates (either implicitly or explicitly), parameterize the following objects.
	\begin{enumerate}[(a)]
		\item A cylinder with radius 3 and height 5 along with end-caps.
		\item An infinite cone with a vertex angle of $\pi/4$.
		\item A helical curve with constant radius 1 and pitch 1.
		\item A hyperboloid of one sheet.
	\end{enumerate}
\end{problem}

\end{document}