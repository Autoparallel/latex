\documentclass[12pt]{amsbook}
\usepackage{preamble}


\begin{document}
\pagenumbering{gobble}       % This kills the page numbering

\begin{center}
   \textsc{\large MATH 272, Exam 1}\\
   \textsc{Take Home Portion}\\
   \textsc{Due February 19$^\textrm{th}$ at the start of class}
\end{center}
\vspace{1cm}

\noindent\textbf{Name} \; \underline{\hspace{8cm}}

\vspace{1cm}

\noindent\textbf{Instructions} \; You are allowed a textbook, homework, notes, worksheets, material on our Canvas page, but no other online resources (including calculators or WolframAlpha). for this portion of the exam.  \textbf{Do not discuss any problem any other person.} All of your solutions should be easily identifiable and supporting work must be shown.  Ambiguous or illegible answers will not be counted as correct. \textbf{Print out this sheet and staple your solutions to it. Use a new page for each problem.}


\vspace{1cm}

\begin{center}\textbf{Problem 1} \; \underline{\hspace{1cm}}/15 \qquad \qquad \textbf{Problem 2} \; \underline{\hspace{1cm}}/10\end{center}

\vspace{1cm}

\hrule

\vspace*{1cm}
\noindent\emph{Note, these problems span two pages.}\\

\newpage
\begin{problem}
Sometimes breaking down operators into smaller components can help one better understand a problem. Given this, let's consider the Hamiltonian operator $\hat{H}$ for the Quantum Harmonic Oscillator (QHO) given by
\[
\hat{H} = \hat{T}+V(x) = \frac{\hat{p}^2}{2m} + \frac{1}{2}m\omega^2 x^2,  \qquad \textrm{where} \qquad \hat{p} = -i\hbar \frac{d}{dx}
\]
where $m$ and $\omega$ are real constants.  Then, the following are states of the system
\[
\psi_0(x) = \left(\frac{m\omega}{\pi \hbar}\right)^{1/4} e^{-\frac{m\omega x^2}{2\hbar}} \qquad \textrm{and} \qquad \psi_1(x) = x \sqrt{\frac{2m\omega}{\hbar}}\left(\frac{m\omega}{\pi \hbar}\right)^{1/4} e^{-\frac{m\omega x^2}{2\hbar}},
\]
where $x \in \R$. 
\begin{enumerate}[(a)]
	\item (\textbf{3 pts.}) One can generate $\psi_1(x)$ from $\psi_0(x)$ by using the \emph{raising operator}
	\[
	\hat{a} = \sqrt{\frac{m\omega }{2\hbar}}\left(x+\frac{i}{m\omega} \hat{p}\right).
	\]  
	Show that $\hat{a}\psi_0 = \psi_1$.
	
	\item (\textbf{2 pts.}) Operators, just like matrices, do not always commute! So, we often want to see how ``far from commuting" two operators are. To this end, let $\Psi(x)$ be an arbitrary function, then the \emph{commutator} $[x,\hat{p}]$ is defined by
	\[
	[x,\hat{p}] \Psi(x) \coloneqq x \left(\hat{p}\left( \Psi(x)\right)\right) - \hat{p}\left(x\left(\Psi(x)\right)\right).
	\]
	Show that $[x,\hat{p}]=i\hbar$. \textit{Note that we also have $[\hat{p},x]=-[x,\hat{p}]$.}
	
	\item (\textbf{2 pts.}) We can define the \emph{lowering operator} $\hat{a}^\dagger$ as the adjoint of $\hat{a}$ which is given by
	\[
	\hat{a}^\dagger = \sqrt{\frac{m\omega }{2\hbar}}\left(x-\frac{i}{m\omega} \hat{p} \right).
	\]
	This allows us to define the \emph{number operator} $\hat{N}=\hat{a}^\dagger \hat{a}$.  Compute $\hat{N}$. \textit{Hint: The commutator $[x,\hat{p}]$ appears in this equation.}
	
	\item (\textbf{2 pts.}) From this, we can define the Hamiltonian operator by $\hat{H}=\hbar \omega \left(N + \frac{1}{2}\right)$. Show that this is true.
	
	\item (\textbf{3 pts.}) Argue that $\hat{H}$ is Hermitian.  \textit{Hint: You can use results from our notes and homework.}
	
	\item (\textbf{3 pts.}) Show that the ground state $\psi_0$ is an eigenfunction of $\hat{H}$ with eigenvalue $\frac{1}{2} \hbar \omega$. (\textit{This means that the lowest energy state of the system has positive (nonzero) energy!})
	
%	\item (\textbf{2 pts.}) Using our Fourier transform table, compute the Fourier transform of the ground state to get
%	\[
%	\fourier{\psi_0(x)} = \phi_0(k).
%	\]
%	\item (\textbf{2 pts.}) Let $p=-2\pi \hbar k$ (this is \underline{not} $\hat{p}$).  Solve for $k$ in terms of $p$ and substitute this in to your function $\phi_0(k)$ to determine the function $\phi_0(p)$.  This new function $\phi_0(p)$ is the ground state written in the \emph{momentum space}.  That is to say that the Fourier transform converts a function in \emph{position space} to a function in momentum space!
	\item (\textbf{Bonus 2pts.}) The fact that $\hat{H}$ is Hermitian implies that $\psi_0$ and $\psi_1$ must be orthogonal with respect to the inner product
	\[
	\innprod{\Psi}{\Phi} = \int_{-\infty}^\infty \Psi(x)\Phi^*(x)dx.
	\]
	Can you argue that this is true for the given $\psi_0$ and $\psi_1$?
	\end{enumerate}
\end{problem}

\newpage
\begin{problem}
What is the Dirac delta? Let us define the function 
\[
S_t(x) = \begin{cases} 0 & x< -t \\ \frac{1}{2t} & -t\leq x \leq t \\ 0 & x>t \end{cases},
\]
where $t$ is some positive real number.
\begin{enumerate}[(a)]
	\item (\textbf{2 pts.}) Let us recall the Dirac delta $\delta(x)$. Given any function $f(x)$, what is
	\[
	\int_{-\infty}^\infty \delta(x) f(x)dx?
	\]
	\item (\textbf{3 pts.}) Show that for any fixed $t>0$ that 
	\[
	\int_{-\infty}^{\infty} S_t(x)dx = 1.
	\]
	\item (\textbf{3 pts.}) Now let $F(x)$ be the antiderivative of $f(x)$.  Evaluate the integral
	\[
	\int_{-\infty}^\infty S_t(x)f(x)dx.
	\]
	Note, your answer should be in terms of the antiderivative $F(x)$ and $t$.
	\item (\textbf{2 pts.}) Recall that by definition, we have that $F'(x)=f(x)$.  In other words,
	\[
	\lim_{\Delta x \to 0} \frac{F(x+\Delta x)-F(x-\Delta x)}{2\Delta x} = F'(x) = f(x).
	\]
	With your answer in (b), take the limit as $t\to 0$ to show that you recover the answer you have in (a).
\end{enumerate}
Visit this Desmos URL to have an interactive look at the function $S_t(x)$ for various values of $t$.  \url{https://www.desmos.com/calculator/eh5jmqeky1}
\end{problem}









\end{document}  