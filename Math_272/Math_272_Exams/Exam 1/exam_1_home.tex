\documentclass[12pt]{amsbook}
\usepackage{preamble}


\begin{document}
\pagenumbering{gobble}       % This kills the page numbering

\begin{center}
   \textsc{\large MATH 272, Exam 1}\\
   \textsc{Take Home Portion}\\
   \textsc{Due February 19$^\textrm{th}$ at the start of class}
\end{center}
\vspace{1cm}

\noindent\textbf{Name} \; \underline{\hspace{8cm}}

\vspace{1cm}

\noindent\textbf{Instructions} \; You are allowed a textbook, homework, notes, worksheets, material on our Canvas page, but no other online resources (including calculators or WolframAlpha). for this portion of the exam.  \textbf{Do not discuss any problem any other person.} All of your solutions should be easily identifiable and supporting work must be shown.  Ambiguous or illegible answers will not be counted as correct. \textbf{Print out this sheet and staple your solutions to it. Use a new page for each problem.}


\vspace{1cm}

\begin{center}\textbf{Problem 1} \; \underline{\hspace{1cm}}/10 \qquad \qquad \textbf{Problem 2} \; \underline{\hspace{1cm}}/15\end{center}

\vspace{1cm}

\hrule

\vspace*{1cm}
\noindent\emph{Note, these problems span two pages.}\\

In the heart of quantum mechanics lies probability theory.  Since we can only compute the likelihood of observing something, we have to start thinking in this framework. 

\newpage
\begin{problem}
Consider a free particle constrained to the infinite line $\R$.  Heisenberg's uncertainty principle states that one can never simultaneously know the position and momentum of a particle.  

Suppose that we measure the \emph{exact} position of a particle at time $t=0$ to be at $x_0$. Then the wavefunction for this particle is given by
\[
\Psi(x) = \delta(x-x_0).  
\]
Thus, since we know the position exactly, we expect to know \emph{nothing} about the particle's momentum. 
\begin{enumerate}[(a)]
	\item
\end{enumerate}
\end{problem}

\newpage
\begin{problem}
Probability is described via probability distribution functions (PDFs) $p(x)$. Intuitively, the value of $p(x)$ at the point $x$ describes the weighting of that point $x$. We require that
\begin{itemize}
	\item $p(x)\geq 0$ for all $x\in \R$. (\emph{There is no negative probability.)}
	\item $\displaystyle{\int_{-\infty}^\infty p(x)dx=1}$. (\emph{The total probability is 1.)}
\end{itemize}
We can then define the probability of measuring an outcome in the region $[a,b]$ by
\[
\mathcal{P}_{[a,b]}[p(x)] = \int_a^b p(x)dx.
\]
\begin{enumerate}[(a)]
	\item Consider the Gaussian blah blah limit to delta funciton
	\item
\end{enumerate}
\end{problem}








\end{document}  