\section{Fourier Series}

With our newfound toolbox, we will investigate solving types of differential equations with an underlying periodic structure.  Specifically, we will consider now linear differential equations on the region $\Omega = [0,L]$ with periodic boundary conditions.  That is, we will be given some differential operator $\linop$, a function $g$, and we will be asked to solve for an $f$ such that
\begin{equation}
\label{eq:operator_diffeq}
\linop f = g,
\end{equation}
with $f$ subject to the boundary conditions $f(0)=f(L)$.  We refer to these boundary conditions as \boldgreen{periodic boundary conditions}.  Another way to view this problem is as defining the function $f$ to be continuous on a circle domain.  This point of view shows its head in the solution for the hydrogen atom.

For the sake of example, we will take the operator $\linop = -\frac{d^2}{dx^2}$ which is referred to as the 1-dimensional \boldgreen{Laplace operator} or \boldgreen{Laplacian}.  To follow the methodology outlined in the previous chapter, before we approach \ref{eq:operator_diffeq}, we must find the spectrum of $\linop$.  Hence, we must solve the eigenvalue equation
\[
\linop f = \omega^2 f,
\]
where $\omega^2$ is some constant that was chosen to simplify the later notation.  Thus, we arrive at the differential equation
\begin{equation}
\label{eq:harmonic_osc_operator}
-\frac{d^2}{dx^2}f(x) = \omega^2 f(x) \qquad\textrm{with~} f(0)=f(L).
\end{equation}
We have in fact solved this equation before! This is the harmonic oscillator equation with angular frequency $\omega$.  However, there is a difference.  In the previous case, we had a distinct and predetermined value for $\omega$ that was given by the system we were investigating (e.g., the stiffness of a spring).  Now, we allow $\omega$ to be a parameter for the problem as well.  

When we solved the matrix eigenvalue problem
\[
[A]\evec = \lambda \evec
\]
we wanted to take the determinant of $[A]-\lambda [I]$ to find what the eigenvalues were and then use this information to determine the eigenvectors.  Differential operators work a bit differently. Instead, we would find a general solution to the differential equation and continue on from there to refine our answer.

In our equation \ref{eq:harmonic_osc_operator}, we know the general solution is
\[
f(x) = C_1 \sin(\omega x)+C_2 \cos(\omega x).  
\]

\begin{exercise}
	Determine the above general solution using either the characteristic polynomial (which in reality comes from a determinant) or by using a power series.
\end{exercise}

With our general solution in hand, we can determine what eigenvalues for $\omega$ are reasonable.  Specifically, we use our boundary conditions and we force
\[
f(0)=C_1 \sin(\omega \cdot 0) + C_2 \cos(\omega \cdot 0) = C_2
\]
and
\[
f(L) = C_1 \sin(\omega L) + C_2 \cos(\omega L).
\]
In order for the $\sin$ term above to disappear, we must have
\[
\omega L = n \pi,
\]
for any integer value $n$. But we also require that $\cos(\omega L)=1$ which means that we restrict further to
\[
\omega L = 2n\pi.  
\]
We can then solve for $\omega$ to find
\[
\boxed{\omega = \frac{2n\pi}{L}.}
\]

\begin{exercise}
	Review the analysis above for determining $\omega$.
\end{exercise}

Then, what we have found is the spectrum is discrete, and there is an eigenvalue 
\[
\omega^2 = \frac{4 n^2 \pi^2}{L^2}
\]
for each value of $n$.  But, note that since $n$ is an integer, $n^2$ is always positive, and so we require $n\geq 0$ since we wish to remove the redundant results.  We can then list off the eigenfunctions
\[
\boxed{1,~\sin\left(\frac{2 \pi x}{L}\right),~ \cos\left(\frac{2 \pi x}{L}\right), ~\sin\left(\frac{4 \pi x}{L}\right),~ \cos\left(\frac{4 \pi x}{L}\right), ~ \dots.}
\]

\subsection{Solving Equations}

The equation \ref{eq:operator_diffeq} before is linear, and hence any sum of solutions is also a solution.  Thus, we can write a solution $f(x)$ as a series by putting
\[
f(x) = \sum_{n=1}^\infty a_n \sin\left(\frac{2n\pi x}{L}\right) + \sum_{n=0}^\infty b_n \cos\left(\frac{2n\pi x}{L} \right).
\]
However, we can then consider if we can take any $g$ and write this as a series as well. 

Let us define the following inner product
\[
\innprod{F}{G} = \frac{1}{L} \int_0^L F(x)G(x)dx,
\]
which we will refer to as the \boldgreen{Fourier inner product}.  Notice, this inner product is very similar to the inner products we have used for functions previously (we just divide by the length of the interval).

Though we won't prove it here, the eigenfunctions found above serve as a basis for the solutions for the equation \ref{eq:operator_diffeq}.  To simplify future work, we will normalize our eigenfunctions. Thus, we must solve
\[
\innprod{c1}{c1} = 1, \quad \innprod{c_o \sin\left(\frac{2n \pi x}{L}\right)}{c_o \sin\left(\frac{2n \pi x}{L}\right)}=1, \quad \innprod{c_e \cos\left(\frac{2n \pi x}{L}\right)}{c_e \cos\left(\frac{2n \pi x}{L}\right)}=1
\]
Solving for the constants, we get
\[
c=1, \quad c_o = \sqrt{2}, \quad c_e = \sqrt{2}.
\]

%fourier series for a dangling wire (telphone wire) by letting g(x)=g be constant and have 0 boundary values.
%the inner product for fourier series is different because it arises from group theory. Specifically, the inner product is a matter of a representation of a group.  So we have something like 
%\[
%\innprod{f}{g} = \frac{1}{|G|} \int_G f(x)g(x)dx
%\]
%So for this on $[0,L]$, we get
%\[
%\innprod{f}{g} = \frac{1}{L} \int_0^L f(x)g(x)dx.
%\]
%The orthonormal basis elements are the functions
%\[
%1, \qquad \sqrt{2}\cos\left(\frac{2n \pi x }{L}\right), \qquad \sqrt{2}\sin\left(\frac{2n \pi x}{L} \right).
%\]
%\textcolor{red}{Example in Desmos}
%
%\section{Fourier Transforms}
%\textcolor{red}{can find fundamental solution for the laplace operator at some point for electricity magnetism}
%
%We can find the spectrum of the Laplace operator on $\R$ by
%\[
%\frac{d^2f}{dx^2} = kf
%\]
%Then the characteristic polynomial is
%\[
%\lambda^2=k
%\]
%thus $\lambda = \pm \sqrt{k}$. Without going into extra detail, we know that $k\leq 0$ (this can be shown via integration by parts).  Thus, we have that the an eigenfunction for the Laplace operator can be written as
%\[
%f(x)=Ce^{\pm i\omega x}
%\]
%where $C$ is a constant.  Hence
%\section{Distributions}