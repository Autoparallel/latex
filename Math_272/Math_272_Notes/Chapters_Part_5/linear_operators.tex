\section{Matrices}
Review stuff briefly. Talk about Hermitian matrices and their eigenvectors and we can prove why the eigenvectors are orthogonal later when we do adjoints. 

\section{Linear Operators}
Introduce operators on hilbert spaces.

\subsection{Adjoints}
mention the unitary stuff from before as a starting point
Let $H$ be a Hilbert space. Then if we have a function $U\colon H \to H$ such that for $\Psi,\Phi \in H$
\[
\langle
\]

What is actually happening here is due to the adjoint operation we briefly discussed in the prequel. Recall that if we were given a matrix $[A]$ with entries $a_{ij}$, we can define the adjoint matrix $[A]^\dagger$ by taking the entries to be $a_{ji}^*$. In other words, we take the transpose of the matrix $[A]$ and take the complex conjugate of each entry to form $[A]^\dagger$.  Another way we could have defined the adjoint has to do with the inner product structure.  Specifically, let

\subsection{Hermitian Operators}
These are the self adjoint operators.

\section{Differential Operators}
Different differential operators in 1-dimension but mention the analogs in higher dimensions we shall see which is why this becomes more interesting. 

\section{integral operators}

can only understand certain quantities like the probability a particle is in a region which is integral and not pointwise (hw 0)

\subsection{Spectra}

Things like spectrum of laplace operator and the 1-dimensional box again. Laplace operator on the circle $[0,L]$ is self adjoint and so it's spectrum is integers and it gives rise to the Fourier series.  Spectrum on Laplace operator for $\R$ should give rise to the Fourier transform.  