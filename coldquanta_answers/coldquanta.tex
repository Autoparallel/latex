%%%%%%%%%%%%%%%%%%%%%%%%%%%%%%%%%%%%%%%%%%%%%%%%%%%%%%%%%%%%%%%%%%%%%%%%%%%%%%%%%%%%
% Document data
%%%%%%%%%%%%%%%%%%%%%%%%%%%%%%%%%%%%%%%%%%%%%%%%%%%%%%%%%%%%%%%%%%%%%%%%%%%%%%%%%%%%
\documentclass[12pt]{article} %report allows for chapters
%%%%%%%%%%%%%%%%%%%%%%%%%%%%%%%%%%%%%%%%%%%%%%%%%%%%%%%%%%%%%%%%%%%%%%%%%%%%%%%%%%%%
\usepackage{preamble}
\usepackage{braket}
\newcommand{\curvegamma}{\boldsymbol{\vec{\gamma}}}
\newcommand{\tangentgamma}{\boldsymbol{\dot{\vec{\gamma}}}}
\newcommand{\normalgamma}{\boldsymbol{\ddot{\vec{\gamma}}}}
\newcommand{\rhat}{\boldsymbol{\hat{r}}}
\newcommand{\thetahat}{\boldsymbol{\hat{\theta}}}
\newcommand{\phihat}{\boldsymbol{\hat{\phi}}}
\newcommand{\rhohat}{\boldsymbol{\hat{\rho}}}
\newcommand{\vecfieldB}{\boldsymbol{\vec{B}}}
\newcommand{\vecfieldJ}{\boldsymbol{\vec{J}}}
\newcommand{\vecfieldF}{\boldsymbol{\vec{F}}}
\newcommand{\vecx}{\boldsymbol{\vec{x}}}
\usepackage{siunitx}

\newcommand{\veclaplace}{\boldsymbol{\vec{\Delta}}}

\begin{document}

\begin{problem}
What is the laser power required to drive the transition in a Cesium atom from the $\ket{6S_{1/2},~F=4,~mF=4}$ ground state to the $\ket{6P_{1/2}, ~F=4, ~mF=4}$ excited state with a Rabi frequency of 100Mhz if the atom is in the waist of a Gaussian laser beam with $1/e^2$ intensity radius of 100 microns.
\end{problem}
\begin{solution}
First take cylindrical coordinates $\rho$, $\theta$, $z$ and let the Gaussian beam be such that its electric field is aligned along the $z$-direction so that we write the electric field strength as
\begin{equation}
E(\rho) = E_0 \exp \left( -\frac{\rho^2}{\sigma_0^2}\right).
\end{equation}
Therefore the intensity is
\begin{equation}
I=E^2.
\end{equation}
Given that the $1/e^2$ intensity radius is 100 microns (i.e., when $\frac{2\rho}{\sigma_0}=1$), in these units $\sigma_0=2\cdot 10^4 \si{\mu.m}$. 

Since the atom is contained in the waist, the power the atom is exposed to is given by integrating the intensity over the cross-sectional area of waist of the beam $A$
\begin{align}
P = \int_{A} I dA &= \int_{0}^{2\pi} \int_{0}^{\frac{\sigma_0}{2}} E_0^2 \exp\left( -\frac{2\rho^2}{\sigma_0^2}\right) d\rho d \theta \\
&= E_0^2 \sqrt{\frac{\pi^3}{2}} \operatorname{erf}\left(\frac{1}{\sqrt{2}} \right) \sigma_0 \\
&\approx 13653.8 \cdot E_0^2 \cdot 10^{6} \si{W}
\end{align}
since we used $\si{\mu m}$ and we keep $E_0$ in units of $\si{V.m^{-1}}$.

Let $\ket{0}=\ket{6S_{1/2},~F=4,~mF=4}$ and $\ket{1}=\ket{6P_{1/2}, ~F=4, ~mF=4}$ for notational simplicity. In Meystre \& Sargent, they derive the dynamic equations for Rabi flopping. These equations come from time-dependent perturbation of a two-state system where we build a perturbed Hamiltonian $H = H_0 + \mathcal{V}$ where $\mathcal{V}$ is the interaction energy. In the interaction picture, we can then suppose that the eigenstates of $H_0$ (that is $\ket{0}$ and $\ket{1}$) are not themselves perturbed but instead we have
\begin{equation}
\ket{\psi(t)} = C_0(t) e^{-i\omega_0 t} \ket{0} + C_1(t) e^{-i\omega_1 t} \ket{1}.
\end{equation}
To first order, we get Fermi's golden rule that the transition rate $\ket{0}\mapsto \ket{1}$ is $\Gamma_{0\mapsto 1} = \frac{dP_T}{dt} \frac{2\pi}{\hbar}\left| \bra{1}\mathcal{V}\ket{0}\right|^2 \rho$. 

In this situation, the interaction energy $\mathcal{V}$ is due to the dipole moment of the Cesium atom coupled to the electric field hence the relevant interaction energy with an electron the electric dipole approximation and gauge transformation is
\begin{equation}
\mathcal{V} = e\mathbf{r}\cdot \mathbf{E}(\mathbf{R},t),
\end{equation}
where $e$ is the charge of a proton, $\mathbf{r}$ is the dipole moment, $\mathbf{E}$ is the incident electric field, and  $\mathbf{R}$ is the center of mass of the atom.  Hence, if we orient all of this along the $z$-axis and assume $E(t)=E_0\cos(\nu t)$ and further assume the beam is linearly polarized to drive a $\pi$ transition, then from Table 16 of Steck's \emph{Cesium D Line Data} the dipole matrix element for this transition $\ket{0} \mapsto \ket{1}$ is $\wp = \sqrt{\frac{1}{3}} e$.
\begin{equation}
	\mathcal{V}_{01} = \wp E_0 \cos(\nu t) = \sqrt{\frac{1}{3}}e E_0 \cos(\nu t),
\end{equation}
where in general $\wp$ is the component of $e\mathbf{r}_{01}$ along $\mathbf{E}$. If the driving frequency of the laser is near the transition frequency $\omega = \omega_2-\omega_1$ (given numerical values before), then we take the rotating-wave approximation so that
\begin{equation}
	\mathcal{V}_{01} = \frac{1}{2} \wp E_0 \exp(-i \nu t)= \frac{1}{2}\sqrt{\frac{1}{3}} e E_0 \exp(-i \nu t).
\end{equation}
Letting $\delta = \omega - \nu$ we have
\begin{align}
\begin{pmatrix} 
	\dot{C}_0\\ \dot{C}_1
\end{pmatrix} 
= 
\frac{i}{2} \begin{pmatrix} -\delta & R_0 \\
		R_0^*  & \delta \end{pmatrix}
\end{align}
and $|R_0|$ is the Rabi flopping frequency given by
\begin{equation}
	R_0 = \frac{\wp E_0}{\hbar}.
\end{equation}

Now, to get a specific answer, we want $|R_0| = 100\si{MHz}$ given an input power $P$ computed before. Hence
\begin{align}
	\frac{1}{\sqrt{3}} \frac{e E_0}{\hbar} \si{Hz} &= E_0 \cdot \frac{1}{\sqrt{3}}\frac{1}{6.582119569 \cdot 10^{-16}} \si{Hz}\\
	&= E_0 \cdot 8.771494701 \cdot 10^{8} \si{MHz}.
\end{align}
Thus,
\begin{equation}
100\si{MHz} \approx E_0 \cdot 8.771494701 \cdot 10^{8} \si{MHz}
\end{equation}
which implies that $E_0 \approx 1.140056551 \cdot 10^{-7} \si{V.m^{-1}}$ hence 
\begin{equation}
	 \boxed{P \approx 0.1774623 \si{mW}.}
\end{equation}

\textcolor{red}{In doing the other problems, I now think I may be off by factors of 2 and/or $\pi$ from what you were expecting. I think this is just our difference in defining Rabi frequency. Hopefully I'm not too far off due to something else.}
\end{solution}

\newpage
\begin{problem}
In the lab, I can prepare the $F=4$, $mF=0$ sublevel of a Cesium atom's $6S$ ground state and then drive a complete transition to the $F=3$, $mF=0$ sublevel by applying microwave radiation at a frequency of $9.192631770$Ghz for 15 microseconds as demonstrated by a measurement that distinguishes between the $F=3$ and $F=4$ levels (this measurement defines the SI second). What should I expect if instead I apply microwave radiation with a frequency that is increased by $33.3 \si{KHz}$ to $9.192665070 \si{GHz}$ for 21.2 microseconds?
\end{problem}
\begin{solution}
We can use the same set up from the previous problem since now we can assume the light is detuned $\delta = 33.3\si{KHz}$ so we have
\begin{equation}
\begin{pmatrix} 
	\dot{C}_0\\ \dot{C}_1
\end{pmatrix} 
= 
\frac{i}{2} \begin{pmatrix} -\delta & R_0 \\
		R_0^*  & \delta \end{pmatrix}
\end{equation}
where $|C_0|$ is the population of the $F=4$ and $mF=0$ state and $|C_1|$ is the population of the $F=3$ and $mF=0$ state, and $\delta$ is the detuning. 

Since the general solution to the ODE above yields
\begin{equation}
	C_0(0)\cos \left( \frac{R}{2}t\right) + A \sin\left(\frac{R}{2} t\right)
\end{equation} 
where we including possible detuning in the generalized Rabi frequency
\begin{equation}
	R = \sqrt{\delta^2 + |R_0|^2}
\end{equation}
We can prepare the state $C_0(0)=1$ in the lab and hence we have $C_0(t) =\cos \left( \frac{R}{2}t\right)$.

I am going to assume that the factors of $\pi$ and $2$ are different from my Rabi frequency references I used in the previous problem, which leads me to the following change:
\begin{equation}
C_0(t) =\cos \left( 2\pi R t\right)
\end{equation}
If the complete transition occurs at $15 \si{\mu s}$ with no detuning, then we want $C_0(15 \si{\mu s})=0$ and so Rabi frequency is
\begin{equation}
	R_0 \cdot 15\cdot 10^{-6} = \frac{1}{2}
\end{equation}
so $R_0 = 33.3\si{KHz}$. 

If we add in detuning now, we get
\begin{equation}
	R = \sqrt{2\cdot 33.3\si{KHz}} = 47.1\si{KHz}
\end{equation}
and this corresponds to $21.2 \si{\mu s}$. If our population for our detuned problem is now $\tilde{C}(t)=\cos \left( 2\pi R t\right)$ we have $\tilde{C}(21.1\si{\mu s}) \approx 1$ so our population was driven back to the original state past the transition.
\end{solution}

\newpage
\begin{problem}
Assign labels to the states above as $\ket{F=4}=\ket{1}$ and $\ket{F=3}=\ket{0}$ and let's assume that we have a third state $\ket{2}$. Suppose we prepare an entangled state of two atoms $\ket{\psi}=\frac{1}{\sqrt{2}}\left( \ket{11} + \ket{22}\right)$. What state do we have after applying the first microwave pulse in the preceding problem to $\ket{\psi}$ if the microwave radiation irradiates only the first of the two atoms in the entangled state? What state do we have if we apply the second microwave pulse to $\ket{\psi}$.
\end{problem}
\begin{solution}
First, we can write
\begin{equation}
\ket{\psi(t)} = \sum_{i,j=0}^2 C_{i}(t) \ket{i} \otimes C_{j}'(t) \ket{j} = \sum_{ij}^2 C_i(t)C_j'(t) \ket{ij},
\end{equation}
and we can put $C_{ij}(t)=C_i(t)C'_j(t)$. 

Given only the pulse is applied only to the first atom, we can say $C_{1j}(t)=C_1(0)\cos \left( \frac{R}{2}t\right) + A \sin\left(\frac{R}{2} t\right)$ and $|C_{0j}(t)|^2 = 1-|C_{1j}(t)|^2$ are constant over $j$. Also, $C_j'(t) = \textrm{constant}$. Then I will assume that $\ket{2}$ is decoupled from the pulse all together so for both atoms $C_{22}(t)=\frac{1}{\sqrt{2}}$. We have that $C_{11}(0)=\frac{1}{\sqrt{2}}$ by our initial state $\ket{\psi}$. Now, 
\begin{align}
\ket{\psi(t)} &= \sum_{ij}^2 C_i(t)C_j'(t) \ket{ij}  \\
&= \frac{1}{\sqrt{2}}\sin\left(\frac{R}{2} t\right) \ket{01} + \frac{1}{\sqrt{2}} \cos\left(\frac{R}{2} t\right) \ket{11} + \frac{1}{\sqrt{2}}\ket{22}
\end{align}
and it is clear this state is normalized and that $\ket{\psi(0)}=\ket{\psi}$. After the first pulse, we would have the state
\begin{equation}
\ket{\psi(15\si{\mu s})}_{0\si{Hz}} = \frac{1}{\sqrt{2}} \ket{01} + \frac{1}{\sqrt{2}} \ket{22}
\end{equation}
and after the other pulse
\begin{equation}
\ket{\psi(21.2\si{\mu s})}_{33.3\si{MHz}} = \frac{1}{\sqrt{2}} \ket{11} + \frac{1}{\sqrt{2}} \ket{22}
\end{equation}
\begin{remark}
This seems interesting. It is like a swap gate. By choosing the entangled pair to have $\ket{22}$ there is no control over the population in $\ket{2}$. Now this has me reading about quantum gates and some about Clifford gates since I'm a big fan of Clifford algebras.
\end{remark}

Okay, it could also be that I am off by some factor of 2 or 4 with the second laser pulse and I don't want to ignore another interesting case. Namely, maybe that second pulse was supposed to take $\ket{0} \mapsto \frac{1}{\sqrt{2}} \left( \ket{0} + \ket{1} \right)$ which would occur with a half-transition. In this case, I believe what we just saw here was an example of a Hadamard gate.
\end{solution}



\newpage
I want to investigate this a bit further. Suppose we have just the states before $\ket{0}$ and $\ket{1}$. Now, the pulses before should be able to be represented as a controlled unitary gate. To this end, let me take 
\begin{equation}
\ket{\psi} = C_{ij}(0) \ket{ij}
\end{equation}
where summation is implied. Suppose I now choose a matrix representation for this space like so:
\begin{equation}
\ket{00} = \begin{pmatrix} 1 \\ 0 \\ 0 \\ 0 \end{pmatrix} \qquad \ket{01} \begin{pmatrix} 0 \\ 1 \\ 0 \\ 0 \end{pmatrix} \quad \cdots \quad \ket{11} = \begin{pmatrix} 0 \\ 0 \\ 0 \\ 1 \end{pmatrix}
\end{equation}
For sake of notation, let me just apply a laser pulse only to the second atom, I believe I'd receive the gate with a matrix representation
\begin{equation}
CU = \begin{pmatrix} 					1 & 0 & 0 & 0 \\
					0 & 1 & 0 & 0\\
					 0 & 0 & \cos\left(\frac{R}{2} t\right) & \sin\left(\frac{R}{2} t\right)\\
					 0 & 0 & -\sin\left(\frac{R}{2} t\right) & \cos\left(\frac{R}{2} t\right)\\
 \end{pmatrix}
\end{equation}
The choice of laser pulse duration and detuning decides the bottom right block unitary transformation you apply to any initial state $\ket{\psi}$. 

\newpage
\noindent\textbf{Notes to and from myself:}

In problem 1, We have that $\ket{0}$ is a $S$-orbital state so the electron orbital angular momentum $L=0$ and for $\ket{1}$ we have $L=1$ since this is a $P$-orbital state. For any given $L$, the magnitude of the orbital angular momentum vector is $\mathbf{L}=\hbar^2 L(L+1)$. This and the electron spin couple add a coupling term in the atomic Hamiltonian
\[
H=H_{\textrm{Coulomb}}+H_{\textrm{SO}}
\]
where
\[
H_{SO} = \left(\frac{Ze^2}{4 \pi \epsilon_0} \right) \left(\frac{g_s-1}{2m_e^2c^2} \right)\frac{\bf{L}\cdot \bf{S}}{r^3}.
\]
We have $\mathbf{J} = \mathbf{L}+\mathbf{S}$ and $|L-S|\leq J \leq L+S$. For $\ket{0}$ we have $S=1/2$ so $J=1/2$ and for $\ket{1}$ we have $S=1/2$ and so $J=1/2$ or $J=3/2$ \textcolor{red}{Fine structure splits due to $J$. There are more terms and more ways to compute the energy shift that I didn't write down (e.g., Darwin term only for $S$-orbitals).} This transition we want in problem 1 is the $D_1$ line (the data is in Steck's paper in the tables and figures). The frequency of transition is 
\[
335.116048807(120)\si{THz}- 4.021776399375\si{GHz} + 510.860(34)\si{MHz} \approx 3.351125378911\si{THz}
\]
which is very roughly a $895\si{nm}$ wavelength beam.

Hyperfine structure couples to the angular momentum (multipole) of the nucleus. For $\ket{0}$ we have $\mathbf{F}=4$ which is the total  angular momentum of the electron along with the atomic angular momentum $I$. For the Cesium ground state $I=7/2$ and $J=1/2$ so $F=3$ or $F=4$. Now, this adds a new term to the Hamiltonian
\[
H_{\textrm{hfs}} = A_{\textrm{hfs}} \mathbf{I}\cdot \mathbf{J} + B_{\textrm{hfs}} \frac{3(\mathbf{I}\cdot \mathbf{J})^2 + \frac{3}{2} \mathbf{I}\cdot \mathbf{J} - I(I+1)J(J+1)}{2I(2I-1)J(2J-1)}
\]
\textcolor{red}{There are energy shifts too.} There are $2F+1$ different magnetic $mF$ sublevels depending on electron spin that can be seen by applying an external magnetic field (Zeeman effect). Each has a different dipole matrix elements that I found in 


\noindent \textbf{Some sources I used:}
\begin{itemize}
\item \url{https://steck.us/alkalidata/cesiumnumbers.1.6.pdf} Steck's \emph{Cesium D Line Data}.
\item Meystre \& Sargent \emph{Elements of Quantum Optics}. I really enjoyed this text. I mostly used Chapter 3 but there is more I would like to read.
\item A brief chat with Jacob Roberts at CSU.
\item \url{https://en.wikipedia.org/wiki/Fine_structure}
\item \url{https://en.wikipedia.org/wiki/Hyperfine_structure}
\item \url{https://users.physics.ox.ac.uk/~Steane/teaching/rabi_logic08.pdf}. This seemed super closely related to everything and I'd be curious to go through and answer these questions and read their sources.
\end{itemize}

I also will be chatting with some friends and colleagues at NASA about lasers and their projects involving them. I feel like I never quite learned enough about them and they are obviously important. I guess I have more reading to do...





\end{document}