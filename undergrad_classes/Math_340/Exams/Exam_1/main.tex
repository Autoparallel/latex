\documentclass[11pt]{article}
\usepackage{enumerate,amsmath,amssymb,amsthm,color,wasysym,float}
\usepackage{multicol}
\usepackage{epsfig}
\usepackage{graphicx}	
\usepackage{fancyhdr}
\usepackage{ifpdf}
%\usepackage{pifont}
%\usepackage{systeme}
%\usepackage{xstring}

\newtheorem{theorem}{Theorem}[section]
\newtheorem*{thmnon}{Theorem}
\newtheorem{algorithm}[theorem]{Algorithm}

\newtheorem*{definition}{Definition}
\newtheorem*{soln}{Solution}

\ifpdf
  \usepackage[pdftex]{hyperref}
\else
  \usepackage[hypertex]{hyperref}
\fi

\renewcommand{\theenumi}{\textbf{\Alph{enumi}}}

\newcommand{\comment}{$\rhd$\ \ }
\newcommand{\rarrow}{\rightarrow}
\newcommand{\cond}{\mbox{cond}}

% Pseudocode line numbering
\newcounter{pseudocode}
\newcommand{\firstline}{\setcounter{pseudocode}{0}\linenumber}
\newcommand{\linenumber}{\refstepcounter{pseudocode}\thepseudocode}

\newcommand{\cmark}{\ding{51}}%
\newcommand{\xmark}{\ding{55}}%

%%%%%%%%%%%%%
\oddsidemargin -0.1in
\evensidemargin -0.1in
\textwidth 6.6in
\headheight 0.4in
\topmargin -0.6in
\textheight 9.0in %\footheight 1.0in %%%%%%%%%%%% 

\renewcommand{\baselinestretch}{1.2}

\def\c{\textcircled}
\def\det{\text{det}}
\def\vecb{\mathbf{b}}
\def\vecv{\mathbf{v}}
\def\vecu{\mathbf{u}}
\def\vecx{\mathbf{x}}
\def\vec0{\mathbf{0}}

\def\<{\langle}
\def\>{\rangle}

\def\dd#1{\displaystyle{#1}}

\begin{document}
\pagestyle{fancy}
\fancyhead[LO,LE]{\bf{MATH 340, Summer 2018}}
\fancyhead[CO,CE]{\bf{EXAM I}}
\fancyhead[RO,RE]{\bf{Colorado State University}}

%%%%%%%%%%%%%%% Title Page %%%%%%%%%%%%%%%

\begin{center}
{\Large \bf {\sc EXAM I}} \\
\textcolor{red}{July 3, 2018} \\
\vspace{3mm}
Name: \underline{\hspace{5cm}} \hspace{3mm}
Instructor: \underline{\hspace{2cm}} \hspace{3mm}
Time your class meets: \underline{\hspace{1cm}}
\end{center}

\noindent
{\bf HONOR PLEDGE} \hspace{5mm} I have not given, received, or used any unauthorized assistance on this exam. Furthermore, I 
agree that I will not share any information about the questions on this exam with any other student before graded exams are returned. 
\noindent 
\begin{center}
Signature: \underline{\hspace{4cm}} Date: \underline{\hspace{3cm}} \hspace{5mm}
\end{center}

\begin{itemize}
\item There are 6 problems, but only the highest scoring 5 problems will be counted.
\item You have one hour and thirty minutes to complete this exam.
\item No notes, books, or other references are allowed during this exam.  
\item Calculators are not allowed during the exam.
\item A one-sided note sheet of $8\frac{1}{2}'' \times 11''$ or smaller is allowed during the exam. 
\item There are questions on the front of the page. You can use the back of each
page if you need more space. If you do use any extra paper, make sure that your name is 
on it and that you attach it to your exam.
\item You must show all work to receive credit. Answers for which no work is shown
will receive no credit unless specifically stated otherwise. 
\end{itemize}

\begin{table}[!ht] 
\begin{center}
\begin{tabular}{|c|c|c|}  \hline
Question & Score & Maximum \\ \hline \hline
1 &  & 20  \\ [2ex] \hline 
2 &  & 20  \\ [2ex]\hline
3 &  & 20  \\ [2ex]\hline
4 &  & 20  \\ [2ex]\hline
5 &  & 20  \\ [2ex]\hline
6 &  & 20 \\ [2ex] \hline
Total &  & 100 \\[2ex] \hline
\end{tabular}
\end{center}
\end{table}

%%%%%%%%%%%%%%% Problem 1 %%%%%%%%%%%%%%%

\newpage
\paragraph{1.} Consider the differential equation:
\[
3y'+\frac{\cos t}{y}+\frac{3}{y}=0.
\]
  

\begin{enumerate}[(a)]
    \item Write the differential equation in normal form.
    \item What is the order of this differential equation?  What type of differential equation is it?
    \item Find the general solution to the differential equation.
\end{enumerate}

%%%%%%%%%%%%%%% Problem 2 %%%%%%%%%%%%%%%

\newpage
\paragraph{2.} Consider the differential equation:
\[
tx' = 4x + t^4
\]

\begin{enumerate}[(a)]
    \item Classify this differential equation: state its order, whether or not it is linear and, if it is linear, whether or not it is homogeneous.
    \item Find the general solution of this differential equation.
    \item Find a solution satisfying the initial condition $x(1)=5$.
    \item Is your solution to (c) unique? Explain rigorously.
\end{enumerate}

%%%%%%%%%%%%%%% Problem 3 %%%%%%%%%%%%%%%

\newpage
\paragraph{3.} A population $P(t)$ is found to change based on the following equation:
\[
P'=(P-1)(P-2)(P-3).
\]

\begin{enumerate}[(a)]
    \item What are the equilibrium point(s)?
    \item Which of the equilibrium point(s) are stable, which are unstable?'
    \item Draw a phase plane with the equilibrium point(s) labeled.  Qualitatively draw possible solution curves (in the $P-t$-plane) for the following initial conditions:
\[
P(0)=0, \quad P(0)=\frac{3}{2}, \quad P(0)=\frac{5}{2}, \quad P(0)=4.
\]
    
\end{enumerate}

%%%%%%%%%%%%%%% Problem 4 %%%%%%%%%%%%%%%

\newpage
\paragraph{4.} Consider the first-order ODE:
\[
(x^2+y^2-x)dx - ydy = 0
\]

\begin{enumerate}[(a)]
    \item Is this ODE exact? Use a calculation to justify your answer.
    \item Multiply the ODE by $\mu(x,y)=\frac{1}{x^2+y^2}$. Is this modified ODE exact? Again, use a calculation to justify your answer.
    \item Find the general solution of this differential equation.
\end{enumerate}

%%%%%%%%%%%%%%% Problem 5 %%%%%%%%%%%%%%%

\newpage
\paragraph{5.} You are given a circuit with a capacitor, inductor, and resistor in series.  This gives the differential equation
\[
\frac{d^2 I}{dt^2}+\frac{R}{L}\frac{dI}{dt}+\frac{1}{LC}I=F(t),
\]
where $I(t)$ is current, $F(t)$ is an external power signal, $C$ is capacitance, $L$ is inductance, and $R$ is resistance. (Note: $C,L,$ and $R$ are constant real numbers.)

\begin{enumerate}[(a)]
    \item Classify this differential equation: state its order, and whether or not it is linear.
    \item Letting $C=\frac{1}{6}$, $L=1$ and $R=5$, what is the solution $I(t)$ to the homogenous equation (i.e., when $F(t)=0$).
    \item Keeping the capacitance, inductance, and resistance from (a), an external power supply is attached and outputs $F(t)=\sin(t).$ What is the particular solution for this differential equation?
    \item Write down the general solution obtained from doing parts (b) and (c). What can we expect to see when we measure the current after a \emph{very} long time is elapsed?
    
\end{enumerate}

\newpage
\noindent This page is left blank for Problem 5.

%%%%%%%%%%%%%%% Problem 6 %%%%%%%%%%%%%%%

\newpage
\paragraph{6.} A charged particle with mass $m$ and charge $q$ experiences a force $qE(t)$ where $E(t)$ is the electric field at time $t$.  With no other forces acting, we have the following equation.:
\[
mx''- qE(t)=0
\]
If you have a switch that turns on a constant electric field at time $t=t_0$, you can write
\[
E(t)=H(t-t_0), 
\]
where $x$ is the position of the particle at time $t$ and $H$ is the Heaviside function. 

\begin{enumerate}[(a)]
    \item Given $mx''-qH(t-t_0)=0$ with the initial data: $x(t_0)=0$, $x'(t_0)=0$, find the Laplace transform of this differential equation.
    \item Going from your result in (a), what is the solution $x(t)$ to the differential equation?
    \item If you multiply your answer from (b) by $H(t-t_0)$, then this solution is valid for all times $t$.  What is the particle doing at $t<t_0$? What about for $t>t_0$?
\end{enumerate}

%%%%%%%%%%%%%%% Laplace Table %%%%%%%%%%%%%%%

\newpage
\thispagestyle{empty}
\section*{Table of Laplace Transforms}
\begin{center}
\begin{tabular}{c c c} \hline\vspace{.2cm}
$f(t)$ & $\displaystyle\mathcal{L}\{f\} = F(s)$ & Domain \\ \vspace{.2cm}
$1$ & $\displaystyle\frac{1}{s}$ & $s > 0$ \\ \vspace{.2cm}
$t^n$ & $\displaystyle \frac{n!}{s^{n+1}}$ & $s > 0$ \\ \vspace{.2cm}
$\sin(at)$ & $\displaystyle \frac{a}{s^2 + a^2}$ & $s > 0$ \\ \vspace{.2cm}
$\cos(at)$ & $\displaystyle \frac{s}{s^2 + a^2}$ & $s > 0$ \\ \vspace{.2cm}
$\exp(at)$ & $\displaystyle \frac{1}{s - a}$ & $s > a$ \\ \vspace{.2cm}
$\exp(at)\sin(bt)$ & $\displaystyle \frac{b}{(s-a)^2 + b^2}$ & $s > a$ \\ \vspace{.2cm}
$\exp(at)\cos(bt)$ & $\displaystyle \frac{s-a}{(s-a)^2 + b^2}$ & $s > a$ \\ \vspace{.2cm}
$t^n\exp(at)$ & $\displaystyle \frac{n!}{(s-a)^{n+1}}$ & $s > a$ \\ \vspace{.2cm}
$\delta_p$ & $\exp(-sp)$ & \\ \vspace{.2cm}
$H_c(t)$ & $\displaystyle \frac{\exp(-cs)}{s}$ & $s > 0$ \\ \vspace{.2cm}
$f$ with period $T$ & $\displaystyle \frac{F_T(s)}{1 - \exp(-Ts)}$ & \\ \hline \vspace{.2cm}
$\exp(at)f(t)$ & $F(s - a)$ & $s > a$ \\ \vspace{.2cm}
$\mathrm{H}(t-c)f(t-c)$ & $\exp(-cs)F(s)$ & $c \ge 0$ \\ \vspace{.2cm}
$tf(t)$ & $-F'(s)$ & \\ \vspace{.2cm}
$t^nf(t)$ & $(-1)^nF^{(n)}(s)$ & \\ \vspace{.2cm}
$f(t)$ & $\displaystyle \int_0^{\infty} f(t)e^{-st} \dd{t}$ & \\ \vspace{.2cm}
$\alpha f(t) + \beta g(t)$ & $\alpha F(s) + \beta G(s)$ & \\ \vspace{.2cm}
$f\ast g$ & $F(s)\cdot G(s)$ & \\ \hline \vspace{.2cm}
$y'$ & $sY(s) - y(0)$ & \\ \vspace{.2cm}
$y''$ & $s^2Y(s) - sy(0) - y'(0)$ & \\ \vspace{.2cm}
$y^{(n)}$ & $s^nY(s) - s^{n-1}y(0) - s^{n-2}y'(0) - \cdots - sy^{n-2}(0) - y^{n-1}(0)$ & \\ \hline
\end{tabular}
\end{center}

\end{document}
