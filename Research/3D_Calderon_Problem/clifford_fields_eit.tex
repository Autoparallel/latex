\section{Notes}

\textcolor{red}{Show that a monogenic function $f=f_0 + f_2$ satisfying $\nabla f=0$ has harmonic components}

The inner product is commutative:
\[
A\cdot B = \frac{1}{2}(AB+BA)
\]
but the exterior products
\[
A\wedge B = \frac{1}{2}(AB-BA)
\]
is anticommutative. So there is a commutative subalgebra or something but the whole algebra isn't commutative.

Do we need that? Also we can show the even fields are a $C^*$-algebra by using the pseudoscalar as the involution.

What about Dirichlet to Neumann map for clifford sections

\section{Set Up}
\textcolor{red}{Wikipedia page on Clifford bundle describes the Clifford bundle as an associated bundle. Spin bundle describe spinors differently. Check those out.}
\textcolor{red}{consider defining the basis forms for all grades i.e., $B_1 = dydz$ is the basis 2 form dual to $dx$ in a way.}
Let $(M,g)$ be an $n$-dimensional smooth Riemannian manifold.  Then we can consider an induced Clifford bundle $C\ell(TM,g)$ on $M$ with the quadratic form $Q(v)=g(v,v)$.  This is a natural choice, but not necessarily the ``correct" one.  Specifically, we define $C\ell(T_pM,g_p)$ to be the Clifford algebra on the tangent space $T_pM$ and let
\[
C\ell(TM,g) \coloneqq \dot{\bigcup}_{p\in M} C\ell (T_pM,g_p)
\]
be the Clifford bundle.

We can then define the space of Clifford sections by noting we have a natural projection
\[
\pi \colon C\ell(TM,g) \to M
\]
that maps a Clifford element to the point at which it is based and putting
\[
\Gamma C \ell (TM,g) \coloneqq \left\{ \sigma \colon M \to C\ell(TM,g) ~\vert~ \pi \circ \sigma = \id_M\right\}.
\]
\begin{question}
There's probably some notation for this somewhere. How do we assure that this is a \emph{smooth} section?
\end{question}

\begin{question}
When is this product operation smooth? Is it always smooth? What does smoothness really mean here?
\end{question}

\textcolor{red}{Belishev is looking at quaternionic functions that satisfy both $\Delta \alpha=0$ and $\vec{\Delta}u=0$ in the weak sense. This should be easier to state with the below set up.}

Consider a $3$-dimensional Riemannian manifold $(M,g)$ with the associated \boldblue{Clifford bundle} $\cliffbund$ generated from the metric $g$. We call a $f\in \cliffsect$ a \boldblue{Clifford field} or just field.  

\begin{problem}
Problem statement to recover $g$ from $\Lambda$ and algebra of ``holomorphic" functions.
\end{problem}

Let $i\colon \partial M \to M$ be inclusion of the boundary in $M$.


\begin{figure}[H]
  \centering
  \includesvg[width=.4\textwidth]{figures/manifold_with_boundary.svg}
  \caption{I fixed the caption}
  \label{fig: broken_caption_is_stupid}
\end{figure}

Think of even grade (spinor?) fields $f\in \cliffsect$ as (maybe?) the voltage and current flux multivector. That is
\[
\projzero{f} = f_0 \qquad \textrm{and} \qquad \projtwo{f}=f_2
\]
where the other grades are zero. $f_0$ could represent the voltage scalar and $f_2$ the current flux plane field.

Specifically the problem reads:
\[
\begin{cases}
df=\delta f = 0 &\textrm{in $M$}\\
i^* f = \varphi \qquad i^*(\delta f)=0 & \textrm{on $\partial M$}
\end{cases}
\]
Do we want $f_0$ above and not $f$ since that is the scalar part of this problem? The $f_2$ part comes in when we're talking about conjugate stuff. Belishev defines \boldblue{harmonic quaternion fields} $\{\alpha,u\}$ to be the ones satisfying 
\[
d\alpha = \star d u \qquad \delta u=0
\]
where we're thinking of $u$ as a 1-form (really a purely imaginary quaternion field).

Then the \boldblue{Dirichlet to Neumann operator} for forms gives
\[
\Lambda \varphi = i^*(\star d \varphi) = (-1)^{k+1}i^*(\delta \star \varphi).
\]

It seems like the idea is we know $f_0$ solves the Dirchlet problem and we can relate $f_2$ to $f_0$ by the CREs. Then $f_2$ is related to the Neumann data in the problem.

\section{Cauchy Riemann Equations}

Letting $\nabla$ represent the \boldblue{Dirac operator} or \boldblue{geometric derivative} (which depends on the metric, as we'll see) we have
\[
\nabla = e^i \partial_i = d + \delta,
\]
where the first equality is induced from a basis on the tangent space $e_i$ and $e^i$ represents the dual basis satisfying $e^i \cdot e_j = \delta^i_j$. The second equality is more coordinate free where $d$ and $\delta$ are the \boldblue{exterior derivative} and \boldblue{interior derivative} respectively. One may define these via
\[
\nabla
\]
\begin{align*}
df &= \frac{1}{2}\left( \nabla f - f\nabla\right)\\
\delta f &= \frac{1}{2} \left( \nabla f + f\nabla\right),
\end{align*}
If we consider this field to be \boldblue{harmonic} in the interior of $M$, then
\[
\nabla^2 f =0 
\]
or more ``weakly" (I believe)
\[
d f =0 \qquad \textrm{and} \qquad \delta f = 0.
\]
If we have 
\begin{align*}
    0=\nabla^2 f &= (d\delta +\delta d) f\\
    0&= d\delta f_0 + (d\delta + \delta d)f_2\\
    \implies d\delta f_0 &= -(d\delta + \delta d)f_2
\end{align*}

For any test form $\varphi$ we should have the weak equations
\begin{align*}
    0&=\langle \nabla^2 f, \varphi\rangle &= \langle (d\delta + \delta d)f,\varphi \rangle\\
    &= \langle d \delta f, \varphi\rangle + \langle \delta d f, \varphi \rangle \\
    &= \langle \delta f, \delta \varphi \rangle + \langle df,d\varphi\rangle
\end{align*}
which means we have that $\delta f=df=0$.

Using the second requirement for harmonic, we find the ``\boldblue{Cauchy Riemann equations}" are
\begin{align*}
    0=df&=\delta f\\
    df_0 + df_2 &= \delta f_0 + \delta f_2\\
    df_0 + df_2 &= \delta f_2,
\end{align*}
since $\delta f_0=0$.  Now, if we match grades we arrive at four equations
\begin{align*}
    \projzero{df}&=\projzero{\delta f} &&& \projone{df}&=\projone{\delta f} &&& \projtwo{df}&= \projtwo{\delta f} &&& \projthree{df}&= \projthree{\delta f}\\
    0 &= 0 &&& df_0 &= \delta f_2 &&& 0&=0 &&& df_2 &= 0.
\end{align*}
The ``Cauchy Riemann equations" are then
\[
df_0= \delta f_2
\]
and $df_2 =0$ states that the current flux plane field is closed. (what does this mean physically? Does it make sense? Does it mean that all ``currents" form loops?) 

\section{Dirichlet to Neumann Operator}

The \boldblue{Dirichlet to Neumann operator} for differential forms is defined by
\[
\Lambda \colon \Omega^k(\partial M) \to \Omega^{n-k-1}{\partial M}
\]
given by
\[
\Lambda \varphi = i^* \star d \omega=(-1)^{k+1}i^*(\delta \star \varphi),
\]
where $\varphi$ is the \boldblue{Dirichlet data} on the boundary and $d\omega$ is the \boldblue{Neumann data} on the boundary.

If we define $\star$ for $\cliffsect$ we could use the \boldblue{volume form} or \boldblue{pseudoscalar} $\mu$ corresponding to $g$ at each point (which is global since it is a volume form). The inclusion map is defined in the usual way and $d$ is the exterior derivative.  

\section{Algebras}
Consider the even grade fields $f\in \Spinors$ which we refer to as \boldblue{spinor fields} as a subalgebra of $\cliffsect$.  At each point, this is clearly a vector space of dimension 4 (in the case of a 3 manifold), and it forms an algebra at each point since a product of two multivectors with only even grade elements returns a multivector with only even grade elements.  So by $\Spinors$ we refer to the algebra of spinor fields.

\begin{question}
Does the below norm really come from an inner product? That is, can we define the inner product for $f,g\in \Spinors$ by:
\[
\langle f,g\rangle = \int_M \projzero{fg^\dagger} \mu
\]
If that's the case, then we should have a Hilbert space.
\end{question}

\begin{definition}
A \boldblue{Banach algebra} is an associative algebra $\mathcal{A}$ over $\R$ or $\C$ that at the same time is also a Banach space, i.e. a normed space and complete in the metric induced by the norm. The norm is required to satisfy
\[
\forall x,y \in \mathcal{A}~\colon~ \|xy\|\leq \|x\|\|y\|.
\]
\end{definition}

\begin{prop}{$\Spinors$ form a Banach Algebra}{spinors_banach_alg}
The algebra $\Spinors$ with a norm 
\[
\|f\|_2^2 \coloneqq \int_M ff^\dagger \mu = \int_M \|f\|^2 \mu
\]
forms a Banach algebra.
\tcblower
\begin{proof}
This is indeed a norm at each point as
\begin{align*}
ff^\dagger \mu  &= (f_0 + f_2)(f_0^\dagger+f_2^\dagger)\mu\\
&= f_0 f_0^\dagger \mu + f_0 f_2^\dagger \mu+ f_2 f_0^\dagger \mu + f_2 f_2^\dagger \mu \\
&= f_0^2 \mu + f_0 f_2^\dagger \mu + f_0 f_2 \mu + f_2f_2^\dagger \mu\\
&= f_0^2 \mu - f_0 f_2^\dagger \mu + f_0 f_2 \mu + f_2 f_2^\dagger \mu\\
&= f_0^2 \mu + f_2f_2^\dagger \mu.
\end{align*}
Here we are working with the pointwise norm $\|f\|=ff^\dagger$.  Then let
\[
f_2 = \alpha_1 dydz + \alpha_2 dzdx + \alpha_3 dxdy
\]
and we have
\begin{align*}
    f_2 f_2^\dagger &= (\alpha_1 dydz + \alpha_2 dzdx + \alpha_3 dxdy)(\alpha_1 dzdy + \alpha_2 dxdz + \alpha_3 dydx)\\
    &= [\alpha_1^2 + \alpha_2^2 + \alpha_3^2] +[ \alpha_1 \alpha_2 dydzdxdz + \alpha_1 \alpha_2 dxdzdydz + \alpha_1\alpha_3 dydzdydx + \\
    &\quad +\alpha_1\alpha_3 dydxdyz + \alpha_2 \alpha_3 dzdxdydz + \alpha_2 \alpha_3 dydxdzdx] \\
    &= \alpha_1^2 + \alpha_2^2 + \alpha_3^2 
\end{align*}
See now \cite{hestenes_classical_mechanics} for the norm definition which this is equivalent to.  Now, we have this norm pointwise, we need to show it is a norm on the sections.  So we must satisfy
\begin{enumerate}[(i)]
    \item (Triangle inequality) Take two fields $g$, $f$ then.... \textcolor{red}{Work this out.}
    \item (Scalar) If we take $\alpha \in \R$ then
    \[
    \|\alpha f\| = |\alpha|\|f\|
    \]
    is clear. Hence
    \begin{align*}
    \|\alpha f\|_2^2 &= \int_M \|\alpha f\|^2 \mu\\
    &= |\alpha|^2\int_M \|f\|^2 \mu\\
    &= |\alpha|^2 \|f\|_2^2.
    \end{align*}
    \item Clearly if $f=0$, then $\|f\|_2=0$. Now, suppose $\|f\|_2=0$, then
    \begin{align*}
        0=\|f\|_2^2 &= \int_M \|f\|^2 \mu,
    \end{align*}
    but $\|f(p)\|^2\geq 0$ for every $p\in M$ and hence it must be that $\|f(p)\|^2 = 0$ for each $p\in M$ and thus $f=0$.
\end{enumerate}
Now, let $f,g \in \Spinors$ then we want to show that
\[
\|fg\|_2\leq \|f\|_2\|g\|_2.
\]
Indeed, take
\begin{align*}
    \|fg\|_2^2 &= \int_M \|fg\|^2 \mu\\
    &\leq \int_M \|f\|^2\|g\|^2 \mu
\end{align*}
\end{proof}
\end{prop}



\begin{remark}{}{module}
What if we think of $\cliffsect$ as a $C^\infty(M)$-module?
\end{remark}

\begin{remark}{}{rmk_norm}
This actually defines a norm on spinor fields in any dimension.
\end{remark}



Maybe we can just take the $l^2$ norm of each part individually or something (similar to how Belishev does in the same algebra section.

\begin{question}
Is the subspace of harmonic fields an algebra? Or is it at least dense in some way? See Theorem 1 in \cite{belishev_quaternion}
\end{question}




\begin{question}
If we can make the algebra of Clifford sections a (unital) Banach algebra, can we compute the spectrum? (If it forms a Banach algebra, the scalar element $1$ is likely the unit.)
\end{question}

\begin{definition}
The \emph{spectrum} $\sigma(x)$ of $x\in \mathcal{A}$ is the set
\[
\sigma(x) \coloneqq \{ \lambda ~\vert~ x-\lambda 1 \textrm{ is not invertible.}\}
\]
\end{definition}

Of course this has all been said before, see \url{https://en.wikipedia.org/wiki/Clifford_bundle}. It seems that the signs are flipped from mine.

\begin{question}
Does $|Q(v)|$ form a norm on the Clifford algebra fiber? If so, can we extend this to a norm on the manifold?
\end{question}

\begin{definition}
A \emph{norm} $\|\cdot\|$ on a vector space $V$ over a field $\mathbb{F}$ is a map
\[
\|\cdot \| \colon V \to \mathbb{F}
\]
that satisfies
\begin{enumerate}[(i)]
    \item (Triangle inequality) $\|u+v\|\leq \|u\|+\|v\|~\forall u,v\in V$,
    \item (Scalar behavior) $\|\alpha v\| = |\alpha|\|v\|~\forall v\in V$ and $\alpha \in \mathbb{F}$,
    \item (Unique zero) $\|v\|=0 ~\iff~ v\equiv 0$.
\end{enumerate}
If $\|\cdot \|$ only satisfies (i) and (ii) then we call $\|\cdot \|$ a \emph{seminorm}.
\end{definition}

\begin{proposition}
Let $V$ be a $n$-dimensional vector space over a field $\mathbb{F}$, $Q$ be a nondegenerate quadratic form, and $C\ell(V,Q)$ be the Clifford algebra over $V$ with $Q$, then the function
\[
|Q(\cdot)| \colon V \to \mathbb{F}
\]
is a norm on $C\ell(V,Q)$.
\end{proposition}

\begin{proof}
We show the three properties listed above.  Let $\{e_1,\dots,e_{2^n}\}$ be an arbitrary basis for $C\ell(V,Q)$. We note that given $v \in C\ell(V,Q)$ we can write
\[
v=\sum_{i=1}^{2^n} \alpha_i e_i.
\]
We then take $A$ to be a symmetric $2^n\times 2^n$-matrix (with rank $2^n$ by the nondegeneracy) and define
\[
Q(v)\coloneqq \sum_{i=1}^{2^n} \sum_{j=1}^{2^n} A_{ij}\alpha_i \alpha_j.
\]
\begin{enumerate}[(i)]
    \item Let $v,u \in C\ell(V,Q)$.  Then put
    \begin{align*}
        v &= \sum_{i=1}^{2^n} \alpha_i e_i,\\
        u &= \sum_{i=1}^{2^n} \beta e_i.
    \end{align*}
    \begin{align*}
    |Q(v+u)|&=\left|\sum_{i=1}^{2^n} \sum_{j=1}^{2^n} A_{ij}(\alpha_i+\beta_i) (\alpha_j+\beta_j)\right|\\
    &= \left|\sum_{i=1}^{2^n} \sum_{j=1}^{2^n} A_{ij}(\alpha_i \alpha_j +\beta_i \alpha_j+ \alpha_i\beta_j +\beta_i\beta_j)\right|
    \end{align*}
    
    \[
    |Q(v)|+|Q(u)| = \left|\sum_{i=1}^{2^n} \sum_{j=1}^{2^n} A_{ij}\alpha_i \alpha_j\right| + \left|\sum_{i=1}^{2^n} \sum_{j=1}^{2^n} A_{ij}\beta_i \beta_j\right|
    \]
\end{enumerate}
\end{proof}

\section{Analytic Functions}



\section{Questions and Thoughts}

\begin{itemize}
    \item Do some searching for ``clifford algebras, eit, voltage to current, dirichlet to neumann, ..."
\end{itemize}

\begin{question}
What is the Dirichlet to Neumann operator?
\end{question}

\begin{question}
If we define the Dirichlet to Neumann operator as in \cite{clay_d-to-n_map}, then this only uses the exterior derivative. What if we instead consider the Dirac operator?
\end{question}

\begin{question}
What is the Hilbert transform?
\end{question}

\begin{question}
Can we relate the Hilbert transform to the D to N map?
\end{question}

\begin{question}
Can we recover the algebra of ``holomorphic" functions from the above information?
\end{question}

\begin{question}
Do the $f_0$ and $f_2$ relate to voltage and current? If so, this may be a physical explanation of this conjugate fields business.  Analogously one may wonder if knowing all the currents in the system is equivalent to knowing all the voltages?  That is, knowing the plane field of currents is the same as knowing the scalar field of voltage.
\end{question}

\section{Random Things}


\begin{itemize}
    \item \url{http://math.uchicago.edu/~amathew/dirac.pdf} These seem like useful notes
\end{itemize}

\begin{question}
    Is there some kind of Hodge-Clifford decomposition we can do for clifford sections?
\end{question}

\begin{question}
Is there some kind of Clifford homology/cohomology?
\end{question}

Outside of the realm of the Clifford stuff, what about the ``tensor" dirac derivative operator?  Consider the operator
\[
D = e_i \otimes \nabla_{e_i}
\]
so that, for example, on a vector field we achieve
\[
DF = e_i \otimes \nabla_{e_i}f^j e_j = e_i \otimes e_j \nabla_{e_i} f^j
\]
which we can write as a matrix that we call the Jacobian. This should transform nicely under coordinates more naturally.

Then if we consider this object not in the tensor algebra but in the quotient (the Clifford algebra) we achieve
\[
DF = \textrm{div}(F) + [DF]_2
\]
the superalgebra splitting into the div and curl like components.

Then
\[
D^2F = (e_1 \otimes \nabla_{e_i})(e_j\otimes \nabla_{e_j})f^ke_{k} = e_i \otimes e_j \otimes e_k \nabla_{e_i} \nabla_{e_j} f^k,
\]
is like the Hessian tensor. In other words, if instead $f$ was a scalar (rank 0), then
\[
D^2f = (e_i \otimes e_j) \nabla_{e_i} \nabla_{e_j} f
\]
is the hessian.

In \cite{belishev_two_dimensional} he mentions a related problem in spacetime which may be interesting to look at (see just at section 0.3).


\newpage
\begin{thebibliography}{1}
    
\bibitem{geo_alg} Doran, C. \& Lasenby A. (2003). \emph{Geometric Algebra for Physicists}. Cambridge University Press.

\bibitem{clay_d-to-n_map} The Complete Dirichlet-to-Neumann Map.

\bibitem{belishev_quaternion} \emph{On Algebraic and uniqueness properties of 3d harmonic quaternion fields}

\bibitem{hestenes_classical_mechanics} \emph{New Foundations for Classical Mechanics}

\bibitem{belishev_two_dimensional} \emph{Calderon Problem for Two-Dimensional Manifolds}
	
	
\end{thebibliography}

