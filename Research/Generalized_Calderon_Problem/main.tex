\documentclass[12pt]{article}
\usepackage{import}
\usepackage{preamble}
\usepackage{environments}

%hspace before small will move the keywords around
% \providecommand{\keywords}[1]
% {
%     \hspace*{0pt}\small	
%   \textbf{\textit{Keywords--}} #1
% }

\title{Recovering a Banach Algebra of Multivectors from the Dirichlet-to-Neumann Map}
% \renewcommand{\maketitlehookb}{\centering Solving Problems in Applied Mathematics\\
% Colorado State University}
\author{Colin Roberts}
% \date{October 21$^\textrm{st}$ 2019}


\begin{document}

 \begin{titlingpage}
     \maketitle
     \vfill
     \begin{abstract}
        \textcolor{red}{TODO}
     \end{abstract}
 \end{titlingpage}

\section{Introduction}

In 1980, Alberto Calder\'on proposed an inverse problem in his 1980 paper \emph{On an inverse boundary value problem} \cite{calderon_inverse_2006}. The problem asks if one can determine the conductivity matrix of a medium from the Cauchy data supplied on the boundary.  In practice, one determines the Cauchy data from measurements of of voltage and current which often leads to this being referred to as the Electrical Impedance Tomography (EIT) problem.  In this landscape, this mapping is known as the voltage-to-current map. Very much related is the Calder\'on problem for Riemannian manifolds which is introduced in \cite{feldman_calderproblem_nodate} and \cite{salo_calderon_nodate}.  Rather than determining the conductivity matrix from the voltage-to-current map, one attempts to reconstruct the Riemannian metric from the Dirichlet-to-Neumann (DN) map.  In dimension $n=2$, the EIT and Calder\'on problem are not equivalent but are otherwise identical via a change of variables seen in \cite{uhlmann_inverse_2014}.

One approach to reconstructing the Riemannian metric in dimension $n=2$ appears in \cite{belishev_calderon_2003} where Belishev uses the Boundary Control (BC) method to determine the metric up to conformal class. In dimension $n=2$, the Laplace-Beltrami operator is conformally invariant, and this result cannot be improved.  It's also worth noting that this does not provide a solution to the EIT problem since the Calder\'on problem is not equivalent in dimension 2.  Belishev and Vakulenko attempts to move towards generalizing this approach to dimension $n=3$ in \cite{belishev_algebras_2017} and \cite{belishev_algebraic_2019} by replacing the complex structure with a quaternionic structure but this has not lead to a complete solution.

To extend the BC method to higher dimensions, one essentially needs to generalize two important pieces. First, from the DN map, recover the algebra of holomorphic functions on $M$.  This algebra provides means for constructing the complex structure on the manifold. By knowing the complex structure on a surface, one can then determine the Hodge star operator. Then, using theory from Gelfand for commutative Banach algebras, one can equate the topology of the manifold to the topologized spectrum of the algebra of holomorphic functions on that manifold in order to complete the proof. In this paper, we provide a means for generalizing the first component of the BC method to manifolds of arbitrary dimension. 

Given an inner product, the complex structure is naturally isomorphic to the even sub-Clifford algebra of scalars and bivectors in dimension $n=2$ with the inner product acting as the quadratic form.  Fortunately, this can be generalized for arbitrary dimension.  The notion of holomorphicity is then replaced by monogenicity as the Wirtinger derivative $\frac{\partial}{\partial \overline{z}}$ is exchanged by the more general Dirac operator $D$.  The DN map allows one to recover an algebra of conjugate scalars and bivectors whose sum are monogenic functions. From this set, one can generate a Banach algebra of even monogenic Clifford fields on $M$.


\section{Clifford Analysis on Riemannian Manifolds}

\textcolor{red}{Basing a lot of this off of \url{https://math.uchicago.edu/~amathew/dirac.pdf}. Need to cite this somehow.}

Let $(M,g)$ be an $n$-dimensional smooth oriented Riemannian manifold with boundary $\partial M$.  Then we can consider an induced Clifford bundle $\cliffordbundle$ on $M$ with the quadratic form $Q(v)=g(v,v)$.  Specifically, we define $C\ell(T_pM,g_p)$ to be the Clifford algebra on the tangent space $T_pM$ and let
\[
\cliffordbundle \coloneqq \dot{\bigcup}_{p\in M} C\ell (T_pM,g_p)
\]
be the \emph{Clifford algebra bundle}. Locally at any point $p\in M$, we can take the basis for $T_pM$ as $e_1, \dots, e_n$ and at each point we have the \emph{Clifford multiplication} of vectors given by 
\[
e_j e_k = g_p(e_j,e_k) + e_j \wedge e_k, 
\]
where $\wedge$ is the exterior product and $g_p(\cdot,\cdot)$ is the Riemannian inner product. The Clifford multiplication induces a $\mathbb{Z}$-grading in each fiber of the Clifford bundle giving us a vector space of dimension $2^n$.  In particular, we refer to a scalar as a grade-0 element, a vector as a sum of grade-1 elements, a \emph{bivector} is a sum of grade-2 elements, and the \emph{pseudoscalar} as the single grade-$n$ element. A \emph{multivector} is a sum of different grade elements and a \emph{$k$-blade} is a simple grade-$k$ element. Clifford multiplcation extends to vectors $v$ acting on multivectors $w$ via
\[
i_v(w)+v\wedge w,
\] 
where $i$ is the interior multiplication. \textcolor{red}{This is not quite flushed out yet but it may not really be that concerning anyway.} Finally, this is further extended to blades by defining the product on a $p$-blade $A$ and a $q$-blade $B$ by
\[
AB = \proj{p-q}{AB} + \proj{p+q}{AB},
\]
where $p-q$ and $p+q$ are to be taken modulo $n$. Finally, one extends this multiplication to general multivectors and, hence, at each point on $M$ we have attached a Clifford algebra $C \ell (T_pM,g)$ of vector space dimension $2^n$ over $\R$.  We can then define the space of $C^\infty$-smooth sections of the Clifford bundle, $\cliffordfields$, and refer to its elements as the \emph{Clifford fields}. The space of Clifford fields then forms a both a $\mathbb{Z}$- and $\mathbb{Z}/2\mathbb{Z}$-graded $\R$-algebra with an inherited Clifford multiplication from the pointwise product.

Given the metric $g$, we have an isomorphism between $TM$ and $T^*M$ which extends to an isomorphism between the Clifford fields and the corresponding \emph{Clifford forms} $\cliffordforms$.  Then, we can specify the space of grade-$k$ Clifford forms by $\cliffordkforms{k}$. Let $\nabla$ be the Levi-Civita connection extended to forms on $M$, $\omega \in \cliffordforms$, and choose local orthonormal coordinates $x^1,\dots,x^n$ to induce $\partial_j = \frac{\partial}{\partial x_j}$ and $dx^j$.  Then the \emph{exterior derivative} $d$ is given by
\[
d\omega = dx^j \wedge \nabla_{\partial j} \omega,
\]
and the \emph{interior derivative}
\[
d^* \omega = i_{\partial_j} \nabla_{\partial_j}\omega.
\]
Note that we have that $d\colon \cliffordkforms{k} \to \cliffordkforms{k+1}$ and $d^* \colon \cliffordkforms{k}\to \cliffordkforms{k-1}$. Then this gives rise to the \emph{Dirac operator} $D \colon \cliffordforms \to \cliffordforms$ as $D=d+d^*$ and we can note that $D$ is the square root of the Hodge Laplacian $\Delta = D^2 = dd^*+d^*d$. Fields in the kernel of $D$ are referred to as \emph{monogenic}.

The metric also induces the Riemannian volume form $\mu \in \cliffordforms$ on $M$ which is of top degree and hence a pseudoscalar as well as a inner product $\innerproduct{\cdot}{\cdot}$ on each fiber of $\cliffordkforms{k}$. The \emph{Hodge star} operator is then defined to be the unique operator $\star \colon \cliffordkforms{k} \to \cliffordkforms{n-k}$ satisfying
\[
\omega_p \wedge \star \eta_p = \innerproduct{\omega_p}{\eta_p} \mu_p,
\]
at any point $p \in M$ and for any $\omega,\eta \in \cliffordkforms{k}$.  Note that this extends to all of $M$ as
\[
\kforminnerproduct{\omega}{\eta} = \int_M \omega \wedge \star \eta.
\]
Formally, this allows us to see that the interior derivative is the formal adjoint to the exterior derivative as
\[
\kforminnerproduct{d\alpha}{\beta} = \kforminnerproduct{\alpha}{d^*\beta},
\]
and hence $d^* = (-1)^{n(k-1)+1}\star d \star$ where $k$ is the degree of the field being acted on.  Moreover, this shows that both $D$ and $\Delta$ are self adjoint operators.

The algebra of Clifford forms contains the exterior algebra bundle where we can properly recover and extend the \emph{Hodge star} operator $\star \colon C\ell^k(T^*M,g) \to C\ell^{n-k}(T^*M,g)$ such that for any Clifford blades $\alpha$ and $\beta$, we have
\[
\alpha \wedge \star \beta = \langle \alpha,\beta \rangle \omega, 
\]
where $\mu$ is the Riemannian volume form. Note that the Hodge star acts as a \emph{grade reversal} operator on $k$-blades.

As well, there exists the \emph{reversal} operation which reverses the order of the vector elements that build a $k$-blade. That is, for example, given a $3$-blade $\eta = dx^1 \wedge dx^2 \wedge dx^3$, we have the reversal
\[
\eta^t = dx^3 \wedge dx^2 \wedge dx^1.
\]

As well, we have the \emph{main involution} $\alpha\colon C\ell^{k}(T^*M,g) \to C\ell^{k}(T^*M,g)$ which we define by 
\[
\alpha(\eta) = (-1)^k \eta.
\]
Note that this decomposes the space of Clifford fields into two ($\pm 1$) eigenspaces.  This then gives us a $\mathbb{Z}_2$-grading on the space of Clifford fields.  We then refer to the $+1$ eigenspace as the \emph{even Clifford fields} and the $-1$ eigenspace as the \emph{odd Clifford fields}.

From the reversal and main involution, one can generate the \emph{Clifford conjugation} by
\[
\overline{\eta} = \alpha(\eta^t) = \alpha(\eta)^t.
\]

\section{Dirichlet to Neumann Operator and Hilbert Transform}

Let $\iota \colon \partial M \hookrightarrow M$ be the inclusion of the boundary of $M$ into $M$ which induces the pullback $\iota^* \colon T^*M \to T^* \partial M$.  Let $u^\phi \in \cliffordkforms{0}$ be a smooth scalar function that is a solution to the following Dirichlet boundary value problem
\[
\begin{cases} \Delta u^\phi = 0 & \textrm{ in $M$} \\  \iota^* u = \phi. \end{cases},
\]
where $\Delta$ be the Hodge Laplacian.  In the case for the Calder\'on problem, the metric $g$ is unknown and one seeks to determine as much as possible about $(M,g)$ from measurements along the boundary. For any given solution to the boundary value problem, there is the corresponding Neumann data $J=\iota^*(\star d u)$ along the boundary.  The set of both boundary conditions $(\phi, J)$ is the \emph{Cauchy data} and the \emph{Dirichlet-to-Neumann (DN) map} $\Lambda \phi = \iota^*(\star d u^\phi)$. Note that $\Lambda \colon C\ell_0^*(\partial M) \to C\ell_{n-1}^*(\partial M)$. The Calder\'on problem for Riemannian manifolds is then to recover the pair $(M,g)$ up to isometry. 

\textcolor{red}{This is going off of Belishev and Sharafutdinov's definition.}
$\harmonicfunctions = \{u \in \cliffordkforms{0} ~\vert~ du=0\}$ as the space of harmonic functions.  From the DN map, one can define the \emph{Hilbert transform} $T\colon \iota^* \harmonicfunctions \to \iota^* \harmonicfunctions$ as 
\[
T = \star d\Lambda^{-1} \star.
\]
An analogous transform is defined (up to the additional Hodge stars) seen in \cite{belishev_dirichlet_2008} where the authors prove this operator is well defined. As well, the authors show benefit to defining the Hilbert transform as it provides the ability to generate so called conjugate forms.  There, Theorem 5.1 implies that if we have a solution $u^\phi$ to the Dirichlet problem and
\begin{equation}
\label{eq:conjugate_requirement}
\left( \Lambda + (-1)^{n}d\Lambda^{-1}d\right)\phi = 0, 
\end{equation}
then there exists a \emph{conjugate} $b^\psi \in \cliffordkforms{2}$ with boundary trace $\psi = \iota^* g$ satisfying $Td\phi = d \phi$. As well, $b$ is also closed in that $db=0$. \textcolor{red}{double check this because in Sharafitdinov $b$ would be co-closed, but we have an additional star here}  

\begin{lemma}
Given $u^\phi$ and $b^\psi$ conjugate as above, the field
\[
u^\phi + b^\psi
\]
is monogenic.
\end{lemma}
\begin{proof}
\textcolor{red}{There are minus sign issues.} The proof follows immediately from Theorem 5.1.  Indeed, note that they take $\epsilon \in \cliffordkforms{n-2}$ to be co-closed so that $d^*\epsilon =0$.  Let $\star b^\psi = \epsilon$ and note that $b^\psi$ is closed since
\[
0=d^* \epsilon = d^*\star b^\psi = (-1)^{-n+1} \star d \star \star b^\psi = (-1)^{-n+1} \star d b^\psi.
\]
Then, it must be that $du^\phi = \star d \star b^\psi$. It follows that
\[
D(u^\phi + b^\psi) = 0
\]
since
\[
D(u^\phi + b^\psi) = du^\phi +(-1)^{-n+1} \star d \star b,
\]
\end{proof}

\begin{remark}
The above equation $D(u^\phi + b^\psi)=0$ is analogous to the Cauchy-Riemann equations found from the Wirtinger derivative
\[
\frac{\partial}{\partial \overline{z}} (u+ib) = 0.
\]
In the case where $M$ is a region of $\C$, $b$ can be decomposed as some function $v(x_1,x_2)e_1e_2$ where $(e_1e_2)^2=-1$.  
\end{remark}



\section{Recovering a Subalgebra of Monogenic Fields}

Supposing that $u^\phi$ satisfies \ref{eq:conjugate_requirement} one can generate a conjugate field $b^\psi$ since the sum of the two fields must be monogenic.  We can then define the space
\begin{align*}
\conjugates &= \{ u + b ~ \vert ~ \Delta u = 0, ~ (\Lambda +(-1)^n d\Lambda^{-1}d)\iota^* u =0, ~D(u+b)=0\}\\
\end{align*}
where the later set is typically referred to as a \emph{Hardy space of monogenic fields} \textcolor{red}{is this correct?}. Indeed, one has that the DN map reconstructs a Hardy space of monogenic functions defined on $M$.

Given the algebra of Clifford forms $\cliffordforms$, it is quick to show that $\conjugates$ generate an even-graded subalgebra of $\cliffordforms$. Indeed, let 
\[
\alg = \{\textrm{algebra generated by $\conjugates$}\},
\]
and note that $D(af+g)=0$ for any scalar $a\in \R$, and by the Liebniz rule for the Dirac operator, $D(fg)=\dot{D}\dot{f}g+\dot{D}f\dot{g}$, the product of two monogenic fields is also a monogenic field.  Finally, equipped with the norm defined in \ref{eq:clifford_norm} \textcolor{red}{need to define this norm. The inner product of forms should extend to a norm as they give a completion in \cite{belishev_dirichlet_2008} Does this actually give an inner product?}, we have that $\conjugates$ form a Banach algebra. In fact, due to the even grading of each element, the Hodge star forms an involution on $\conjugates$ and hence we have a Banach$^*$ algebra.

\begin{question}
Can we realize this algebra of monogenic even functions via some type of power series? Relate this to spinors?
\end{question}


\section{Relation to the BC Method}

\textcolor{red}{Describe how this process can lead to the BC method in dimension $n=2$}


\section{Conclusion}



\bibliographystyle{siam}
\bibliography{calderon-problem}





\end{document}
