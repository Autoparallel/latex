\documentclass[12pt]{article}
\usepackage{import}
\usepackage{preamble}
\usepackage{environments}
% package for todo 
\setlength{\marginparwidth}{2cm}
\usepackage[colorinlistoftodos]{todonotes}
\setuptodonotes{size=\scriptsize,backgroundcolor=red!15!white} 

%\usepackage{showkeys} %show cites and refs

\title{Commutative Banach Algebras of Multivectors from the Scalar Dirichlet-to-Neumann Operator}
\author{Colin Roberts}



\begin{document}

 \begin{titlingpage}
     \maketitle
     \vfill
     \begin{abstract}
        The problem of determining an unknown Riemannian manifold given the Dirichlet-to-Neumann (DN) operator is known as the Calder\'on problem.  One method of solving this problem in the two dimensional case is through the Boundary Control method.  There, one uses the DN operator to construct a Banach algebra of holomorphic functions on the manifold. The Gelfand transform of this algebra is then homeomorphic to the manifold. In higher dimensions, we replace the complex field with a Clifford algebra and use the DN operator to determine a $\spingroup$ invariant space of monogenic multivector fields. Using a power series representation for monogenic fields, one decomposes the space of monogenics into products of commutative algebras of $(0+2)$-vector fields constant on translations of planes and monogenic in $\R^n$. Using this decomposition, we define spinor characters on the space of monogenic fields that correspond to Dirac measures on the manifold.  The set of these Dirac measures is then homeomorphic to the underlying manifold with the Gelfand topology.
     \end{abstract}
 \end{titlingpage}

\tableofcontents

%\section{Introduction}
%In 1980, Alberto Calder\'on proposed an inverse problem in his paper \emph{On an inverse boundary value problem} \cite{calderon_inverse_2006} where he asks if one can determine the electrical conductivity matrix of some Ohmic medium from knowledge of voltage and current measurements on the boundary of the given domain. This problem goes under the name of Electrical Impedance Tomography (EIT). Physically, the EIT problem is a static 3-dimensional boundary value inverse problem wherein the practicioner has access to voltage to a discrete subset of the boundary of a body and make noisy measurements of the outgoing current flux along the same discrete subset. In other words, a partial and noisy version of the voltage-to-current map is known.

One notes that the voltage-to-current map inputs a scalar potential (the Dirichlet data) on the boundary which, if the conductivity matrix was known, would allow one to determine the potential on the interior by solving a second order elliptic partial differential equation with coefficients defined by the conductivity. In an Ohmic material, the current field is induced by from the electric field (the gradient of the potential) and the conductivity, and hence for a fixed conductivity, utilizing different Dirichlet data can induce different current fields in the body. The practicioner has the ability to measure the outgoing current flux along the boundary (the Neumann data) and thus, they have access to the voltage-to-current map.   

This problem can be generalized naturally into dimensions $\geq 2$ as a geometric inverse problem. One replaces the medium with a manifold and the conductivity becomes the Riemannian metric. Thus, the second order elliptic equation from before amounts to finding scalar fields in the kernel of the Laplace-Beltrami operator. This leads to a small caveat that in dimension 2 since the Laplace-Beltrami operator is conformally invariant. At any rate, the EIT problem is equivalent to determining an unknown Riemannian manifold up to isometry from the classical Dirichlet-to-Neumann (DN) map which inputs a scalar field and outputs the outward normal component of the derivative of the solution \cite{feldman_calderproblem_nodate, salo_calderon_nodate, uhlmann_inverse_2014}. 

There are a handful of approaches to solving this problem, but it remains unsolved. In order to make progress, theorists have allowed themselves access to larger sets of data, for example, complete knowledge of a generalized DN map on differential forms \cite{krupchyk_inverse_2011,sharafutdinov_complete_2013,belishev_dirichlet_2008,joshi_inverse_nodate}. In dimension 2, the smooth problem has been solved up to conformal invariance and in dimension $\geq 3$, the problem has been solved for analytic manifolds \cite{lassas_determining_2001}. Another approach that is unique to a manifold of dimension 2 appears in \cite{belishev_calderon_2003}. In this paper, Belishev determines the algebra of holomorphic functions from the DN map and realizes the spectrum of this algebra homeomorphic to the underlying manifold by Gelfand. The metric $g$ is then recovered up to conformal class by extracting the complex structure from this algebra as well. An attempt to generalize this approach to dimension $n=3$ can be found in by replacing the complex structure with a quaternionic structure but this has not lead to a complete solution \cite{belishev_algebras_2017, belishev_algebraic_2019}. It has been shown that the 3-dimensional round ball can be determined up to homeomorphism from a quaternionic spectrum. Belishev and Vakulenko ask whether this can be extended to higher dimensions and to other spaces. An answer to this question is provided by \cref{thm:gelfand}.

In this work, I first introduce the geometric algebras $\G$ as special cases of more general Clifford algebras in \cref{subsec:clifford_and_geometric_algebras}. Following this, I take a smooth, oriented, Riemannian manifold $M$ and construct a Clifford algebra bundle whose sections lie in the space $\G(M)$ and are referred to as multivector fields in \cref{sec:geometric_manifolds}. The graded algebraic structure of $\G(M)$ expands upon the exterior algebra of forms $\Omega(M)$, and moreover, there exists a natural differential structure via the gradient operator $\grad$, which finds similarities to the Hodge-Dirac operator $d+\delta$. The space $\G(M)$ proves to be more rich than $\Omega(M)$ since one can realize $\Omega(M)$ as a trivial case of some $\G(M)$. Moreover, it is quite natural to look at multivector fields that consist of many differently graded elements at once. For example, multivector fields that lie in the kernel of $\grad$ are called monogenic and these fields share many of the same properties as holomorphic functions on $\C$ including, but not limited to, a Cauchy integral formula \cref{eq:cauchy_integral}. However, unlike holomorphic functions, the space of monogenic fields $\monogenicfields{}$ is not, in general, commutative or an algebra.

A useful version of a Green's formula is shown in \cref{thm:multivector_greens_formula}, and this allows us to prove a multivector version of the Hodge-Morrey decomposition that we realize in the following theorem.
\begin{customthm}{3.1.1}[Monogenic Hodge Decomposition]
The space of multivector fields $\G(M)$ has the $L^2$-orthogonal decomposition
\begin{equation}
\G(M) = \monogenicfields{} \oplus \pseudoscalar^{-1} \grad \G(M).
\end{equation}
\end{customthm}

The space $\monogenicfields{}$ is a right module over the constant multivectors $\G$. For the special case of the Euclidean geometric algebra $\G_n$, I define a space of module homomorphisms from $\monogenicfields{}$ to $\G_n$ and refer to these morphisms $\G_n$-functionals. Inside $\monogenicfields{}$ lie commutative subalgebras $\algebra{\bivector}(M)$ that are analogs of $\C$ and on these algebras we can define $\G_n$-characters as the $\G_n$-functionals that are also algebra morphisms on each $\algebra{\bivector}(M)$ into $\G_n$. The space of $\G_n$-characters, $\characters(M)$, with the weak-$\ast$ topology, is shown to be homeomorphic to $M$ in the special case where $M$ is a region of $\R^n$ and inherits the Euclidean metric. This is summarized in the following theorem.
\begin{customthm}{3.3.1}
For any $\delta \in \characters(M)$, there is a point $x^\delta \in M$ such that $\delta(f) = f(x^\delta)$ for any $f\in \monogenics(M)$ a monogenic field. Given the weak-$\ast$ topology on $\dualmonogenics(M)$, the map
\[
\gamma \colon \characters(M) \to M, \quad \delta \mapsto x^\delta
\]
is a homeomorphism. 
%The Gelfand transform 
%\[
%\widehat{~} \colon \monogenics(M) \to C(\characters(M); \G_n), \quad \widehat{f}(\delta) \coloneqq \delta(f), \quad \delta \in \characters(M),
%\]
%is an isometry onto its image, so that $\characters(M)$ is isomorphic to $\widehat{\monogenics(M)}$ as algebras.
\end{customthm}

Owing to the original intention of this work, I consider physical and geometric inverse boundary value problems related to the Calder\'on problem. For example, I discuss the electric and magnetic impedance tomography problems and their statements in terms of multivector fields \cref{sec:tomography}. Relationships between the two problems are established, and one finds that the Ohmic property of a medium couples together the scalar potential $u$ and the magnetic bivector field $b$ into a single monogenic field. Given knowledge of the electrostatic and magnetostatic version of the DN map alongside this new relationship, can one determine the underlying conductivity? Likewise, in higher dimensions, do \cref{thm:gelfand,thm:monogenic_hodge} provide new tools for solving the Calder\'on or other related inverse problems? Finally, there also exists a Hilbert transform in two guises via \cite{belishev_dirichlet_2008,brackx_hilbert_2008}. Are these two notions equivalent? Does either add any more useful information for solving boundary inverse problems?

\section{Preliminaries}
\subsection{Clifford algebras}
The complex algebra $\C$ can be generalized in a handful of ways.  Some of which can be found through the use of Clifford algebras and, more specifically, in geometric algebras.  We define the more general Clifford algebras first and realize geometric algebras as particularly nice Clifford algebras with a quadratic form arising from an inner product. Elements of a geometric algebra are known as multivectors and these multivectors carry a wealth of geometric information in their algebraic structure. $\C$ itself can be realized as a special subalgebra of biparavectors in the geometric algebra on $\R^2$ with the Euclidean inner product and the quaternions $\quat$ are realized as an analogous algebra on $\R^3$. In particular, both $\C$ and $\quat$ arise as the 2- and 3-dimensional even Clifford groups $\Gamma^+$ respectively.

Formally, we let $(V,Q)$ be an $n$-dimensional vector space $V$ over some field $K$ with an arbitrary quadratic form $Q$.  The tensor algebra is given by
\[
\mathcal{T}(V) \coloneqq \bigoplus_{j=0}^\infty V^{\otimes j} = K \bigoplus V \oplus (V\otimes V) \oplus (V\otimes V \otimes V) \oplus \cdots,
\]
where the elements (tensors) inherit a multiplication $\otimes$ (the tensor product). From the tensor algebra $\mathcal{T}(V)$, we can quotient by the ideal generated by $v\otimes v - Q(v)$ to define \emph{Clifford algebra} $C\ell(V,Q)$. That is, 
\[
C\ell(V,Q) = \mathcal{T}(V) ~ / ~ \langle v \otimes v - Q(v) \rangle.
\]
To see how the tensor product descends to the quotient, we let $e_1, \dots, e_n$ be an arbitrary basis for $V$, then we can consider the tensor product of basis elements $e_i \otimes e_j$ which induces a product in the quotient $C\ell(V,Q)$ which we refer to as the \emph{Clifford multiplication}. In this basis, we write this product as concatenation $e_ie_j$ and define the multiplication by
\[
e_i e_j = \begin{cases} Q(e_i) & \textrm{if $i=j$}, \\ e_i \wedge e_j & \textrm{if $i\neq j$},\end{cases}
\]
where $\wedge$ is the typical exterior product satisfying $v\wedge w = - w\wedge v$ for all $v,w\in V$.  As a consequence, the exterior algebra $\bigwedge(V)$ can be realized as a subalgebra of any Clifford algebra over $V$ or as a Clifford algebra with a trivial quadratic form $Q=0$.  

Note that $C\ell(V,Q)$ is a $\mathbb{Z}$-graded algebra with elements of grade-0 up to elements of grade-$n$. We refer to grade-0 elements as scalars, grade-1 elements as vectors, grade-2 elements as \emph{bivectors}, grade-$k$ elements as \emph{$k$-vectors}, and grade-$n$ elements as \emph{pseudoscalars}. We denote the space of $k$-vectors by $C\ell(V,Q)^k$. For each grade there is a basis of ${n\choose k}$ \emph{$k$-blades} which are $k$-vectors of the form
\[
A_k = \prod_{j=1}^k v_j, ~\textrm{for}~ v_j \in V.
\]
For example, if $\dim(V)=3$, then there are ${3\choose 2}=3$ 2-blades that form a basis for the bivectors. One particular choice given our vector basis of $V$ would be the following list of 2-blades
\[
\label{eq:3_dim_basis}
B_{12} = e_1 \wedge e_2, \quad B_{13} = e_1 \wedge e_3, \quad B_{23} = e_e \wedge e_3.
\]
We refer to an $(n-1)$-blade as a \emph{pseudovector} and it should be noted that every $(n-1)$-vector is a pseudovector. In other literature, some will refer to a $k$-blade as a \emph{simple} or a \emph{decomposable} $k$-vector. 

In general, an element $A \in C\ell(V,Q)$ is written as a linear combination of basis elements of all possible grades and we refer to $A$ as a \emph{multivector}.  To extract the grade-$k$ components of $A$, we use the notation
\[
\proj{k}{A}
\]
to denote the grade-$k$ components of the multivector $A$. Any multivector $A$ can then be given by
\[
A = \sum_{k=0}^n \proj{k}{A}
\]
which shows the decomposition
\[
C\ell(V,Q) = \bigoplus_{j=0}^n C\ell(V,Q)^j.
\]
For example, $A\in C\ell(\R^3, \|\cdot \|)$ is given by
\[
A= a + \alpha_1 e_1 + \alpha_2 e_2 + \alpha_3 e_3 + \beta_{12} B_{12} + \beta_{13} B_{13} + \beta_{23} B_{23} + r e_1 \wedge e_2 \wedge e_3
\]
in general, and we have
\[
\proj{0}{A}=a, \quad \proj{1}{A}=\alpha_1 e_1 + \alpha_2 e_2 + \alpha_3 e_3, \quad \proj{2}{A}=\beta_{12} B_{12} + \beta_{13} B_{13} + \beta_{23} B_{23}, \quad \proj{3}{A} = re_1 \wedge e_2 \wedge e_3.
\]
If $A$ contains only grade-$k$ components, then we say that $A$ is \emph{homogeneous}.  For example, when we refer to vectors we realize them as homogeneous grade-1 multivectors and likewise we realize bivectors as homogeneous grade-2 multivectors. We also refer elements in
\[
 C\ell(V,Q)^{0+2} = C\ell(V,Q)\oplus C\ell(V,Q)^2
\]
as \emph{biparavectors}.

The Clifford multiplication of vectors can be extended to multiplication of vectors with homogeneous grade-$k$ multivectors.  In particular, given a vector $v \in C\ell(V,Q)$ and a homogeneous grade-$k$ multivector $A_k \in C\ell(V,Q)$, we have
\begin{equation}
\label{eq:vector_multiplication}
aA_k = \proj{k-1}{aA_k} + \proj{k+1}{aA_k},
\end{equation}
which decomposes the multiplication into a grade lowering \emph{interior product} and a grade raising \emph{exterior product}.  This allows us to extend the Clifford multiplication further. Given a homogeneous grade-$s$ multivector $B_s$, we have
\begin{equation}
\label{eq:general_clifford_multiplication}
A_k B_s = \proj{|r-s|}{A_rB_s} + \proj{|r-s|+2}{A_rB_s} + \cdots + \proj{r+s}{A_rB_s}.
\end{equation}
This rule for multiplication then allows for the multiplication of two general multivectors in $C\ell(V,Q)$. 

Some specific graded elements of the above product are worth noting here, 
\begin{equation}
    A_k \cdot B_s \coloneqq \proj{|k-s|}{A_k B_s}
\end{equation}
\begin{equation}
    A_k \wedge B_s \coloneqq \proj{k+s}{A_k B_s}
\end{equation}
\begin{equation}
\label{eq:left_contraction}
    A_k \rfloor B_s \coloneqq \proj{s-k}{A_k B_s}
\end{equation}
\begin{equation}
\label{eq:right_contraction}
    A_k \lfloor B_s \coloneqq \proj{k-s}{A_k B_s}.
\end{equation}
These products are particularly emphasized as many helpful identities used in this paper are phrased using these notions. Another key reason behind the additional multiplication symbols $\rfloor$ and $\lfloor$ is to avoid needing to pay special attention to the specific grade of each multivector in a product.  The product $\cdot$ on $A_k$ and $B_s$ depends on $k$ and $s$ and as such given by either $\rfloor$ or $\lfloor$ but one must know $k$ and $s$ in order to define this product exactly. 

We also have the identities
\begin{equation}
\label{eq:left_contraction_dot}
A_r \cdot B_s = A_r \rfloor B_s \qquad \textrm{if $k\leq s$}
\end{equation}
\begin{equation}
\label{eq:right_contraction_dot}
A_r \cdot B_s = A_r \lfloor B_s \qquad \textrm{if $k\geq s$}.
\end{equation}
For homogeneous $k$-vectors $A_k$ and $B_k$, the products above simplify to 
\begin{equation}
\label{dot_equivalent_contraction}
    A_k \lfloor B_k = A_k \rfloor B_k = A_k \cdot B_s.
\end{equation}
Using this notation, for a vector $\alpha$ we have
\begin{equation}
\alpha A_k = \alpha \rfloor A_k + \alpha \wedge A_k,
\end{equation}
so the $\cdot$ and $\lfloor$ notation coincide for left multiplication by vectors. If we are given two $k$-blades $A_k = \alpha_1 \wedge \cdots \wedge \alpha_k$ and $B_k = \beta_1 \wedge \cdots \wedge \beta_k$ we have 
\begin{equation}
\label{eq:dot_product}
A_k \cdot B_k^\dagger = \det(\alpha_i \cdot \beta_j )_{i,j=1}^k,
\end{equation}
which is equivalent to $A_k \rfloor B_k$ and $A_k \lfloor B_k$ through \ref{dot_equivalent_contraction} and this is extended to all $k$-vectors as is typically seen when constructing the inner product of $k$-vectors (see \cite{hestenes_clifford_1986}. If we are given two bivectors $B$ and $B'$, then we have another special multiplication
\begin{equation}
\label{eq:bivector_product}
B\times B' \coloneqq \proj{2}{BB'} = \frac{1}{2} (BB' - B'B),
\end{equation}
which is the grade preserving anti-symmetric portion of the product $BB'$.

As discussed, $C\ell(V,Q)$ is naturally a $\mathbb{Z}$-graded algebra but we also find that it carries a $\mathbb{Z}/2\mathbb{Z}$-grading as well. This additional grading can be realized by sorting $k$-vectors in $C\ell(V,Q)$ into the sets where $k$ is even or odd.  We say a $k$-vector is \emph{even} (resp. \emph{odd}) $k$ is even (resp. odd) and in general if a multivector $A$ is a sum of only even (resp. odd) grade elements we also refer to $A$ as even (resp. odd).  Taking note of the multiplication defined in \ref{eq:general_clifford_multiplication}, one can see that the multiplication of even multivectors with another even multivectors outputs an even multivector.  Thus, the even multivectors form closed subalgebra of $C\ell(V,Q)$ which we denote by $C\ell(V,Q)^+$.

\begin{example}
\label{ex:complex_representation}~
\begin{itemize}
    \item Let $V=\R^2$ and let the quadratic form $Q$ be given by the Euclidean norm $Q(\cdot)=\|\cdot\|$.  Let $e_1$ and $e_2$ be the standard unit vectors and note that we have $1$ as the basis scalar, and $B_{12} = e_1\wedge e_2 = e_1e_2$ as the basis pseudoscalar.  Thus, a general multivector $m$ and $r$ can be written as
\[
m = m_0 + m_1 e_1 + m_2 e_2 + m_{12} B_{12}, \qquad r = r_0 +r_1 e_1 + r_2 e_2 + r_{12}B_{12}.
\]
We can then multiply $mr$ and find
\[
\proj{0}{mr} = m_0r_0 + m_1 r_1 + m_2 r_2 - m_{12}r_{12},
\]
\[
\proj{1}{mr} = (m_0 r_1 + m_1 r_0 - m_2 r_{12} + m_{12} r_2) e_1 + (m_0 r_2 + m_2 r_0 + m_1r_{12} - m_{12} r_1) e_2,
\]
and
\[
\proj{2}{mr} = (m_1r_2 - m_2 r_1)B_{12}.
\]

Most notably, we see that $B_{12}^2=-1$ and this allows us to consider a biparavector
\[
z = x + y B_{12} 
\]
as a representation of the complex number $\zeta = x+ iy$ in $\G_n^{0+2}$.  Thus, the even subalgebra of this Clifford algebra is indeed isomorphic to the complex numbers $\C$. 

    \item If $V =\R^n$, with $n\geq 2$, and with the analogous $Q$, then there are natural copies of $\C$ contained inside of $C\ell(V,Q)$. In particular, we have the isomorphism
    \[
        \C \cong \{\lambda + \beta B ~\vert~ \lambda,\beta \in C\ell(V,Q)^0,~ B \in C\ell(V,Q)^2,~ B^2=-1. \},
    \]
   which shows that complex numbers arise as biparavectors. Given the standard basis $e_1,\dots,e_n$ we have copies of $\C$ for each of the ${ n \choose 2}$ unit bivectors $B_{jk}$ with $k=2,\dots,n$ and $j<k$. Note that $B_{jk}B_{jk}=-1$ and we have the representation of $\C$ since
    \[
        \zeta = x + yB,
    \]
    behaves as a complex number $z=x+iy$.
\end{itemize}
\end{example}

\begin{example}
\label{ex:quaternions}
Let $V=\R^3$ and $Q(\cdot)=\|\cdot \|$.  Then, let
\[
B_{23} = e_2 e_3, \quad B_{31} = e_3 e_1, \quad B_{12} = e_1 e_2,
\]
and note that we can write a even multivector as
\[
q = a + \beta_{23} B_{23} + \beta_{31} B_{31} + \beta_{12} B_{12}.
\]
Note as well that
\[
B_{23}^2 = B_{31}^2 = B_{12}^2 = -1,
\]
and
\[
B_{23}B_{31}B_{12} = +1.
\]
In this case, this even subalgebra is extremely close to being a copy of the quaternion algebra $\quat$. Indeed, one can arrive at a representation of the quaternions by taking
\[
\boldsymbol{i} \leftrightarrow B_{23}, \quad \boldsymbol{j} \leftrightarrow -B_{31}=B_{13}, \quad \boldsymbol{k} \leftrightarrow B_{12},
\]
and noting that we then have $ijk=-1$ as well as $i^2=j^2=k^2=-1$. A more in depth explanation is provided in \cite{doran_geometric_2003}.

Once again, quaternions arise naturally as parabivectors since we can put
\[
q= \alpha + u_1B_{23} - u_2 B_{13} + u_3 B_{12},
\]
and recover the necessary arithmetic seen in $\quat$.
\end{example}

In the case where $V$ has a (pseudo) inner product $g$, we can induce a quadratic form $Q$ by $Q(v)=g(v,v)$ and give rise to a Clifford algebra $C\ell(V,Q)$.  This is a special case and we refer to this type of Clifford algebra as a \emph{geometric algebra}. We generally put $\geometricalg$ and assume the inner product will be given alongside or will be clear from context.  For example, when $V=\R^n$ and we define $Q$ from the Euclidean inner product, we have $C\ell(V,Q)=\mathcal{G}(\R^n)$ and moreover we put $\mathcal{G}(\R^n)=\mathcal{G}_n$. For more information on the topic of geometric algebras see the classical text \cite{hestenes_clifford_1986} or the text \cite{doran_geometric_2003} which also provides a wide range of applications to physics problems. Both these sources include much of the other necessary preliminaries I cover in the remainder of this section. Finally, the paper \cite{chisolm_geometric_2012} proves many of the useful identities I claimed above.

\begin{example}
\label{ex:spacetime_algebra}
If instead we take $V=\R^4$ we take the vector basis $e_t,e_1,e_2,e_3$ with the pseudo inner products
\[
e_t \cdot e_t = -1 \qquad e_t \cdot \gamma_i =0 ~i=1,2,3, \quad e_i \cdot e_j =\delta_{ij}.
\]
For this basis, we can denote the matrix for this inner product $\eta =\operatorname{diag}(-+++)$ and define $Q$ from $\eta$. Then, we have for a vector $A = A_t e_t +A_1 e_1 + A_2 e_2 + A_3 e_3$ we have
\[
A\cdot A = -A_t^2 + \sum_{i=1}^3 A_i^2.
\]
\end{example}

For the cases with pseudo inner products with $p$ vectors satisfying $e_i^2 = -1$ for $i=1,\dots, p$ and $q$ vectors satisfying $e_j^2=1$ for $q=p+1,\dots,p+q$, we will denote the algebras by $\G_{p,q}$.

\subsubsection{Duality and pseudoscalars}
\label{subsection:duality_and_pseudoscalars}

For the remainder of this paper we will be working with geometric algebras with a positive definite inner product $g$. Given access to an inner product we have a natural isomorphism between $V$ and $V^*$ by the Riesz representation.  Namely, given an arbitrary basis $e_i$ for $V$ there exists the dual basis $f_i$ for $V^*$ such that $f_i(e_j)=\delta_{ij}$.  This dual basis resides inside $V$ itself in the following manner. There is then a unique map $\sharp \colon V^* \to V$ with $f\mapsto f^\sharp$ such that
\[
f_i^\sharp \cdot e_j = \delta_{ij},
\]
where $\delta_{ij}$ is the Kronecker delta symbol. In terms of the geometric algebra, we put $e^i \coloneqq f_i^\sharp$ and can note that $e^i$ is simply a vector in the geometric algebra. For an arbitrary basis $e_1,\dots,e_n$ for $V$, the coefficients for the inner product $g$ are given by $g_{ij}=e_i\cdot e_j$ and we can put $e^i = g^{ij}e_j$ where $g^{ij}$ is the coefficients to matrix inverse of $g_{ij}$.  There is inverse isomorphism $\flat \colon V \to V^*$ given by $e \mapsto e^\flat$ satisfying
\[
e_i^\flat (e_j)= \delta_{ij}.
\]
Given these identifications, there is no need to distinguish between the vector space $V$ and its dual $V^*$ as it suffices to consider $V$ itself with reciprocal basis elements $e^i$ with the application of the scalar product.

A volume element can be defined by $\mu=e_1 \wedge e_2 \wedge \cdots \wedge e_n = \sqrt{|g|} I$ where $\sqrt{|g|}$ is the square root of the determinant of the matrix $g_{ij}$ and $I$ is the unit pseudoscalar. It follows that the unit pseudoscalar is given by $I=\frac{1}{\sqrt{|g|}} e_1 \wedge e_2 \wedge \cdots e_n$. We can define $\mu^{-1}$ such that $\mu^{-1}\mu = 1 = \mu \mu^{-1}$ and analogously $I^{-1}$.  One can equivalently put $e^j = (-1)^{j-1} e_1 \wedge e_2 \wedge \cdots \wedge \breve{e_j} \wedge \cdots \wedge e_n \mu^{-1}$ and note that this gives $\mu^{-1} = e^n \wedge \cdots \wedge e^1$.  Conveniently, the unit pseudoscalar satisfies the relation
\[
IA_k = (-1)^{k(n-1)} A_k I.
\]
Thus, $I$ commutes with the even subalgebra, and anticommutes with the odd subalgebra.  Moreso, the pseudoscalar allows one to exchange the interior and exterior products as
\begin{equation}
\label{eq:wedge_to_dot}
 (A_k \wedge B_s) I = A_k \cdot (B_s I)
\end{equation}
for homogeneous $k$ and $s$-vectors $A_k$ and $B_s$. The above holds true if we replace $I$ with $I^{-1}$ when working in spaces where $g$ is positive definite due to the fact that $I^{-1}$ differs only by a sign. If $B_s = C_{n-s}I$ then,
\begin{align*}
 (A_k \cdot B_s)I^{-1} = A_k \cdot (C_{n-s}I) = (A_k \wedge C_{n-s})I = (A_k \wedge (B_sI))I,
\end{align*}
and in particular
\begin{equation}
\label{eq:dot_to_wedge}
    (A_k \cdot B_s)I^{-1} = A_k \wedge (B_s I).
\end{equation}
This shows the duality between the inner and exterior products. The duality extends further to provide an isomorphism between the spaces of $k$-vectors and $(n-k)$-vectors. For any $k$-vector $A_k$, we can take $A_k I^{-1}=B_{n-k}$ to get the corresponding $(n-k)$-vector $B_{n-k}$. It is under this isomorphism one can see that all pseudovectors are $(n-1)$-blades. 

\begin{example}
\label{ex:cross_product}
Consider $\spacealg$ with the standard orthonormal vector basis $e_1,\dots,e_n$ and Euclidean inner product.  Then, we can define the \emph{cross product} of two vectors $u$ and $v$ by
\[
u \times v = (u\wedge v)I^{-1}.
\]
The special fact of $\spacealg$ is that vectors and bivectors (pseudoscalars in 3-dimensions) are dual to one another. One can also note that the vector $w=u\times v$ is sometimes refered to as axial and in other cases the pseudovector $u\wedge v$ is referred to as axial. 

The $\times$ symbol is now overloaded from the bivector definition we saw prior to this example.  But, referring back to Example \ref{ex:quaternions}, we can realize the cross product of vectors as the antisymmetric product of bivectors
\[
(uI^{-1})\times (vI^{-1}). 
\]
The necessary relationships for the cross product are seen clearly on the products of the basis blades $B_{23}, B_{31}$, and $B_{12}$. In particular, $e_1 = B_{23}I^{-1}$, $e_2 = B_{31} I^{-1}$, and $e_3 = B_{12} I^{-1}$.
\end{example}

\subsubsection{Reverse, inverses, and the Clifford and spin groups}

We had used the notation $~^{-1}$ to denote the inverse for the pseudoscalar, but there are other invertible elements in a geometrical algebra.  In particular, all blades are invertible. From this, we can construct a group of all invertible elements referred to as the \emph{Clifford group} $\Gamma$ for a geometric algebra $\G$ by
\[
\Gamma \coloneqq \left\{ \prod_{j=1}^k v_i ~\vert~ k\in \mathbb{Z}^+,~ \forall j~\colon1\leq j \leq k~\colon~v_i \in \R^n ~\textrm{such that}~|v_i|\neq 0\right\}.
\]
We refer to elements of the Clifford group as \emph{Clifford multivectors}. For any Clifford multivectors $A = v_1 \cdots v_k$ in the group $\Gamma$, we have that multiplicative inverse $A^{-1}$ is given by $A^{-1} = v^k \dots v^1$ as we can see that $A^{-1}A=AA^{-1} = 1$ by construction.  Of note is the fact that all scalars, vectors, pseudovectors, and pseudoscalars are always in the Clifford group and have multiplicative inverses. The inverse of a vector $v$ is given by $\frac{v}{v\cdot v}$. It becomes useful to define the \emph{reverse} $\dagger$ such that $A^\dagger = v_k \cdots v_1$. For a $k$-blade $A_k$, the reverse also satisfies the relationship
\begin{equation}
\label{eq:reverse_sign}
A_k^\dagger = (-1)^{k(k-1)/2} A_k.
\end{equation}
One can then see that the inverse for the unit pseudoscalar is $I^{-1}=I^\dagger$ which is an identification I will often use. One can see that the multiplicative inverse of an element of the Clifford group $A$ is the reverse of the corresponding product of reciprocal vectors since $A_k^{-1} = (v^1 \cdots v^k)^\dagger$. Note as well that elements $s \in \Gamma^+$ act as rotations on $A\in \G_n$ given the conjugate action
\[
A \mapsto s A s^{-1}.
\]
In fact, all nonzero vectors $v\in\Gamma$ define a reflection in the hyperplane perpendicular to $v$ via the same conjugation action above.

Following these realizations, one can see that the Clifford group contains important subgroups such as the orthogonal and special orthogonal groups as quotients
\[
\operatorname{O}(n) \cong \Gamma/\R \qquad \textrm{and} \qquad \operatorname{SO}(n) \cong \Gamma^+ /\R.
\]
This motivatives the following definition.
\begin{definition}
    The \emph{Clifford norm} $\| \cdot \|$ for $s \in \Gamma$ is given by
    \[
    \|s\|^2 \coloneqq ss^\dagger.
    \]  
\end{definition}
Note that for vectors the Clifford norm corresponds with the norm induced from the inner product in that with a vector $v$ we have $\|v\|=vv^\dagger = v\cdot v$. We also give the name \emph{unit} to $k$-blades $A_k$ with unit spinor norm $1=\|A_k\|$. We can also see that 
\begin{equation}
\label{eq:pseudoscalar_norm}
\|\mu\| = \sqrt{|g|},
\end{equation}
and so
\[
\|I\| = 1.
\]

With this, we have the \emph{pin} and \emph{spin groups}
\begin{align*}
    \operatorname{Pin}(n) &\coloneqq \{s\in \Gamma ~\vert~ \|s\|=1\}.\\
    \operatorname{Spin}(n) &\coloneqq \{s\in \Gamma^+ ~\vert~ \|s\|=1\}.
\end{align*}
Moreover,
\[
\operatorname{Pin}(n) \cong \Gamma/\R^+ \qquad \textrm{and} \qquad \operatorname{Spin}(n) \cong \Gamma^+/\R^+.
\]

The spin group $\spingroup$ is a Lie group and its associated Lie algebra is denoted by $\spinalgebra$. In particular, the $\spinalgebra$ is isomorphic to the algebra of bivectors with the antisymmetric product $\times$ \todo[inline]{provide a citation.}.  Then, for any bivector $B$, we have an element in the spin group given by
\[
e^{B} = \sum_{j=0}^\infty \frac{B^n}{n!}.
\]
Fundamentally, $\spingroup$ acts on the even subalgebra $\G_n^+$. A \emph{spinor} $\psi$ is an element that transforms under a left action of an element of $\spingroup$ to produce another spinor.  In terms of geometric algebra, a spinor is simply an even multivector (i.e., an element of $\G_n^+$).  Of note are the two cases we have had as examples before. 

\begin{example}
\label{ex:exponential_of_bivector}
    Consider $\G_2$ and note that we have shown the algebra of spinors $\G_2^+$ is isomorphic to the complex numbers $\C$.  Indeed, there is one unit 2-blade $B_{12}$ in $\G_2$ to form the spin algebra $\mathfrak{spin}(2) \cong \R$ and as a consequence all unit norm elements in $\G_2^+$ can be written as
    \[
       e^{\theta B_{12}} = \sum_{n=0}^\infty \frac{\theta B_{12}}{n!} = \cos(\theta)+B_{12}\sin(\theta),
    \]
    where $\theta B_{12}$ is a general bivector in $\G_2$.  Hence, we arrive at $\operatorname{Spin}(2)\cong \operatorname{U}(1)$. Any element in $\C$ is also a scaled version of an element of the spin group $\operatorname{Spin}(2)$. Hence, we can use a spin representation for an element in $\C$ via $z=re^{\theta B_{12}} \in \R\times \operatorname{Spin}(2)$.  This special case shows that parabivectors in $\G_2$ have a unique spin representation and they are spinors as well.
\end{example}

\begin{example}
    Consider $\G_3$ and note that we have shown the spinors $\G_3^+$ are isomorphic to the quaternion $\quat$ algebra.  We also realize $\quat$ as scalar copies of elements of $\operatorname{Spin}(3) \cong \operatorname{Sp}(1)$.  That is to say that $\quat \cong \R \times \operatorname{Spin}(3)$. Indeed, since elements of $\G_3^+$ are simply biparavectors, the biparavectors once again admit a natural spin representation. Likewise, \todo[inline]{finish this.}
\end{example}

\subsubsection{Projection and rejection}

There is a direct relationship between unit $k$-blades and $k$-dimensional subspaces.  Indeed, each unit $k$-blade $B_k$ ($\|B_k\|=1$) corresponds to a $k$-dimensional subspace.  That is, each point in $\Grassmannian{k}{n}$ corresponds to a unit $k$-blade.  Since blades represent subspaces, they also give us a compact way of projecting multivectors into subspaces.  In general, given an multivector $A$ the \emph{projection} onto the subspace spanned by $B_k$ is
\begin{equation}
\label{eq:projection}
\projection{B_k}{A} \coloneqq (A\rfloor B_k)B_k^{-1}.
\end{equation}
By definition, we have
\[
\projection{B_k}{A} \in \bigoplus_{j=0}^k \G_n^j = \G_n^{0+\cdots + k}
\]
Specifically,
\[
\projection{B_k}{\proj{j}{A}} \in G_n^j,
\]
shows the projection preserves grades.

G a vector $v$, the projection onto the subspace spanned by the $k$-blade $A_k$ is given by the identity
\begin{equation}
\label{eq:vector_projection}
(v\rfloor A_k )A_k^{-1} = (v\rfloor A_k)\rfloor A_k^{-1} = (v\cdot B_k)\cdot B_k^{-1}.
\end{equation}
and more enlightening is to take a projection of a vector $v$ onto another vector $u$
\[
(v\rfloor u)u^{-1} = (v \cdot u) \frac{u}{\|u\|^2},
\]
which is the expected result. 

A dual notion also exists.  We define the \emph{rejection} of a multivector from the subspace spanned by $B_k$ as
\begin{equation}
\label{eq:rejection}
\rejection_{B_k}(A) = (A\wedge B_k)B_k^{-1}.
\end{equation}
In the case we have a vector $v$, we can note
\begin{equation}
\label{eq:projection+rejection_vector}
\projection{B_k}{v} + \rejection_{B_k}(v) = v.
\end{equation}

To see this in action, we let $v=v^1 e_1 + v^2 e_2 + v^3 e_3$ and let $B_{12}=e_1 e_2$ and note
\begin{align*}
    \rejection_{B_{12}}(v) &= [(v^1 e_1 + v^2 e_2 + v^3 e_3)\wedge (e_1 e_2)]B_{12}^{-1}\\
    &= v^3 e_3 e_1 e_2 e^2 e^1 \\
    &= v^3 e_3.
\end{align*}
Both the notion of projection and rejection prove to be useful. For vectors $u$ and $v$, we can find
\begin{equation}
\label{eq:projection_rejection_vector}
\projection{uI^{-1}}{v} = \rejection_u(v).
\end{equation}

\todo[inline]{Change the above equation slightly for the use in the Calderon section. Can we use rejection to simplify the $B$-planar proofs}


\subsection{Multivector fields}

We want to generalize the setting of geometric algebra to include a smooth structure. One can take the work above for $\mathcal{G}_n$ and consider a $C^{\infty}$-module structure as opposed to the $\R$-algebra structure in the proceeding section. For brevity, we put $\mathcal{G}_n(\R^n)$ for the $C^\infty$-module and $\G_n$ for the $\R$-algebra. The multivectors themselves can be realized as constant multivector fields so that $\G_n \subset \G_n(\R^n)$. This smooth setting simply makes the coefficients of the global basis blades given by $C^\infty$ functions as opposed to $\R$ scalars.  In this case, we refer to a generic element in the $C^{\infty}$-module $\mathcal{G}_n$ as a \emph{multivector field}. We take $\Omega \subset \R^n$ as a connected region in $\R^n$ for the entirety of this paper and we put
\[
\G_n(\Omega) \coloneqq \{ f \colon \Omega \to \G_n ~\vert~ \textrm{$f$ is $C^\infty$-smooth}\},
\]
where smoothness is meant in terms of the $C^\infty$-module structure.

Perhaps the $C^\infty$-module structure obfuscates the point slightly.  Instead, one should think of the fields in $\G_n(\Omega)$ as multivector valued functions on $\Omega \subset \R^n$.  Taking this identification allows for an extended toolbox at our disposal.  In particular, points in $\Omega$ are uniquely identified with constant vector fields in $\G_n^1$ and one can consider endomorphisms living in $\G_n$ (acting on $\G_n^1$) as acting on the input of fields in $\G_n(\Omega)$ as well.  Thus, there is not only an algebraic structure on the fields themselves, but on the point in which the field is evaluated.  This is perhaps the key insight on why authors developed the so-called vector manifolds widely used in the geometric algebra landscape.

\begin{example}
    Consider a multivector field $f$ valued in $\G_n(\R^n)$.  With $x\in \R^n$ being identified with the vector in $\G_n^1$, we output a multivector $f(x) \in \G_n$ at each point $x$.  One may be interested in the restriction of $f$ to a vector subspace of $\R^n$ which amounts to using projection on the input.  For example, perhaps we wish to know how $f$ behaves on the subspace generated by some $k$-blade $A_k$.  As such, it suffices to then study $f(\projection{A_k}{x})$.  
\end{example}

We refer to smooth fields valued in $\G_n^+$ as \emph{spinor fields} and put $\G_n^+(\Omega)$ to refer to the $C^\infty$-module counterpart. These fields will be shown to carry a Banach algebra structure. 


\subsubsection{Directional derivative and gradient}

Note that $\R^n$ has global coordinates and thus we can choose a global constant vector field basis $e_1,\dots,e_n$ and we generate $\G_n$ from this basis. Note that we will adopt the Einstein summation convention when needed. With respect to these fields, we have the \emph{directional derivative} $\nabla_\omega$ with $\omega = \omega^i e_i$. The \emph{gradient} (or \emph{Dirac operator}) is defined as $\grad = \sum_{i} e^i \nabla_{e_i}$ and it acts a grade-1 element in the algebra.   Note then that $\omega \cdot \grad = \nabla_\omega$ defines the directional derivative via the gradient. The directional derivative is also grade preserving in that for a multivector $A$
\[
\nabla_\omega \proj{k}{A} = \proj{k}{\nabla_\omega A}.
\]  

This structure defined above is typically referred to as \emph{geometric calculus}.  The setting for geometric calculus extends the setting of differential forms and reduces some of the complexity with tensor computations.  Since $\grad$ is a grade-1 object, it acts on a homogeneous $k$-vector $A_k$ by
\[
\grad A_k = \proj{k-1}{\grad A_k} + \proj{k+1}{\grad A_k} \coloneqq \grad \cdot A_k + \grad \wedge A_k.
\]
Thus, the gradient splits into two operators $(\grad \cdot) \colon \G_n^k(\Omega) \to \G_n^{k-1}(\Omega)$ (or $\grad \lfloor$) and $(\grad \wedge) \colon \G_n^k \to \G_n^{k+1}$.  Here, $\grad \wedge$ can be identified with the exterior derivative $d$ and $\grad \cdot$ can be identified with the codifferential $\delta$ on differential forms up to a sign (see \cite{schindler_geometric_2020} \textcolor{red}{There are more citations to use}). This of course means the standard properties that apply to $d$ and $\delta$ apply to $\grad \wedge$ and $\grad \cdot$. Namely, we have
\begin{equation}
\label{eq:differential_properties}
(\grad \wedge)^2=0 \qquad (\grad \cdot)^2 = 0,
\end{equation}
when acting on a homogeneous $k$-vector. Since \ref{eq:differential_properties} holds, the gradient operator gives rise to the grade preserving \emph{Laplace-Beltrami operator}
\[
\Delta = \grad \grad = \grad \cdot \grad \wedge + \grad \wedge \grad \cdot,
\]
which is manifestly coordinate invariant by definition.  It also motivates the use of the physicist notation $\grad^2=\Delta$, but we do not adopt this here.  We refer to multivector fields $f$ in the kernel of the Laplace-Beltrami operator \emph{harmonic}.

\subsubsection{Monogenic fields}

Geometric calculus includes another definition for multivectors that is a big motivation for those who study Clifford analysis. 
\begin{definition}
 Let $f \in \G_n(\Omega)$. Then we say that $f$ is \emph{monogenic} if $f \in \ker(\grad)$.
\end{definition}

\todo[inline]{What monogenics are we really caring about here? Just the ones we can recover which are...}
Monogenic fields are of utmost importance as they have many beautiful properties. One should find them as a suitable generalization of the notion of complex holomorphicity. For example, in regions of Euclidean spaces, a monogenic field $f$ can be completely determined by its Dirichlet boundary values through a generalized Cauchy integral formula. For any even monogenic field, the each of the graded components of $f$ are harmonic.  

We put 
\[
\monogenics(\Omega) \coloneqq \{f \in \G_n(\Omega) ~\vert~ \grad f =0\}
\]
to refer to elements of this set as \emph{monogenic fields} on $\Omega$. As a subset, we also have the \emph{monogenic spinors} $\monogenics^+(\Omega)$, which are simply the even monogenic fields and the \emph{monogenic parabivectors} $\monogenics^{0+2}(\Omega)$. Though these spaces do not form algebras in their own right, they do indeed form a vector space as sums of monogenic functions are monogenic due to the linearity of the gradient.  Moreover, the monogenic spinors are invariant under multiplication from the Clifford group $\Gamma^+$.

\begin{lemma}
\label{lem:clifford_invariant}
Let $s\in \spingroup$ then $\grad \circ s = s \circ \grad$.  In particular, the space of monogenic spinors $\monogenics^+(\Omega)$ is $\spingroup$ invariant.
\end{lemma}
This lemma is classical in the theory of the Dirac operator, Clifford analysis, and harmonic analysis so we omit a proof.  One can see \cite{janssens_special_nodate} for example.

\subsection{Differential forms and integration}
\label{subsec:diff_forms}

\subsubsection{Directed measures and $k$-forms}
Naturally, we would also like to be able to integrate multivectors.  In order to do so, we appeal to the language of differential forms and build a relationship between multivectors and forms. Forms have their appeal in global understanding via their properties through integration (e.g., Stokes' and Green's theorems).  What we build here provides us a way to carry out a full multivector treatment of boundary value problems.

Given the coordinate system $x^i$ on $\R^n$, we form the basis of tangent vector fields $\partial_i = \frac{\partial}{\partial x^i}$ with the reciprocal 1-forms $dx^i$ (a section of $T^*\Omega$) which are the differentials of the coordinate functions.  Thinking of 1-forms as linear functions on tangent vectors, we have $dx^i  (\partial_j) = \delta^i_j$.  The benefit of this definition is that the 1-forms $dx^i$ carry a natural measure and we can form product measures via the exterior product.  For example, we have the surface \emph{directed measure} $d\Sigma = e_i \wedge e_j dx^i dx^j$ and we can note that $(e^j \wedge e^i)\cdot d\Sigma = dx^idx^j - dx^j dx^i$ is antisymmetric and provides us a surface measure we can integrate; this is a differential form.

In an $n$-dimensional space with a position dependent inner product $g$, we have the $n$-dimensional volume directed measure $d\Omega = \sqrt{|g|} dx^1\dots dx^n$. If we then define $dX_n = e^n \wedge \cdots \wedge e^1 dx^1 \dots dx^n$ we then find that
\[
d\Omega = I^\dagger \cdot dX_n.
\]
Here $I$ is the pseudoscalar field defined on $\Omega$ with respect to $g$. Similarly, for $k<n$, we can define the $k$-dimensional volume measure as 
\[
dX_k = \frac{1}{k!}(e^{i_k}\wedge \cdots \wedge e^{i_1}) dx^{i_1} \cdots dx^{i_k}.
\]
We can now write a $k$-form $\alpha_k$ as $\alpha_k = A_k \cdot dX_k$. In this sense, a differential form is made up of two essential components namely the multivector field and the $k$-dimensional volume directed measure. For example, if we wish to write a 2-form $\alpha_2$ we take $dX_2 = \frac{1}{2!} e^j \wedge e^i dx^i dx^j$ and $A_2 = a_{ij} e_i \wedge e_j$ to yield
\[
\alpha_2 = A_2 \cdot dX_2 = \frac{a_{ij}}{2!} (e_i \wedge e_j) \cdot (e^j \wedge e^i) dx^i dx^j = \frac{a_{ij}}{2!} (dx^i dx^j - dx^j dx^i)
\]
Thus, we arrive at an isomorphism between $k$-forms and $k$-vectors as a contraction with the $k$-dimensional volume directed measure $dX_k$ since
\[
\alpha_k = A_k \cdot dX_k.
\]
Hence, we can see now how a differential form simply appends the measure attached to the underlying space. We can also see how this generalizes the musical isomorphisms $\sharp$ and $\flat$ by taking a vector field $a$ and noting
\begin{equation}
\label{eq:line_element}
a \cdot dX_1 = a^i e_i \cdot e^j dx^j = a^i dx^i,
\end{equation}
corresponds to the usual $\flat$ map on vector fields.


\subsubsection{Exterior algebra}
The exterior algebra of differential forms comes with an addition $+$ and exterior multiplication $\wedge$.  We note that the sum of two $k$-forms $\alpha_k$ and $\beta_k$ that $\alpha_k+\beta_k$ is also a $k$-form which we can see by letting $\alpha_k = A_k \cdot dX_k$ and $\beta_k = B_k \cdot dX_k$ and putting
\[
\alpha_k + \beta_k = (A_k \cdot dX_k)+(B_k \cdot dX_k) = (A_k + B_k) \cdot dX_k,
\]
due to the linearity of $\cdot$.  If instead had an $s$ form $\beta_s$ then we have the exterior product
\[
\alpha_k \wedge \beta_s = (A_k \wedge B_k) \cdot dX_{k+s},
\]
where $dX_{k+s}=0$ if $k+s>n$.  


\subsubsection{Exterior derivative}
With differential forms one also has the exterior derivative $d$ giving rise to the calculus of forms.  Given we can write a differential $k$-form as $\alpha_k = A_k \wedge dX_k$,  In particular, we have
\[
d \alpha_k = (\grad \wedge A_k) \cdot dX_{k+1},
\]
which realizes the exterior derivative as the grade raising component of the gradient $\grad$. Of course, for scalar fields, this returns the gradient as desired. 


\subsubsection{Integration on submanifolds}

Given a $k$-dimensional submanifold of $K \subset \Omega$ with a $k$-form $\alpha_k$ defined on $K$, we can integrate the $k$-form. Using the multivector equivalents leads us to the $k$-dimensional directed measure $dK$ for the submanifold $K$.  Given $K$ is a submanifold of $\Omega$, for any $x \in K$ we have tangent space $T_x K$ which corresponds to a tangent $k$-blade $I_K(x)$.  We put $I_K$ as the smooth $k$-blade field everywhere tangent to $K$. Then we have the directed volume measure on $K$ given by
\[
dK = I_K^\dagger \cdot dX_k.
\]
For a tangent $k$-vector field $A_k$ on $K$, we must have for any $x \in K$ that $f = \operatorname{P}_{I_K} \circ f$ so that these fields lie purely tangent to $K$. In particular, we can always put $A_k = A I_k^\dagger$ for a scalar field $A$. These fields can contract with the directed measure $dX_k$ to create a $k$-form on $K$ by $\alpha_k = A_k \cdot dX_k = A dK$ which can be integrated as
\[
\int_K \alpha = \int_K A dK.
\]
Hence, on $\Omega$ itself, we can integrate top forms $\omega = W I^\dagger$ for a scalar field $W$ by
\[
\int_\Omega \omega = \int_\Omega W d\Omega.
\]

There is also the normal space $N_x K$ that is everywhere orthogonal (with respect to $g$ on $\Omega$) to $T_x K$.  In particular, we have the normal $(n-k)$-blade field $\nu = I_K^\dagger I$. Note that for a unit $k$-blade $I_K$ we have $I_K^{-1}=I_K^\dagger$ and we see $I_K \nu = I$. Since $K$ is a submanifold of $\Omega$ we have the inclusion $\iota \colon K \to \Omega$ and the induced pullback on forms $\iota^*$ which is equivalent to the tangent projection operator $\tangent_K$ seen in \cite{schwarz_hodge_1995}. Given a $p$-form $\omega$ defined on $\Omega$, we have that $\tangent_K \omega = \omega \circ \operatorname{P}_{I_K}$. Specifically, $\omega = W \cdot dX_p$ we have 
\[
\iota^* \omega  = W \cdot (dX_p \circ \operatorname{P}_{I_K}) = \operatorname{P}_{I_K}(W) \cdot dX_p.
\]
The normal projection $\normal_K$ is then $\normal_K \omega = \omega - \tangent_K \omega$. 

This is pertinent when we take the submanifold $\Sigma = \partial \Omega$. There, $I_\Sigma$ yields the directed measure
\[
d\Sigma \coloneqq I_\Sigma^\dagger \cdot dX_{n-1}.
\]
The normal space is 1-dimensional and $\nu$ is the unit normal vector to the boundary. The pullback coincides with projection into the tangent space given by $I_\Sigma$.  Then, for 1-forms $\alpha = \cdot dX_1$ it is apparent that $\tangent_\Sigma \alpha = \projection{I_\Sigma}{a} \cdot dX_1$ and $\normal_\Sigma \alpha = \rejection_{I_\Sigma}(a) \cdot dX_1$ by Equations \ref{eq:projection+rejection_vector} and \ref{eq:projection_rejection_vector}. One can then find the flux of a vector field through $\Sigma$ arises as an $(n-1)$-form $\projection{\nu}{a} I^{-1} \cdot dX_{n-1}$. Once again we see that the flux is determined both by the vector field $a$ and the local geometry of $\Sigma$ captured by $d\Sigma$ in the following way. Note that  $\nu^{-1}=\nu$ since $\|\nu\|=1$ everywhere on $\Sigma$ and so $\projection{\nu}{a} I^{-1}= a \cdot \nu \nu I^{-1} = a \cdot \nu I_\Sigma^\dagger$ which gives us the corresponding form $a \cdot \nu d\Sigma$ and the total flux of $a$ through $\Sigma$ is then
\[
\int_\Sigma (\projection{\nu}{a} I^{-1}) \cdot dX_{n-1} = \int_\Sigma a \cdot \nu d\Sigma.
\]

\subsubsection{$k$-form inner product}
\todo[inline]{Show that this relates back to the spinor norm.}
For smooth $k$-forms $\alpha_k = A_k \cdot dX_k$ and $\beta_k = B_k \cdot dX_k$, we have an inner product 
\[
\langle \alpha_k, \beta_k \rangle = \int_\Omega \alpha_k \wedge \star \beta_k 
\]
where $\star$ is the Hodge star. The Hodge star on $k$-forms inputs a $k$-form and outputs a a specific dual $(n-k)$-form so that we always have $\alpha_k \wedge \star \beta_k  = (A_k\cdot B_k^\dagger)d\Omega$ as we note Equation \ref{eq:dot_product}. Thus, we can realize how $\star$ acts on multivector representative. We let $\star \beta_k = B_k^\star \cdot dX_{n-k}$ by $B_k^\star = (I^{-1} B_k)^\dagger$.  Indeed, we have
\begin{align*}
    \alpha_k \wedge \star \beta_k &= (A_k \wedge B_k^\star) \cdot dX_n\\
    &= A_k \cdot B_k^\dagger d\Omega.
\end{align*}

\subsubsection{Stokes' and Green's theorem}

For regions $\Omega$ with boundary $\Sigma$, we have a compact form of Stokes' theorem
\[
\int_\Omega d \omega = \int_\Sigma \tangent \omega,
\]
for sufficiently smooth $(n-1)$-forms $\omega$.

\todo[inline]{More on integration and do Ohms law as an example of some of this stuff. Do all of hodge decomposition and stuff?}

%\section{Algebras of multivector fields}
%\subsection{Banach algebras of Clifford fields}

\todo[inline]{Finish this section. I'm saying this here but it should go later on, but this should lead to the weak formulation for the laplace equation??? Does there exist an inner product instead of just a norm? }

Letting $\Omega$ be a region in $\R^n$, recall that the space of monogenic fields $\monogenics(\Omega)$ is not an algebra. However, $\monogenics(\Omega)$ does contain algebras that are commutative Banach algebras. For example, we always have the following algebra.

The Clifford fields are given as functions $s \colon \Omega \to \Gamma$ for which we say $s\in \Gamma(\Omega)$. This space $\Gamma(\Omega)$ has a norm induced from the spinor norm in the $L_2$ sense by
\[
\spinnorm{s} = \int_\Omega ss^\dagger d\Omega
\]
This gives us a normed algebra of Clifford fields. One can see that we have the unit $1$ in this algebra. We also have for multivectors $s,r \in \Gamma(\Omega)$ (constant Clifford fields)
\[
\|sr\| = \|s\|\|r\|
\]
since
\[
\|sr\|^2 = (sr)(sr)^\dagger = srr^\dagger s^\dagger = s\|rr^\dagger\|^2 s^\dagger = \|s\|^2 \|r\|^2.
\]
It follows for non-constant $C^\infty$-fields $s$ and $r$
\[
\spinnorm{sr} \leq \spinnorm{s}\spinnorm{r}.
\]
This shows the algebra is uniform. Identifying the constant fields in the algebra $\Gamma(\Omega)$ with $\R^{2^n}$ we see that the algebra is also complete. Thus we have shown that the space $\Gamma(\Omega)$ is a (noncommutative) Banach algebra. The subspace $\monogenics(\Omega)\cap \Gamma(\Omega)$ is thus a Banach algebra contained inside of $\monogenics(\Omega)$. There are more algebras to discover.

\begin{remark}
It is worth noting that while the constant elements in $\Gamma(\Omega)$ form a group, the elements in general do not.  That is to say that not every element in $\Gamma(\Omega)$ is invertible.  This not only expected since functions in general are not invertible, but this is also a nonissue as the algebra structure of the Clifford fields is what remains important.
\end{remark}

\subsubsection{Planar monogenic fields}

Generically, if I take some multivector $A$ times a monogenic field $f$, $Af$ need not be monogenic. This is exactly why $\monogenics(\Omega)$ fails to be an algebra. But, there are certain types of monogenic fields in which this property is true. We describe a set of parabivectors that operate entirely on a plane given by a unit bivector $B$. These specific fields will be of great utility for the remainder of this paper.
\begin{definition}
    Let $f$ be a parabivector and $B$ a unit $2$-blade. Then $f$ is a \emph{$B$-planar field} if $f = \operatorname{P}_B \circ f \circ \operatorname{P}_B$.
\end{definition} 
We then refer to the \emph{$B$-planar monogenic fields} $f$ when $f$ is both $B$-planar and monogenic. Planar monogenic fields will serve as a realization of complex valued functions since they carry over some additional nice properties and admit a nice representation.
\begin{lemma}
    Let $f$ be a $B$-planar monogenic field, then:
\begin{itemize}
    \item The directional derivatives in all directions other than in the $B$ plane are zero;
    \item We have the representation $f=u+\beta B$ for a $u,\beta \in G_n^0$ and $B$ the given unit bivector.
\end{itemize}
\end{lemma}
\begin{proof}
~
    \begin{itemize}
    \item Let $v$ be a unit vector not in the $B$ plane so that $\projection{B}{v}=0$. Since $f$ is $B$-planar, we know $f=f \circ \operatorname{P}_B$ which shows that $f(x+\epsilon v)= f(x)$.  It follows that $\nabla_v f=0$.
    \item Let $f=u+b$ for $u\in \G_n^0$ and $b\in G_n^2$. Then $f=\projection{B}{v} \circ f$ and so $\projection{B}{u+b}=u+b$. In particular, $\operatorname{P}_B=b$ and thus $b=\beta B$ for a scalar $\beta \in \G_n^0$.
\end{itemize}
\end{proof}
To get a geometric interpretation of $B$-planar fields we can note that they are constant on translations of the $B$-plane.  It follows that 
\begin{equation}
\label{eq:exterior_b_derivative}
(\grad \wedge B)f = 0.
\end{equation}
In $\R^3$, for example, this amounts to fields constant along an axis $\omega=IB^{-1}$ perpendicular to $B$ as
\begin{equation}
\label{eq:omega_axial_equivalence}
\grad \wedge B = \grad \wedge \omega I =\grad \cdot \omega = \nabla_\omega.
\end{equation}

\todo[inline]{Rephrase this with rejection?}

Recall from Example \ref{ex:complex_representation} that multivectors in the form $\zeta=x+yB$ mimic the complex number $\zeta$ when $B$ is a unit $2$-blade since $B^2=-1$.  Planar monogenic fields are thus a direct analog of $\C$-holomorphic functions.  Indeed, for simplicity take the orthonormal basis $e_i$ and the blade $B=B_{12}$ and for scalar fields $u$ and $\beta$ put
\[
f=u+\beta B_{12}
\]
and note
\[
\grad f = 0 
\]
yields the Cauchy-Riemann equations
\[
\nabla_{e_1} u = \nabla_{e_2} \beta \qquad \textrm{and} \qquad \nabla_{e_2}u = -\nabla_{e_1} \beta.
\]
Holomorphic functions form an algebra and we shall show the $B$-planar monogenic fields do as well. 

We let 
\[
\algebra{B}(\Omega) = \{f ~\vert~ \textrm{$f$ is $B$-planar and monogenic}\}
\]
be the space of $B$-planar monogenic fields. For any $2$-blade $B$ in $\Grassmannian{2}{n}$, we have a copy of $\algebra{B}(\Omega)$. Multiplication of two $B$-planar fields $f=u_f+\beta_f B$ and $g=u_g+\beta_g B$ is given by
\begin{equation}
\label{eq:axial_multiplication}
fg = u_f u_g - \beta_f \beta_g + B (u_f b_g + u_g b_f) = gf.
\end{equation}

Another property mimics $\C$-holomorphicity.  Namely, scaling a holomorphic function by constant complex numbers remains holomorphic. We realize this for $B$-planar fields as $\operatorname{Spin}(2)$ invariance (really $\R \times \operatorname{Spin}(2)$ invariant).  This corollary follows from Lemma \ref{lem:clifford_invariant} since $\operatorname{Spin}(2)$ is a subgroup of $\Gamma^+$ 
\begin{corollary}
    \label{cor:mult_by_i_monogenic}
    Let $f=u+\beta B$ be an $B$-planar monogenic field and let $\zeta=x+yB$ for constant scalars $x$ and $y$. Then $\zeta f$ is a $B$-planar monogenic.
\end{corollary}
\begin{proof}
    Note that $\zeta$ is in $\Gamma^+(\Omega)$, and utilize Lemma \ref{lem:clifford_invariant}.
\end{proof}
The point here is that we have now effectively found functions that can be scaled by $B$-planar constants $\zeta$ and remain monogenic. 
 
With the above, we show the space $\algebra{B}(\Omega)$ is closed under multiplication and is in fact abelian.
\begin{lemma}
\label{lem:product_of_monogenics}
    Let $f$ and $g$ be monogenic and $B$-planar. Then $fg=gf$, and $fg$ is a $B$-planar monogenic.
\end{lemma}
\begin{proof}~
    \begin{itemize}
        \item First, it is clear that $fg=gf$ by Equation \ref{eq:axial_multiplication}.
        \item The product $fg$ is $B$-planar since $u_f,u_g,\beta_f$, and $\beta_g$ are all constant on translations of the $B$-plane, i.e. that $fg = fg \circ \operatorname{P}_B$.  Due again to Equation \ref{eq:axial_multiplication} we have $fg = \operatorname{P}_B \circ fg$ as well.  
    \item To see that the product is monogenic, we have
    \[
        \grad(fg) = \grad(u_fu_g - b_f b_g +  B(u_f b_g + u_g b_f)).
    \]
    Then the grade-1 components are
    \[
        \proj{1}{\grad(fg)}=\grad \wedge (u_f u_g - b_f b_g) + \grad \cdot B(u_f b_g + u_g b_f),
    \]
    and note that we have
    \begin{align*}
        \grad(u_f u_g - b_f b_g) &= (\grad u_f) u_g + u_f (\grad u_g) - (\grad b_f) b_g - b_f (\grad b_g)\\
        \grad \cdot B(u_f b_g + u_g b_f) &= (\grad \cdot B u_f) b_g + u_f (\grad \cdot B b_g) + b_f(\grad \cdot B u_g) + (\grad \cdot B b_f) u_g,
    \end{align*}
    and since $f$ and $g$ are both monogenic we have
    \begin{align*}
        \proj{1}{\grad(fg)} &= (\grad \cdot B u_f - \grad  b_f)b_g + (\grad \cdot B u_g - \grad  b_g)b_f.
    \end{align*}
    \[
        0=\proj{1}{\grad Bf} = \grad \cdot B u_f - \grad b_f
    \]
    by Corollary \ref{cor:mult_by_i_monogenic} and likewise for $\proj{1}{\grad Bg}$. Thus,
    \[
        \proj{1}{\grad(fg)}=0.
    \]
    
    The grade-3 components for the gradient are
    \[
        \proj{3}{\grad(fg)} = \grad \wedge B (u_f b_g + u_g b_f),
    \]
    and we can note that $\grad \wedge B=0$ since $u_f,b_g,u_g,$ and $b_f$ are all $B$-planar.
\end{itemize}
\end{proof}

From the above work, we realize that for each $\algebra{B}(\Omega)$ we have a well defined multiplicative structure. But, we need to show that inverses also exist. Doing show realizes that $\algebra{B}(\Omega)$ sits inside of Clifford fields $\Gamma^+(\Omega)$. This is clear as any constant field $\zeta = x+yB$ is invertible. Thus we arrive at the following corollary.
\begin{corollary}
The space $\algebra{B}$ is a commutative unital Banach algebra.
\end{corollary}
\begin{proof}
Let $f$ and $g$ be $B$-planar monogenic fields. It is clear that the sum $f+g$ is a $B$-planar monogenic by the linearity of $\grad$ and the projection. Since $fg=gf$ is $B$-planar and monogenic we find that each $\algebra{B}(\Omega)$ is an algebra. Since $\algebra{B}(\Omega)$ is a commutative subalgebra of $\Gamma(\Omega)$ (really of $\Gamma^+(\Omega)$), it is also a commutative Banach algebra.
\end{proof}

\subsubsection{$\omega$-axial fields}
The authors in \cite{belishev_algebraic_2019,belishev_algebras_2019} give a thorough treatment of an analogous story but with quaternion fields.  We show the relationship between the two stories in this section and we find them to be entirely equivalent. As in Example \ref{ex:quaternions}, we can see these quaternion fields as parabivector fields.  The authors work exclusively in 3-dimensions and quickly specialize to the fields which are $\omega$-axial due to their rich algebraic structure. There, $\omega$ is a purely imaginary unit quaternion. Their harmonic $\omega$-axial fields are equivalent to monogenic $B$-planar fields if we take the axis $\omega = BI^{-1}$. First, note we define $\omega$-axial in the same way.
\begin{definition}
    Let $A \in \G_3$ be a multivector field then $A$ is \emph{$\omega$-axial} if $A(x+t\omega) = A(x+t\omega)$.  
\end{definition}

This definition allows us to perfectly coincide the notions of $B$-planar monogenic fields with $\omega$-axial harmonic quaternion fields.
\begin{proposition}
    In $\R^3$, every $B$-planar monogenic field is in correspondence with an $\omega$-axial harmonic quaternion field $h = \varphi + \psi \omega$. 
\end{proposition}
\begin{proof}
    Let $f$ be a $B$-planar monogenic field with $\tilde{\omega}=BI^{-1}$ and note that $f(x+t\tilde{\omega)}=f(x)$ since $\projection{B}{t\omega}=0$. Thus, $f$ is $\tilde{\omega}$-axial.
    
    Given the quaternion multiplication is a left handed bivector multiplication (see Example \ref{ex:quaternions}, we can replace the purely imaginary quaternion $\omega$ and get a vector in $\G_3^1$ by using the correspondence $\boldsymbol{i} \leftrightarrow e_1$, $\boldsymbol{j}\leftrightarrow e_2$, and $\boldsymbol{k}\leftrightarrow e_3$ we generate $\tilde{\omega} \in \G_3^1$. We then have the $2$-blade $B=\tilde{\omega} I$ such that
    \[
        \tilde{h} = \varphi + \psi B,
    \]
    is the corresponding parabivector in $\G_3$. It's clear that $\operatorname{P}_B \circ \tilde{h} = \tilde{h}$. Likewise, since $\varphi$ and $\psi$ were constant on the axis given by $\omega$, then by the previous work $\varphi \circ \operatorname{P}_B$ and $\psi \circ \operatorname{P}_B$ implies that $\tilde{h} \circ \operatorname{P}_B$ and so $\tilde{h}$ is a $B$-planar. Hence, setting $\varphi = u$ and $\psi=\beta$, we recover a unique $f$ from a given $h$.

Then, if $h=\varphi + \psi \omega$ is harmonic, we know
\[
\grad \psi = \omega \times \grad \varphi,
\]
where we take the vector cross product $\times$.  Based on Example \ref{ex:cross_product}, we can see that corresponding $B$-planar field $f=u+\beta B$ yields the analogous equation
\[
\grad u = \grad \cdot \beta B = (\grad \wedge \tilde{\omega})I = \tilde{\omega } \times \grad \beta.
\]
Thus, the notions of an $\omega$-axial harmonic quaternion field coincides with $B$-planar monogenic fields in $\R^3$ so long as $B=\tilde{\omega}I$.
\end{proof}

The $\omega$-axial fields do not generalize properly and this definition is solely a happy circumstance seen in $\R^3$ given the duality between vectors and bivectors.  In higher dimensions, the notion of $B$-planar retains all the desired properties that let us define a notion of a Gelfand spectrum.



\subsubsection{Spinor spectrum}

This story no longer continues in higher dimensions and one can find the two and three dimensional cases to be happy accidents.  Instead, now we must deal fully with the situation at hand to dissect the relevant algebras. At our disposal are the algebras $\algebra{B}(\Omega)$ of $B$-planar monogenic fields. Take the case where the domain $\ball \subset \R^n$ is the unit $n$-ball and moreover let $\disk$ be the unit disk in $\C \cong \R^2$.  By Gelfand, the maximal ideal space of the commutative Banach algebra $\algebra{B}(\ball)$ is homeomorphic to the disk given the isomorphism mapping the blade $B \leftrightarrow i$ in the complex plane. Since the space $\monogenics$ is no longer an algebra or even commutative, we are at a loss to determine maximal ideals.  Instead, one can note that maximal ideals of a commutative Banach algebra correspond to the multiplicative linear functions.  Using this identification, we carry on and describe functionals on the monogenics.

\begin{definition}
    Define the \emph{spinor dual} $\dualmonogenics(\Omega)$ as
    \[
        \dualmonogenics(\Omega) \coloneqq \{ l \in \mathcal{L}(\monogenics(\Omega); \Gamma^+) ~\vert~ l(sf) = sl(f), ~\forall f \in \monogenics, ~s \in \spinalgebra \}
    \]
\end{definition}
$\dualmonogenics(\Omega)$ are the spinor valued functionals or \emph{spin functionals}. Similarly, we have the definition for the spinor functionals that are multiplicative on the $B$-planar monogenics. In other words, spin characters are simply algebra homomorphisms from $\algebra{B}(\Omega)$ to $\Gamma^+$.
\begin{definition}
    The \emph{spinor spectrum} is the set
    \[
        \characters(\Omega) \coloneqq \{ \mu \in \dualmonogenics(\Omega) ~\vert~ \mu(fg) = \mu(f)\mu(g),~ \forall f,g \in \algebra{B}, ~ B \in \Grassmannian{2}{n}\},
    \]
    and we refer to the elements as \emph{spin characters}.
\end{definition}

In the case where $\Omega$ itself is 2-dimensional and compact, we realize $\Gamma^+$ is isomorphic to $\C$ and we find that these match the typical definition for characters $\mu\in \characters(\Omega)$.  These spin characters each amount to function evalation. Take $f\in \monogenics(\Omega)$ and note that $f \in \algebra{B}(\Omega)$ as well.  $f$ is then a holomorphic function when we identify $B \leftrightarrow i$ and as such the spin character $\mu$ acts by $\mu(f)=f(x_\mu)$ for some point $x_\mu \in \Omega$ showing the correspondence of points in $\Omega$ with spin characters in $\characters(\Omega)$. Hence, with the weak-$\ast$ topology, the space $\characters(\Omega)$ is homeomorphic to $\Omega$. 

\todo[inline]{There is the question now on what is the homeomorphism type of $\algebra{B}(\Omega)$ for an arbitrary $\Omega$ and for a given $B$. Use 2d Belishev somehow? Describe the weak-$\ast$ topology here to use later.} 

%
%\section{Gelfand theory}
%\subsection{$\G_n$-spectrum}

We turn our focus to the geometric content of the algebras $\algebra{B}(M)$ of monogenic subsurface spinor fields. Take the case where the domain $\disk$ be the unit disk in $\C \cong \R^2$.  By Gelfand, the maximal ideal space of the commutative Banach algebra $\algebra{B}(\disk)$ is homeomorphic to the disk since the algebra $\algebra{B}(\disk)$ is exactly the algebra of holomorphic functions in $\disk$. Naively attempting to generalize this notion leads on to consider the maximal ideal space of $\monogenicfields{}$ or $\monogenicfields{+}$, but no such maximal ideal space can be determined. Instead, one can note that maximal ideals of a commutative Banach algebra $\mathcal{A}$ correspond to the algebra morphisms $\mathcal{A} \to \C$.  Using this as our guiding intuition, we carry on and describe the relevant morphisms of the monogenic fields.

\begin{definition}
    Define the \emph{$\G_n$-dual} $\dualmonogenics(M)$ as the continuous right $\G_n$-module homomorphisms
    \begin{equation}
        \dualmonogenics(M) \coloneqq \{ l \colon \monogenics(M) \to \G_n ~\vert~ l(fs+g) = l(f)s+l(g), ~\forall f,g \in \monogenics(M), ~r,s \in \G_n \},
    \end{equation}
    and refer to elements of $\dualmonogenics(M)$ are \emph{$\G_n$-functionals}.
\end{definition}
Similarly, we will now define the $\G_n$-functionals that are multiplicative, and therefore constitute algebra morphisms, on the monogenic subsurface spinor fields. In other words, spin characters are simply algebra homomorphisms from $\algebra{\bivector}(M)$ to $\G_n^+$.
\begin{definition}
    The \emph{$\G_n$-spectrum} $\characters(M)$ is the set of algebra homomorphisms
    \[
        \characters(M) \coloneqq \{ \delta \in \dualmonogenics(M) ~\vert~ \delta(fg) = \delta(f)\delta(g),~ \forall f,g \in \algebra{\bivector}(M),~  \bivector \in \Grassmannian{2}{n}\},
    \]
    and we refer to the elements as \emph{$\G_n$-characters}.
\end{definition}
One choice of $\G_n$-characters is point evaluation. Take $\delta(f)=f(x^\delta)$ for some $x^\delta \in M$. We find that these characters exhaust $\characters(M)$. Along with this, we define \emph{weak-$\ast$} topology on $\characters(M)$, which is defined to be the coarsest topology so that every $x \in M$ corresponds to a continuous map on $\dualmonogenics(M)$. 

\subsection{Topology from monogenics}

We seek to determine that the space $\characters(M)$ is homeomorphic to $M$ in the case that $M$ is $n$-dimensional, smooth, oriented, imbedded manifold in $\R^n$ inheriting the Euclidean metric. 

\begin{theorem}
\label{thm:gelfand}
For any $\delta \in \characters(M)$, there is a point $x^\delta \in M$ such that $\delta(f) = f(x^\delta)$ for any $f\in \monogenics(M)$ a monogenic field. Given the weak-$\ast$ topology on $\dualmonogenics(M)$, the map
\[
\gamma \colon \characters(M) \to M, \quad \delta \mapsto x^\delta
\]
is a homeomorphism. 
%The Gelfand transform 
%\[
%\widehat{~} \colon \monogenics(M) \to C(\characters(M); \G_n), \quad \widehat{f}(\delta) \coloneqq \delta(f), \quad \delta \in \characters(M),
%\]
%is an isometry onto its image, so that $\characters(M)$ is isomorphic to $\widehat{\monogenics(M)}$ as algebras.
\end{theorem}

We prove this theorem in three main components. First, we can realize a power series representation for elements in a ball $\ball$ and denote this set as $\monogenics^\mathcal{P}(\ball)$ which is dense in $\monogenics(\ball)$. This power series is constructed using specific monogenic subsurface spinor fields. Finally, we constructively show a correspondence between $\delta \in \characters(\ball)$ with $x^\delta \in \ball$. Then, we can take a cover $M$ generated by unions of balls to complete the proof.

\subsubsection{Taylor series}
%\url{https://math.stackexchange.com/questions/811248/wedge-product-between-nonorthogonal-basis-and-its-reciprocal-basis-in-geometric}
Fix an orthonormal basis $\blade{e}_1,\dots,\blade{e}_n$ for $\R^n$ and define the functions $z_{ij} = x_j - x_i \blade{e}_i \blade{e}_j$. Recall that for an orthonormal basis the reciprocal basis elements $\blade{e}^i=\blade{e}_i$. To further condense notation, we define $\bivector_{ij}\coloneqq \blade{e}_i \blade{e}_j$ for $i\neq j$ as necessary. The functions $z_{ij}$ are the analogs to $z$ in $\C$ but specifically in the $\bivector_{ij}$ plane. We then note
\begin{equation}
z_{ij} = x_j - x_i \bivector_{ij} = \blade{e}_j\projection_{\bivector_{ij}}(x),
\end{equation}
for $x=(x_1,\dots,x_n)\in \R^n$. Hence, this is but a special case of \cref{eq:z} and we note that each $z_{ij}$ is monogenic and hence belongs in $\algebra{\bivector_{ij}}$ so long as $i \neq j$. One can quickly verify $z_{ii}$ is not monogenic. These functions find their use in a power series representation for monogenic fields $f$. Specifically, let $\sigma$ be a permutation of the set $\{2,3,\dots,n\}$, then we have the polynomials
\begin{equation}
        p_{j_2 \dots j_n}(x) = \frac{1}{j!} \sum_{\textrm{permutations}}z_{1\sigma(1)}(x) \cdots z_{1\sigma(j)}(x),
\end{equation}
each of which is monogenic, linearly independent (\cite[Proposition 1]{ryan_clifford_2004}) and formed by products of elements $z_{ij} \in \algebra{\blade{B}_{ij}}$. We put
\begin{equation}
    \monogenics^\mathcal{P}(M) = \left\{\sum_{j=0}^\infty \left(\sum_{\substack{{j_2 \cdots j_n} \\ {j_2 + \cdots j_n = j}}}p_{j_2 \cdots j_n}a_{j_2 \cdots j_n}\right) ~\vert~ j_2+\cdots+j_n = j, ~0\leq j < \infty, ~ a_{j_2\cdots j_n} \in \G_n\right\}
\end{equation}
to refer to the set of \emph{monogenic polynomials}. 

\begin{lemma}
\label{lem:density}
The space $\monogenics^\mathcal{P}(\ball_{R,w})$ is dense in $\monogenics(\ball_{R,w})$.
\end{lemma}
\begin{proof}
We can center a ball of radius $R$ at $w$ to get the monogenic polynomials $p_{j_2 \dots j_n}(x-w)$. Then, let $f\in \monogenics(\ball_{R,w})$ and define the coefficients $a_{j_2 \cdots j_n}\in \G_n$ by
\begin{equation}
a_{j_2 \cdots j_n} = \int_{\partial B(w,R)} \frac{\partial^j G(w,y)}{\partial y_2^{j_2} \cdots \partial y_n^{j_n}} \normal(y) f(y) \mu_\partial(y),
\end{equation}
where $G$ is the Cauchy kernel and we have used the Cauchy integral formula with the fact $\pseudoscalar$ is constant. By \cite[Theorem 4]{ryan_clifford_2004}, we have
\begin{equation}
        f(x) = \sum_{j=0}^\infty \left(\sum_{\substack{{j_2 \cdots j_n} \\ {j_2 + \cdots j_n = j}}} p_{j_2 \cdots j_n} (x-w) a_{j_2 \cdots j_n} \right),
\end{equation}
which converges pointwise to $f$ for points $x\in \ball_{R,w}$.
\end{proof}

We have found that all monogenic fields are generated as power series of homogeneous polynomials in the variables $z_{ij}$. Thus, we have a form for which we can determine the action of a $\G_n$-character on a monogenic field. Specifically, take the series for $f(x)$ above and note for $\delta \in \characters(\ball_{R,w})$ that
\begin{align}
\delta(f(x)) &= \sum_{j=0}^\infty \sum_{j=0}^\infty \left(\sum_{\substack{{j_2 \cdots j_n} \\ {j_2 + \cdots j_n = j}}} \delta(p_{j_2 \cdots j_n} (x-w)) a_{j_2 \cdots j_n} \right)
\end{align}
and on each monogenic polynomial
\begin{align}
\delta(p_{j_2 \dots j_n}(x)) &= \frac{1}{j!} \sum_{\textrm{permutations}}\delta\left((z_{1\sigma(1)}(x)\right) \cdots \delta\left(z_{1\sigma(j)}(x)\right),
\end{align}
by definition. Hence, we now need to determine the action of $\delta$ on the variables $z_{ij}$.

\subsubsection{Correspondence}

The functions $z_{ij}$ played a crucial role in the above power series representation and they also play a key part in determining the behavior of the $\G_n$-characters on monogenic fields  Deducing the action of $\delta(z_{ij})$ will allow us to extend this to any monogenic $f$ via \cref{lem:density}. 

\begin{lemma}
Let $\delta \in \characters(\ball_{R,w})$ and $z_{ij} \in \algebra{B_{ij}}$ then $\delta(z_{ij})=z_{ij}(x^\delta)$ for some $x^\delta \in \R^n$.
\end{lemma}
\begin{proof}
Let $\delta \in \characters(\ball_{R,w})$ and note that $\delta(1)=1$ since $\delta$ is an algebra homomorphism. Hence, let $\bivector$ be an arbitrary unit $2$-blade then 
\begin{equation}
\delta(\alpha + \beta \bivector) = \delta(\alpha) + \delta(\beta \bivector)=\alpha \delta(1) + \delta(1)\beta \bivector = \alpha + \beta \bivector.
\end{equation}
by definition. Applying $\delta$ to $z_{ij}$ yields
\begin{equation}
\delta(z_{ij}) = \alpha_{ij} + \beta_{ij} \bivector_{ij},
\end{equation}
for some constants $\alpha_{ij}$ and $\beta_{ij}$. Then, note that we have two key relationships
\begin{equation}
\label{eq:z_reciprocal_relationship}
z_{ij} \bivector_{ji}  = -z_{ji}
\end{equation}
\begin{equation}
\label{eq:z_relationship}
z_{ij} = z_{kj} + z_{ik} \bivector_{kj}.
\end{equation}

Applying $\delta$ to \cref{eq:z_reciprocal_relationship}
\begin{equation}
\delta(z_{ij} \bivector_{ji}) = \delta(z_{ij}) \bivector_{ji} = -\delta(z_{ji})
\end{equation}
yields
\begin{equation}
(\alpha_{ij} + \beta_{ij} \bivector_{ij})\bivector_{ji} = \beta_{ij} + \alpha_{ij} \bivector_{ji} = - \alpha_{ji} - \beta_{ji} \bivector_{ji}.
\end{equation}
Hence, $\alpha_i^j = -\beta_j^i$ for all $i \neq j$.

Applying $\delta$ to \cref{eq:z_relationship}
\begin{equation}
\delta(z_{ij}) = \delta(z_{kj} + z_{ik} \bivector_{kj}) = \delta(z_{kj})+\delta(z_{ik})\bivector_{kj}
\end{equation}
yields
\begin{equation}
a_{ij} + b_{ij} \bivector_{ij} = \alpha_{kj} + \beta_{kj} \bivector_{kj} + (\alpha_{ik} + \beta_{ik} \bivector_{ik})B_{kj} = \alpha_{kj} + \beta_{ik} B_{ij} + (\alpha_{ik} + \beta_{kj})\bivector_{kj}
\end{equation}
yields the relationships $\alpha_{ij} = \alpha_{kj}$, $\beta_{ij} = \beta_{ik}$, and $\alpha_{ik}=-\beta_{kj}$.

These relationships allow us to achieve our proof. Briefly, picture $\alpha_{ij}$ and $\beta_{ij}$ as components of the $n \times n$ matrices $\alpha$ and $\beta$ where we index row by column. Note that $\alpha$ and $\beta$ both have zero diagonal since the functions $z_i^i$ are not monogenic. The relationship $\alpha_{ji} = -\beta_{ij}$ for $i\neq j$ then shows that $\alpha = -\beta^\top$.  Then we have $\alpha_{ij} = \alpha_{kj}$ for $i\neq j \neq k$ shows that $\alpha$ is constant along rows and hence $\beta$ is constant along columns. This is consistent with $\alpha = -\beta^\top$, $\beta_j^i = \beta_k^i$, and the final relationship $\alpha_{ik} = -\beta_{kj}$. The matrices $\alpha$ and $\beta$ are thus uniquely determined by $n$ numbers.  Moreover, treating $\delta(z_{ij})=z_{ij}(x_\delta)$ for some $x^\delta \in \R^n$ satisfies the relationships granted above. Thus, we simply find the $x^\delta$ such that we retrieve the desired components for $\alpha$ and $\beta$.  
\end{proof}


\begin{lemma}
Let $f\in \monogenics(\ball_{R,w})$, then $\delta(f)=f(x^\delta)$ for some $x^\delta \in \ball_{R,w}$.
\end{lemma}
\begin{proof}
To see that $x^\delta \in \ball_{R,w}$, take $G_0 \in \monogenics(\ball_{R,w})$ by $G_0(x)\coloneqq G(x-x_0)$ where $G$ is the Green's function in \cref{eq:greens_function} and take some $x'\not\in \ball_{R,w}$. Fix $\delta \in \characters(\ball_{R,w})$, then, 
\begin{align}
\delta(G_0)=G_0(x^\delta).
\end{align}
If $x^\delta  \notin \ball_{R,w}$, then take a sequence $x_n \to x^\delta$ and note that the sequence of functions $G_n(x)\coloneqq G(x-x_n) \in \monogenics(\ball_{R,w})$ and the sequence converges to a monogenic function $G_\infty = G(x-x^\delta)$ but
\begin{align}
\lim_{n\to \infty} \delta(G_n) = \lim_{n\to \infty} G_n(x^\delta),
\end{align}
does not converge. Hence, it must be that $x^\delta \in \ball_{R,w}$ via continuity of $\delta$.
\end{proof}

Take an arbitrary open cover of $M$ with balls $\ball_{R,w}$. Via this cover, we extend the lemmas to prove \cref{thm:gelfand}.
\begin{proof}
\textcolor{red}{This theorem is not yet proven, but the previous lemmas prove the theorem in the case where $M= \ball_{R,w}$.}
\end{proof}
%
%\section{Calder\'on problem}
%\todo[inline]{Okay, we can surely recover $\monogenics^{0+2}(\Omega)$ which is $\monogenics^+(\Omega)$ when $\Omega$ is dimension 3 or less.  Is this all we really need? Otherwise, we may be at a loss here.}

\todo[inline]{Go over Ohm's law (or do it in the forms and integration section) but relate it back to the stuff here so that the conjugate field gets some interpretation.}

\todo[inline]{Explain this using the variational approach and explain that $\Omega$ is ohmic where Ohm's law is a linearization of conductivity and such (just like linear elasticity). The electromagnetic potential (or something) is a monogenic spinor?}
\subsection{Electromagnetism}

Consider the spacetime algebra $\G_{1,3}$ seen in Example \ref{ex:spacetime_algebra}. Then the spacetime multivector fields on $\Omega$ are $\G_{1,3}(\Omega)$ with the basis vector fields $e_t,e_1,e_2,e_3$.  A vector field $F$ on $\Omega$ is then given by
\[
A= A_t e_t + A_1 e_1 + A_2 e_2 + A_3 e_3,
\]
where each coefficient is a smooth scalar field. We denote now by $\grad_{st} = e^t \nabla_{e_t} + \sum_{j=1}^3 e^j \nabla_{e_j}$ the spacetime gradient and take $\grad = \sum_{j=1}^3 e^j \nabla_{e_j}$ as the spatial gradient. Let $\vectorpotential = A_1 e_1 + A_2 e_2 + A_3 e_3$ be the spacelike part of the spacetime vector $A$ and let $\phi = A_t$ be the timelike part. If the vector field does not depend on the temporal variable $t$ we have $\nabla_{e_t} A=0$ and thus
\[
\grad_{st} A = \grad u e_t + \grad \wedge \vectorpotential + \grad \cdot \vectorpotential.
\]
If we then take the Lorenz gauge condition $\grad \cdot \vectorpotential =0$, we have
\[
\|\grad_{st}A\|^2 = \| \grad u\|^2 + \|\grad \wedge \vectorpotential\|^2.
\]
The Lagrangian for this field is then
\[
\mathcal{L}(A) = \|\grad_{st}A\|^2 - A \cdot J,
\]
where $J = \rho e_t + J_1 e_1 + J_2 e_2 + J_3 e_3$ is a spacetime vector field \textcolor{red}{A lagrange multiplier?}. The Euler-Lagrange equations with the gauge condition yields
\[
\grad_{st}^2 A = J.
\]
Let $\current = J_1 e_1 + J_2 e_2 + J_3 e_3$ and if we take a static four current $\nabla_{e_t} J=0$ we must have $\nabla_{e_t} A =0$ and we arrive at two equations
\[
\grad \cdot \grad \wedge u e_t = \rho e_t \qquad \textrm{and} \qquad \grad \cdot \grad \wedge \vectorpotential = \current, 
\]
of course one can take $\Delta u = \rho$, but we this equation arises from the spacetime formulation itself. Note that we did not force an inner product on the spatial vectors $e_1,e_2,e_3$ other than they are orthogonal with the temporal vector $e_t$.  These equations we have are the invariant forms of the equations with respect to any (positive definite) spatial inner product. This will be important momentarily.

In this, we have realized the electric and magnetic fields
\[
\grad \wedge u e_t = e \qquad \textrm{and} \qquad \grad \wedge \vectorpotential = b,
\]
and note the electric field $e$ is a spacetime bivector and the magnetic field $b$ is a purely spatial bivector that $\grad \wedge e = 0$ and $\grad \wedge b =0$ are satisfied. The fact that $e$ is a spacetime bivector means it behaves like a spacelike vector when acted on by spatial gradient $\grad$ owing to the static Faraday's law $\grad \times E =0$. Since $b$ is purely spatial, we see $\grad \wedge b = 0$ mimics the Gauss's law for magnetism if we take the unit spacelike trivector $I$ and let $B=bI$ be the magnetic vector field we have $\grad \cdot B = 0$.

In the EIT problem, we begin with a region $\Omega$ with unknown symmetric positive definite conductivity matrix $\gamma$. We apply a static scalar potential $\phi$ on $\Sigma$ which produces the potential $u^\phi$ in the interior. We assume $\Omega$ is an ohmic material in that Ohm's law $-\gamma \grad \wedge u^\phi = \current$ is satisfied.  It follows that 
\[
-\gamma \grad \wedge u^\phi = \grad \cdot b.
\]
The conservation law
\[
\int_\Sigma J \cdot \nu d\Sigma = 0,
\]
implies $\grad \cdot \current= 0$ and we arrive at $\grad \cdot (\gamma \grad \wedge u) = 0$. See, for example, \cite{feldman_calderproblem_nodate}. 

The conductivity matrix was given in an terms of an orthonormal spatial basis under the Euclidean inner product and we can write the components $\gamma^{ij}$ for $i,j=1,2,3$.  In \cite{uhlmann_inverse_2014} we find a relationship between the intrinsic Riemannian metric on a space and the conductivity by
\begin{equation}
\label{eq:conductivity_metric}
    g_{ij} = (\det \gamma^{k\ell} )^{\frac{1}{n-2}} (\gamma^{ij})^{-1}, \quad \gamma^{ij} = (\det g_{k\ell})^{\frac{1}{2}} (g_{ij})^{-1}.
\end{equation}
If we impose the inner product on the spatial components come from $g$ with coefficients in the basis given by the above equations, we can note we have Ohm's law by
\[
-\grad \wedge u^\phi = \current,
\]
In the static case when there are no free charges inside $\Omega$, we have
\[
\Delta_{st} A = \current \quad \textrm{in $\Omega$},
\]
and we arrive at $\Delta u = 0$ for the scalar potential and $\Delta \vectorpotential = \current$ for the magnetic vector potential. In terms of the magnetic field bivector, we have $\grad \cdot b = \current$ and once again by Ohm's law we have $-\grad \wedge u^\phi = \grad \cdot b$. This leads us to consider the parabivector field $f=u+b$. We can note that $f$ is (spatially) monogenic since 
\[
\grad f = 0 ~\iff~ -\grad \wedge u^\phi =  \grad \cdot b ~\textrm{and}~ \grad \wedge b = 0,
\]
is satisfied. We see now that the fact that the body $\Omega$ is ohmic gives us a necessary coupling between the scalar potential and the magnetic field.

\subsection{Biot Savart Law}

Recall the Biot Savart law from electromagnetic theory
\[
\vec{B}(y)=\frac{1}{4\pi} \int_\Omega \current \times \frac{y-x}{|y-x|^3} d\Omega(x),
\]
which satisfies 
\[
\grad \times \vec{B} = \current + \frac{1}{4\pi} \grad \wedge \int_\Omega \frac{\grad \cdot \current}{|y-x|}d\Omega(x) -\frac{1}{4\pi} \grad \wedge \int_\Sigma \frac{\current \cdot \nu}{|y-x|}d\Sigma(x)
\]
In the EIT problem we do not allow charges to accumulate in the interior and so we must have
\[
\grad \cdot \current = 0,
\]
so long as $\grad \cdot \current$ is continuous \cite{feldman_calderproblem_nodate}. Hence we are left with
\[
\grad \times \vec{B} = \current -\frac{1}{4\pi} \grad \wedge \int_\Sigma \frac{\Lambda(\phi)\\}{|y-x|}d\Sigma(x),
\]
where $\Lambda$ is the DN map.

\begin{remark}
    It seems like this now says that $u$ has a conjugate field $B$ if and only if
\[
\frac{1}{4\pi} \grad \wedge \int_\Sigma \frac{\Lambda(\phi)\\}{|y-x|}d\Sigma(x) = 0.
\]
\end{remark}
Assuming we can swap differentiation and integration we have
\begin{align*}
\grad \wedge \int_\Sigma \frac{\Lambda(\phi)}{|y-x|} d\Sigma(x) &= \int_\Sigma \frac{\Lambda(\phi)(y-x)}{|y-x|^3}d\Sigma(x),
\end{align*}
since $\grad \wedge \Lambda = 0$ \textcolor{red}{In B.V. DN-Forms}.

\begin{remark}
Perhaps we can just rearrange to see:
\[
\cauchy [\current] = \frac{1}{4\pi} \int_\Sigma \frac{y-x}{|y-x|^3} (\nu \cdot \current + \nu \wedge \current) d\Sigma(x) 
\]
and we note $\current \cdot \nu = \Lambda(\phi)$ for which we have found
\[
 \frac{1}{4\pi} \int_\Sigma \frac{\Lambda(\phi)(y-x)}{|y-x|^3}=\grad \times \vec{B}-\current,
\]
Hence 
\[
\cauchy[\current] = \grad \times \vec{B} - \current + \frac{1}{4\pi} \int_\Sigma \frac{y-x}{|y-x|^3} \nu \wedge \current d \Sigma(x)
\]
\end{remark}

In terms of geometric algebra, we wish to show the analogous statement for the magnetic bivector field
\[
B(y) = \frac{1}{4\pi} \int_\Omega \current \times \frac{y-x}{|y-x|^3} d\Omega(x),
\] 
in that
\[
\grad \cdot \frac{1}{4\pi} \int_\Omega \current \wedge \frac{y-x}{|y-x|^3} d\Omega(x) = \current.
\]



\subsubsection{Discussion}

The scalar potential in the EIT problem arises inside of a four vector potential for the electromagnetic field.  The electromagnetic potential satisfies Maxwell's equations which can be succinctly stated as $\grad_{st}^2 A = J$, for the four current $J$.  When the four current $J$ does not depend on time, we arrive at the static equations where the electrostatic potential $u$ and magnetic spatial vector potential $\vectorpotential$ are split into separate equations. Removal of time dependence decouples these potentials. We realize the magnetic field as the bivector $b=\grad \wedge \vectorpotential$ and the electric vector field $E=\grad \wedge u$. 

These fields interact with materials which carry an intrinsic inner product related to the conductivity by \ref{eq:conductivity_metric}. If the material is ohmic, we have Ohm's law given by $\grad \wedge u = \grad \cdot b$ which leads to the parabivector field $f=u+b$ to be monogenic. This relationship is important and is not fully realized without the proper treatment of the electromagnetic potential. 

In an electrostatic boundary value problem, one can supply the scalar potential $\phi$ on the boundary of a region. This forced scalar potential induces the scalar potential inside of the region and the scalar potential is harmonic when the interior is free of charges.  This scalar potential drives a current $\current$ via Ohm's law, and this current is related to the magnetic bivector field $b$. One may only have access to the boundary of the region and can make measurements of the resulting current flux $\projection{\nu}{\current}$ that corresponds to a given input scalar potential $\phi$. Is this enough to determine the underlying inner product of the region?


\subsection{Generalization}
\todo[inline]{Explain how we can put $\gamma$ as a spatial metric and incorporate this into the geometric algebra for $\G_{1,n}(\Omega)$ stuff. Contract away time part again and we get the same equations.}
What we have seen for the electromagnetic field is there is a coupling between the electric bivector field and the magnetic bivector field via the four vector potential.  This can be generalized to fields in $\G_{1,n}(\Omega)$ to produce analogous equations.

\subsection{Inverse problem}

A particular application for the work we have done thus far is with the Calder\'on inverse problem. One can work with differential forms, but we have found forms to be rooted in multivectors contracted with a directed measure.  We also note the previous portion on electromagnetism provides a convenient understanding for this problem. The forward problem in terms of geometric calculus is given by the following scenario. We have an ohmic $\Omega$ and we find the electrostatic potential $u$ satisfying the Dirichlet problem
\begin{equation}
\label{eq:dirichlet_problem}
\begin{cases} \Delta u^\phi = 0 & \textrm{ in $\Omega$} \\  u^\phi \vert_\Sigma = \phi & \textrm{ on $\Sigma$}. \end{cases}.
\end{equation}
In the realm of Electrical Impedence Tomography (EIT), the Dirichlet data $\phi$ amounts to an input voltage along the boundary and by Ohm's law $\current=\grad \wedge u^\phi$ provides us the current. For any given solution to the boundary value problem, there is the corresponding Neumann data $\current^\perp=\projection{\nu}{\grad u^\phi}$ where $\nu$ is the normal to the boundary $\Sigma$ defined by $\nu = I_\Sigma I$ for the oriented boundary pseudoscalar $I_\Sigma$. This motivates the so called Voltage-to-Current (VC) operator $\phi \mapsto \current^\perp$. In general, we refer to set of both boundary conditions $(\phi, \current^\perp)$ $\forall \phi$ as the \emph{Cauchy data} and define the \emph{Dirichlet-to-Neumann (DN) operator} $\Lambda$ such that $\Lambda \phi = \current^\perp$. This mimics the VC operator in EIT. With our notation from before we have
\[
\Lambda \phi = \projection{\nu}{\grad u^\phi} = \current^\perp.
\] 
Note that this operator $\Lambda$ is often referred to as the \emph{scalar} DN operator since the input is the scalar field $\phi$ whereas a more general operator on differential $k$-forms has been described in \cite{belishev_dirichlet_2008,sharafutdinov_complete_2013}. The inverse problem follows.

\vspace*{5pt}
\noindent\textbf{Calder\'on problem.} Let $\Omega$ be an unknown Riemannian manifold with unknown metric $g$ and with known boundary $\Sigma$ and known DN operator $\Lambda$. Can one recover $\Omega$ and the spatial inner product $g$ from knowledge of $\Sigma$ and $\Lambda$?
\vspace*{5pt}

\subsection{Recovering monogenic fields from $\Lambda$}

With the DN operator, we can reconstruct the boundary four current $J$.  On $\Sigma$, we have the gradient $\grad_\Sigma$ inherited from $\grad$ on $\Omega$.  In particular, we have the relationship
\[
\grad_\Sigma \phi = \projection{I_\Sigma}{\grad \phi},
\]
which is accessible with our knowledge of $\phi$ and $\Sigma$. The boundary current is then
\[
\current\vert_{\Sigma} = \grad_\Sigma \phi + \Lambda(\phi).
\]
Though we do not have access to $u^\phi$ directly, we do know that $\Delta u^\phi = \rho$ and as such we have the boundary four current by
\[
J\vert_\Sigma = \Delta u^\phi\vert_\Sigma \gamma_0 + \current\vert_\Sigma
\]
as well as the interior four current $J = \current$ since the interior is free of charges.  Defining the the four vector potential as before, we arrive at the extra equation $\Delta \vectorpotential = \current$ in $\Omega$. Once again define the magnetic bivector field $b=\grad \wedge \vectorpotential$ and we note that Ohm's law implies $\grad \cdot b = -\grad \wedge u^\phi$ in $\Omega$ and so the parabivector field $f=u^\phi + b$ is spatially monogenic since we also have $\grad \wedge b = 0$.  This all holds assuming that we can solve the electromagnetic Neumann boundary value problem
\[
\begin{cases} \Delta A = \current & \textrm{in $\Omega$}\\ A = A_\Sigma & \textrm{on $\Sigma$} \end{cases}
\]
\todo[inline]{Show that we can determine the magnetic potential $A_\Sigma$ on the boundary. This may also show that the two notions of the DN operator are equivalent. That'd be nice.}

\todo[inline]{If we show there is always a unique monogenic conjugate $b$ for any harmonic $u$ then this must be what we are doing here. Is this gauranteed by the Cauchy integral?}

Though briefly we mentioned $\Omega$ as a Riemannian manifold, we now take $\Omega$ to be a region in $\R^n$ for brevity. Using the DN operator, one can define a \emph{Hilbert transform} by
\[
T \phi  = d\Lambda^{-1} \phi,
\]
as in \cite{belishev_dirichlet_2008}. It has yet to be shown that this definition coincides with the definition in \cite{brackx_hilbert_2008}, but there is reason to believe they are related. The classical Hilbert transform on $\C$ inputs a harmonic function and outputs another harmonic function $v$ such that $u+iv$ is holomorphic. Essentially, this translates into finding a conjugate bivector field $b$ to $u^\phi$ such that $u^\phi +b$ is monogenic. First, we require $\phi$ satisfies
\todo[inline]{This statement should come from the lagrangian perspective hopefully.}
\begin{equation}
\label{eq:conjugate_requirement}
\left( \Lambda + (-1)^{n}d\Lambda^{-1}d\right)\phi = 0,
\end{equation}
where $d$ is the exterior derivative on forms. \textcolor{red}{They show how to find the image of this, perhaps I can show what the kernel is.} As shown earlier in Section \ref{subsec:diff_forms}, $d$ amounts to $\grad \wedge$ on the multivector field constituent of a form.  When condition \ref{eq:conjugate_requirement} is met, there exists a \emph{conjugate form} $\epsilon \in \Omega^{n-2}(M)$. As well, $\epsilon$ is also coclosed in that $\delta \epsilon=0$. To retrieve the constituent $(n-2)$-vector $E$, we just note $\epsilon = E \cdot dX_k$. Given Hodge duality, we have a 2-form $\beta$ such that $\star\beta = \epsilon$ and the corresponding bivector $b^\star=E$.  Combining the fields $u^\phi$ and $b$ into the parabivector $f=u^\phi+b \in \G_n^{0+2}(\Omega)$. We then note that $f$ is monogenic if and only if
\[
\grad \wedge u = -\grad \cdot b \qquad \textrm{and} \qquad \grad \wedge b = 0.
\]

\begin{lemma}
Given the fields $u^\phi$ and $b$ as above, the corresponding parabivector field
\[
f=u^\phi +b
\]
is monogenic.
\end{lemma}
\begin{proof}
Let $\star \beta^\psi = \epsilon$ as before and note that 
\begin{equation}
\label{eq:conjugate_belishev}
d u^\phi = \star d \epsilon = \star d \star \beta^\psi,  
\end{equation}
as shown in Theorem 5.1 in \cite{belishev_dirichlet_2008}. The multivector equivalent of the right hand side of Equation \cite{eq:conjugate_belishev} yields
\begin{align*}
(\grad \wedge b^\star )^\star &= [(\grad \cdot b^\dagger) I]^\star\\
    &= [I^{-1} ((\grad \cdot b^\dagger) I)]^\dagger\\
    &= ((\grad \cdot b^\dagger)I)^\dagger I\\
    &= \grad \cdot b^\dagger && \textrm{since $\dagger$ of a vector is trivial}\\
    &= -\grad \cdot b. && \textrm{since $\dagger$ of a bivector is -1}
\end{align*}
\textcolor{red}{Perhaps I should just show this property in the differntial forms section.} Thus, we have $\grad \wedge u + \grad \cdot b = 0$. Since $\epsilon$ is coclosed we have
\begin{align*}
0=\grad \cdot b^\star &= \grad \cdot (I^{-1} b)^\dagger \\
    &= \grad \cdot (b^\dagger I)\\
    &= (\grad \wedge b^\dagger) I\\
  \implies ~0  &= \grad \wedge b.
\end{align*}
\textcolor{red}{Perhaps I should just show this property in the differntial forms section.} Thus $\grad f =0$ and $F$ is monogenic.
\end{proof}

We have shown that conjugate forms give rise to monogenic fields.  We now seek to determine for what boundary conditions $\phi$ we have at our disposal. Let $E^\parallel \coloneqq \projection{I_\Sigma}{E}$, with $I_\Sigma$ the boundary pseudoscalar satisfying $\nu I_\Sigma = I$. Hence by Equation \ref{eq:projection_rejection_vectors} we have $E^\parallel = \rejection_{\nu}(E)$ then in investigating the requirement from Equation \ref{eq:conjugate_requirement} we find the multivector equivalent
\begin{align*}
    (\Lambda + (-1)^n (\grad \wedge) \Lambda^{-1} (\grad \wedge))\phi &= E^\perp + (-1)^n T E^\parallel
\end{align*}
so we arrive at the fact that we must have
\[
E^\perp = (-1)^{n-1} T E^\parallel.
\]
In other words,
\[
T  \rejection_\nu(E)= (-1)^{n-1}\projection{\nu}{E}.
\]
Thus, the Hilbert transform maps tangential components of $\grad u^\phi = E$ to nontangential boundary components on the boundary.


%
%\section{Conclusion}
%\todo[inline]{Write a conclusion.}
%
%\section{Extras}
%
%Journals:
%\begin{itemize}
%    \item Advances in Applies Clifford Algebras.
%\end{itemize}

% EXTRA STUFF BELOW
%\section{Further questions}
%\subsection{Generating axial monogenics}

The following questions remain for a domain in $\R^3$.

\begin{question}
    For what boundary values $\varphi \in C_\infty(\Sigma)$ can we generate axial monogenics?
\end{question}

\begin{question}
    Do these boundary values exhaust the whole axial algebra $\algebra{\omega}$?
\end{question}

Fix an axis $\omega$ which defines the blade $B = \omega I$ and thus defines the $B$-plane in $\R^3$.  Then, let $f=u+\beta B$ be an $\omega$-axial monogenic.  We can then determine the boundary values for $f$ on $\Sigma$ by orthogonal projection onto the $B$-plane.  That is, we care only about the components of $f$ perpendicular to the axis $\omega$ and hence we take for $\zeta \in \Sigma$
\[
\zeta^\perp = \omega \omega \wedge \zeta = (x\cdot B)B^{-1}.
\]
showing the relationship between projection onto a plane and being orthogonal to an axis in $\R^3$. Specifically, this means that the relationship $f(x)=f(x+t\omega)$ can be written as
\[
f(x)=f((x\cdot B)B^{-1}),
\]
in that we only care about the portion of $x$ along the plane given by $B$.  Thus, for $\xi \in \Sigma$ we have
\[
f(\xi) = f((\xi \cdot B)B^{-1}).
\]

\begin{figure}[H]
	\centering
	%\def\svgwidth{\columnwidth}
	\resizebox{\columnwidth}{!}{\input{omega_axial.pdf_tex}}
\end{figure}

\textcolor{red}{So boundary values of axial monogenics are axial and...?.}

\begin{example}
    Consider the 3-dimensional example with $M=B_3$ and $\Sigma=S^2$.  Let $e_1,e_2,e_3$ be a global orthonormal basis and let $g_{ij}=\delta_{ij}$.  Then let $B=e_1 \wedge e_2$.  Then the paravector field $f(x^1,x^2,x^3)=x^1+x^2B$ is $e_3$-axial. Clearly we can see that $f(x^1,x^2,x^3+t)=f(x^1,x^2,x^3)$ for any $t$.  $f$ is also monogenic as one can show
    \[
        \grad f = e_1 + (e_2 \wedge e_3)I = e_1 - e_1 = 0.
    \]
    Indeed, this $f$ is none other than the complex function $f(z)=z$ with $B$ taking the role of the imaginary unit $i$. 

    Let $x=x^1e_1 + x^2e_2 + x^3e_3$.  Then, 
    \[
        B (x\cdot B) = (e_1e_2)( x^1e_2 -x^2 e_1 ) = x^1 e_1 + x^2 e_2.
    \] 
    Thus, for $\xi \in S^2$, we have $f(\xi)=\xi^1 +\xi^2 B$.
\end{example}

\todo[inline]{If we consider now every $\omega$-axial monogenic can be written as a power series, if we can construct $z$ we should be done...?}

It is clear that we can define a monogenic field $f=u+b$ via the Cauchy integral, but we then require $\nabla_\omega f = 0$.  Let $f=\cauchy[\varphi](x)$, then we must have
\[
\nabla_\omega \proj{0}{\cauchy[\varphi](x)} = 0 \qquad \textrm{and} \qquad \nabla_\omega \proj{2}{\cauchy[\varphi](x)}=0.
\]
The first condition yields
\[
0 = \int_\Sigma \frac{(\nu(\zeta)\cdot x) (\omega \cdot x)}{|x-\zeta|^2} \phi(\zeta) d\Sigma(\zeta).
\]


\begin{theorem}
    For any $\omega \in Gr(1,3)$ we have that $\algebra{\omega}\subset \monogenics$. 
\end{theorem}
\begin{proof}
    \textcolor{red}{This seems to be saying that we need boundary values in some hardy space or something. They defined this conjugacy thing as $G$.}
    Fix a unit vector $\omega$.  We want to show that for any $f=u+b\in \algebra{\omega}$ that $\iota^* u=\phi$ satisfies \ref{eq:conjugacy_requirement}.  That is,
    \[
        G\phi = (\Lambda - d\Lambda^{-1}d) \phi = 0.
    \]
    Note that $\phi$ is the trace of a harmonic function, so this operator is well defined.  Note that the equation
    \[
        \Lambda \psi = d \phi
    \]
    has a solution
\end{proof}

\section{Radon transform and integral geometry}

I feel like there is some way to go from projection onto subspaces as a map to grassmannians and reconstructing the manifold.  It's like a morse function type of thing.  Radon transforms also come to mind.

\section{Relation to the BC Method}

\textcolor{red}{Describe how this process can lead to the BC method in dimension $n=2$}


\section{Conclusion}


\appendix
\section{Appendix}

\todo[inline]{Put axial condition for cauchy integral and some other quick proofs in here.}

\subsection{Spin fibration}
maybe pose this as a question in relation to using the 2d belishev stuff.

\textcolor{red}{The inner product for characters is what you use for fourier theory, maybe we can do something here with characters as maps to the grassmannian? Do these form some kind of orthogonal basis? Also, the Dirac operator and Laplacian are spin invariant! This is what they use the $\mathbb{H}$ module structure for!}

A main question to answer now is how the $B$-planar algebras $\algebra{B}$ relate to the space of monogenic functions $\monogenics$.  In particular, this question seems analogous to the invertibility of a $2$-plane x-ray transform.  Let $f$ be a monogenic, can $f$ be generated by $B$-planar monogenics? Noting that each unit 2-blade corresponds to a unique 2-plane in $\R^n$, we can realize every $B$ as a point in $\Grassmannian{2}{n}$.  Letting $f_B$ be some $B$-planar axial monogenic, is
\[
f = \int_{B \in \Grassmannian{2}{n}} a(B) f_B d \lambda,
\]
where $a(B)$ is a scalar function on $\Grassmannian{2}{n}$ and $d\lambda$ is the Haar measure on $\Grassmannian{2}{n}$ monogenic? Moreover, can any monogenic $f$ be constructed in this manner? First, we start with a lemma describing the form of $f_B$.


\begin{lemma}
    Let $f$ be a monogenic (0+2)-vector and define $f_B \coloneqq \projection{B}{f(\projection{B}{x})}$. Then $f_B$ is $B$-planar and monogenic.  
\end{lemma}
\begin{proof}
    It is clear by definition that $f_B$ is constant along translations of the $B$-plane and can be written as $u_B+\beta b_B$ and so $f_B$ is $B$-planar.  To see $f_B$ is monogenic, let $e_1,\dots,e_n$ be a basis such that $B=e_1e_2$ and $e_i \cdot B = 0$ for $i\neq 1,2$. Then note $\nabla_{e_i} f_B =0$ when $i\neq 1,2$ as well leading to
    \[
        \grad f_B = e^1 \nabla_{e_1}f_B + e^2 \nabla_{e_2}f_B
    \]
    Recall that $f=u+b$ must satisfy
    \[
        \grad \wedge u = \grad \cdot b \qquad \textrm{and} \qquad \grad \wedge b = 0.
    \]
    Specifically,
    \[
        e^1 \wedge \nabla_{e_1} u + e^2 \wedge \nabla_{e_2}u + \cdots + e^n \wedge \nabla_{e_n} = e^1 \cdot \nabla_{e_1} b + e^2 \cdot \nabla_{e_2}b + \cdots + e^n \cdot \nabla_{e_n} b
    \]
    Clearly, $\grad \wedge b_B = 0$, thus we need only show
    \[
        \grad \wedge u_B = \grad \cdot b_B.
    \]
    In particular
\end{proof}


We can note that the $B$-planar monogenics are given by a power series $\sum_{n=0}^\infty a_n (x+yB)^n$ due to the isomorphism of algebras $\mathfrak{spin}(2)\cong \C$ \textcolor{red}{This shouldn't be hard to show without appealing to this isomorphism.} In particular, any $B$-planar monogenic is approximated arbitrarily closely by a homogeneous polynomial of degree $n$ in the variables $x$ and $y$. Moreover, $1$ and $x+yB$ generate the $B$-planar monogenics. $\spingroup$ then acts on $B$. \textcolor{red}{Okay, well maybe there's some nice way to talk about characters as mappings to the grassmannian instead of the circle? Should read more about characters and maybe they are really maps to spin group? They are for the 2d case. Structure space and stuff. Should probably rename some of these things I have.s}


\textcolor{red}{Countable basis for $\monogenics$ ?}


\bibliographystyle{siam}
\bibliography{calderon_problem}





\end{document}
