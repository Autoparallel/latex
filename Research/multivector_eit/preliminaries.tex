The complex algebra $\C$ can be generalized in a handful of ways.  Some of which can be found through the use of Clifford algebras and, more specifically, in geometric algebras.  We define the more general Clifford algebras first and realize geometric algebras as particularly nice Clifford algebras with a quadratic form arising from an inner product. Elements of a geometric algebra are known as multivectors and these multivectors carry a wealth of geometric information in their algebraic structure. $\C$ itself can be realized as a special subalgebra of biparavectors in the geometric algebra on $\R^2$ with the Euclidean inner product and the quaternions $\quat$ are realized as an analogous algebra on $\R^3$. In particular, both $\C$ and $\quat$ arise as the 2- and 3-dimensional even Clifford groups $\Gamma^+$ respectively.

\subsection{Clifford and geometric algebras}

Formally, we let $(V,Q)$ be an $n$-dimensional vector space $V$ over some field $K$ with an arbitrary quadratic form $Q$.  The tensor algebra is given by
\[
\mathcal{T}(V) \coloneqq \bigoplus_{j=0}^\infty V^{\otimes j} = K \bigoplus V \oplus (V\otimes V) \oplus (V\otimes V \otimes V) \oplus \cdots,
\]
where the elements (tensors) inherit a multiplication $\otimes$ (the tensor product). From the tensor algebra $\mathcal{T}(V)$, we can quotient by the ideal generated by $v\otimes v - Q(v)$ to create a new algebra.
\begin{definition}
The \emph{Clifford algebra} $C\ell(V,Q)$ is the quotient algebra
\begin{equation}
C\ell(V,Q) = \mathcal{T}(V) ~ / ~ \langle v \otimes v - Q(v) \rangle.
\end{equation}
\end{definition}
To see how the tensor product descends to the quotient, we let $e_1, \dots, e_n$ be an arbitrary basis for $V$, then we can consider the tensor product of basis elements $e_i \otimes e_j$ which induces a product in the quotient $C\ell(V,Q)$ which we refer to as the \emph{Clifford multiplication}. In this basis, we write this product as concatenation $e_ie_j$ and define the multiplication by
\begin{equation}
\label{eq:clifford_multiplication}
e_i e_j = \begin{cases} Q(e_i) & \textrm{if $i=j$}, \\ e_i \wedge e_j & \textrm{if $i\neq j$},\end{cases}
\end{equation}
where $\wedge$ is the typical exterior product satisfying $v\wedge w = - w\wedge v$ for all $v,w\in V$.  As a consequence, the exterior algebra $\bigwedge(V)$ can be realized as a subalgebra of any Clifford algebra over $V$ or as a Clifford algebra with a trivial quadratic form $Q=0$.  

In the case where $V$ has a (pseudo) inner product $g$, we can induce a quadratic form $Q$ by $Q(v)=g(v,v)$ and give rise to a special type of Clifford algebra which motivates the following definition.
\begin{definition}
Let $V$ be a vector space with an (pseudo) inner product $g(\cdot,\cdot)$. Then taking $Q(\cdot) = g(\cdot,\cdot)$, the Clifford algebra $C \ell(V,Q)$ is called a \emph{geometric algebra}.
\end{definition}
In general, we put $\geometricalg$ and assume the inner product will be given alongside or will be clear from context.  For example, when $V=\R^n$ and we define $Q$ from the Euclidean inner product $|\cdot|$, we have $C\ell(V,Q)=\mathcal{G}(\R^n)$ and moreover we let $\mathcal{G}_n \coloneqq \mathcal{G}(\R^n)$. Geometric algebras are an old and widely studied topic. For more information, see the classical text \cite{hestenes_clifford_1986} or the more modern text \cite{doran_geometric_2003} which also provides a wide range of applications to physics problems. Both these sources include much of the other necessary preliminaries I cover in the remainder of this section. Finally, the paper \cite{chisolm_geometric_2012} proves many of the useful identities and notation used throughout this paper.

\subsubsection{Grading and multivectors}
Note that $C\ell(V,Q)$ is a $\mathbb{Z}$-graded algebra with elements of grade-0 up to elements of grade-$n$. We refer to grade-0 elements as scalars, grade-1 elements as vectors, grade-2 elements as \emph{bivectors}, grade-$r$ elements as \emph{$r$-vectors}, and grade-$n$ elements as \emph{pseudoscalars}. We denote the space of $r$-vectors by $C\ell(V,Q)^r$. For each grade there is a basis of ${n\choose r}$ \emph{$r$-blades} which are $r$-vectors of the form
\begin{equation}
\blade{A_r} = \prod_{j=1}^k v_j, ~\textrm{for linearly independent}~ v_j \in V,
\end{equation}
and we use a boldface to specify that a $r$-vector is a $r$-blade (except for the special case of vectors which can be thought of as $1$-blades). For example, if $\dim(V)=3$, then there are ${3\choose 2}=3$ 2-blades that form a basis for the bivectors and one particular choice of a bivector basis would be the following list of 2-blades
\begin{equation}
\label{eq:3_dim_basis}
\blade{B}_{12} = e_1 \wedge e_2, \quad \blade{B}_{13} = e_1 \wedge e_3, \quad \blade{B}_{23} = e_e \wedge e_3.
\end{equation}
We refer to an $(n-1)$-blade as a \emph{pseudovector} and it should be noted that every $(n-1)$-vector is a pseudovector. In other literature, some will refer to a $r$-blade as a \emph{simple} or a \emph{decomposable} $r$-vector\todo{citations}. 

In general, an element $A \in C\ell(V,Q)$ is written as a linear combination of basis elements of all possible grades and we refer to $A$ as a \emph{multivector}.  To extract the grade-$r$ components of $A$, we use the notation
\begin{equation}
\proj{r}{A}
\end{equation}
to denote the grade-$r$ components of the multivector $A$. Any multivector $A$ can then be given by
\begin{equation}
A = \sum_{r=0}^n \proj{r}{A}
\end{equation}
which shows the decomposition
\begin{equation}
C\ell(V,Q) = \bigoplus_{j=0}^n C\ell(V,Q)^j.
\end{equation}
If $A$ contains only components of a single grade, then we say that $A$ is \emph{homogeneous} and if the components are grade-$r$ we write $A_r$ and refer to $A_r$ as a \emph{homogeneous $r$-vector} or simply a \emph{$r$-vector}.  For example, when we refer to vectors we realize them as 1-vectors and likewise we realize bivectors as 2-vectors. Also of interest will be the elements in
\begin{equation}
 C\ell(V,Q)^{0+2} = C\ell(V,Q)\oplus C\ell(V,Q)^2
\end{equation}
which we refer to as \emph{biparavectors}.

The Clifford multiplication of vectors defined in \ref{eq:clifford_multiplication} can be extended to multiplication of vectors with homogeneous $r$-vectors.  In particular, given a vector $v \in C\ell(V,Q)$ and a homogeneous $r$-vector $A_r \in C\ell(V,Q)$, we have
\begin{equation}
\label{eq:vector_multiplication}
aA_r = \proj{r-1}{aA_r} + \proj{r+1}{aA_r},
\end{equation}
which decomposes the multiplication into a grade lowering \emph{interior product} and a grade raising \emph{exterior product}.  This allows us to extend the Clifford multiplication further. Given a homogeneous grade-$s$ multivector $B_s$, we have
\begin{equation}
\label{eq:general_clifford_multiplication}
A_k B_s = \proj{|r-s|}{A_rB_s} + \proj{|r-s|+2}{A_rB_s} + \cdots + \proj{r+s}{A_rB_s}.
\end{equation}
This rule for multiplication then allows for the multiplication of two general multivectors in $C\ell(V,Q)$. For this multiplication, specific grades of the product are worth noting.
\begin{equation}
\label{eq:dot}
    A_r \cdot B_s \coloneqq \proj{|r-s|}{A_r B_s}
\end{equation}
\begin{equation}
\label{eq:wedge}
    A_r \wedge B_s \coloneqq \proj{r+s}{A_r B_s}
\end{equation}
\begin{equation}
\label{eq:left_contraction}
    A_r \rfloor B_s \coloneqq \proj{s-r}{A_r B_s}
\end{equation}
\begin{equation}
\label{eq:right_contraction}
    A_r \lfloor B_s \coloneqq \proj{r-s}{A_r B_s}.
\end{equation}
These products are particularly emphasized as many helpful identities used in this paper are phrased using these notions. Taking \cref{eq:vector_multiplication,eq:wedge,eq:left_contraction} into mind, we see that we have the grade lowering interior product can be written as
\begin{equation}
    \proj{r-1}{aA_r} = a\rfloor A_r = a \cdot A_r
\end{equation}
and the grade raising exterior product can be written as
\begin{equation}
    \proj{r+1}{aA_r} = a \wedge A_r.
\end{equation}

As discussed, $C\ell(V,Q)$ is naturally a $\mathbb{Z}$-graded algebra but we also find that it carries a $\mathbb{Z}/2\mathbb{Z}$-grading as well. This additional grading can be realized by sorting $k$-vectors in $C\ell(V,Q)$ into the sets where $k$ is even or odd.  We say a $k$-vector is \emph{even} (resp. \emph{odd}) $k$ is even (resp. odd) and in general if a multivector $A$ is a sum of only even (resp. odd) grade elements we also refer to $A$ as even (resp. odd).  Taking note of the multiplication defined in \ref{eq:general_clifford_multiplication}, one can see that the multiplication of even multivectors with another even multivectors outputs an even multivector.  Thus, the even multivectors form closed subalgebra of $C\ell(V,Q)$ which we denote by $C\ell(V,Q)^+$. We end this subsection with a few examples.

\begin{example}
\label{ex:complex_representation}~
Consider $\G_2$ with $e_1$ and $e_2$ the standard vector basis and note that we have $1$ as the basis scalar, and $\blade{B_{12}} = e_1\wedge e_2 = e_1e_2$ as the basis pseudoscalar.  Then, an arbitrary multivector $A$ and $B$ can be specified by
\[
A = a_0 + a_1 e_1 + a_2 e_2 + a_{12} B_{12}, \qquad B = b_0 +b_1 e_1 + b_2 e_2 + b_{12}\blade{B_{12}},
\]
and the graded elements of $A$, for example, can be extracted as
\begin{subequations}
\begin{align}
\proj{0}{A}&=a_0\\
\proj{1}{A}&=a_1 e_1 + a_2 e_2\\
\proj{2}{A}&=a_{12} \blade{B_{12}}.
\end{align}
\end{subequations}
We can then take the product $AB$ to yield
\begin{subequations}
\begin{align}
\proj{0}{AB} = a_0b_0 + a_1 b_1 + a_2 b_2 - a_{12}b_{12}\\
\proj{1}{AB} = (a_0 b_1 + a_1 b_0 - a_2 b_{12} + a_{12} b_2) e_1 + (a_0 b_2 + a_2 b_0 + a_1b_{12} - a_{12} b_1) e_2\\
\proj{2}{AB} = (a_1b_2 - a_2 b_1)\blade{B_{12}}.
\end{align}
\end{subequations}
Most notably, we see that $\blade{B_{12}}^2=-1$ and this allows us to consider a biparavector
\begin{equation}
z = x + y \blade{B_{12}}
\end{equation}
as a representation of the complex number $\zeta = x+ iy$ in $\G_n^{0+2}$.  Thus, the even subalgebra of this Clifford algebra is indeed isomorphic to the complex numbers $\C$. 
\end{example}

\begin{example}
\label{ex:quaternions}
Next, take $\G_3$ with the standard vector basis $e_1,e_2,e_3$, then an arbitrary multivector $A$ is specified by
\begin{equation}
A= a + \alpha_1 e_1 + \alpha_2 e_2 + \alpha_3 e_3 + \beta_{12} \blade{B_{12}} + \beta_{13} \blade{B_{13}} + \beta_{23} \blade{B_{23}} + \mu e_1 \wedge e_2 \wedge e_3
\end{equation}
in general, and we have
\begin{subequations}
\begin{align}
\proj{0}{A}&=a\\
\proj{1}{A}&=\alpha_1 e_1 + \alpha_2 e_2 + \alpha_3 e_3\\
\proj{2}{A}&=\beta_{12} \blade{B_{12}} + \beta_{13} \blade{B_{13}} + \beta_{23} \blade{B_{23}}\\
\proj{3}{A}&= \mu e_1 \wedge e_2 \wedge e_3.
\end{align}
\end{subequations}
Then, let
\begin{equation}
\blade{B_{23}} = e_2 e_3, \quad \blade{B_{31}} = e_3 e_1, \quad \blade{B_{12}} = e_1 e_2,
\end{equation}
and note that we can write a even multivector as
\begin{equation}
q = a + \beta_{23}\blade{B_{23}} + \beta_{31} \blade{B_{31}} + \beta_{12} \blade{B_{12}}.
\end{equation}
Note as well that
\begin{equation}
\blade{B_{23}}^2 = \blade{B_{31}}^2 = \blade{B_{12}}^2 = -1,
\end{equation}
and
\begin{equation}
\blade{B_{23}}\blade{B_{31}}\blade{B_{12}} = +1.
\end{equation}
In this case, this even subalgebra is extremely close to being a copy of the quaternion algebra $\quat$. Indeed, one can arrive at a representation of the quaternions by taking
\begin{equation}
\boldsymbol{i} \leftrightarrow \blade{B_{23}}, \quad \boldsymbol{j} \leftrightarrow -\blade{B_{31}}=\blade{B_{13}}, \quad \boldsymbol{k} \leftrightarrow \blade{B_{12}},
\end{equation}
and noting that we then have $ijk=-1$ as well as $i^2=j^2=k^2=-1$. A more in depth explanation is provided in \cite{doran_geometric_2003}.

Once again, quaternions arise naturally as parabivectors since we can put
\begin{equation}
q= \alpha + u_1 \blade{B_{23}} - u_2 \blade{B_{13}} + u_3 \blade{B_{12}},
\end{equation}
and recover the necessary arithmetic seen in $\quat$.
\end{example}

\begin{remark}
If we take $\G_n$ for $n\geq 2$, then there are natural copies of $\C$ contained inside of $C\ell(V,Q)$. In particular, we have the isomorphism
    \[
        \C \cong \{\lambda + \beta \blade{B} ~\vert~ \lambda,\beta \in C\ell(V,Q)^0,~ \blade{B} \in C\ell(V,Q)^2,~ \blade{B}^2=-1. \},
    \]
   which shows that complex numbers arise as biparavectors. Given the standard basis $e_1,\dots,e_n$ we have copies of $\C$ for each of the ${ n \choose 2}$ unit bivectors $B_{jk}$ with $k=2,\dots,n$ and $j<k$. Note that $\blade{B_{jk}}\blade{B_{jk}}=-1$ and we have the representation of $\C$ since
    \[
        \zeta = x + y\blade{B},
    \]
    behaves as a complex number $z=x+iy$.
\end{remark}

\begin{example}
\label{ex:spacetime_algebra}
We shall not rule out the utility of geometric algebras with pseudo inner products. The classical example is the \emph{spacetime algebra} defined by taking $V=\R^4$ with a vector basis $\gamma_0,\gamma_1,\gamma_2,\gamma_3$ satisfying
\begin{subequations}
\begin{align}
\gamma_0 \cdot \gamma_0 &= -1\\
\gamma_0 \cdot \gamma_i &= 0  &i=1,2,3\\
\gamma_i \cdot \gamma_j &= \delta_{ij}, &i,j=1,2,3.
\end{align}
\end{subequations}
We refer to $\gamma_0$ as \emph{temporal} and $\gamma_i$ for $i=1,2,3$ as \emph{spatial}. For this basis, we can denote the matrix for this inner product $\eta =\operatorname{diag}(-+++)$ and define $Q$ from $\eta$. Then, we have for a vector $A = a_0 \gamma_0 +a_1 \gamma_1 + a_2 \gamma_2 + a_3 \gamma_3$ we have
\[
A\cdot A = -a_0^2 + \sum_{i=1}^3 a_i^2.
\]
\end{example}

\begin{remark}
For the cases with pseudo inner products with $p$ vectors satisfying $e_i^2 = -1$ for $i=1,\dots, p$ and $q$ vectors satisfying $e_j^2=1$ for $q=p+1,\dots,p+q$, we will denote the algebras by $\G_{p,q}$. The spacetime algebra is thus $\G_{1,3}$. 
\end{remark}

\subsubsection{Duality and pseudoscalars}
\label{subsection:duality_and_pseudoscalars}

For the remainder of this paper we will be mostly working with geometric algebras with a positive definite inner product $g$. Given access to an inner product we have a natural isomorphism between $V$ and $V^*$ by the Riesz representation.  Namely, given an arbitrary basis $e_i$ for $V$ there exists the dual basis $f_i$ for $V^*$ such that $f_i(e_j)=\delta_{ij}$.  This dual basis resides inside $V$ itself in the following manner. There is then a unique map $\sharp \colon V^* \to V$ with $f\mapsto f^\sharp$ such that
\[
f_i^\sharp \cdot e_j = \delta_{ij},
\]
where $\delta_{ij}$ is the Kronecker delta symbol. In terms of the geometric algebra, we put $e^i \coloneqq f_i^\sharp$ and can note that $e^i$ is simply a vector in the geometric algebra. For an arbitrary basis $e_1,\dots,e_n$ for $V$, the coefficients for the inner product $g$ are given by $g_{ij}=e_i\cdot e_j$ and we can put $e^i = g^{ij}e_j$ where $g^{ij}$ is the coefficients to matrix inverse of $g_{ij}$.  There is inverse isomorphism $\flat \colon V \to V^*$ given by $e \mapsto e^\flat$ satisfying
\[
e_i^\flat (e_j)= \delta_{ij}.
\]
Given these identifications, there is no need to distinguish between the vector space $V$ and its dual $V^*$ as it suffices to consider $V$ itself with reciprocal basis elements $e^i$ with the application of the scalar product.

A volume element can be defined by $\mu=e_1 \wedge e_2 \wedge \cdots \wedge e_n = \sqrt{|g|} I$ where $\sqrt{|g|}$ is the square root of the determinant of the matrix $g_{ij}$ and $I$ is the unit pseudoscalar. It follows that the unit pseudoscalar is given by $I=\frac{1}{\sqrt{|g|}} e_1 \wedge e_2 \wedge \cdots e_n$. We can define $\mu^{-1}$ such that $\mu^{-1}\mu = 1 = \mu \mu^{-1}$ and analogously $I^{-1}$.  One can equivalently put $e^j = (-1)^{j-1} e_1 \wedge e_2 \wedge \cdots \wedge \breve{e_j} \wedge \cdots \wedge e_n \mu^{-1}$ and note that this gives $\mu^{-1} = e^n \wedge \cdots \wedge e^1$.  Conveniently, the unit pseudoscalar satisfies the relation
\[
IA_k = (-1)^{k(n-1)} A_k I.
\]
Thus, $I$ commutes with the even subalgebra, and anticommutes with the odd subalgebra.  

Note that for a homogeneous $r$-rector $A_r$ we have that $A_r^\perp$ is an $n-r$-vector. Indeed, if we take an invertible $r$-blade $\blade{A_r}$, then we can find the \emph{$\blade{A_r}$-subspace dual} of a multivector $B$ by
\[
B \rfloor \blade{A_r}^{-1}.
\]
The notions of duality here give us geometrical insight. Taking an $s$-blade $\blade{B_s}$ we can note:
\begin{itemize}
    \item If $s>r$, the $\blade{A_r}$-subspace dual of $\blade{B_s}$ vanishes.
    \item If $s=r$, the $\blade{A_r}$-subspace dual of $\blade{B_s}$ is a scalar and is zero if $\blade{B_s}$ contains a vector orthogonal to $\blade{A_r}$.
    \item If $s<r$, the $\blade{A_r}$-subspace dual of $\blade{B_s}$ represents the orthogonal complement of the subspace corresponding to $\blade{B_s}$ in the subspace corresponding to $\blade{A_r}$.
\end{itemize}  
Since the pseudoscalar is a blade representing the entire vector space, this allows one to create dual elements within the entire vector space. Given a multivector $B$, we define the \emph{dual} of $B$ to be
\[
B^\perp \coloneqq B \rfloor I^{-1} = AI^{-1}.
\]
The dual allows one to exchange interior and exterior products in the following way.
\begin{equation}
\label{eq:wedge_to_dot}
 (A \wedge B)^\perp  = A\rfloor B^\perp
\end{equation}
\begin{equation}
\label{eq:dot_to_wedge}
    (A\rfloor B)^\perp = A \wedge B^\perp
\end{equation}
This shows the natural duality between the inner and exterior products and their interpretations as subspace operations. The duality extends further to provide an isomorphism between the spaces of $r$-vectors and $n-r$-vectors since for any $r$-vector $A_r$ we have $A_r^\perp$ is an $n-r$-vector. It is under this isomorphism one can realize that all pseudovectors are $n-1$-blades. 

\begin{example}
\label{ex:cross_product}
Consider $\spacealg$ with the standard orthonormal vector basis $e_1,\dots,e_n$ and Euclidean inner product.  Then, we can define the \emph{cross product} of two vectors $u$ and $v$ by
\[
u \cross v = (u\wedge v)I^{-1},
\]
where we use the bold notation for $\cross$ to distinguish between the bivector commutator product. The special fact of $\spacealg$ is that vectors and bivectors (pseudoscalars in 3-dimensions) are dual to one another. One can also note that the vector $w=u\times v$ is sometimes refered to as axial and in other cases the pseudovector $u\wedge v$ is referred to as axial. 

Referring back to Example \ref{ex:quaternions}, we can realize the cross product of vectors as the bivector commutator product
\[
u \cross v = (u^\perp)\times (v^\perp), 
\]
for which the similar product notation of $\times$ and $\cross$ now becomes transparent. The necessary relationships for the cross product are seen clearly on the products of the basis blades $\blade{B_{23}}, \blade{B_{31}}$, and $\blade{B_{12}}$. In particular, $e_1 = \blade{B_{23}}^\perp$, $e_2 = \blade{B_{31}}^\perp$, and $e_3 = \blade{B_{12}}^\perp$. This also shows that every bivector in a 3-dimensional space is a 2-blade.
\end{example}


\subsubsection{Blades and subspaces}

Each unit $r$-blade $\blade{A_r}$ ($\|\blade{A_r}\|=1$) corresponds to a $r$-dimensional subspace and can be identified with a point in $\Grassmannian{r}{n}$.
There are a handful of reasons to adopt the additional multiplication symbols $\rfloor$ and $\lfloor$. 
\begin{itemize}
    \item The products $\rfloor$ and $\lfloor$ allow us to avoid needing to pay special attention to the specific grade of each multivector in a product. The product $\cdot$ on $A_r$ and $B_s$ depends on $k$ and $s$ and as such given by either $\rfloor$ or $\lfloor$ but one must know $k$ and $s$ in order to define this product exactly. 
    \item We gain geometrical insight on the structure of $r$-blades in terms of their corresponding subspaces. Let $\blade{A_k}$ and $\blade{B_s}$ be nonzero blades with $r,s\geq 1$ then
    \begin{itemize}
        \item $\blade{A_r} \rfloor \blade{B_s} =0$ iff $\blade{A_r}$ contains a nonzero vector orthogonal to $\blade{B_s}$.
        \item If $r<s$ then if $\blade{A_r}\rfloor \blade{B_s} \neq 0$ then the result is a $s-r$-blade representing the orthogonal complement of $\blade{A_r}$ in $\blade{B_s}$.
        \item If $\blade{A_r}$ is a subspace of $\blade{B_s}$ then $\blade{A_r}\blade{B_s} = \blade{A_r}\rfloor \blade{B_s}$.
        \item If $\blade{A_r}$ and $\blade{B_s}$ are orthogonal, then $\blade{A_r}\blade{B_s} = \blade{A_r} \wedge \blade{B_s}$.
    \end{itemize}
\end{itemize}
See \cite{chisolm_geometric_2012} theorem 16.


We also have the identities
\begin{equation}
\label{eq:left_contraction_dot}
A_r \cdot B_s = A_r \rfloor B_s \qquad \textrm{if $k\leq s$}
\end{equation}
\begin{equation}
\label{eq:right_contraction_dot}
A_r \cdot B_s = A_r \lfloor B_s \qquad \textrm{if $k\geq s$}.
\end{equation}
For homogeneous $r$-vectors $A_r$ and $B_r$, the products above simplify to 
\begin{equation}
\label{dot_equivalent_contraction}
    A_r \lfloor B_r = A_r \rfloor B_r = A_r \cdot B_r.
\end{equation}
Using this notation, for a vector $\alpha$ we have
\begin{equation}
\alpha A_k = \alpha \rfloor A_k + \alpha \wedge A_k,
\end{equation}
so the $\cdot$ and $\lfloor$ notation coincide for left multiplication by vectors. If we are given two $k$-blades $A_k = \alpha_1 \wedge \cdots \wedge \alpha_k$ and $B_k = \beta_1 \wedge \cdots \wedge \beta_k$ we have 
\begin{equation}
\label{eq:dot_product}
A_k \cdot B_k^\dagger = \det(\alpha_i \cdot \beta_j )_{i,j=1}^k,
\end{equation}
which is equivalent to $A_k \rfloor B_k$ and $A_k \lfloor B_k$ through \ref{dot_equivalent_contraction} and this is extended to all $k$-vectors as is typically seen when constructing the inner product of $k$-vectors (see \cite{hestenes_clifford_1986}. If we are given two bivectors $B$ and $B'$, then we have another special multiplication
\begin{equation}
\label{eq:bivector_product}
B\times B' \coloneqq \proj{2}{BB'} = \frac{1}{2} (BB' - B'B),
\end{equation}
which is the grade preserving anti-symmetric portion of the product $BB'$ which we refer to as the \emph{bivector commutator product}.


\subsubsection{Projection and rejection}

Given the direct relationship between unit $r$-blades and $r$-dimensional subspaces we can also form a compact way of projecting multivectors into subspaces in a manner closely related to the subspace dual.  In general, given an multivector $B$ the \emph{projection} onto the subspace corresponding to the invertible $\blade{A_r}$ is
\begin{equation}
\label{eq:projection}
\projection_{B}(\blade{A_r}) \coloneqq B\rfloor \blade{A_r} \blade{A_r}^{-1} = (B\rfloor \blade{A_r})\rfloor \blade{A_r}^{-1}
\end{equation}
By definition, we have
\[
\projection_{\blade{A_r}}(B) \in \bigoplus_{j=0}^r \G_n^j = \G_n^{0+\cdots + r},
\]
since the subspace corresponding to $\blade{A_r}$ is $r$-dimensional and moreover the operation preserves grades since
\[
\projection_{\blade{A_r}}(B) \in G_n^j,
\]
shows the projection preserves grades.

Given vectors $u$ and $v$ we retrieve the familiar statement 
\[
\projection_u (v) = (v \cdot u) \frac{u}{\|u\|^2}.
\]

If instead we wish to project onto the subspace perpendicular to $\blade{A_r}$ we can use the \emph{rejection} operation which we define by
\begin{equation}
\label{eq:rejection}
\rejection_{\blade{A_r}}(B) \coloneqq B \wedge \blade{A_r} \blade{A_r}^{-1} = (B\wedge \blade{A_r})\lfloor \blade{A_r}^{-1}.
\end{equation}
Note that this operation is also grade preserving. In the case we have a vector $v$, we can note
\begin{equation}
\label{eq:projection+rejection_vector}
\projection_{\blade{A_r}}(v) + \rejection_{\blade{A_r}}(v) = v.
\end{equation}

To rehash the geometric notions of the interior and exterior products, note that
\begin{align}
    B \rfloor \blade{A_r} &= \projection_{\blade{A_r}} (B) \blade{A_r}\\
    B \wedge \blade{A_r} &= \rejection_{\blade{A_r}}(B) \blade{A_r}.
\end{align}

To see this in action, we let $v=v^1 e_1 + v^2 e_2 + v^3 e_3$ and let $\blade{B_{12}}=e_1 e_2$ and note
\begin{align*}
    \rejection_{\blade{B_{12}}}(v) &= [(v^1 e_1 + v^2 e_2 + v^3 e_3)\wedge (e_1 e_2)]\blade{B_{12}}^{-1}\\
    &= v^3 e_3 e_1 e_2 e^2 e^1 \\
    &= v^3 e_3.
\end{align*}
Both the notion of projection and rejection prove to be useful and behave nicely with the dual. Take, for example, vectors $u$ and $v$ and note
\begin{equation}
\label{eq:projection_rejection_vector}
\projection_{u^\perp}(v) = \rejection_u(v).
\end{equation}

Finally, the exterior product of orthogonal blades gives us a direct sum of subspaces. Let $\blade{A_r}$ and $\blade{B_s}$ and be orthogonal so that $\blade{A_r}\wedge \blade{B_s}=\blade{A_r}\blade{B_s}$, then we can note
\begin{equation}
    \projection_{\blade{A_r}\wedge \blade{B_s}} (v) = \projection_{\blade{A_r}}(v) + \projection_{\blade{B_s}}(v).
\end{equation}


\subsection{Multivector fields}

We want to generalize the setting of geometric algebra to include a smooth structure. One can take the work above for $\mathcal{G}_n$ and consider a $C^{\infty}$-module structure as opposed to the $\R$-algebra structure in the proceeding section. For brevity, we put $\mathcal{G}_n(\R^n)$ for the $C^\infty$-module and $\G_n$ for the $\R$-algebra. The multivectors themselves can be realized as constant multivector fields so that $\G_n \subset \G_n(\R^n)$. This smooth setting simply makes the coefficients of the global basis blades given by $C^\infty$ functions as opposed to $\R$ scalars.  In this case, we refer to a generic element in the $C^{\infty}$-module $\mathcal{G}_n$ as a \emph{multivector field}. We take $\Omega \subset \R^n$ as a connected region in $\R^n$ for the entirety of this paper and we put
\[
\G_n(\Omega) \coloneqq \{ f \colon \Omega \to \G_n ~\vert~ \textrm{$f$ is $C^\infty$-smooth}\},
\]
where smoothness is meant in terms of the $C^\infty$-module structure.

Perhaps the $C^\infty$-module structure obfuscates the point slightly.  Instead, one should think of the fields in $\G_n(\Omega)$ as multivector valued functions on $\Omega \subset \R^n$.  Taking this identification allows for an extended toolbox at our disposal.  In particular, points in $\Omega$ are uniquely identified with constant vector fields in $\G_n^1$ and one can consider endomorphisms living in $\G_n$ (acting on $\G_n^1$) as acting on the input of fields in $\G_n(\Omega)$ as well.  Thus, there is not only an algebraic structure on the fields themselves, but on the point in which the field is evaluated.  This is perhaps the key insight on why authors developed the so-called vector manifolds widely used in the geometric algebra landscape.

\begin{example}
    Consider a multivector field $f$ valued in $\G_n(\R^n)$.  With $x\in \R^n$ being identified with the vector in $\G_n^1$, we output a multivector $f(x) \in \G_n$ at each point $x$.  One may be interested in the restriction of $f$ to a vector subspace of $\R^n$ which amounts to using projection on the input.  For example, perhaps we wish to know how $f$ behaves on the subspace corresponding to some $r$-blade $\blade{A_r}$.  As such, it suffices to then study $f(\projection_{\blade{A_r}}(x))$.  
\end{example}

We refer to smooth fields valued in $\G_n^+$ as \emph{spinor fields} and put $\G_n^+(\Omega)$ to refer to the $C^\infty$-module counterpart. These fields will be shown to carry a Banach algebra structure. 


\subsubsection{Directional derivative and gradient}

Note that $\R^n$ has global coordinates and thus we can choose a global constant vector field basis $e_1,\dots,e_n$ and we generate $\G_n$ from this basis. Note that we will adopt the Einstein summation convention when needed. With respect to these fields, we have the \emph{directional derivative} $\nabla_\omega$ with $\omega = \omega^i e_i$. The \emph{gradient} (or \emph{Dirac operator}) is defined as $\grad = \sum_{i} e^i \nabla_{e_i}$ and it acts a grade-1 element in the algebra.   Note then that $\omega \cdot \grad = \nabla_\omega$ defines the directional derivative via the gradient. The directional derivative is also grade preserving in that for a multivector $A$
\begin{equation}
\nabla_\omega \proj{r}{A} = \proj{r}{\nabla_\omega A}.
\end{equation}

This structure defined above is typically referred to as \emph{geometric calculus}.  The setting for geometric calculus extends the setting of differential forms and reduces some of the complexity with tensor computations.  Since $\grad$ is a grade-1 object, it acts on a homogeneous $r$-vector $A_r$ by
\begin{equation}
\grad A_r = \proj{r-1}{\grad A_r} + \proj{r+1}{\grad A_r} \coloneqq \grad \rfloor A_r + \grad \wedge A_r.
\end{equation}
Thus, the gradient splits into two operators, 
\begin{align}
\grad \rfloor &\colon \G_n^r(\Omega) \to \G_n^{r-1}(\Omega), \\
\grad \wedge &\colon \G_n^r(\Omega) \to \G_n^{r+1}(\Omega),
\end{align}
which satisfy the properties
\begin{align}
\label{eq:differential_properties}
(\grad \wedge)^2=0,\\
(\grad \rfloor)^2 = 0,
\end{align}
when acting on a homogeneous $r$-vector. Since \ref{eq:differential_properties} holds, the gradient operator gives rise to the grade preserving \emph{Laplace-Beltrami operator}
\[
\Delta = \grad^2 = \grad \rfloor \circ \grad \wedge + \grad \wedge \circ \grad \rfloor,
\]
which is manifestly coordinate invariant by definition.  It also motivates the use of the physicist notation $\grad^2=\Delta$, but we do not adopt this here.  We refer to multivector fields $f$ in the kernel of the Laplace-Beltrami operator \emph{harmonic}.

\subsubsection{Monogenic fields}

Geometric calculus includes another definition for multivectors that is a big motivation for those who study Clifford analysis. 
\begin{definition}
 Let $f \in \G_n(\Omega)$. Then we say that $f$ is \emph{monogenic} if $f \in \ker(\grad)$.
\end{definition}

Monogenic fields are of utmost importance as they have many beautiful properties. One should find them as a suitable generalization of the notion of complex holomorphicity. For example, in regions of Euclidean spaces, a monogenic field $f$ can be completely determined by its Dirichlet boundary values through a generalized Cauchy integral formula. For any even monogenic field, the each of the graded components of $f$ are harmonic.  

We put 
\[
\monogenics(\Omega) \coloneqq \{f \in \G_n(\Omega) ~\vert~ \grad f =0\}
\]
to refer to elements of this set as \emph{monogenic fields} on $\Omega$. As a subset, we also have the \emph{monogenic spinors} $\monogenics^+(\Omega)$, which are simply the even monogenic fields and the \emph{monogenic parabivectors} $\monogenics^{0+2}(\Omega)$. Though these spaces do not form algebras in their own right, they do indeed form a vector space as sums of monogenic functions are monogenic due to the linearity of the gradient.  Moreover, the monogenic spinors are invariant under multiplication from the Clifford group $\Gamma^+$.

\begin{lemma}
\label{lem:clifford_invariant}
Let $s\in \spingroup$ then $\grad \circ s = s \circ \grad$.  In particular, the space of monogenic spinors $\monogenics^+(\Omega)$ is $\spingroup$ invariant.
\end{lemma}
This lemma is classical in the theory of the Dirac operator, Clifford analysis, and harmonic analysis so we omit a proof.  One can see \cite{janssens_special_nodate}, for example.

\subsection{Differential forms}
\label{subsec:diff_forms}

It has become clear that geometric algebra and geometric calculus combine into a single toolbox of multivector field analysis that is useful for vector space algebra and the calculus of $\R^n$. Conveniently, the language of differential forms rests neatly inside this toolbox as well. As such, we will also develop a means of integrating multivector fields. In this subsection we connect the two together into a single framework and note the additional benefits geometric algebra and calculus provide over forms. In order to do so, we appeal to the language of differential forms and build a relationship between multivector fields and forms through measures. Forms have their appeal in global understanding via their properties through integration (e.g., Stokes' and Green's theorems) and their utility extends to boundary value problems \cite{schwarz_hodge_1995}.  

Given that there exists a global coordinate system $x^i$ on $\R^n$, we can place this set of coordinates on any region $\Omega$. Then, we form the basis of tangent vector fields $\partial_i = \frac{\partial}{\partial x^i}$ with the reciprocal 1-forms $dx^i$ that are each global sections of $T^*\Omega$ and are the exterior derivatives (or gradients) of the coordinate functions.  Typically, 1-forms are viewed as linear functionals on tangent vector fields and in these coordinates we have $dx^i  (\partial_j) = \delta^i_j$.  The benefit of this definition is that the 1-forms $dx^i$ carry a natural measure and we can form product measures via the exterior product $\wedge$.  For example, for a 2-dimensional surface $\Sigma$ we have the \emph{directed measure} $d\Sigma = e_i \wedge e_j dx^i dx^j$ and we can note that $(e^i \wedge e^j)\cdot d\Sigma^\dagger = dx^idx^j - dx^j dx^i$ is completely antisymmetric and provides us a surface measure we can integrate; this is a differential 2-form.

In an $n$-dimensional space with a position dependent inner product $g$, we have the $n$-dimensional volume directed measure $d\Omega = \sqrt{|g|} dx^1\dots dx^n$. If we then define $dX_n = e^n \wedge \cdots \wedge e^1 dx^1 \dots dx^n$ we then find that
\[
d\Omega = I^\dagger \cdot dX_n.
\]
Here $I$ is the pseudoscalar field defined on $\Omega$ with respect to $g$. Similarly, for $k<n$, we can define the $k$-dimensional volume measure as 
\[
dX_k = \frac{1}{k!}(e^{i_k}\wedge \cdots \wedge e^{i_1}) dx^{i_1} \cdots dx^{i_k}.
\]
We can now write a $k$-form $\alpha_k$ as $\alpha_k = A_k \cdot dX_k$. In this sense, a differential form is made up of two essential components namely the multivector field and the $k$-dimensional volume directed measure. For example, if we wish to write a 2-form $\alpha_2$ we take $dX_2 = \frac{1}{2!} e^j \wedge e^i dx^i dx^j$ and $A_2 = a_{ij} e_i \wedge e_j$ to yield
\[
\alpha_2 = A_2 \cdot dX_2 = \frac{a_{ij}}{2!} (e_i \wedge e_j) \cdot (e^j \wedge e^i) dx^i dx^j = \frac{a_{ij}}{2!} (dx^i dx^j - dx^j dx^i)
\]
Thus, we arrive at an isomorphism between $k$-forms and $k$-vectors as a contraction with the $k$-dimensional volume directed measure $dX_k$ since
\[
\alpha_k = A_k \cdot dX_k.
\]
Hence, we can see now how a differential form simply appends the measure attached to the underlying space. We can also see how this generalizes the musical isomorphisms $\sharp$ and $\flat$ by taking a vector field $a$ and noting
\begin{equation}
\label{eq:line_element}
a \cdot dX_1 = a^i e_i \cdot e^j dx^j = a^i dx^i,
\end{equation}
corresponds to the usual $\flat$ map on vector fields.

\subsubsection{Exterior algebra and calculus}
The exterior algebra of differential forms comes with an addition $+$ and exterior multiplication $\wedge$.  We note that the sum of two $k$-forms $\alpha_k$ and $\beta_k$ that $\alpha_k+\beta_k$ is also a $k$-form which we can see by letting $\alpha_k = A_k \cdot dX_k$ and $\beta_k = B_k \cdot dX_k$ and putting
\[
\alpha_k + \beta_k = (A_k \cdot dX_k)+(B_k \cdot dX_k) = (A_k + B_k) \cdot dX_k,
\]
due to the linearity of $\cdot$.  If instead had an $s$ form $\beta_s$ then we have the exterior product
\[
\alpha_k \wedge \beta_s = (A_k \wedge B_k) \cdot dX_{k+s},
\]
where $dX_{k+s}=0$ if $k+s>n$.  

With differential forms one also has the exterior derivative $d$ giving rise to the calculus of forms.  Given we can write a differential $k$-form as $\alpha_k = A_k \wedge dX_k$,  In particular, we have
\[
d \alpha_k = (\grad \wedge A_k) \cdot dX_{k+1},
\]
which realizes the exterior derivative as the grade raising component of the gradient $\grad$. Of course, for scalar fields, this returns the gradient as desired. 

Here, $\grad \wedge$ can be identified with the exterior derivative $d$ and $\grad \rfloor$ can be identified with the codifferential $\delta$ on differential forms up to a sign \cite{schindler_geometric_2020}. This of course means the standard properties that apply to $d$ and $\delta$ apply to $\grad \wedge$ and $\grad \rfloor$.


\subsubsection{Integration on submanifolds}

Given a $k$-dimensional submanifold of $K \subset \Omega$ with a $k$-form $\alpha_k$ defined on $K$, we can integrate the $k$-form. Using the multivector equivalents leads us to the $k$-dimensional directed measure $dK$ for the submanifold $K$.  Given $K$ is a submanifold of $\Omega$, for any $x \in K$ we have tangent space $T_x K$ which corresponds to a tangent $k$-blade $I_K(x)$.  We put $I_K$ as the smooth $k$-blade field everywhere tangent to $K$. Then we have the directed volume measure on $K$ given by
\[
dK = I_K^\dagger \cdot dX_k.
\]
For a tangent $k$-vector field $A_k$ on $K$, we must have for any $x \in K$ that $f = \operatorname{P}_{I_K} \circ f$ so that these fields lie purely tangent to $K$. In particular, we can always put $A_k = A I_k^\dagger$ for a scalar field $A$. These fields can contract with the directed measure $dX_k$ to create a $k$-form on $K$ by $\alpha_k = A_k \cdot dX_k = A dK$ which can be integrated as
\[
\int_K \alpha = \int_K A dK.
\]
Hence, on $\Omega$ itself, we can decompose top grade forms by taking
\[
\alpha_n = A_n \cdot dX_n = A I^\dagger \cdot dX_n
\]
for a scalar field $A$ satisfying $A_n = AI^\dagger$. Then this form can be integrated by
\[
\int_\Omega \alpha_n = \int_\Omega A d\Omega.
\]

There is also the normal space $N_x K$ that is everywhere orthogonal (with respect to $g$ on $\Omega$) to $T_x K$.  In particular, we have the normal $(n-k)$-blade field $\nu = I_K^\dagger I$. Note that for a unit $k$-blade $I_K$ we have $I_K^{-1}=I_K^\dagger$ and we see $I_K \nu = I$. Since $K$ is a submanifold of $\Omega$ we have the inclusion $\iota \colon K \to \Omega$ and the induced pullback on forms $\iota^*$ which is equivalent to the tangent projection operator $\tangent_K$ seen in \cite{schwarz_hodge_1995}. Given a $p$-form $\alpha_p$ defined on $\Omega$, we have that $\tangent_K \alpha_p = \alpha_p \circ \operatorname{P}_{I_K}$. Specifically, $\alpha_p = A_p \cdot dX_p$ we have \todo[inline]{I should probably work through this to show it's true}
\[
\tangent_K \alpha_p  = A_p \cdot (dX_p \circ \operatorname{P}_{I_K}) = \operatorname{P}_{I_K}(A_p) \cdot dX_p = \rejection_{\nu}(A_p)\cdot dX_p.
\]
The normal projection $\normal_K$ is then $\normal_K \alpha_p = \alpha_p - \tangent_K \alpha_p$ and moreover
\[
\normal_K \alpha_p = \rejection_{I_K}(A_p)\cdot dX_p = \operatorname{P}_{\nu}(A_p)\cdot dX_p.
\]
\todo[inline]{Not sure this is true. but \url{https://en.wikipedia.org/wiki/Geometric_algebra} talks about projection and rejection.}

This is pertinent when we take the submanifold $\Sigma = \partial \Omega$. There, $I_\Sigma$ yields the directed measure
\[
d\Sigma \coloneqq I_\Sigma^\dagger \cdot dX_{n-1}.
\]
The normal space is 1-dimensional and $\nu$ is the unit normal vector to the boundary. The pullback coincides with projection into the tangent space given by $I_\Sigma$.  Then, for 1-forms $\alpha = \cdot dX_1$ it is apparent that $\tangent_\Sigma \alpha = \projection{I_\Sigma}{a} \cdot dX_1$ and $\normal_\Sigma \alpha = \rejection_{I_\Sigma}(a) \cdot dX_1$ by Equations \ref{eq:projection+rejection_vector} and \ref{eq:projection_rejection_vector}. One can then find the flux of a vector field through $\Sigma$ arises as an $(n-1)$-form $\projection{\nu}{a} I^{-1} \cdot dX_{n-1}$. Once again we see that the flux is determined both by the vector field $a$ and the local geometry of $\Sigma$ captured by $d\Sigma$ in the following way. Note that  $\nu^{-1}=\nu$ since $\|\nu\|=1$ everywhere on $\Sigma$ and so $\projection{\nu}{a} I^{-1}= a \cdot \nu \nu I^{-1} = a \cdot \nu I_\Sigma^\dagger$ which gives us the corresponding form $a \cdot \nu d\Sigma$ and the total flux of $a$ through $\Sigma$ is then
\[
\int_\Sigma (\projection{\nu}{a} I^{-1}) \cdot dX_{n-1} = \int_\Sigma a \cdot \nu d\Sigma.
\]

\subsubsection{$k$-form inner product}
\todo[inline]{Show that this relates back to the spinor norm.}
For smooth $k$-forms $\alpha_k = A_k \cdot dX_k$ and $\beta_k = B_k \cdot dX_k$, we have an inner product 
\[
\langle \alpha_k, \beta_k \rangle = \int_\Omega \alpha_k \wedge \star \beta_k 
\]
where $\star$ is the Hodge star. The Hodge star on $k$-forms inputs a $k$-form and outputs a a specific dual $(n-k)$-form so that we always have $\alpha_k \wedge \star \beta_k  = (A_k\cdot B_k^\dagger)d\Omega$ as we note Equation \ref{eq:dot_product}. Thus, we can realize how $\star$ acts on multivector representative. We let $\star \beta_k = B_k^\star \cdot dX_{n-k}$ by $B_k^\star = (I^{-1} B_k)^\dagger$.  Indeed, we have
\begin{align*}
    \alpha_k \wedge \star \beta_k &= (A_k \wedge B_k^\star) \cdot dX_n\\
    &= A_k \cdot B_k^\dagger d\Omega.
\end{align*}

\subsubsection{Stokes' and Green's theorem}

For regions $\Omega$ with boundary $\Sigma$, we have a compact form of Stokes' theorem
\[
\int_\Omega d \alpha_{n-1} = \int_\Sigma \tangent_\Sigma \alpha_{n-1},
\]
for sufficiently smooth $(n-1)$-forms $\alpha_{n-1}$. Taking the multivector equivalent $\alpha_{n-1} = A_{n-1} \cdot dX_{n-1}$ we retrieve \emph{Stokes' theorem} as
\[
\int_\Omega (\grad \wedge A_{n-1}) \cdot dX_n = \int_\Sigma \projection{I_\Sigma}{A_{n-1}} \cdot dX_{n-1}  = \int_\Sigma \rejection_{\nu}(A_{n-1}) \cdot dX_{n-1}
\]
Let $v = A_{n-1} I$ denote the vector field dual to the pseudovector $A_{n-1}$, then we have the more recognizable \textcolor{red}{(Helmholtzian?)} form of Stokes' theorem
\[
\int_\Omega \grad \cdot v d\Omega = \int_\Sigma v \cdot \nu d\Sigma.
\]
\todo[inline]{Show this work?} This provides a compact relationship for those who choose to work with vector fields and those who choose to work with forms. \textcolor{red}{Pause and reflect here about what Green's theorem and Stokes' theorem are really saying. Use DUAL notation!}

Finally,




\todo[inline]{More on integration and do Ohms law as an example of some of this stuff. Do all of hodge decomposition and stuff?}