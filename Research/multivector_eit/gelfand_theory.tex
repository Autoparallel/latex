\subsection{Topology from monogenics}

We seek to determine that the space $\characters(\Omega)$ is homeomorphic to $\Omega$.  Thinking of the Calder\'on problem, we may only have access to functions defined on $\Omega$ and not the whole of $\Omega$ itself.  If one can recover the spinor characters $\characters(\Omega)$, we can utilize the following result.

\begin{theorem}
For any $\mu \in \characters(\Omega)$, there is a point $x^\mu \in \Omega$ such that $\mu(f) = f(x_\mu)$ for any $f\in \monogenics(\Omega)$ a monogenic spinor field. Given the weak-$\ast$ topology on $\characters(\Omega)$, the map
\[
\gamma \colon \characters(\Omega) \to \Omega, \quad \mu \mapsto x^\mu
\]
is a homeomorphism. The Gelfand transform 
\[
\widehat{~} \colon \monogenics(\Omega) \to C(\characters(\Omega); \G_n), \quad \widehat{f}(\mu) \coloneqq \mu(f), \quad \mu \in \characters(\Omega),
\]
is an isometry onto its image, so that $\characters(\Omega)$ is isomorphic to $\widehat{\monogenics(\Omega)}$ as algebras.
\end{theorem}

We prove this theorem in two main parts and discuss the result in this section. First, we can realize a power series representation for elements in a ball $\ball$ and denote this sit as $\monogenics(\ball)$. This power series is constructed using specific $B$-planar monogenic fields. Finally, we constructively show a correspondence between $\mu \in \characters(\ball)$ with $x^\mu \in \ball$. \textcolor{red}{Then we can use these to cover $\Omega$ or something?}

\subsubsection{Power series}

\todo[inline]{This really is a honest to god Taylor series so I should call it that.}

One beautiful result in Clifford analysis is the celebrated generalization of the Cauchy integral formula for $\C$-holomorphic functions. Details of the Cauchy integral formula and Hilbert transform for multivector fields can be found in \cite{brackx_hilbert_2008}. We have the fundamental solution to $\grad$ is a vector field given by
\[
E(x) = \frac{1}{a_m} \frac{x}{\|x\|^m},
\]
for $x\in \R^n$. That is to say that $\grad E(x) = \delta(x)$. For any region $\Omega \subset \R^n$ with boundary $\Sigma$, we define the \emph{Cauchy kernel} for $x\in \R^n$ and $y \in \Sigma$ using the fundamental solution $E$ as
\[
C(y, x) = -\frac{1}{a_n} \nu(x_0) E(x-y),
\]
where $a_n$ is the surface area of the $n$-ball and $\nu(x_0)$ is the outward normal at $x_0$. The \emph{Cauchy integral} for $\phi \in L_2(\Sigma)$ is then
\[
\cauchy[\phi](x) = \frac{1}{a_n} \int_{\Sigma} \frac{y-x}{\|x-y\|^n} \nu(y) \phi(y) d\Sigma(y).
\]
The Cauchy integral is indeed a monogenic function and note that for a scalar $\phi$ we have $\cauchy[\phi] \in \monogenics(\Omega)$ since it must be a parabivector as well.

Fix a basis $e_1,\dots,e_n$ in $\R^n$ and we can define the functions $z_j^i = x^j - x^i e^i e_j$. Recall that for an orthonormal basis the reciprocal basis elements $e^i=e_i$ satisfy $e^i \cdot e_j = 1$. \textcolor{red}{Ryan uses $e_i^{-1}$ actually. Are the reciprocal basis elements the inverses? Yes see \url{https://math.stackexchange.com/questions/811248/wedge-product-between-nonorthogonal-basis-and-its-reciprocal-basis-in-geometric}} To further condense notation, we let $B_{ij}=e_i e_j$ be the 2-blade acting as the pseudoscalar for the $e_i e_j$-plane and likewise put $B_j^i = e^ie_j$ and $B^{ij}=e^i e^j$ as necessary. In the same vein, the functions $z_j^i$ are very analogous to $z$ in $\C$ but rather in the $B_j^i$ plane.  We then note
\[
z_j^i = x^j - x^i B_j^i = e_j\projection{B_j^i}{x}.
\]
One can quickly confirm that the $z_j^i$ are monogenic and are indeed $B_j^i$-planar by construction. These functions find their use in a power series representation for monogenic fields $f$.
\begin{itemize}
    \item Consider the function $z_{\sigma(j)}^1(x)=x^{\sigma(j)} - x^1 B_{\sigma(j)}^i$ for $\sigma \in \{2,\dots,n\}$ a permutation.  
    \item Let $f \in \monogenics^+(\Omega)$.  Then by Theorem 4 in \cite{ryan_clifford_2004}, we can center a ball of radius $R$ at $w$ to get the monogenic polynomials
    \[
        P_{j_2 \dots j_n}(x) = \frac{1}{j!} \sum_{\textrm{permutations}}z_{\sigma(1)}^1(x-w) \cdots z_{\sigma(j)}^1(x-w).
    \]
    Each polynomial in the collection
    \[
    \mathcal{P}(\Omega) = \{P_{j_2 \cdots j_n} ~\vert~ j_2+\cdots+j_n = j, ~0\leq j < \infty\}
    \]
    is monogenic and linearly independent.
    These polynomials generate $f$ as a power (Taylor) series as
    \[
        f(x) = \sum_{j=0}^\infty \left(\sum_{{j_2 \cdots j_n}_{j_2 + \cdots j_n = j}} P_{j_2 \cdots j_n} (x-w) a_{j_2 \cdots j_n}(w) \right),
    \]
    where the coefficients are found using the Cauchy integral
    \[
        a_{j_2 \cdots j_n} = \frac{1}{a_n} \int_{\partial B(w,R)} \frac{\partial^j G(x-w)}{\partial x_2^{j_2} \cdots \partial x_n^{j_n}} \nu(x) f(x) d\Sigma(x).
    \]
    Each coefficient $a_{j_2 \cdots j_n} \in \G_n^+$. \textcolor{red}{Yes but these are coming in as a right module multiplication. So this should be noted and checked}
    \item This series converges uniformly to $f$ for points $x\in \ball$.
\end{itemize}


We have now found that all monogenic fields are generated as power series of homogeneous polynomials in the variables $z_j^i$. Thus, we have a direct route between the algebras $\algebra{B_j^i}(\ball)$ and the monogenic spinor fields $\monogenics(\ball)$.  In each algebra $\algebra{B_j^i}(\ball)$ the $z_j^i$ act much like a realization of $z\in \C$.  We will find that the action of the spin characters on $z_j^i$ can be understood and extended through the power series to all monogenic spinors. The power series representation seen here is one of the strong reasons to utilize geometric calculus and study the results of Clifford analysis. 



\subsubsection{Correspondence}

The functions $z_j^i$ play a crucial role in the above power series representation but they also play a key part in determining the behavior of the spin characters $\mu \in \characters$.  If we are able to deduce the action $\mu(z_j^i)$, then we can extend this to any monogenic $f$ via the power series representation. Note that $\mu(1)=1$ since it is an algebra homomorphish and so for any $2$-blade $B$ and $\mu \in \characters(\ball)$ that the image of the axial algebras $\mathbb{A}_B=\mu(\algebra{B}(\ball))$ are all commutative subalgebras of $\G_n^+$.  In particular, for a constant $\alpha+\beta B \in \algebra{B}(\ball)$, $\mu(\alpha+\beta B)=\alpha+\beta B$ by definition and so we retrieve $\mathbb{A}_B$ must be generated by linear combinations of the scalar $1$ and the bivector $B$.  Thus, $\mathbb{A}_B$ is an isomorphic copy of $\G_2^+ \cong \C$ as the even subalgebra of the $B$-plane.

Working in terms of an arbitrary basis and applying $\mu$ yields
\[
\mu(z_j^i) = \alpha_j^i + \beta_j^i B_j^i,
\]
for some constants $\alpha_j^i$ and $\alpha_j^i$.  The $z_j^i$ are not independent from one another.  In fact, we have two key relationships in that
\begin{equation}
\label{eq:z_reciprocal_relationship}
z_j^i B_i^j  = -z_i^j.
\end{equation}
Similarly, we have
\begin{equation}
\label{eq:z_relationship}
z_j^i = z_j^k + z_k^i B_j^k.
\end{equation}

Thus, we can take $\mu$ of Equations \ref{eq:z_reciprocal_relationship} and \ref{eq:z_relationship} and determine a relationship on the constants $\alpha_j^i$ and $\beta_j^i$. First, using Equation \ref{eq:z_reciprocal_relationship}
\[
\mu(z_j^i B_i^j) = \mu(z_j^i) B_i^j = -\mu(z_i^j)
\]
yields
\[
(\alpha_j^i + \beta_j^iB_j^i)B_i^j = \beta_j^i + \alpha_j^i B_i^j = - \alpha_i^j - \beta_i^j B_i^j 
\]
and so $\alpha_i^j = -\beta_j^i$ for all $i \neq j$. Next, using Equation \ref{eq:z_relationship}
\[
\mu(z_j^i) = \mu(z_j^k + z_k^i B_j^k) = \mu(z_j^k)+\mu(z_k^i)B_j^k
\]
and so
\[
a_j^i + b_j^i B_j^i = \alpha_j^k + \beta_j^kB_j^k + (\alpha_k^i + \beta_k^i B_k^i)B_j^k = \alpha_j^k + \beta_k^i B_j^i + (\alpha_k^i + \beta_j^k)B_j^k
\]
yields the relationships $\alpha_j^i = \alpha_j^k$, $\beta_j^i = \beta_k^i$, and $\alpha_k^i=-\beta_j^k$. 

Briefly, picture $\alpha_j^i$ and $\beta_j^i$ as components of the $n \times n$ matrices $\alpha$ and $\beta$.  We can index rows by the superscript and columns by the subscript and see that $\alpha$ and $\beta$ both have zero diagonal (since we do not have functions $z_i^i$). The relationship $\alpha_i^j = -\beta_j^i$ for $i\neq j$ then shows that $\alpha = -\beta^\top$.  Then we have $\alpha_j^i = \alpha_j^k$ for $i\neq j \neq k$ shows that $\alpha$ is constant along rows and hence $\beta$ is constant along columns (which shows $\alpha = -\beta^\top$ is consistent with the additional relationship $\beta_j^i = \beta_k^i$). The final relationship $\alpha_k^i = -\beta_j^k$ is consistent as well. The matrices $\alpha$ and $\beta$ are thus uniquely determined by $n$ numbers.  Moreover, treating $\mu(z_j^i)=z_j^i(x_\mu)$ for some $x_\mu \in \R^n$ satisfies the relationships granted above. Thus, we simply find the $x_\mu$ such that we retrieve the desired components for $\alpha$ and $\beta$.  

Using the power series representation for a monogenic spinor $f$ we can extend $\mu$ to act on $\monogenics^+(\ball)$ by the multiplicative and $\G_n^+$ linear nature of $\mu$ since we also note again that the coefficients $a_{j_2 \cdots j_n} \in \G_n^+$. Using the correspondence, we then realize $\mu(f)=f(x_\mu)$ for the corresponding $x_\mu \in \R^n$. To see that this point $x_\mu \in \ball$, we take a field defined on $\G_n(\R^n)$ and monogenic in $\G_n(\Omega)$. For any $x_0 \in \R^n \setminus \ball$ we have the field $E(x_0 - x)$ is monogenic for $x\in \ball$. Then for a spin character $\mu$ we have a sequence of functions $E_n \to E(x_\mu - x)$ such that $\mu(E_n)$ is bounded for all $n$ but diverges in the limit.   \textcolor{red}{Can we actually just argue that we can determine all $x_0$ such that $E(x_0-x)$ is monogenic on $\Omega$ therefore we can determine $\R^n \setminus \Omega$?}

\todo[inline]{Make thie more explicit and do an example or something in 3D. Show that $x\mu$ is in the ball.Finish this and note that this proves the theorem.}

\todo[inline]{I'm not even sure we need to do this with $\Omega=\ball$ other than for part of the proof with the power series. But if $\Omega$ is compact, it fits inside a ball of some radius $r$ and so we should still be able to represent all the monogenics on $\Omega$ with this. The trick is we have a function that is monogenic except at a point. }

\todo[inline]{If work with weak monogenic functions then we can probably use mollifiers and stitch together monogenics on $\Omega$ from various open balls in $\Omega$ that are monogenic except at some set of measure zero. Then this should allow us to probably speak more accurately about the delta function and $E$ and probably suup this all up to determine the homeomorphism type of any embedded manifold.}

\subsection{Discussion}

Perhaps the above result should not be so surprising.  One could venture to the Atiyah-Singer index theorem which relates the topological information of a manifold with the elliptic operators.  In particular, the Dirac operator (the gradient $\grad$) is indeed elliptic. Indeed, this seemingly sparks the motivation for the Calder\'on problem.  There, the elliptic operator is the Laplace-Beltrami operator $\Delta$.  However, this is an inverse problem in which we do not know the space (or the metric) and are asked to, in a sense, determine the Laplace-Beltrami operator from information on the boundary of a Riemannian manifold.  With this boundary data, one would hopefully be able to decipher $\Delta$ and as such, construct a copy of the desired Riemannian manifold.

