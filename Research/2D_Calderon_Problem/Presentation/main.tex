\documentclass[aspectratio=169]{beamer}

\usepackage[utf8]{inputenx} % For æ, ø, å
\usepackage{csquotes}       % Quotation marks
\usepackage{microtype}      % Improved typography
\usepackage{amssymb}        % Mathematical symbols
\usepackage{mathtools}      % Mathematical symbols
\usepackage[absolute, overlay]{textpos} % Arbitrary placement
\setlength{\TPHorizModule}{\paperwidth} % Textpos units
\setlength{\TPVertModule}{\paperheight} % Textpos units
\usepackage{tikz}
\usetikzlibrary{overlay-beamer-styles}  % Overlay effects for TikZ

\AtBeginSection{\frame{\sectionpage}}
\AtBeginSubsection{\frame{\subsectionpage}}

\usepackage{hyperref}
\usepackage{svg}
%\usefonttheme{serif}

\usepackage{color, soul, xcolor} % Colored text and highlighting, respectively
\usepackage{tikz-cd} % For commutative diagrams
\usepackage{tikz-3dplot}
\usetikzlibrary{angles}
\RequirePackage{pgfplots}
\usepackage{mathtools}
\usepackage{answers}
\usepackage{setspace}
\usepackage{graphicx}
\usepackage{enumerate}
\usepackage{multicol}
\usepackage{mathrsfs}
\usepackage{amsmath,amsthm,amssymb}
\usepackage{marvosym,wasysym} %fucking smileys
\usepackage{float}
\usepackage{morefloats}
\usepackage{pgf,tikz}
\pgfplotsset{compat=1.15}
\usepackage{mathrsfs}
\usetikzlibrary{arrows}
\usepackage{subcaption}
\usepackage[most]{tcolorbox}
\tcbuselibrary{theorems}
\usepackage{fancyvrb}
\usepackage{longtable,booktabs}
\usepackage{stackrel}
\usepackage{animate}
\usepackage[percent]{overpic}
\definecolor{lighter_csu_green}{RGB}{60,133,77}
\newcommand\boldgreen[1]{\textcolor{lighter_csu_green}{\emph{\textbf{#1}}}}
\usepackage{MnSymbol}

%Commands
\newcommand{\R}{\mathbb{R}}
\newcommand{\opens}{\mathcal{O}}

%border matrix
\makeatletter
\newif\if@borderstar
\def\bordermatrix{\@ifnextchar*{%
\@borderstartrue\@bordermatrix@i}{\@borderstarfalse\@bordermatrix@i*}%
}
\def\@bordermatrix@i*{\@ifnextchar[{\@bordermatrix@ii}{\@bordermatrix@ii[()]}}
\def\@bordermatrix@ii[#1]#2{%
\begingroup
\m@th\@tempdima8.75\p@\setbox\z@\vbox{%
\def\cr{\crcr\noalign{\kern 2\p@\global\let\cr\endline }}%
\ialign {$##$\hfil\kern 2\p@\kern\@tempdima & \thinspace %
\hfil $##$\hfil && \quad\hfil $##$\hfil\crcr\omit\strut %
\hfil\crcr\noalign{\kern -\baselineskip}#2\crcr\omit %
\strut\cr}}%
\setbox\tw@\vbox{\unvcopy\z@\global\setbox\@ne\lastbox}%
\setbox\tw@\hbox{\unhbox\@ne\unskip\global\setbox\@ne\lastbox}%
\setbox\tw@\hbox{%
$\kern\wd\@ne\kern -\@tempdima\left\@firstoftwo#1%
\if@borderstar\kern2pt\else\kern -\wd\@ne\fi%
\global\setbox\@ne\vbox{\box\@ne\if@borderstar\else\kern 2\p@\fi}%
\vcenter{\if@borderstar\else\kern -\ht\@ne\fi%
\unvbox\z@\kern-\if@borderstar2\fi\baselineskip}%
\if@borderstar\kern-2\@tempdima\kern2\p@\else\,\fi\right\@secondoftwo#1 $%
}\null \;\vbox{\kern\ht\@ne\box\tw@}%
\endgroup
}
\makeatother

\usetheme{UiB}

%For easier reading
\setbeamersize{text margin left=40pt,text margin right=40pt}
\renewcommand{\baselinestretch}{1.3}


%% FONT STUFF
\usepackage{amsmath}
\usepackage{amsfonts}
\usefonttheme[onlymath]{serif}


\author{Colin Roberts}
\setbeamercolor{title}{fg=white} 
\title{The Calder\'on Problem}
\setbeamercolor{subtitle}{fg=white} 
\subtitle{on Riemannian Manifolds}



\begin{document}


\section{Introduction}

\begin{frame}{The Two Sides to the Problem}
    \begin{itemize}
        \item Practitioners: Work with sparse and noisy data to recover information in the real world.
        \item Theorists: Work in ideal scenarios with chosen information to see the scope of possibilities.
    \end{itemize}
\end{frame}

\subsection{Electrical Impedence Tomography}

\begin{frame}{EIT}
    \textbf{\underline{Idea:}} Given a domain $\Omega$ which has an interior $\Omega^+$ we cannot probe, what can we learn from studying the boundary $\partial \Omega$? In particular...
    \begin{itemize}
        \item $\Omega^+$ is free of charges, hence $\Delta u = 0$ in $\Omega^+$ where $u$ is the electrostatic potential.
        \item Apply a known voltage $f$ at the boundary $\partial \Omega$. Hence $f = u \vert_{\partial \Omega}$.
        \item Measure the current flux $g$ through the boundary $\partial \Omega$. Hence, $g = \frac{\partial u}{\partial \nu}$.
        \item This defines the voltage-to-current map $\Lambda$ so that $\Lambda(f) = g$.
        \item What can we learn about $\Omega^+$ from $\Lambda$?
        \item Can we determine the conductivity matrix $\gamma$ from $\Lambda$?
    \end{itemize}
\end{frame}

\begin{frame}{EIT}
    Use cases:
    \begin{itemize}
        \item Medical Imaging: $\Omega^+$ could be a portion of a human body. (AC Method)
        \item Geophysical Imaging: $\Omega^+$ could be a below the Earth's surface. (DC Method)
        \item 
    \end{itemize}
\end{frame}

\subsection{Riemannian Manifolds}


\subsection{Challenges}



\section{Preliminaries}

\begin{frame}{Key Tools}
    Clifford algebras/differential forms, hodge decomposition, spectral theory for commutative banach algebras
\end{frame}


\subsection{Smooth Manifolds}

\end{document}

