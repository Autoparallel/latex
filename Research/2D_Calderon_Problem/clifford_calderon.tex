\section{Introduction}
The Calder\'on problem for requires reconstructing the conformal structure of a manifold $\Omega$ given the Dirchlet-to-Neumann map on the boundary.  It has been demonstrated in dimensions up to two in various ways. The method in which we emply is the Boundary Control (BC) method \cite{belishev_complex}.  

\section{Outline}

\subsection{Belishev's Complex Idea}
The following facts are in use:
\begin{enumerate}[(i)]
	\item $(\Omega,g)$ has the complex structure of a Riemann surface; this structure determines
the class of metrics conformally equivalent to $g$;

	\item the algebra $\mathcal{A}(\Omega)$ of functions continuous in $\Omega$ and holomorphic in $\mathrm{int}(\Omega)$ is
nontrivial; functions $w\in \mathcal{A}(\Omega)$ (as local homeomorphisms $\Omega''\to \C$) determine the
complex structure;

	\item  algebra $\mathcal{A}(\Omega)$ is generic: its (topologized) spectrum is homeomorphic to the
manifold, $\mathrm{sp}\mathcal{A}(\Omega) \asymp \Omega$, whereas the algebra itself is identical to its Gelfand transform,
$\hat{\mathcal{A}}(\Omega) \equiv \mathcal{A}(\Omega)$;

	\item  the algebra of traces $\algebra(\Gamma) \coloneqq \{w\vert_{\Gamma} ~\vert~ w \in \algebra(\Omega)\}$ is isometrically isomorphic
to $\algebra(\Omega)$ (through the map $\trace \colon w'' \to w\vert_\Gamma$); the isometry yields $\spectrum \algebra(\Gamma) \asymp \spectrum \algebra(\Omega)$,
$\hat{\algebra}(\Gamma) \equiv \hat{\algebra}(\Omega)$ that leads to $\spectrum \algebra(\Gamma) \asymp \Omega$ and $\hat{\algebra}(\Gamma) \equiv \algebra(\Omega)$;

	\item the algebra $\alg(\Gamma)$ is determined by the DN-map $\Lambda_g$.
\end{enumerate}


To solve the Calderon problem we use these facts in reverse order:
\begin{enumerate}[(a)]
	\item from the operator $\Lambda_g$, one recovers the trace algebra $\algebra(\Gamma)$;
	\item Finding its spectrum and Gelfand transform, we get $\Omega \asymp \spectrum \algebra(\Gamma)$ and $\algebra(\Omega)\equiv \hat{\algebra}(\Gamma)$;
	\item Using functions of the algebra $\algebra(\Omega)$ we endow $\Omega$ with the complex structure;
	\item introducing a metric $g$ on $\Omega$ conformal to the complex structure we get the manifold $(\Omega,g)$ whose DN-map coincides with $\Lambda_g$ by construction.
\end{enumerate}
This procedure (a)-(d) gives a canonical representative of the class of conformally equivalent manifolds which has the given DN-map. The assertion of Theorem 1 is a simple corollary of determinacy of this procedure.

\subsection{Rephrasing with the Geometric Algebra}

We will want to prove these theorems/lemmas

\begin{thm}{}{theorem1}
	The algebra $C(\Omega)$ is dense in the holomorphic functions on $\Omega$. (or something like this.)
\end{thm}

\begin{thm}{}{theorem2}
	The algebra $\algebra(\Omega)$ is generic.
\end{thm}

\begin{thm}{}{theorem3}
	The trace algebra $\algebra(\Gamma)$ is isometrically isomorphic to $\algebra(\Omega)$.
\end{thm}

\begin{thm}{}{theorem4}
	The algebra $\algebra(\Gamma)$ is determined by the DN-map.
\end{thm}

Note that Theorems 1-2 combine to tell us that it suffices to understand the holomorphic function in $\Omega$ to determine an isometric copy of $\Omega$ through an algebra.  Theorems 3-4 relate the boundary information to the algebra on all of $\Omega$. Combined, Theorems 1-4 yield

\begin{thm}{}{theorem5}
	The DN-map determines the conformal class of  a smooth 2-dimensional manifold.
\end{thm}

\section{Ideas}
We can consider the $2D$ region embedded in some $\R^n$ or something and use directed integration theory.

\section{Harmonic Functions and Fields}
Consider the boundary value problem
\begin{align}
	\Delta_g u &= 0 \quad \textrm{in $\interior \Omega$}\\
	u&=f \quad \textrm{on $\Gamma$}.
\end{align}
We refer to solutions of the above problem as \boldblue{potentials} and we put $\potentials \coloneqq \{u^f~\vert~f\in C^\infty(\Gamma)\}$ (\textcolor{red}{there is a use of smoothness here}).

Let $(\Omega,g)$ be a oriented 2-dimensional compact Riemannian manifold with boundary $\Gamma$. Then the set of \boldblue{Harmonic functions} on $\Omega$ is
  \[
  \harmonic_g \Omega \coloneqq \{u~\vert~\Delta_g=0 ~\textrm{in}~\interior \Omega\}
  \]
  Note that $\potentials = \harmonic_g \Omega \cap C^\infty (\Omega)$. \textcolor{red}{Is this fact obvious?}
    
  \subsection{Clifford Fields}
  Denote by $v\in \evenfields$ an even (spinor?) field in the clifford bundle on $\Omega$.  Given that $\Omega$ is 2-dimensional, we can write $v=v_0+v_2$ where $v_0=\projzero{v}$ and $v_2=\projtwo{v}$ and we note that $\projone{v}=0$. \textcolor{red}{Think of the even clifford fields as even dimensional forms endowed with the interior and exterior multiplication in \cite{dirac-spectral}}
    
  \textcolor{red}{there is the definition of rotation operator here. Maybe this isn't too hard to define, but I'm not seeing the use of it yet.}
  
  Let $\harmonicfields\coloneqq \{v \in \evenfields ~\vert~ dv=\delta v = 0 ~\textrm{in $\interior \Omega$}\}$ denote the space of \boldblue{harmonic even fields}.  \textcolor{red}{Harmonic fields are smooth in $\interior \Omega$; the set $\harmonicfields^\infty \coloneqq \harmonicfields \cap \vec{C}^\infty(\Omega)$ is dense in $\harmonicfields$}. The space $\harmonicfields$ contains a subspace of \boldblue{potential fields} $\potentialfields \coloneqq \{d u \in \harmonicfields ~\vert~ u \in \harmonic_g\Omega \cap H^1(\Omega)\}$.  \textcolor{red}{Okay, but $d u$ is not an even field.  What is meant here then? Maybe $u$ should be an honest vector (1-form) field. Maybe it's really just a form that's exact.} Note that every $h\in \harmonicfields$ is locally potential.
  
  We let $\trace$ denote the boundary value for fields in $\Omega$.  The important fact is that a harmonic field is determined by its trace (that is the Cauchy integral formula in effect here): $h\in \harmonicfields$ and $\trace h =0$ implies $h=0$ in $\interior \Omega$ (this is the maximum modulus principle proven by Cauchy integral formula).  Indeed, fixing an $x_0\in \Gamma$, in $U_{x_0}$ we have $h=d u$ that leads to $\Delta_g u =0$ and $\trace du = 0$, yielding $v=\textrm{const.}$ and $h=0$ in $U_{x_0}$ due to the uniqueness of the solution to the Cauchy problem for the second order elliptic equation. We can then cover the manifold in neighborhoods and use the same uniqueness theorem to obtain $h=0$ everywhere. 
  
  \textcolor{red}{This may be worthy of a picture and intuition via directed integration and generalization of Cauchy}
  
  Let $\difference \coloneqq \harmonicfields \ominus \potentialfields$. We have
  \[
  \dim \difference = \dim \harmonicfields / \potentialfields = \beta_1 (\Omega)=1-\chi(\Omega)
  \]
  which (\textcolor{red}{I believe}) follows from the Hodge decomposition. The fields $b\in \difference$ are tangent on $\Gamma$ since for any $f\in C^{\infty}(\Gamma)$ we have
  \[
  0=\int_\Omega d\Omega \langle \nabla u^f,b\rangle = \int_\Gamma d\gamma f \left\langle b,\frac{\partial}{\partial \nu}\right\rangle,
  \]
  which yields $\left\langle b,\frac{\partial}{\partial \nu}\right\rangle=0$ or equivalently, $\trace b = \kappa \frac{\partial}{\partial \gamma}$ with smooth $\kappa = \left\langle b,\frac{\partial}{\partial \gamma}\right\rangle$.
  
  \textcolor{red}{It would be worth writing the above in terms of differential forms. Specifically as inner products of forms and such. Note that we have $\alpha \wedge \star \beta = \langle \alpha, \beta \rangle \omega$ where $\omega$ is the volume form.}  
  
  \begin{prop}{$\difference$ fields are tangent on $\Gamma$}{difference_tangent}  
  We have that for any $b\in \difference$ that $b=\kappa d \gamma$ on $\Gamma$.
  \tcblower
  \begin{proof}	
  By definition of $\difference$ we have that any $b\in \difference$ is $L^2$ orthogonal to any element in $\potentialfields$. Hence, we have
   \[
  0=(du^f,b)_\Omega=\int_\Omega \langle du^f,b\rangle \omega
   \]
   and
   \[
   (du^f,b)_\Omega = (u^f,\delta b)_\Omega =0,
   \]
  where $\omega$ is the Riemannian volume form on $\Omega$ and for any $f\in C^\infty (\Gamma)$.   Note that
  \[
  \langle du^f,b\rangle \omega = du^f \wedge \star b,
  \]
  and
  \[
  d(u^f \wedge \star b) = du^f \wedge \star b + (-1)^k u^f \wedge d\star b.
  \]
  Hence by Stokes' theorem
  \begin{align*}
  	\int_\Omega d(u^f \wedge \star b) &= \int_\Omega du^f \wedge \star b + (-1)^k \int_\Omega u^f \wedge d \star b\\
  	\int_{\partial \Omega} u^f \wedge \star b &=\underbrace{\int_\Omega du^f \wedge \star b}_{=0} + (-1)^k \underbrace{\int_\Omega u^f \wedge d \star b}_{=0}\\
  	\int_\Gamma f \star b &= 0,
  \end{align*}
  for any $f$. Hence, it must be that $\star b = (-1)^k\kappa d\nu$ and hence $b=\kappa d\gamma$.
  \end{proof}
  \end{prop}

Let $\dot{C}^\infty(\Gamma)\coloneqq \{f\in C^\infty (\Gamma) ~\vert~ (1,f)_\Gamma=\int_\Gamma f(\gamma)d\gamma=0\}$ be the subset of smooth functions with zero mean value. For $f\in\zeromean$, we denote by $Jf$ the primitive function with zero mean value, i.e., $Jf\in \zeromean$, $\frac{d}{d\gamma}Jf=f$. \textcolor{red}{This is an issue in higher dimensions since the boundary is not 1-dimensional and hence the primitive is not determined up to a constant, it is determined up to some closed form.}

Note that $\ker \Lambda_g = \{\textrm{const.}\}$, $\im \Lambda_g \subset \zeromean$. Also, by standard elliptic theory the operator $\Lambda_g J \colon L_2(\Gamma) \mapsto L_2(\Gamma)$, $\dom \Lambda_g J = \zeromean$ is continuous.

\textcolor{red}{There is a Lemma 1 that I don't quite get the purpose of or how to prove on my own let alone the proof given.}

\section{Algebras}

We let $\algebra$ denote a Commutative Banach Algebra (CBA). Let $\algebra'$ be the space of continuous linear functionals on $\algebra$.  We call a functional $\delta\in \algebra'$ \boldblue{multiplicative} if $\delta(ab)=\delta(a)\delta(b)$  (that is, an algebra homomorphism into $\C$?). We denote the set of multiplicative functionals by $\multifuncts$.

 A subspace $I\neq \algebra$ is an \boldblue{ideal} if $ja\in I$ for any $j\in I$, $a\in I$.  An ideal is \boldblue{maximal} if for any ideal $\overline{I}\subset \algebra$ with $I\subset \overline{I}$ means $I=\overline{I}$. We let $\maxideals$ denote the set of maximal ideals of an algebra $\algebra$. \textcolor{red}{there are other facts about ideals given.}

There is a canonical bijection between the sets $\multifuncts$ and $\maxideals$. If we have $\delta \in \multifuncts$, then we can create $I_\delta\coloneqq \ker \delta \in \maxideals$. Then the projection $\delta_I\colon \algebra \to \algebra/I=\C$ is an element of $\multifuncts$.  

The \boldblue{Gelfand transform} maps an element $a\in \algebra$ into a function $\hat{a}$ on $\multifuncts$ by the rule $\hat{a}(\delta)\coloneqq \delta(a)$. The \boldblue{Gelfand topology} is defined as the weakest topology on $\multifuncts$ in which all $\hat{a}$ are continuous. The set $\multifuncts$ with this topology is compact and called the \boldblue{spectrum} (or the maximal ideal space) of the algebra $\algebra$ and we denote by $\spectrum \algebra$.

\textcolor{red}{There are more definitions given}

\subsection{Algebra $\algebra (\Omega)$}

Functions $v_0,v_2^* \in \harmonic_g \Omega$ are said to be \boldblue{conjugate} if they satisfy the CREs. That is, if we let $v=v_0+v_2$ be an even clifford field, if $v$ is monogenic in that
\[
(\delta + d)v = 0,
\]
then it must be that $v_0$ and $v_2$ satisfy
\[
dv_0 = -\delta v_2.
\]
\textcolor{blue}{See notes below.} Moreover, monogenic even clifford fields have harmonic components. We define the algebra $\algebra(\Omega)=\{v\in \evenfields ~\vert~ (\delta+d)v=0\}$ to be the set of monogenic even clifford fields which decompose into harmonic conjugates.  \textcolor{blue}{It is clear this is an algebra. Indeed, see my notes later.}

\textcolor{red}{Some facts from complex analysis are stated which I believe can be done using the generalized Cauchy integral formula.}

\newpage
\section*{Notes}
\textcolor{red}{Does the fact that a spinor can be written as an exponential of a bivector do anything?}
The key seems to be that for a complex function $\Psi=u(z)+iv(z)$ that $\frac{\partial}{\partial z} \Psi =0$ implies that both $u(z)$ and $v(z)$ are harmonic functions. This is true for even grade multivectors $f$ as well. So, if we take $f=f_0+f_2$ then $\deriv f =0$ (i.e., $f$ is \boldblue{monogenic}) means that $f_0$ and $f_2$ are harmonic.  Hence, by knowing $f$ is monogenic and knowing either $f_0$ or $f_2$, we know $f$ all together.  So the goal is then to recover the algebra of monogenic functions on $\Omega$ via this connection.

In section 6.3 of Geometric Algebra for Physicists, the eigenvalue equation arises. This might connect nicely to the spectral theory somehow.

We can talk about the notation for taking derivatives which is:
\begin{itemize}
    \item In the absence of brackets, $\deriv$ acts on the object to its immediate right.
    \item When $\deriv$ is followed by brackets, the derivative acts on all of the terms in the brackets.
    \item When the $\deriv$ acts on a multivector to which it is not adjacent, we use overdots to describe the scope.
\end{itemize}
To construct the \boldblue{algebra of monogenic multivectors} we can take two monogenic multivectors $A$ and $B$ and then we have
\begin{align*}
\deriv (AB) &= \deriv AB + \dot{\deriv}A\dot{B}\\
&= 0
\end{align*},
since $\deriv A=\deriv B=0$. Then let $\monogenicfxns$ be the algebra of monogenic multivectors on $\Omega$.

We also want to show that if a even field $f$ is monogenic, then its components are harmonic. So take $f=f_0+f_2$ to be monogenic, then
\[
\deriv f = \deriv f_0 + \deriv f_2 = 0.
\]
Thus we have
\begin{align*}
    (\delta + d)f_0 + (\delta+d)f_2&= df_0 + \delta f_2 + df_2 = 0. 
\end{align*}
Hence it must be that 
\[
df_0 = - \delta f_2 \qquad \textrm{and} \qquad df_2 = 0,
\]
which are like the CREs (shown below also).  Then we want to show that
\[
(d\delta +\delta d)f_0 = 0 \qquad \textrm{and} \qquad (d\delta + \delta d) f_2 = 0.
\]
For the first we have $\delta f_0=0$ and this means that we just want
\[
\delta d f_0 =0 
\]
which is true since we know that 
\[
df_0 = -\delta f_2 ~\implies~ \delta df_0 = -\delta^2 f_2 = 0.
\]
So $f_0$ is harmonic.  Then for $f_2$ we have that $df_2=0$, and hence we need only show that
\[
d\delta f_2 =0
\]
which is true since
\[
\delta f_2 = df_0 ~\implies~ d\delta f_2 = d^2 f_0 = 0.
\]
Thus, if $f$ is an even monogenic field, its components are harmonic.



%% Stuff to pull from
\section*{Set Up}
Let $(M,g)$ be an $n$-dimensional smooth Riemannian manifold.  Then we can consider an induced Clifford bundle $C\ell(TM,g)$ on $M$ with the quadratic form $Q(v)=g(v,v)$.  This is a natural choice, but not necessarily the ``correct" one.  Specifically, we define $C\ell(T_pM,g_p)$ to be the Clifford algebra on the tangent space $T_pM$ and let
\[
C\ell(TM,g) \coloneqq \dot{\bigcup}_{p\in M} C\ell (T_pM,g_p)
\]
be the Clifford bundle.

We can then define the space of Clifford sections by noting we have a natural projection
\[
\pi \colon C\ell(TM,g) \to M
\]
that maps a Clifford element to the point at which it is based and putting
\[
\Gamma C \ell (TM,g) \coloneqq \left\{ \sigma \colon M \to C\ell(TM,g) ~\vert~ \pi \circ \sigma = \id_M\right\}.
\]
\begin{question}
There's probably some notation for this somewhere. How do we assure that this is a \emph{smooth} section?
\end{question}

\begin{question}
When is this product operation smooth? Is it always smooth? What does smoothness really mean here?
\end{question}

\textcolor{red}{Belishev is looking at quaternionic functions that satisfy both $\Delta \alpha=0$ and $\vec{\Delta}u=0$ in the weak sense. This should be easier to state with the below set up.}

Consider a $3$-dimensional Riemannian manifold $(M,g)$ with the associated \boldblue{Clifford bundle} $\cliffbund$ generated from the metric $g$. We call a $f\in \cliffsect$ a \boldblue{Clifford field} or just field.  

\begin{problem}
Problem statement to recover $g$ from $\Lambda$ and algebra of ``holomorphic" functions.
\end{problem}

Let $i\colon \partial M \to M$ be inclusion of the boundary in $M$.


 \begin{figure}[H]
   \centering
   \def\svgwidth{0.75\columnwidth} 
   \resizebox{75mm}{!}{\input{figures/manifold_with_boundary.pdf_tex}}
   \caption{I fixed the caption}
   \label{fig: broken_caption_is_stupid}
 \end{figure}

Think of even grade (spinor?) fields $f\in \cliffsect$ as (maybe?) the voltage and current flux multivector. That is
\[
\projzero{f} = f_0 \qquad \textrm{and} \qquad \projtwo{f}=f_2
\]
where the other grades are zero. $f_0$ could represent the voltage scalar and $f_2$ the current flux plane field.

Specifically the problem reads:
\[
\begin{cases}
df=\delta f = 0 &\textrm{in $M$}\\
i^* f = \varphi \qquad i^*(\delta f)=0 & \textrm{on $\partial M$}
\end{cases}
\]
Do we want $f_0$ above and not $f$ since that is the scalar part of this problem? The $f_2$ part comes in when we're talking about conjugate stuff. Belishev defines \boldblue{harmonic quaternion fields} $\{\alpha,u\}$ to be the ones satisfying 
\[
d\alpha = \star d u \qquad \delta u=0
\]
where we're thinking of $u$ as a 1-form (really a purely imaginary quaternion field).

Then the \boldblue{Dirichlet to Neumann operator} for forms gives
\[
\Lambda \varphi = i^*(\star d \varphi) = (-1)^{k+1}i^*(\delta \star \varphi).
\]

It seems like the idea is we know $f_0$ solves the Dirchlet problem and we can relate $f_2$ to $f_0$ by the CREs. Then $f_2$ is related to the Neumann data in the problem.

\section*{Cauchy Riemann Equations}

Letting $\nabla$ represent the \boldblue{Dirac operator} or \boldblue{geometric derivative} (which depends on the metric, as we'll see) we have
\[
\nabla = e^i \partial_i = d + \delta,
\]
where the first equality is induced from a basis on the tangent space $e_i$ and $e^i$ represents the dual basis satisfying $e^i \cdot e_j = \delta^i_j$. The second equality is more coordinate free where $d$ and $\delta$ are the \boldblue{exterior derivative} and \boldblue{interior derivative} respectively. One may define these via
\[
\nabla
\]
\begin{align*}
df &= \frac{1}{2}\left( \nabla f - f\nabla\right)\\
\delta f &= \frac{1}{2} \left( \nabla f + f\nabla\right),
\end{align*}
If we consider this field to be \boldblue{harmonic} in the interior of $M$, then
\[
\nabla^2 f =0 
\]
or more ``weakly" (I believe)
\[
d f =0 \qquad \textrm{and} \qquad \delta f = 0.
\]
If we have 
\begin{align*}
    0=\nabla^2 f &= (d\delta +\delta d) f\\
    0&= d\delta f_0 + (d\delta + \delta d)f_2\\
    \implies d\delta f_0 &= -(d\delta + \delta d)f_2
\end{align*}

For any test form $\varphi$ we should have the weak equations
\begin{align*}
    0&=\langle \nabla^2 f, \varphi\rangle &= \langle (d\delta + \delta d)f,\varphi \rangle\\
    &= \langle d \delta f, \varphi\rangle + \langle \delta d f, \varphi \rangle \\
    &= \langle \delta f, \delta \varphi \rangle + \langle df,d\varphi\rangle
\end{align*}
which means we have that $\delta f=df=0$.

Using the second requirement for harmonic, we find the ``\boldblue{Cauchy Riemann equations}" are
\begin{align*}
    0=df&=\delta f\\
    df_0 + df_2 &= \delta f_0 + \delta f_2\\
    df_0 + df_2 &= \delta f_2,
\end{align*}
since $\delta f_0=0$.  Now, if we match grades we arrive at four equations
\begin{align*}
    \projzero{df}&=\projzero{\delta f} &&& \projone{df}&=\projone{\delta f} &&& \projtwo{df}&= \projtwo{\delta f} &&& \projthree{df}&= \projthree{\delta f}\\
    0 &= 0 &&& df_0 &= \delta f_2 &&& 0&=0 &&& df_2 &= 0.
\end{align*}
The ``Cauchy Riemann equations" are then
\[
df_0= \delta f_2
\]
and $df_2 =0$ states that the current flux plane field is closed. (what does this mean physically? Does it make sense? Does it mean that all ``currents" form loops?) 

\section*{Dirichlet to Neumann Operator}

The \boldblue{Dirichlet to Neumann operator} for differential forms is defined by
\[
\Lambda \colon \Omega^k(\partial M) \to \Omega^{n-k-1}{\partial M}
\]
given by
\[
\Lambda \varphi = i^* \star d \omega=(-1)^{k+1}i^*(\delta \star \varphi),
\]
where $\varphi$ is the \boldblue{Dirichlet data} on the boundary and $d\omega$ is the \boldblue{Neumann data} on the boundary.

If we define $\star$ for $\cliffsect$ we could use the pseudoscalar $I$ at each point (which is global since it is a volume form). The inclusion map is defined in the usual way and $d$ is the exterior derivative.  

\section{Algebras}
Even (spinor?) fields form a subalgebra of $\cliffbund$.  This is similar to the result for Section 2 in Belishev \cite{belishev_quaternion}. 
\begin{question}
Can we place a norm on it to form it into a Banach algebra? 
\end{question}

Maybe we can just take the $l^2$ norm of each part individually or something (similar to how Belishev does in the same algebra section.

\begin{question}
Is the subspace of harmonic fields an algebra? Or is it at least dense in some way? See Theorem 1 in \cite{belishev_quaternion}
\end{question}

\begin{question}
Does this algebra of Clifford sections form a Banach algebra?
\end{question}

\begin{definition}
A \emph{Banach algebra} is an associative algebra $\mathcal{A}$ over $\R$ or $\C$ that at the same time is also a Banach space, i.e. a normed space and complete in the metric induced by the norm. The norm is required to satisfy
\[
\forall x,y \in \mathcal{A}~\colon~ \|xy\|\leq \|x\|\|y\|.
\]
\end{definition}

\begin{question}
If we can make the algebra of Clifford sections a (unital) Banach algebra, can we compute the spectrum? (If it forms a Banach algebra, the scalar element $1$ is likely the unit.)
\end{question}

\begin{definition}
The \emph{spectrum} $\sigma(x)$ of $x\in \mathcal{A}$ is the set
\[
\sigma(x) \coloneqq \{ \lambda ~\vert~ x-\lambda 1 \textrm{ is not invertible.}\}
\]
\end{definition}

Of course this has all been said before, see \url{https://en.wikipedia.org/wiki/Clifford_bundle}. It seems that the signs are flipped from mine.

\begin{question}
Does $|Q(v)|$ form a norm on the Clifford algebra fiber? If so, can we extend this to a norm on the manifold?
\end{question}

\begin{definition}
A \emph{norm} $\|\cdot\|$ on a vector space $V$ over a field $\mathbb{F}$ is a map
\[
\|\cdot \| \colon V \to \mathbb{F}
\]
that satisfies
\begin{enumerate}[(i)]
    \item (Triangle inequality) $\|u+v\|\leq \|u\|+\|v\|~\forall u,v\in V$,
    \item (Scalar behavior) $\|\alpha v\| = |\alpha|\|v\|~\forall v\in V$ and $\alpha \in \mathbb{F}$,
    \item (Unique zero) $\|v\|=0 ~\iff~ v\equiv 0$.
\end{enumerate}
If $\|\cdot \|$ only satisfies (i) and (ii) then we call $\|\cdot \|$ a \emph{seminorm}.
\end{definition}

\begin{proposition}
Let $V$ be a $n$-dimensional vector space over a field $\mathbb{F}$, $Q$ be a nondegenerate quadratic form, and $C\ell(V,Q)$ be the Clifford algebra over $V$ with $Q$, then the function
\[
|Q(\cdot)| \colon V \to \mathbb{F}
\]
is a norm on $C\ell(V,Q)$.
\end{proposition}

\begin{proof}
We show the three properties listed above.  Let $\{e_1,\dots,e_{2^n}\}$ be an arbitrary basis for $C\ell(V,Q)$. We note that given $v \in C\ell(V,Q)$ we can write
\[
v=\sum_{i=1}^{2^n} \alpha_i e_i.
\]
We then take $A$ to be a symmetric $2^n\times 2^n$-matrix (with rank $2^n$ by the nondegeneracy) and define
\[
Q(v)\coloneqq \sum_{i=1}^{2^n} \sum_{j=1}^{2^n} A_{ij}\alpha_i \alpha_j.
\]
\begin{enumerate}[(i)]
    \item Let $v,u \in C\ell(V,Q)$.  Then put
    \begin{align*}
        v &= \sum_{i=1}^{2^n} \alpha_i e_i,\\
        u &= \sum_{i=1}^{2^n} \beta e_i.
    \end{align*}
    \begin{align*}
    |Q(v+u)|&=\left|\sum_{i=1}^{2^n} \sum_{j=1}^{2^n} A_{ij}(\alpha_i+\beta_i) (\alpha_j+\beta_j)\right|\\
    &= \left|\sum_{i=1}^{2^n} \sum_{j=1}^{2^n} A_{ij}(\alpha_i \alpha_j +\beta_i \alpha_j+ \alpha_i\beta_j +\beta_i\beta_j)\right|
    \end{align*}
    
    \[
    |Q(v)|+|Q(u)| = \left|\sum_{i=1}^{2^n} \sum_{j=1}^{2^n} A_{ij}\alpha_i \alpha_j\right| + \left|\sum_{i=1}^{2^n} \sum_{j=1}^{2^n} A_{ij}\beta_i \beta_j\right|
    \]
\end{enumerate}
\end{proof}



\section*{Questions and Thoughts}

\begin{itemize}
    \item Do some searching for ``clifford algebras, eit, voltage to current, dirichlet to neumann, ..."
\end{itemize}

\begin{question}
What is the Dirichlet to Neumann operator?
\end{question}

\begin{question}
If we define the Dirichlet to Neumann operator as in \cite{clay_d-to-n_map}, then this only uses the exterior derivative. What if we instead consider the Dirac operator?
\end{question}

\begin{question}
What is the Hilbert transform?
\end{question}

\begin{question}
Can we relate the Hilbert transform to the D to N map?
\end{question}

\begin{question}
Can we recover the algebra of ``holomorphic" functions from the above information?
\end{question}

\section*{Random Things}


\begin{itemize}
    \item \url{http://math.uchicago.edu/~amathew/dirac.pdf} These seem like useful notes
\end{itemize}

\begin{question}
    Is there some kind of Hodge-Clifford decomposition we can do for clifford sections?
\end{question}

\begin{question}
Is there some kind of Clifford homology/cohomology?
\end{question}

Outside of the realm of the Clifford stuff, what about the ``tensor" dirac derivative operator?  Consider the operator
\[
D = e_i \otimes \nabla_{e_i}
\]
so that, for example, on a vector field we achieve
\[
DF = e_i \otimes \nabla_{e_i}f^j e_j = e_i \otimes e_j \nabla_{e_i} f^j
\]
which we can write as a matrix that we call the Jacobian. This should transform nicely under coordinates more naturally.

Then if we consider this object not in the tensor algebra but in the quotient (the Clifford algebra) we achieve
\[
DF = \textrm{div}(F) + [DF]_2
\]
the superalgebra splitting into the div and curl like components.

Then
\[
D^2F = (e_1 \otimes \nabla_{e_i})(e_j\otimes \nabla_{e_j})f^ke_{k} = e_i \otimes e_j \otimes e_k \nabla_{e_i} \nabla_{e_j} f^k,
\]
is like the Hessian tensor. In other words, if instead $f$ was a scalar (rank 0), then
\[
D^2f = (e_i \otimes e_j) \nabla_{e_i} \nabla_{e_j} f
\]
is the hessian.




