%\documentclass[aspectratio=169]{beamer}
\documentclass[aspectratio=169,handout]{beamer}

\usepackage[utf8]{inputenx} % For æ, ø, å
\usepackage{csquotes}       % Quotation marks
\usepackage{microtype}      % Improved typography
\usepackage{amssymb}        % Mathematical symbols
\usepackage{mathtools}      % Mathematical symbols
\usepackage[absolute, overlay]{textpos} % Arbitrary placement
\setlength{\TPHorizModule}{\paperwidth} % Textpos units
\setlength{\TPVertModule}{\paperheight} % Textpos units
\usepackage{tikz}
\usetikzlibrary{overlay-beamer-styles}  % Overlay effects for TikZ

\AtBeginSection{\frame{\sectionpage}}
\AtBeginSubsection{\frame{\subsectionpage}}

\usepackage{hyperref}
\usepackage{svg}
%\usefonttheme{serif}

\usepackage{xfrac}
\usepackage{soul} % Colored text and highlighting, respectively
\usepackage{xcolor}
\usepackage{tikz-cd} % For commutative diagrams
\usepackage{tikz-3dplot}
\usetikzlibrary{angles}
\RequirePackage{pgfplots}
\usepackage{mathtools}
\usepackage{answers}
\usepackage{setspace}
\usepackage{graphicx}
\usepackage{enumerate}
\usepackage{multicol}
\usepackage{mathrsfs}
\usepackage{amsmath,amsthm,amssymb}
\usepackage{marvosym,wasysym} %fucking smileys
\usepackage{float}
\usepackage{morefloats}
\usepackage{pgf,tikz}
\pgfplotsset{compat=1.15}
\usepackage{mathrsfs}
\usetikzlibrary{arrows}
\usepackage{subcaption}
\usepackage[most]{tcolorbox}
\tcbuselibrary{theorems}
\usepackage{fancyvrb}
\usepackage{longtable,booktabs}
\usepackage{stackrel}
\usepackage{animate}
\usepackage[percent]{overpic}
\definecolor{lighter_csu_green}{RGB}{60,133,77}
\definecolor{gray}{RGB}{60,60,60}
\newcommand\boldgreen[1]{\textcolor{lighter_csu_green}{\emph{\textbf{#1}}}}
\usepackage{MnSymbol}
%border matrix
\makeatletter
\newif\if@borderstar
\def\bordermatrix{\@ifnextchar*{%
\@borderstartrue\@bordermatrix@i}{\@borderstarfalse\@bordermatrix@i*}%
}
\def\@bordermatrix@i*{\@ifnextchar[{\@bordermatrix@ii}{\@bordermatrix@ii[()]}}
\def\@bordermatrix@ii[#1]#2{%
\begingroup
\m@th\@tempdima8.75\p@\setbox\z@\vbox{%
\def\cr{\crcr\noalign{\kern 2\p@\global\let\cr\endline }}%
\ialign {$##$\hfil\kern 2\p@\kern\@tempdima & \thinspace %
\hfil $##$\hfil && \quad\hfil $##$\hfil\crcr\omit\strut %
\hfil\crcr\noalign{\kern -\baselineskip}#2\crcr\omit %
\strut\cr}}%
\setbox\tw@\vbox{\unvcopy\z@\global\setbox\@ne\lastbox}%
\setbox\tw@\hbox{\unhbox\@ne\unskip\global\setbox\@ne\lastbox}%
\setbox\tw@\hbox{%
$\kern\wd\@ne\kern -\@tempdima\left\@firstoftwo#1%
\if@borderstar\kern2pt\else\kern -\wd\@ne\fi%
\global\setbox\@ne\vbox{\box\@ne\if@borderstar\else\kern 2\p@\fi}%
\vcenter{\if@borderstar\else\kern -\ht\@ne\fi%
\unvbox\z@\kern-\if@borderstar2\fi\baselineskip}%
\if@borderstar\kern-2\@tempdima\kern2\p@\else\,\fi\right\@secondoftwo#1 $%
}\null \;\vbox{\kern\ht\@ne\box\tw@}%
\endgroup
}
\makeatother

\usetheme{UiB}

%For easier reading
\setbeamersize{text margin left=40pt,text margin right=40pt}
\renewcommand{\baselinestretch}{1.3}


%% FONT STUFF
\usepackage{amsmath}
\usepackage{amsfonts}
\usefonttheme[onlymath]{serif}

%Commands
\newcommand{\R}{\mathbb{R}}
\newcommand{\C}{\mathbb{C}}
\newcommand{\opens}{\mathcal{O}}
\newcommand{\hilbert}{\mathcal{H}}
\newcommand{\algebra}{\mathcal{A}}
\newcommand{\ideals}{\mathcal{I}}
\newcommand{\functionals}{\mathcal{M}}
\newcommand{\spec}{\mathrm{spec}}
\newcommand{\clifford}{\mathrm{C}\ell}
    \newcommand\quotient[2]{
        \mathchoice
            {% \displaystyle
                \text{\raise1ex\hbox{$#1$}\Big/\lower1ex\hbox{$#2$}}%
            }
            {% \textstyle
                #1\,/\,#2
            }
            {% \scriptstyle
                #1\,/\,#2
            }
            {% \scriptscriptstyle  
                #1\,/\,#2
            }
    }

% Special Commands
\newcommand{\bigslant}[2]{{\raisebox{.2em}{$#1$}\left/\raisebox{-.2em}{$#2$}\right.}}
\newcommand{\RE}{\mathrm{Re}}
\newcommand{\IM}{\mathrm{Im}}
\newcommand{\multivectorbundle}{\mathcal{G}(M,g)}
\newcommand{\multivectorfields}{\mathcal{G}(M)}
\newcommand{\differentialforms}{\mathcal{G}^*(M)}
\newcommand{\kvectorfields}[1]{\mathcal{G}^{#1}(M)}
\newcommand{\evenfields}{\mathcal{G}^{[0]}(M)}
\newcommand{\oddfields}{\mathcal{G}^{[1]}(M)}
\newcommand{\id}{\mathrm{id}}
\newcommand{\innerproduct}[2]{\left\langle #1, #2 \right\rangle_{L_2(\Sigma)}}
%\newcommand{\algebra}[1]{\mathcal{A}_{#1}}
\newcommand{\grad}{\boldsymbol{\nabla}}
\newcommand{\dirac}{\boldsymbol{D}}
\newcommand{\geometricalg}{\mathcal{G}(V)}
\newcommand{\spacealg}{\mathcal{G}_3}
\newcommand{\dual}[1]{\overset{$\star$}&{#1}}
\newcommand{\G}{\mathcal{G}}
\newcommand{\cauchy}{\mathcal{C}}
\newcommand{\poisson}{\mathcal{P}}
\newcommand{\tangent}{\boldsymbol{t}}

\newcommand{\openO}{\mathcal{O}}
\newcommand{\openU}{\mathcal{U}}
\newcommand{\characters}{\mathfrak{M}}
\newcommand{\Span}{\operatorname{Span}}
\newcommand{\paravectors}{\mathcal{P}}
\newcommand{\paravectorfields}{\mathcal{P}(M)}
\newcommand{\monogenics}{\mathcal{M}}
\newcommand{\dualmonogenics}{\mathcal{M}^\times}
\newcommand{\Grassmannian}[2]{\operatorname{Gr}(#1,#2)}
\newcommand{\spinalgebra}{\mathfrak{spin}(n)}
\newcommand{\spingroup}{\operatorname{Spin}(n)}
\newcommand{\ball}{\mathbb{B}}
\newcommand{\disk}{\mathbb{D}}
\newcommand{\projection}{\mathsf{P}}
\newcommand{\vectorspace}{\mathbb{V}}
\newcommand{\multivectorfieldson}[1]{\mathcal{G}(#1)}
\newcommand{\biparavectorfieldson}[1]{\mathcal{G}_n(#1)^{0+2}}
\newcommand{\spinnorm}[1]{\|#1\|_{L_2}}
\newcommand{\rejection}{\mathsf{R}}
\newcommand{\vectorpotential}{\mathbf{A}}
\newcommand{\current}{\blade{j}}
\newcommand{\magneticbivector}{\blade{b}}
\newcommand{\cross}{\boldsymbol{\times}}
\newcommand{\blade}[1]{\boldsymbol{#1}}
\newcommand{\intcurrent}[1]{\left[ #1 \right]}
\newcommand{\multivecinnerproduct}[2]{\ll #1, #2\gg}
\newcommand{\spacetime}{\boldsymbol{\gamma}}
\newcommand{\gradst}{\boldsymbol{\nabla}_{\textrm{ST}}}
\newcommand{\boundary}{{\partial M}}
\newcommand{\normalcurrent}{\blade{j}^{\blade{\nu}}}
\newcommand{\tangentialcurrent}{\blade{j}^{\blade{I}_\partial}}
\newcommand{\normal}{\blade{\nu}}
\newcommand{\pseudoscalar}{\blade{I}}

\newcommand{\kforminnerproduct}[2]{\llangle #1, #2 \rrangle}

\DeclarePairedDelimiter\angles{\langle}{\rangle}
\newcommand{\proj}[2]{\angles*{#2}_{#1}}

% Forms stuff
\newcommand{\harmonicfields}[1]{\mathcal{H}^{#1}(M)}
\newcommand{\monogenicex}[1]{\mathcal{M}^{#1}_{\textrm{ex}}(M)}
\newcommand{\monogenicco}[1]{\mathcal{M}^{#1}_{\textrm{co}}(M)}
\newcommand{\monogenicdirichlet}[1]{\mathcal{M}^{#1}_D(M)}
\newcommand{\monogenicneumann}[1]{\mathcal{M}^{#1}_N(M)}
\newcommand{\exactfields}[1]{\mathcal{E}^{#1}(M)}
\newcommand{\coexactfields}[1]{\mathcal{C}^{#1}(M)}
\newcommand{\monogenicfields}[1]{\mathcal{M}^{#1}(M)}
\newcommand{\cliffordoplus}{\stackrel{C\ell}{\oplus}}
\newcommand{\bivector}{\blade{B}}

\author{Colin Roberts}
\setbeamercolor{title}{fg=white} 
\title{Clifford Analysis and a Noncommutative Gelfand Representation}
\setbeamercolor{subtitle}{fg=white} 
\subtitle{}



\begin{document}

\begin{frame}{Outline}
\vfill
    \begin{itemize}
    \pause
        \item Introduce geometric algebra and calculus.
        
        \pause
        \item Describe the toolbox in comparison to differential forms.
        
        \pause
        \item Prove a multivector version of the Hodge-Morrey decomposition.
        
        \pause
        \item Prove a noncommutative version of the Gelfand representation.
    \end{itemize}
\vfill    
\end{frame}

\section{Introduction}

\subsection{Motivation}
\begin{frame}{Electrical Impedance Tomography}
\vfill
\boldgreen{Electrical Impedance Tomography (EIT)} asks whether one can determine the conductivity of a medium based on measurements along the boundary.
\vfill
\end{frame}

\begin{frame}{Calder\'on problem}
\vfill 
    \begin{itemize}
        \pause 
        \item Let $M$ be a smooth, connected, oriented Riemannian manifold with boundary $\partial M$ with metric $g$.
        
        \pause
        \item Conductivity is represented by $g$.
        
        \pause
        \item \underline{Forward problem:} Let $\Delta u = 0$ in $M$ and $u=\phi$ on $\partial M$.
        
        \pause
        \item \underline{Inverse problem:} Given the \boldgreen{Dirichlet-to-Neumann map} $\Lambda \phi = \frac{\partial u}{\partial \nu}$, can we recover $(M,g)$?
    \end{itemize}
\vfill
\end{frame}

\subsection{Preliminaries}


\begin{frame}{}
\vfill
\begin{itemize}
    \item \boldgreen{Clifford algebra} originated in 1878 with William Kingdon Clifford's work that extends Hermann Grassmann's \boldgreen{exterior algebra}.
    \pause
    \item \boldgreen{Clifford analysis} arrived in the 1980's due to Hestenes, Sobczyk, Sommen, Brackx, and Delenghe in order to enrich \'Ellie Cartan's \boldgreen{differential forms}.
    \begin{itemize}
    \pause
        \item Atiyah-Singer Dirac operator and spin manifolds.
    \end{itemize}
\end{itemize}
\vfill
\end{frame}

\begin{frame}{Clifford algebras}
\vfill
Let $V$ be a vector space over a field $\mathbb{F}$ with quadratic form $Q$.
\begin{itemize}
    \pause
        \item Define the tensor algebra
        \[
        \mathcal{T}(V) \coloneqq \bigoplus_{j=0}^\infty V^{\otimes_j} = \mathbb{F} \oplus (V \otimes V) \oplus (V \otimes V \otimes V) \oplus \cdots.
        \]
    \pause
        \item Form the \boldgreen{Clifford algebra} via a quotient
        \[
        C\ell(V,Q) \coloneqq \mathcal{T}(V)/ \langle \blade{v} \otimes \blade{v} - Q(V)\rangle.
        \]
\end{itemize}
\vfill
\end{frame}

\begin{frame}{Geometric and exterior algebras}
\vfill
Let $V$ be a vector space over a field $\mathbb{F}$ with quadratic form $Q$.
\begin{itemize}
    \pause
        \item Given a (pseudo) inner product $g$, we set $Q(\cdot)=g(\cdot,\cdot)$ and define a \boldgreen{geometric algebra}
        \[
        \G \coloneqq C\ell(V,g).
        \]
    \pause
        \item The \boldgreen{exterior algebra} is given by
        \[
        \bigwedge(V) \coloneqq C\ell(V,0).
        \]
\end{itemize}
\vfill
\end{frame}

\begin{frame}{Algebra structure}
\vfill
We define a multiplication in $\G(V)$ by noting how the product $\otimes$ acts in the quotient.
\begin{itemize}
\pause
    \item Given $\blade{u}, \blade{v} \in \G(V)$ we can take the product
    \[
    \blade{u}\blade{v} = \underbrace{\blade{u}\cdot \blade{v}}_{\textrm{scalar}} + \underbrace{\blade{u}\wedge \blade{v}}_{\textrm{bivector}}.
    \]
\pause
\begin{itemize}
    \item The scalar part is symmetric: $\blade{u}\cdot \blade{v} = g(\blade{u},\blade{v})$.
\pause
    \item The bivector part is antisymmetric: $\blade{u}\wedge \blade{v} = -\blade{v}\wedge \blade{u}$.
\end{itemize}
\end{itemize}
\vfill
\end{frame}

\begin{frame}{Multivectors}
\vfill
\begin{itemize}
    \pause
        \item $\G$ is graded and of dimension $2^n$.
    \begin{itemize}
    \pause
        \item There are ${n \choose r}$ elements of grade $r$ called \boldgreen{$r$-vectors}. 
        \item Those that are exterior products of $r$ independent vectors are \boldgreen{$r$-blades}. E.g., $\blade{v}_1 \wedge \cdots \wedge\blade{v}_r$.
    \end{itemize}
    \pause
        \item The most general elements are \boldgreen{multivectors} and are given by
        \[
        A = \sum_{r=0}^n \proj{r}{A},
        \]
        where $\proj{r}{A}$ extracts the grade $r$ part of $A$.
\end{itemize}
\vfill
\end{frame}

\begin{frame}{Examples}
$\C$ and quaternions, spacetime algebra even subalgebras
\end{frame}

\begin{frame}{The playing field}
\vfill
\pause
We let $M$ be a smooth, compact, connected, and oriented $n$-dimensional Riemannian manifold with metric $g$ \textcolor{gray}{(unless otherwise stated)}.
\begin{figure}[H]
	\centering
	\def\svgwidth{.6\columnwidth}
	\input{figures/sphere_charts_maps.pdf_tex}
\end{figure}
\vfill
\end{frame}

\begin{frame}{The playing field}
\vfill
At each point on $M$, we have the tangent space $T_pM$.
\begin{figure}[H]
	\centering
	\def\svgwidth{.6\columnwidth}
	\input{figures/pushforward.pdf_tex}
\end{figure}
\vfill    
\end{frame}

\begin{frame}{The playing field}
\vfill
From $M$, we create the tangent bundle $TM$ whose sections are vector fields.
\begin{figure}[H]
	\centering
	\def\svgwidth{.6\columnwidth}
	\input{figures/section.pdf_tex}
\end{figure}
\vfill
\end{frame}

\begin{frame}{}
\vfill
    \textbf{\underline{Idea:}} On each tangent space, let us construct a manner in which to multiply vectors.
    \pause
    \begin{itemize}
        \item 
    \end{itemize}
\vfill
\end{frame}

\begin{frame}{}
\vfill
    \textbf{\underline{Idea:}} Glue together geometric tangent spaces.
    \pause
    \begin{itemize}
        \item Each $C\ell(T_pM,g_p)$ is a \boldgreen{geometric tangent space} which we glue together to form
        \[
        C\ell(TM,g) \coloneqq \bigsqcup_{p\in M} C\ell(T_pM,g_p).
        \]
    
        \item The space of \boldgreen{(smooth) multivector fields} is
        \[
        \G(M) \coloneqq \{\textrm{$C^\infty$-smooth sections of $C\ell(TM,g)$}\}.
        \]
    \end{itemize}
\vfill
\end{frame}

\begin{frame}{Clifford Algebraic Structure}
\vfill
    \pause
    \begin{itemize}
        \item How do we add and multiply vector fields.
    
        \item Extend this to products on multivectors
    \end{itemize}
\vfill
\end{frame}



\subsection{Preliminaries}

%\begin{frame}{Geometry}
%\vfill
%    \pause
%    \begin{itemize}
%        \item \boldgreen{Smooth $n$-dimensional manifold}: A space that locally looks like (is $C^\infty$ diffeomorphic to) an open subset of $\R^n$.
%        
%        \pause 
%        \item \boldgreen{Riemannian metric}: A smoothly varying inner product defined on $\Omega$.  In coordinates, $g$ takes the form of a symmetric and positive definite matrix with entries $g_{jk}$ with inverse $g^{jk}$.
%        
%        \pause
%        \item \boldgreen{Exterior algebra}: Differential forms with the wedge product $\wedge$.
%        
%        \pause
%        \item \boldgreen{Hodge Star}: Attached to the exterior algebra when we also have a Riemannian metric.  Gives an isomorphism between $k$ and $n-k$-forms.
%    \end{itemize}
%\vfill
%\end{frame}

%\begin{frame}{}
%\vfill
%
%\vfill
%\end{frame}
%
%\begin{frame}{}
%\vfill
%
%\vfill
%\end{frame}
%
%\begin{frame}{}
%\begin{figure}[H]
%	\hspace*{-2cm}
%	\def\svgwidth{1.2\columnwidth}
%	\input{area.pdf_tex}
%\end{figure}
%\vfill
%\end{frame}
%
%\begin{frame}{1-Forms}
%\begin{figure}[H]
%    \centering
%	\includegraphics[width=.8\columnwidth]{1forms.png}
%\end{figure}
%\vfill
%\end{frame}
%
%\begin{frame}{2-Forms}
%\begin{figure}[H]
%    \centering
%	\includegraphics[width=.8\columnwidth]{2forms.png}
%\end{figure}
%\vfill
%\end{frame}
%
%\begin{frame}{3-Forms}
%\vfill
%\begin{figure}[H]
%    \centering
%	\includegraphics[width=.8\columnwidth]{3forms.png}
%\end{figure}
%\vfill
%\end{frame}
%
%
%
%
%\begin{frame}{Partial Differential Equations}
%    \vfill
%    \pause
%    \begin{itemize} 
%    \item \boldgreen{$k$-Form Inner Product}: $\displaystyle{\langle\alpha,\beta\rangle = \int_\Omega \alpha \wedge \star \beta}$.
%    \pause
%    
%        \item \boldgreen{Exterior Derivative}: Derivative operator $d$ defined on $k$-forms.  
%        
%        \pause 
%        \item \boldgreen{Codifferential}: Formal adjoint to $d$ written as $\delta$.
%        
%        \pause
%        \item \boldgreen{Dirac Operator}: $D=d+\delta$.
%        
%        \pause
%        \item \boldgreen{Laplace-Beltrami Operator}: $\Delta = d\delta +\delta d=D^2$ and in coordinates
%        \[
%        \Delta f = \frac{1}{\sqrt{|g|}} \sum_{j=1}^n \sum_{k=1}^n \frac{\partial}{\partial x^i} \sqrt{|g|} g^{ij} \frac{\partial}{\partial x^j} f 
%        \]
%    \end{itemize}
%\end{frame}
%
%\subsection{Raphrasing EIT Problem in a Geometrical Language}
%

%
%\subsection{Expected and Current Results}
%
%\begin{frame}{Formal Variable Count}
%    \pause $f \mapsto \Lambda(f)$ approximated by
%    \[
%    \Lambda(f)_j = \sum_{k=1}^m \lambda_{jk} f_k.
%    \]
%    
%    \pause In the smooth setting,
%    \[
%    \Lambda(f) = \int_{\partial \Omega} \lambda(x,y) f(y)dS(y).
%    \]
%    
%    \pause So, $g$ is a function of $n$ variables that needs to be determined by the kernel $\lambda$ which is $2n-2$ variables.
%\end{frame}
%
%\begin{frame}{Formal Variable Count}
%\vfill
%\begin{itemize}
%    \pause 
%    \item $n=1$ gives us an undetermined system.
%    
%    \pause
%    \item $n=2$ is well determined.
%    
%    \pause
%    \item $n\geq 3$ is overdetermined.
%   \end{itemize}
%\vfill   
%\end{frame}
%
%\begin{frame}{Issue In 2-Dimensions}
%    \pause In 2D, $\Delta$ is conformally invariant.  \\
%    
%    \vspace*{.5cm}
%    \pause Indeed, let $\tilde{g}=e^{2\phi}g$, then
%    \[
%    \Delta_{\tilde{g}} = e^{-2\phi}\Delta_g +(n-2)e^{-2\phi} g^{jk} \frac{\partial \phi}{\partial x^k} \frac{\partial}{\partial x^j}
%    \]
%    
%    \pause When $n=2$, the extra term cancels.
%\end{frame}
%
%\begin{frame}{1 Dimension}
%\vfill
%\pause
%    \begin{itemize}
%        \item The problem in 1 dimension is trivial.
%        
%        \pause
%        \item Can only know the total impedence between the two electrodes.
%        
%    \end{itemize}
%\vfill
%\end{frame}
%
%\begin{frame}{Isotropic Case}
%    For $g$ isotropic and $n\geq 3$ one can determine $g$ from $\Lambda$. \\
%    (\emph{Sylvester-Uhlmann 1987})
%    
%\end{frame}
%
%\begin{frame}{2 Dimensional Anisotropic}
%\vfill
%    \begin{itemize}
%        \item Can recover $g$ up to conformal class and can't do better.
%        \item Proven by Lassas and Uhlmann in \emph{On Determining the Riemannian manifold from the Dirichlet to Neumann map}.
%    \end{itemize}
%\vfill
%\end{frame}
%
%\begin{frame}{3+ Dimensional Anisotropic}
%\vfill
%\pause
%    \begin{itemize}
%        \item For real analytic manifolds, the (scalar/classical) DN map determines the manifold up to isometry.  This gives the topological information as well. (Lassas, Taylor, Uhlmann \emph{The Dirichlet-to-Neumann map for complete Riemannian manifolds with boundary}.
%        
%        \pause
%        \item Can determine the boundary $C^\infty$-jet of $g$ in Lee and Uhlmann's \emph{Determining anisotropic real-analytic conductivities by boundary measurements.}
%        
%        \pause
%        \item For smooth manifolds, the anisotropic problem is open. The goal is to recover the metric up to isometry.
%    \end{itemize}
%\end{frame}
%
%
%
%\section{Boundary Control Method in 2 Dimensions}
%
%\begin{frame}{Theorem}
%    \pause
%    \emph{Two $2$-dimensional compact orientable manifolds with single common boundary are conformally equivalent iff their DN-maps coincide.}\\
%    
%    \vspace*{1cm}
%    Belishev's \emph{The Calder\'on Problem for Two-Dimensional Manifolds by the BC-Method.}
%\end{frame}
%
%
%\begin{frame}{Key Ideas}
%\vfill
%\pause
%    \begin{itemize}
%        \item Surfaces are two dimensional and can be related to $\C$.
%        
%        \pause
%        \item Holomorphic functions have components that are harmonic.
%        
%        \pause
%        \item Hilbert transform converts one harmonic function to another by connecting them via a single holomorphic function.
%        
%        \pause
%        \item Gelfand transform relates an algebra $\algebra$ to the algebra of continuous functions on the spectrum of that algebra, $C(\spec \algebra)$.
%        
%        \pause
%        \item This gives us a way to realize $\Omega$ from functions defined on $\Omega$ that we have access to.
%    \end{itemize}
%\vfill
%\end{frame}
%
%\begin{frame}{Outline of the Proof}
%\vfill
%\pause
%    \begin{itemize}
%        \item From DN map, recover the algebra of holomorphic functions. (Lemma 1)
%        
%        \pause
%        \item Show that this algebra is generic. (Lemma 2)
%        
%        \pause
%        \item Represent the trace algebra with the DN map. (Lemma 3)
%        
%        \pause
%        \item Construct the manifold. (Theorem)
%    \end{itemize}
%\vfill
%\end{frame}
%
%\begin{frame}{Some Notes}
%    \vfill
%    \begin{itemize}
%    \pause
%    \item We have the inclusion of the boundary $\iota \colon \partial \Omega \to \Omega$.
%    
%    \pause
%    \item The pullback $\iota^* \colon T^*\Omega \to T^*\partial \Omega$.
%    
%    \pause
%    \item Define $\Lambda$ by $\iota^*(\star d u)$ for a harmonic $u$.
%    
%    \pause
%    \item $\Lambda$ maps boundary $k$-forms to boundary $n-k-1$ forms.
%    \end{itemize}
%\end{frame}
%
%\begin{frame}{}
%\begin{figure}[H]
%    \centering
%	\includegraphics[width=.8\columnwidth]{2forms.png}
%\end{figure}
%\vfill
%\end{frame}
%
%\subsection{Lemma 1}
%
%\begin{frame}{Lemma 1}
%\vfill
%        \begin{itemize}
%            \pause
%            \item A function $u$ satisfying $\Delta u =0$ has a conjugate function $v$ if and only if the trace $\iota^* u$ satisfies
%            \[
%                \left[ \Lambda + d\Lambda^{-1} d\right] \iota^*u = 0.
%            \]
%            
%            \pause
%            \item $\dim\textrm{Ran}\left[ \Lambda + d \Lambda^{-1} d \right] = \beta_1(\Omega).$
%        \end{itemize}
%        \vfill
%\end{frame}
%
%\begin{frame}{Corollary}
%\vfill
%    $\Lambda$ completely determines the topology of $\Omega$.
%\vfill
%\end{frame}
%
%\begin{frame}{Proof}
%\vfill
%\pause
%    \begin{itemize}
%        \item Since $\Omega$ is a single connected component, $\beta_0(\Omega)=1$.
%        \item We have $\beta_1(\Omega)$ from before.
%        \item Since $\Omega$ is a surface with boundary, $\beta_2(\Omega)=0$.
%        \item Since $\Omega$ is two dimensional, $\beta_n(\Omega)=0$ for $n\geq 3$.
%    \end{itemize}
%\vfill
%\end{frame}
%
%\begin{frame}{Conjugate Function Intuition}
%\vfill
%\begin{itemize}
%   \pause
%   \item Suppose we have homorphic complex function $w = u + i v$.  
%   
%   \pause
%   \item Let $u$ be a 0-form and $v$ as a 2-form.
%   
%   \pause
%   \item Then $\frac{\partial}{\partial \overline{z}}$ is given by $D=d+\delta$.
%   
%   \pause
%   \item $Dw=0$ gives us the Cauchy-Riemann equations
%    
%   \item We call $u$ and $v$ conjugate by CREs.    
%\end{itemize}
%\vfill
%\end{frame}
%
%\begin{frame}{Hilbert Transform}
%\vfill
%\begin{itemize}
%    \pause
%    \item We can get $v$ from $u$ via the Hilbert transform.
%    
%    \pause
%    \item Define $\hilbert = d \Lambda^{-1}$.
%\end{itemize}
%\vfill
%\end{frame}
%
%\begin{frame}{Construct the Algebra $\algebra(\Omega)$}
%\vfill
%\begin{itemize}
%    \pause
%    \item By Lemma 1 we can now create the algebra $\algebra(\Omega)\subset C(\Omega)$ from harmonic functions $u$ with conjugates $v$ by
%    \[
%    \algebra(\Omega) \coloneqq \{ w = u+iv\}.
%    \]
%    Algebra since product of two holomorphic functions is holomorphic.
%    
%    \pause
%    \item In isothermal coordinates, each $w\in \algebra(\Omega)$ is holomorphic.
%    
%    \pause
%    \item This gives $\Omega$ a complex structure.
%    
%    \pause
%    \item This is analogous to having the Hodge star on a surface.
%\end{itemize}
%\vfill
%\end{frame}
%
%\subsection{Lemma 2}
%
%
%\begin{frame}{Gelfand Representation}
%\vfill
%    \begin{itemize}
%    \pause
%            \item Let $\functionals$ be the set of multiplicative linear functionals on a commutative Banach algebra $\algebra$.
%            
%        \pause
%        \item The Gelfand transform gives a way of representing an algebra $\algebra$ as a function algebra $\hat{\algebra}$.
%        
%        \pause
%        \item The \boldgreen{Gelfand transform} maps $a\in \algebra$ to a function $\hat{a}$ on $\functionals$ by
%        \[
%        \hat{a}(\delta) \coloneqq \delta(a), \quad \delta \in \functionals.
%        \]
%        
%        \pause
%        \item The \boldgreen{Gelfand topology} is the weakest topology on $\functionals$ in which all $\hat{a}$ are continuous. This makes $\functionals$ compact.
%        
%        \pause
%        \item $\functionals$ with this topology is called the \boldgreen{spectrum} $\spec \algebra$.
%    \end{itemize}
%\vfill
%\end{frame}
%
%\begin{frame}{Generic Algebras}
%\vfill
%    \begin{itemize}
%        \pause
%        \item The Gelfand transform $\hat{\algebra}$ is a subalgebra of $C(\spec \algebra)$ and $a\mapsto \hat{a}$ is an isometric isomorphism.
%        
%%        \pause
%%        \item An isomorphism $t\colon A(X) \to B(Y)$ between two function algebras is \boldgreen{spatial} if there exists a bijection $b \colon X \to Y$ such that $tw=w\circ b^{-1}$.  
%        
%        \pause
%        \item For a function algebra $\algebra \subset C(X)$, take $\epsilon \colon X \to \spec \algebra$ with $\epsilon(x)=\delta_x$.
%        
%        \pause
%        \item A function algebra $\algebra \subset C(X)$ is \boldgreen{generic} if $\epsilon$ is a homeomorphism. 
%        
%        \pause
%        \item A generic algebra is (spatially) isomorphic to its Gelfand transform.
%        
%    \end{itemize}
%\vfill
%\end{frame}
%
%\begin{frame}{Lemma 2}
%\vfill
%\pause
%    The algebra of holomorphic functions $\algebra(\Omega)$ is generic.
%\vfill
%\end{frame}
%
%\begin{frame}{Importance}
%\vfill
%    \begin{itemize}       
%        \pause
%        \item $\hat{\algebra}(\partial \Omega)$ is (spatially) isomorphic to $\algebra(\Omega)$ by taking the Gelfand transform of the trace.
%        
%        \pause
%        \item The lemma shows that $\epsilon \colon \Omega \to \spec \algebra(\Omega)$ is a homeomorphism, so we have determined $\Omega$ up to homeomorphism.
%    \end{itemize}
%\vfill
%\end{frame}
%
%\begin{frame}{What's Left?}
%\vfill
%\pause
%
%    We can only have hope access to the trace algebra $\algebra(\partial \Omega)$. So we need to determine this to reach our goal.
%    \vfill
%\end{frame}

%\begin{frame}{Commutative Banach Algebras (CBA)}
%\vfill
%    \begin{itemize}
%        \pause
%        \item An algebra $\algebra$ is a \boldgreen{commutative Banach Algebra} if 
%        \begin{itemize}
%            \item $\algebra$ is a Banach space.
%            \item $a,b \in \algebra$ satisfy $\|ab\|\leq \|a\|\|b\|$.
%        \end{itemize}
%        
%        \pause
%        \item $\algebra(\Omega) \subset C(\Omega)$.
%        
%        \pause
%        \item $C(\Omega)$ is a CBA and $\algebra(\Omega)$ is a subalgebra.
%        
%        \pause
%        \item $C(\Omega)$ is a \boldgreen{uniform} algebra so $\|a^2\|=\|a\|^2$.
%    \end{itemize}
%\vfill
%\end{frame}
%
%\begin{frame}{Ideals and Functionals}
%\vfill
%    \begin{itemize}
%        \pause
%        \item A subspace $I\neq \algebra$ is an \boldgreen{ideal} if $ja \in I$ for $j\in I$ and $a\in \algebra$.
%        
%        \pause
%        \item An ideal is \boldgreen{maximal} if $\tilde{I}\subset \algebra$ and $I\subset \tilde{I}$ implies $I=\tilde{I}$.
%        
%        \pause 
%        \item A functional $\delta \in \algebra'$ is \boldgreen{multiplicative} if $\delta(ab)=\delta(a)\delta(b)$. 
%        
%        \pause
%        \item Ex: Dirac measure since $\delta_x(ab)=a(x)b(x)$.  
%    \end{itemize}
%\vfill
%\end{frame}

%\begin{frame}{Ideals and Functionals}
%\vfill
%    \begin{itemize}
%        \pause
%        \item Let $\ideals$ be the set of maximal ideals of $\algebra$.
%        
%        \pause
%        \item Let $\functionals$ be the set of multiplicative functions on $\algebra$.
%        
%        \pause
%        \item These sets are in bijection. 
%        
%        \begin{itemize}
%            \item If $\delta \in \functionals$ then $I_\delta \coloneqq \ker \delta \in \ideals$.  
%            \item If $I\in \ideals$ then $\delta_I \colon \algebra \to \algebra / I = \C$ is in $\functionals$.
%        \end{itemize}
%    \end{itemize}
%\vfill
%\end{frame}






%\subsection{Lemma 3}
%
%\begin{frame}{Trace Algebra}
%\vfill
%    \begin{itemize}
%        \pause
%        \item The trace algebra $\algebra (\partial \Omega) \coloneqq \iota^* \algebra(\Omega)$ is isometrically isomorphic to $\algebra(\Omega)$ since
%        \[
%        \|w\|_{\algebra(\Omega)} = \|\iota^* w \|_{\algebra(\partial \Omega)}
%        \]
%        and since a holomorphic function is uniquely determined by its boundary values.
%    \end{itemize}
%\vfill
%\end{frame}
%
%\begin{frame}{Lemma 3}
%\vfill
%    We have the representation
%    \[
%    \algebra(\partial \Omega) = \mathrm{clos}_{C(\partial \Omega)} \{ f +i h\},
%    \]
%    where $h$ is conjugate to $f$ by $\hilbert$.
%\vfill
%\end{frame}
%
%\subsection{Proof of the Main Theorem}
%
%
%\begin{frame}{Construction of the Manifold}
%\vfill
%\pause
%Following these steps yields a manifold $(\Omega, g)$ with the DN map $\Lambda$.
%    \begin{itemize}
%        \pause
%        \item \underline{Step 1:} We know $g|_{\partial \Omega}$ by Lee and Uhlmann, and thus we know $\hilbert$ and $C^\infty(\partial \Omega)$.  This allows us to recover the trace algebra $\algebra(\partial \Omega)$ using the representation in Lemma 3.
%        
%        \pause
%        \item \underline{Step 2:} Then find $\spec \algebra(\partial \Omega)=\Omega$ by Lemma 2.
%        
%        \pause 
%        \item \underline{Step 3:} Next, the Gelfand transform $\hat{\algebra}(\partial \Omega)=\algebra(\Omega)$ by Lemma 2.
%        
%        \pause
%        \item \underline{Step 4:} $\algebra(\Omega)$ gives us the complex structure on $\Omega$ by Lemma 1.
%        
%        \pause
%        \item \underline{Step 5:} Equip $\Omega$ with a metric $g$ conforming to this complex structure.
%    \end{itemize}
%\vfill
%\end{frame}
%
%
%
%
%\section{Generalizing This Method}
%
%\begin{frame}{First Issue}
%\vfill
%\pause
%    \begin{itemize}
%        \item No complex structure in higher dimensions.
%    \end{itemize}
%    \vfill
%\end{frame}
%
%\begin{frame}{Resolution}
%\vfill
%\pause
%    \begin{itemize}
%        \item Use a Clifford algebra/calculus structure to replace $\C$.
%        
%        \pause
%        \item The tools of Clifford analysis allow us to recover a notion of holomorphicity known as \boldgreen{monogenicity}.
%        
%        \pause
%        \item We can recover a similar algebra (Hardy space) of monogenic functions.
%    \end{itemize}
%\vfill    
%\end{frame}
%
%\begin{frame}{Clifford Structure}
%\vfill
%    A Clifford algebra builds upon the exterior algebra of forms.  Given a quadratic space $(V,Q)$, we have
%    \pause
%    \begin{itemize}
%        \item The quotient of the tensor algebra
%        \[
%        \clifford(V,Q) =\quotient{\bigoplus_{j=0}^\infty V^{\otimes j}}{\langle v \otimes v - Q(v)\rangle}.
%        \]
%        
%        \pause
%        \item If $Q=g$ is an inner product, this yields a geometric product on vectors $u,v \in \clifford(V,g)$
%        \[
%        uv = g(u,v) + u\wedge v.
%        \]
%        
%        \pause
%        \item Geometric product can be extended to multivectors.
%    \end{itemize}   
%\vfill
%\end{frame}
%
%\begin{frame}{Familiar Examples}
%\vfill
%\pause
%We have all seen Clifford algebras before.  Indeed, 
%\pause
%\begin{itemize}
%    \item $\C \cong \clifford(\R,-\cdot)$.
%    
%    \pause
%    \item $\C$ also lives inside of $\clifford(\R^2,\cdot)$ as the even subalgebra. That is,
%    \[
%    x+iy \iff 1x + e_1 \wedge e_2 y,
%    \]
%    where $e_1\wedge e_2$ is the bivector (and psuedoscalar).  
%    \pause
%    \item $\mathbb{H}$ lives inside of $\clifford(\R^3,\cdot)$ as the even subalgebra.
%\end{itemize}
%\vfill
%\end{frame} 
%
%\begin{frame}{Clifford Analysis}
%\vfill
%\begin{itemize}
%\pause
%
%    \item We can replace $d+\delta$ with the Dirac operator $D$
%    \[
%    D = \sum_{j=1}^n e_j \frac{\partial}{\partial x^j}.
%    \]
%    
%    \pause
%    \item $D^2 = \Delta$.
%    
%    \pause
%    \item Elements in the kernel of $D$ are monogenic.
%    
%    \pause
%    \item Monogenic objects in even subalgebra have components that are harmonic.
%    
%    \pause
%    \item There are Cauchy integral and Hilbert transform type operators for $D$ in arbitrary dimension.
%    \end{itemize}
%    \vfill
%\end{frame}
%
%\begin{frame}{Second Issue}
%    In dimensions greater than 2, even subalgebra is noncommutative.
%    \begin{itemize}
%        \pause
%        \item The spectral theory for Belishev's solution required a commutative Banach algebra.
%        
%        \pause
%        \item The spectral theory for noncommutative Banach algebras is not as developed.
%        
%        \pause
%        \item There is still work to do here to find some way around this.
%    \end{itemize}
%\end{frame}
%
%\section{Conclusion}
%
%\begin{frame}{Main Story}
%    \begin{itemize}
%        \pause
%        \item Calder\'on proposed a useful and challenging problem for both theorists and practicioners.
%        
%        \pause
%        \item Advances have been made in both areas.
%        
%        \pause
%        \item Ideal results are still not yet obtained.
%    \end{itemize}
%\end{frame}
%
%\begin{frame}{Novel Approach}
%    \begin{itemize}
%        \pause
%        \item Belishev solved the 2D problem using the boundary control method.
%        
%        \pause
%        \item It relies heavily on complex analysis and the spectral theory for commutative Banach algebras.
%        
%        \pause
%        \item These issues remain if we try to naively generalize this approach.
%    \end{itemize}
%\end{frame}
%
%\begin{frame}{New Input}
%    \begin{itemize}
%        \pause
%        \item Clifford algebras and analysis replace the complex structure in arbitrary dimension.
%        
%        \pause
%        \item We can still construct the same algebra of holomorphic functions.
%        
%        \pause
%        \item The relevant algebras are noncommutative for dimensions $\geq$ 3.
%        
%        \pause
%        \item There are possibly other tools at our disposal that may be able to replace the loss of commutivity.
%    \end{itemize}
%\end{frame}
%
%\begin{frame}{}
%    \vfill
%    \centering \huge{Thank you!}
%    \vfill
%\end{frame}


%\section*{Bibliography}
%
%\begin{frame}{References}
%    \begin{itemize}
%        \item \url{http://www.numdam.org/article/SLSEDP_2012-2013____A13_0.pdf}
%        \item Dirichlet to Neumann operator on differential forms
%        \item 2d bc method
%        \item The HIlbert Transform on a smooth closed hypersurface
%    \end{itemize}
%\end{frame}

\end{document}

