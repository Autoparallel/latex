\todo{If we do E\&M over curved spacetime manifolds, can we think of resistance as curvature in spatial only? And impedence would include the spatial direction??}

Spacetime and electromagnetism are intimately coupled structures. Many derivations of the fundamental laws of electromagnetic theory are available to us \cite{delphenich_axioms_2005}. 

\subsection{Electromagnetism}

Consider the spacetime algebra $\G_{1,3}$ seen in \cref{subsubsec:motivating_example} with the spacetime multivector fields on $M$ given by the space $\G_{1,3}(M)$. Locally, we can take the basis vector fields $\spacetime_0,\spacetime_1,\spacetime_2,\spacetime_3$ and we take the spatial pseudoscalar 
\begin{equation}
\blade{I}_S = \spacetime_1 \wedge \spacetime_2 \wedge \spacetime_3.
\end{equation}
Next, we have the \emph{spacetime gradient} given by
\begin{equation}
\gradst = \spacetime^0 \nabla_{\spacetime_0} + \sum_{j=1}^3 \spacetime^j \nabla_{\spacetime_j} = \spacetime^0 \nabla_{\spacetime_0} + \grad,
\end{equation}
where we are taking $\grad = \projection_{\blade{I}_S}(\gradst)$ as the spatial gradient. Note that the spatial gradient is simply the gradient in $\G_3(M)$ whereas $\gradst$ is the gradient in $\G_{1,3}$. We simply impose the naming scheme to refer to the components later on.

Next, we can consider the 4-vector potential $A=\phi \spacetime_0 + A_1 \spacetime_1 + A_2 \spacetime_2 + A_3 \spacetime_3$. The spatial component is then $A_S = \projection_{\blade{I}_S}(A)$ and the temporal component is $\projection_{\spacetime_0}(A) = \phi \spacetime_0$. Then, we have the following grade decomposition of $\gradst A$
\begin{align}
    \proj{0}{\gradst A} &= \gradst \rfloor A= \nabla_{\spacetime_0} \phi + \grad \cdot A_S\\
    \proj{2}{\gradst A} &= \gradst \wedge A = \spacetime_0 (-\grad \wedge \phi + \nabla_{\spacetime_0} A_s) + \grad \wedge A_S.
\end{align}
Taking $\proj{0}{\gradst A} = 0$ is equivalent to the Lorenz gauge condition $\partial_\mu A^\mu = 0$. 



\[
\gradst^2 A = J,
\]
where Let $\current = J_1 e_1 + J_2 e_2 + J_3 e_3$ and if we take a static four current $\nabla_{e_t} J=0$ we must have $\nabla_{e_t} A =0$ and we arrive at two equations
\[
\grad \cdot \grad \wedge u e_t = \rho e_t \qquad \textrm{and} \qquad \grad \cdot \grad \wedge \vectorpotential = \current, 
\]
of course one can take $\Delta u = \rho$, but we this equation arises from the spacetime formulation itself. Note that we did not force an inner product on the spatial vectors $e_1,e_2,e_3$ other than they are orthogonal with the temporal vector $e_t$.  These equations we have are the invariant forms of the equations with respect to any (positive definite) spatial inner product. This will be important momentarily.

In this, we have realized the electric and magnetic fields
\[
\grad \wedge u e_t = e \qquad \textrm{and} \qquad \grad \wedge \vectorpotential = b,
\]
and note the electric field $e$ is a spacetime bivector and the magnetic field $b$ is a purely spatial bivector that $\grad \wedge e = 0$ and $\grad \wedge b =0$ are satisfied. The fact that $e$ is a spacetime bivector means it behaves like a spacelike vector when acted on by spatial gradient $\grad$ owing to the static Faraday's law $\grad \times E =0$. Since $b$ is purely spatial, we see $\grad \wedge b = 0$ mimics the Gauss's law for magnetism if we take the unit spacelike trivector $I$ and let $B=bI$ be the magnetic vector field we have $\grad \cdot B = 0$.



\subsection{Biot Savart Law}

Recall the Biot Savart law from electromagnetic theory
\[
\vec{B}(y)=\frac{1}{4\pi} \int_\Omega \current \times \frac{y-x}{|y-x|^3} d\Omega(x),
\]
which satisfies 
\[
\grad \times \vec{B} = \current + \frac{1}{4\pi} \grad \wedge \int_\Omega \frac{\grad \cdot \current}{|y-x|}d\Omega(x) -\frac{1}{4\pi} \grad \wedge \int_\Sigma \frac{\current \cdot \nu}{|y-x|}d\Sigma(x)
\]
In the EIT problem we do not allow charges to accumulate in the interior and so we must have
\[
\grad \cdot \current = 0,
\]
so long as $\grad \cdot \current$ is continuous \cite{feldman_calderproblem_nodate}. Hence we are left with
\[
\grad \times \vec{B} = \current -\frac{1}{4\pi} \grad \wedge \int_\Sigma \frac{\Lambda(\phi)\\}{|y-x|}d\Sigma(x),
\]
where $\Lambda$ is the DN map.

\begin{remark}
    It seems like this now says that $u$ has a conjugate field $B$ if and only if
\[
\frac{1}{4\pi} \grad \wedge \int_\Sigma \frac{\Lambda(\phi)\\}{|y-x|}d\Sigma(x) = 0.
\]
\end{remark}
Assuming we can swap differentiation and integration we have
\begin{align*}
\grad \wedge \int_\Sigma \frac{\Lambda(\phi)}{|y-x|} d\Sigma(x) &= \int_\Sigma \frac{\Lambda(\phi)(y-x)}{|y-x|^3}d\Sigma(x),
\end{align*}
since $\grad \wedge \Lambda = 0$ \textcolor{red}{In B.V. DN-Forms}.

\begin{remark}
Perhaps we can just rearrange to see:
\[
\cauchy [\current] = \frac{1}{4\pi} \int_\Sigma \frac{y-x}{|y-x|^3} (\nu \cdot \current + \nu \wedge \current) d\Sigma(x) 
\]
and we note $\current \cdot \nu = \Lambda(\phi)$ for which we have found
\[
 \frac{1}{4\pi} \int_\Sigma \frac{\Lambda(\phi)(y-x)}{|y-x|^3}=\grad \times \vec{B}-\current,
\]
Hence 
\[
\cauchy[\current] = \grad \times \vec{B} - \current + \frac{1}{4\pi} \int_\Sigma \frac{y-x}{|y-x|^3} \nu \wedge \current d \Sigma(x)
\]
\end{remark}

In terms of geometric algebra, we wish to show the analogous statement for the magnetic bivector field
\[
B(y) = \frac{1}{4\pi} \int_\Omega \current \times \frac{y-x}{|y-x|^3} d\Omega(x),
\] 
in that
\[
\grad \cdot \frac{1}{4\pi} \int_\Omega \current \wedge \frac{y-x}{|y-x|^3} d\Omega(x) = \current.
\]



\subsubsection{Discussion}

The scalar potential in the EIT problem arises inside of a four vector potential for the electromagnetic field.  The electromagnetic potential satisfies Maxwell's equations which can be succinctly stated as $\grad_{st}^2 A = J$, for the four current $J$.  When the four current $J$ does not depend on time, we arrive at the static equations where the electrostatic potential $u$ and magnetic spatial vector potential $\vectorpotential$ are split into separate equations. Removal of time dependence decouples these potentials. We realize the magnetic field as the bivector $b=\grad \wedge \vectorpotential$ and the electric vector field $E=\grad \wedge u$. 

These fields interact with materials which carry an intrinsic inner product related to the conductivity by \ref{eq:conductivity_metric}. If the material is ohmic, we have Ohm's law given by $\grad \wedge u = \grad \cdot b$ which leads to the parabivector field $f=u+b$ to be monogenic. This relationship is important and is not fully realized without the proper treatment of the electromagnetic potential. 

In an electrostatic boundary value problem, one can supply the scalar potential $\phi$ on the boundary of a region. This forced scalar potential induces the scalar potential inside of the region and the scalar potential is harmonic when the interior is free of charges.  This scalar potential drives a current $\current$ via Ohm's law, and this current is related to the magnetic bivector field $b$. One may only have access to the boundary of the region and can make measurements of the resulting current flux $\projection{\nu}{\current}$ that corresponds to a given input scalar potential $\phi$. Is this enough to determine the underlying inner product of the region?


\subsection{Generalization}
\todo[inline]{Explain how we can put $\gamma$ as a spatial metric and incorporate this into the geometric algebra for $\G_{1,n}(\Omega)$ stuff. Contract away time part again and we get the same equations.}
What we have seen for the electromagnetic field is there is a coupling between the electric bivector field and the magnetic bivector field via the four vector potential.  This can be generalized to fields in $\G_{1,n}(\Omega)$ to produce analogous equations.