\subsection{$\G_n$-spectrum}

We turn our focus to the geometric content of the algebras $\algebra{B}(M)$ of monogenic subsurface spinor fields. Take the case where the domain $\disk$ be the unit disk in $\C \cong \R^2$.  By Gelfand, the maximal ideal space of the commutative Banach algebra $\algebra{B}(\disk)$ is homeomorphic to the disk since the algebra $\algebra{B}(\disk)$ is exactly the algebra of holomorphic functions in $\disk$. Naively attempting to generalize this notion leads on to consider the maximal ideal space of $\monogenicfields{}$ or $\monogenicfields{+}$, but no such maximal ideal space can be determined. Instead, one can note that maximal ideals of a commutative Banach algebra $\mathcal{A}$ correspond to the algebra morphisms $\mathcal{A} \to \C$.  Using this as our guiding intuition, we carry on and describe the relevant morphisms of the monogenic fields.

\begin{definition}
    Define the \emph{spinor dual} $\dualmonogenics(M)$ as the continuous right $\G_n^+$-module homomorphisms
    \begin{equation}
        \dualmonogenics(M) \coloneqq \{ l \colon \monogenics^+(M) \to \G_n^+ ~\vert~ l(fs+g) = l(f)s+l(g), ~\forall f,g \in \monogenics(M), ~r,s \in \G_n^+ \},
    \end{equation}
    and refer to elements of $\dualmonogenics(M)$ are \emph{spin functionals}. \colin{Maybe a better name would be spinor valued currents}
    \end{definition}
Similarly, we will now define the spinor functionals that are multiplicative, and therefore constitute algebra morphisms, on the monogenic subsurface spinor fields. In other words, spin characters are simply algebra homomorphisms from $\algebra{\bivector}(M)$ to the $\R$-algebra of constant surface spinors (i.e., the algebra generated by 1 and $\bivector$) which we refer to as $\mathbb{A}_{\bivector}$. One can think of these as morphisms into the complex algebra since $\mathbb{A}_{\bivector}$ is isomorphic to $\C$.
\begin{definition}
    The \emph{spinor spectrum} $\characters(M)$ is the set of algebra homomorphisms
    \[
        \characters(M) \coloneqq \{ \delta \in \dualmonogenics(M) ~\vert~ \delta(f) \in \mathbb{A}_{\bivector},~\delta(fg) = \delta(f)\delta(g),~ \forall f,g \in \algebra{\bivector}(M),~  \bivector \in \Grassmannian{2}{n}\},
    \]
    and we refer to the elements as \emph{spin characters}.
\end{definition}
One choice of spin characters is point evaluation. Take $\delta(f)=f(x^\delta)$ for some $x^\delta \in M$. We find that these characters exhaust $\characters(M)$. Along with this, we define \emph{weak-$\ast$} topology on $\characters(M)$, which is defined to be the coarsest topology so that every $x \in M$ corresponds to a continuous map on $\dualmonogenics(M)$. 

\subsection{Topology from monogenics}

We seek to determine that the space $\characters(M)$ is homeomorphic to $M$ in the case that $M$ is $n$-dimensional, smooth, oriented, imbedded manifold in $\R^n$ inheriting the Euclidean metric. 

\begin{theorem}
\label{thm:gelfand}
For any $\delta \in \characters(M)$, there is a point $x^\delta \in M$ such that $\delta(f) = f(x^\delta)$ for any $f\in \monogenics(M)$ a monogenic field. Given the weak-$\ast$ topology on $\dualmonogenics(M)$, the map
\[
\gamma \colon \characters(M) \to M, \quad \delta \mapsto x^\delta
\]
is a homeomorphism. 
%The Gelfand transform 
%\[
%\widehat{~} \colon \monogenics(M) \to C(\characters(M); \G_n), \quad \widehat{f}(\delta) \coloneqq \delta(f), \quad \delta \in \characters(M),
%\]
%is an isometry onto its image, so that $\characters(M)$ is isomorphic to $\widehat{\monogenics(M)}$ as algebras.
\end{theorem}

We prove this theorem in three main components. First, we can realize a power series representation for elements in a ball $\ball$ and denote this set as $\monogenics^\mathcal{P}(\ball)$ which is dense in $\monogenics(\ball)$. This power series is constructed using specific monogenic subsurface spinor fields. Finally, we constructively show a correspondence between $\delta \in \characters(\ball)$ with $x^\delta \in \ball$. Then, we can take a cover $M$ generated by unions of balls to complete the proof.

\subsubsection{Taylor series}
%\url{https://math.stackexchange.com/questions/811248/wedge-product-between-nonorthogonal-basis-and-its-reciprocal-basis-in-geometric}
Fix an orthonormal basis $\blade{e}_1,\dots,\blade{e}_n$ for $\R^n$ and define the functions $z_{ij} = x_j - x_i \blade{e}_i \blade{e}_j$. Recall that for an orthonormal basis the reciprocal basis elements $\blade{e}^i=\blade{e}_i$. To further condense notation, we define $\bivector_{ij}\coloneqq \blade{e}_i \blade{e}_j$ for $i\neq j$ as necessary. The functions $z_{ij}$ are the analogs to $z$ in $\C$ but specifically in the $\bivector_{ij}$ plane. We then note
\begin{equation}
z_{ij} = x_j - x_i \bivector_{ij} = \blade{e}_j\projection_{\bivector_{ij}}(x),
\end{equation}
for $x=(x_1,\dots,x_n)\in \R^n$. Hence, this is but a special case of \cref{eq:z} and we note that each $z_{ij}$ is monogenic and hence belongs in $\algebra{\bivector_{ij}}$ so long as $i \neq j$. One can quickly verify $z_{ii}$ is not monogenic. These functions find their use in a power series representation for monogenic fields $f$. Specifically, let $\sigma$ be a permutation of the set $\{2,3,\dots,n\}$, then we have the polynomials
\begin{equation}
        p_{j_2 \dots j_n}(x) = \frac{1}{j!} \sum_{\textrm{permutations}}z_{1\sigma(1)}(x) \cdots z_{1\sigma(j)}(x),
\end{equation}
each of which is monogenic, linearly independent (\cite[Proposition 1]{ryan_clifford_2004}) and formed by products of elements $z_{ij} \in \algebra{\blade{B}_{ij}}$. We put
\begin{equation}
    \monogenics^\mathcal{P}(M) = \left\{\sum_{j=0}^N \left(\sum_{\substack{{j_2 \cdots j_n} \\ {j_2 + \cdots j_n = j}}}p_{j_2 \cdots j_n}a_{j_2 \cdots j_n}\right) ~\vert~ j_2+\cdots+j_n = j,~ j\in \mathbb{N}, ~ a_{j_2\cdots j_n} \in \G_n\right\}
\end{equation}
to refer to the set of \emph{monogenic polynomials}. 

\begin{lemma}
\label{lem:density}
The space $\monogenics^\mathcal{P}(\ball_{R,w})$ is dense in $\monogenics(\ball_{R,w})$.
\end{lemma}
\begin{proof}
We can center a ball of radius $R$ at $w$ to get the monogenic polynomials $p_{j_2 \dots j_n}(x-w)$. Then, let $f\in \monogenics(\ball_{R,w})$ and define the coefficients $a_{j_2 \cdots j_n}\in \G_n^+$ by
\begin{equation}
a_{j_2 \cdots j_n} = \int_{\partial B(w,R)} \frac{\partial^j G(w,y)}{\partial y_2^{j_2} \cdots \partial y_n^{j_n}} \normal(y) f(y) \mu_\partial(y),
\end{equation}
where $G$ is the Cauchy kernel and we have used the Cauchy integral formula with the fact $\pseudoscalar$ is constant. By \cite[Theorem 4]{ryan_clifford_2004}, we have
\begin{equation}
        f(x) = \sum_{j=0}^\infty \left(\sum_{\substack{{j_2 \cdots j_n} \\ {j_2 + \cdots j_n = j}}} p_{j_2 \cdots j_n} (x-w) a_{j_2 \cdots j_n} \right),
\end{equation}
which converges pointwise to $f$ for points $x\in \ball_{R,w}$.
\end{proof}

We have found that all monogenic fields are generated as power series of homogeneous polynomials in the variables $z_{ij}$. Thus, we have a form for which we can determine the action of a $\G_n$-character on a monogenic field. Specifically, take the series for $f(x)$ above and note for $\delta \in \characters(\ball_{R,w})$ that
\begin{align}
\delta(f(x)) &= \sum_{j=0}^\infty \left(\sum_{\substack{{j_2 \cdots j_n} \\ {j_2 + \cdots j_n = j}}} \delta(p_{j_2 \cdots j_n} (x-w)) a_{j_2 \cdots j_n} \right)
\end{align}
and on each monogenic polynomial
\begin{align}
\delta(p_{j_2 \dots j_n}(x)) &= \frac{1}{j!} \sum_{\textrm{permutations}}\delta\left((z_{1\sigma(1)}(x)\right) \cdots \delta\left(z_{1\sigma(j)}(x)\right),
\end{align}
by definition. Hence, we now need to determine the action of $\delta$ on the variables $z_{ij}$.

\subsubsection{Correspondence}

The functions $z_{ij}$ played a crucial role in the above power series representation and they also play a key part in determining the behavior of the spin characters on monogenic fields  Deducing the action of $\delta(z_{ij})$ will allow us to extend this to any monogenic $f$ via \cref{lem:density}. 

\begin{lemma}
Let $\delta \in \characters(\ball_{R,w})$ and $z_{ij} \in \algebra{B_{ij}}$ then $\delta(z_{ij})=z_{ij}(x^\delta)$ for some $x^\delta \in \R^n$.
\end{lemma}
\begin{proof}
Let $\delta \in \characters(\ball_{R,w})$ and note that $\delta(1)=1$ since $\delta$ is an algebra homomorphism. Hence, let $\bivector$ be an arbitrary unit $2$-blade then 
\begin{equation}
\delta(\alpha + \beta \bivector) = \delta(\alpha) + \delta(\beta \bivector)=\alpha \delta(1) + \delta(1)\beta \bivector = \alpha + \beta \bivector.
\end{equation}
by definition. Applying $\delta$ to $z_{ij}$ yields
\begin{equation}
\delta(z_{ij}) = \alpha_{ij} + \beta_{ij} \bivector_{ij},
\end{equation}
for some constants $\alpha_{ij}$ and $\beta_{ij}$ since $\delta(\algebra{\bivector})=\mathbb{A}_{\bivector}$. Then, note that we have two key relationships
\begin{equation}
\label{eq:z_reciprocal_relationship}
z_{ij} \bivector_{ji}  = -z_{ji}
\end{equation}
\begin{equation}
\label{eq:z_relationship}
z_{ij} = z_{kj} + z_{ik} \bivector_{kj}.
\end{equation}

Applying $\delta$ to \cref{eq:z_reciprocal_relationship}
\begin{equation}
\delta(z_{ij} \bivector_{ji}) = \delta(z_{ij}) \bivector_{ji} = -\delta(z_{ji})
\end{equation}
yields
\begin{equation}
(\alpha_{ij} + \beta_{ij} \bivector_{ij})\bivector_{ji} = \beta_{ij}+\alpha_{ij}\bivector_{ji} = - \alpha_{ji} - \beta_{ji} \bivector_{ji}.
\end{equation}
Hence, $\alpha_{ij} = -\beta_{ji}$ for all $i \neq j$.

Applying $\delta$ to \cref{eq:z_relationship}
\begin{equation}
\delta(z_{ij}) = \delta(z_{kj} + z_{ik} \bivector_{kj}) = \delta(z_{kj})+\delta(z_{ik})\bivector_{kj}
\end{equation}
yields
\begin{equation}
a_{ij} + b_{ij} \bivector_{ij} = \alpha_{kj} + \beta_{kj} \bivector_{kj} + (\alpha_{ik} + \beta_{ik} \bivector_{ik})B_{kj} = \alpha_{kj} + \beta_{ik} B_{ij} + (\alpha_{ik} + \beta_{kj})\bivector_{kj}
\end{equation}
yields the relationships $\alpha_{ij} = \alpha_{kj}$, $\beta_{ij} = \beta_{ik}$, and $\alpha_{ik}=-\beta_{kj}$.

These relationships allow us to achieve our proof. Briefly, picture $\alpha_{ij}$ and $\beta_{ij}$ as components of the $n \times n$ matrices $\alpha$ and $\beta$ where we index row by column. Note that $\alpha$ and $\beta$ both have zero diagonal since the functions $z_i^i$ are not monogenic. The relationship $\alpha_{ji} = -\beta_{ij}$ for $i\neq j$ then shows that $\alpha = -\beta^\top$.  Then we have $\alpha_{ij} = \alpha_{kj}$ for $i\neq j \neq k$ shows that $\alpha$ is constant along rows and hence $\beta$ is constant along columns. This is consistent with $\alpha = -\beta^\top$, $\beta_j^i = \beta_k^i$, and the final relationship $\alpha_{ik} = -\beta_{kj}$. The matrices $\alpha$ and $\beta$ are thus uniquely determined by $n$ numbers.  Moreover, treating $\delta(z_{ij})=z_{ij}(x_\delta)$ for some $x^\delta \in \R^n$ satisfies the relationships granted above. Thus, we simply find the $x^\delta$ such that we retrieve the desired components for $\alpha$ and $\beta$.  
\end{proof}


\begin{lemma}
Let $f\in \monogenics(\ball_{R,w})$, then $\delta(f)=f(x^\delta)$ for some $x^\delta \in \ball_{R,w}$.
\end{lemma}
\begin{proof}
To see that $x^\delta \in \ball_{R,w}$, take $G_0 \in \monogenics^+(\ball_{R,w})$ by $G_0(x)\coloneqq G(x,x_0)\blade{e}_1$ where $G$ is the Green's function in \cref{eq:greens_function} and $x_0\not\in \ball_{R,w}$. Fix $\delta \in \characters(\ball_{R,w})$, then, 
\begin{align}
\delta(G_0)=G_0(x^\delta).
\end{align}
Take a sequence $x_n \to x^\delta$ with $x_n \notin \ball_{R,w}$ and note that the sequence of functions $G_n(x)\coloneqq G(x,x_n)\blade{e}_1 \in \monogenics(\ball_{R,w})$ and the sequence converges to a monogenic function $G_\infty(x) = G(x,x^\delta)\blade{e}_1$ but
\begin{align}
\lim_{n\to \infty} \delta(G_n) = \lim_{n\to \infty} G_n(x^\delta),
\end{align}
does not converge due to the singularity at $x^\delta$. Hence, it must be that $x^\delta \in \ball_{R,w}$ by continuity of $\delta$.
\end{proof}

Take an arbitrary open cover of $M$ with balls $\ball_{R,w}$. Via this cover, we extend the lemmas to prove \cref{thm:gelfand}.
\begin{proof}
\textcolor{red}{This theorem is not yet proven, but the previous lemmas prove the theorem in the case where $M= \ball_{R,w}$.}
\end{proof}