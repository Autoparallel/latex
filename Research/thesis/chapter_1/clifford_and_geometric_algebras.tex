
The complex algebra $\C$ can be generalized in a handful of ways.  Some of which can be found through the use of Clifford algebras and, more specifically, in geometric algebras.  We define the more general Clifford algebras first and realize geometric algebras as particularly nice Clifford algebras with a quadratic form arising from an inner product. Elements of a geometric algebra are known as multivectors and these multivectors carry a wealth of geometric information in their algebraic structure. $\C$ itself can be realized as a special subalgebra of parabivectors in the geometric algebra on $\R^2$ with the Euclidean inner product and the quaternions $\quat$ are realized as an analogous algebra on $\R^3$. In particular, both $\C$ and $\quat$ arise as the 2- and 3-dimensional even Clifford groups $\Gamma^+$ respectively. \todo{reword this paragraph}

First, we present a review of Clifford algebras and the relevant notions needed for this work. Those who feel they are familiar with both Clifford and geometric algebras may wish to skim through this subsection and visit \cref{subsubsec:motivating_example} to review the notation used throughout this manuscript. 

Formally, we let $(V,Q)$ be an $n$-dimensional vector space $V$ over some field $K$ with an arbitrary quadratic form $Q$.  The tensor algebra is given by
\begin{equation}
\mathcal{T}(V) \coloneqq \bigoplus_{j=0}^\infty V^{\otimes j} = K \bigoplus V \oplus (V\otimes V) \oplus (V\otimes V \otimes V) \oplus \cdots,
\end{equation}
where the elements (tensors) inherit a multiplication $\otimes$ (the tensor product). From the tensor algebra $\mathcal{T}(V)$, we can quotient by the ideal generated by $\blade{v}\otimes \blade{v} - Q(\blade{v})$ to create a new algebra.
\begin{definition}
The \emph{Clifford algebra} $C\ell(V,Q)$ is the quotient algebra
\begin{equation}
C\ell(V,Q) = \mathcal{T}(V) ~ / ~ \langle \blade{v} \otimes \blade{v} - Q(\blade{v}) \rangle.
\end{equation}
\end{definition}
To see how the tensor product descends to the quotient, we let $\blade{v}_1, \dots, \blade{v}_n$ be an arbitrary basis for $V$, then we can consider the tensor product of basis elements $\blade{v}_i \otimes \blade{v}_j$ which induces a product in the quotient $C\ell(V,Q)$ which we refer to as the \emph{Clifford multiplication}. In this basis, we write this product as concatenation $\blade{v_i}\blade{v_j}$ and define the multiplication by
\begin{equation}
\label{eq:clifford_multiplication}
\blade{v}_i \blade{v}_j = \begin{cases} Q(\blade{v}_i) & \textrm{if $i=j$}, \\ \blade{v}_i \wedge \blade{v}_j & \textrm{if $i\neq j$},\end{cases}
\end{equation}
where $\wedge$ is the typical exterior product satisfying $\blade{v}\wedge \blade{w} = - \blade{w}\wedge \blade{v}$ for all $\blade{v},\blade{w}\in V$.  As a consequence, the exterior algebra $\bigwedge(V)$ can be realized as a subalgebra of any Clifford algebra over $V$ or as a Clifford algebra with a trivial quadratic form $Q=0$.  

In the case where $V$ has a (pseudo) inner product $g$, we can induce a quadratic form $Q$ by $Q(\blade{v})=g(\blade{v},\blade{v})$ and give rise to a special type of Clifford algebra which motivates the following definition.
\begin{definition}
Let $V$ be a vector space with an (pseudo) inner product $g(\cdot,\cdot)$. Then taking $Q(\cdot) = g(\cdot,\cdot)$, the Clifford algebra $C \ell(V,Q)$ is called a \emph{geometric algebra}.
\end{definition}
In general, we put $\G$ and assume the inner product and vector space will be arbitrary, given alongside, or will be clear from context.  For example, when $V=\R^n$ we have the standard orthonormal basis $\blade{e}_1,\dots,\blade{e}_n$ which allows us to neatly define the quadratic form $Q$ from the Euclidean inner product which has coefficients $\delta_{ij}$ with respect to this basis. Since we frequently utilize this geometric algebra, we put $\mathcal{G}_n \coloneqq C\ell(\R^n, |\cdot|)$ to simplify notation. 

Geometric algebras are an old and widely studied topic. For more information, see the classical text \cite{hestenes_clifford_1986} or the more modern text \cite{doran_geometric_2003} which also provides a wide range of applications to physics problems. Both these sources include much of the other necessary preliminaries I cover in the remainder of this section. Finally, the paper \cite{chisolm_geometric_2012} proves many of the useful identities and notation used throughout this paper.

\subsection{Multivectors and grading}
\todo{fix all vector indices to be not bold}
Note that $C\ell(V,Q)$ is a $\mathbb{Z}$-graded algebra with elements of grade-0 up to elements of grade-$n$. We refer to grade-0 elements as scalars, grade-1 elements as vectors, grade-2 elements as \emph{bivectors}, grade-$r$ elements as \emph{$r$-vectors}, and grade-$n$ elements as \emph{pseudoscalars}. For example, the pseudoscalar $\blade{\mu} = \blade{v}_1 \wedge \blade{v}_2 \wedge \cdots \wedge \blade{v}_n$ is an $n$-vector we will frequently return to. We denote the space of $r$-vectors by $C\ell(V,Q)^r$. For each grade there is a basis of ${n\choose r}$ \emph{$r$-blades} which are $r$-vectors of the form
\begin{equation}
\label{eq:blade}
\blade{A_r} = \bigwedge_{j=1}^r \blade{v}_j, ~\textrm{for linearly independent}~ \blade{v}_j \in V,
\end{equation}
and we use a boldface of both the character and its subscript to specify that a $r$-vector is a $r$-blade and we note that vectors (since they are $1$-blades) will not use this subscript. Instead, a vector $\blade{v}$ may use a non-boldfaced subscript to reference an index. Briefly, take for example the case where $\dim(V)=3$, then there are ${3\choose 2}=3$ 2-blades that form a basis for the bivectors and one particular choice of a bivector basis would be the following list of 2-blades
\begin{equation}
\label{eq:3_dim_basis}
\blade{B}_{12} = \blade{v}_1 \wedge \blade{v}_2, \quad \blade{B}_{13} = \blade{v}_1 \wedge \blade{v}_3, \quad \blade{B}_{23} = \blade{v}_2 \wedge \blade{v}_3.
\end{equation}
We will repeatedly use the notation $\blade{B}_{ij} \coloneqq \blade{v}_i\wedge \blade{v}_j$ and the underlying basis will be clear from context. We refer to an $n-1$-vector as a \emph{pseudovector} and it should be noted that every $n-1$-vector is a blade (see \cref{subsubsec:duality_and_pseudoscalars}). In other literature, some will refer to a $r$-blade as a \emph{simple} or a \emph{decomposable} $r$-vector\todo{citations}. 

In general, an element $A \in C\ell(V,Q)$ is written as a linear combination of basis elements of all possible grades and we refer to $A$ as a \emph{multivector}.  To extract the grade-$r$ components of $A$, we use the \emph{grade projection} for which we have the notation
\begin{equation}
\proj{r}{A} \in C\ell(V,Q)^r
\end{equation}
to denote the grade-$r$ components of the multivector $A$ (i.e., $\proj{r}{A} \in C\ell(V,Q)^r$). Any multivector $A$ can then be given by
\begin{equation}
A = \sum_{r=0}^n \proj{r}{A}
\end{equation}
which shows the decomposition via the $\mathbb{Z}$-grading
\begin{equation}
C\ell(V,Q) = \bigoplus_{j=0}^n C\ell(V,Q)^j.
\end{equation}
If $A$ contains only components of a single grade, then we say that $A$ is \emph{homogeneous} and if the components are grade-$r$ we put $A_r$ and refer to $A_r$ as a \emph{homogeneous $r$-vector} or simply an \emph{$r$-vector}.  For example, when we refer to vectors we realize them as 1-vectors and likewise we realize bivectors as 2-vectors. Also of interest will be the elements in
\begin{equation}
 C\ell(V,Q)^{0+2} = C\ell(V,Q)\oplus C\ell(V,Q)^2
\end{equation}
which we refer to as \emph{parabivectors}.

The Clifford multiplication of vectors defined in \ref{eq:clifford_multiplication} can be extended to multiplication of vectors with homogeneous $r$-vectors.  In particular, given a vector $\blade{v} \in C\ell(V,Q)$ and a homogeneous $r$-vector $A_r \in C\ell(V,Q)$, we have
\begin{equation}
\label{eq:vector_multiplication}
\blade{v}A_r = \proj{r-1}{\blade{v}A_r} + \proj{r+1}{\blade{v}A_r},
\end{equation}
which decomposes the multiplication into a grade lowering \emph{interior product} and a grade raising \emph{exterior product}.  This allows us to extend the Clifford multiplication further. Given an $s$-vector $B_s$, we have
\begin{equation}
\label{eq:general_clifford_multiplication}
A_r B_s = \proj{|r-s|}{A_rB_s} + \proj{|r-s|+2}{A_rB_s} + \cdots + \proj{r+s}{A_rB_s}.
\end{equation}
This rule for multiplication then allows for the multiplication of two general multivectors in $C\ell(V,Q)$. For this multiplication, specific grades of the product are worth noting.
\begin{equation}
\label{eq:dot}
    A_r \cdot B_s \coloneqq \proj{|r-s|}{A_r B_s}
\end{equation}
\begin{equation}
\label{eq:wedge}
    A_r \wedge B_s \coloneqq \proj{r+s}{A_r B_s}
\end{equation}
\begin{equation}
\label{eq:left_contraction}
    A_r \rfloor B_s \coloneqq \proj{s-r}{A_r B_s}
\end{equation}
\begin{equation}
\label{eq:right_contraction}
    A_r \lfloor B_s \coloneqq \proj{r-s}{A_r B_s}.
\end{equation}
Finally, we have a special product for bivectors called the \emph{commutator product} given by
\begin{equation}
\label{eq:commutator_product}
    A_2 \times B_2 \coloneqq \proj{2}{A_2 B_2} \equiv \frac{1}{2} (A_2 B_2 - B_2 A_2).
\end{equation}
These products are particularly emphasized as many helpful identities used in this paper are phrased using these notions. Taking \cref{eq:vector_multiplication,eq:wedge,eq:left_contraction} into mind, we see that the grade lowering interior product can be written as
\begin{equation}
    \proj{r-1}{\blade{v}A_r} \equiv \blade{v}\rfloor A_r \equiv \blade{v} \cdot A_r
\end{equation}
and the grade raising exterior product can be written as
\begin{equation}
    \proj{r+1}{\blade{v}A_r} \equiv \blade{v} \wedge A_r.
\end{equation}
Finally, to suppress needless additional parentheses later on, we assert that the above products take precedence over the geometrical product in order of operation. For example, for multivectors $A$, $B$, and $C$, we must take
\begin{equation}
A\cdot B C \equiv (A \cdot B)C,
\end{equation}
and extend this to the other products defined in \cref{eq:wedge,eq:left_contraction,eq:right_contraction,eq:commutator_product} as well. 

As discussed, $C\ell(V,Q)$ is naturally a $\mathbb{Z}$-graded algebra but we also find that it carries a $\mathbb{Z}/2\mathbb{Z}$-grading as well. Some would then refer to $C\ell(V,Q)$ as an \emph{superalgebra} \todo{source}. This additional grading can be realized by sorting $r$-vectors in $C\ell(V,Q)$ into the sets where $r$ is even or odd.  We say a $r$-vector is \emph{even} (resp. \emph{odd}) if $r$ is even (resp. odd) and in general if a multivector $A$ is a sum of only even (resp. odd) grade elements we also refer to $A$ as even (resp. odd).  Taking note of the multiplication defined in \ref{eq:general_clifford_multiplication}, one can see that the multiplication of even multivectors with another even multivectors outputs an even multivector and that motivates the following.
\begin{definition}
The \emph{even subalgebra} $C\ell(V,Q)^+ \subset C\ell(V,Q)$ is the subalgebra of even grade multivectors
\begin{equation}
    C\ell(V,Q)^+ \coloneqq C\ell(V,Q)^0 \oplus C\ell(V,Q)^2 \oplus C\ell(V,Q)^4 \oplus \cdots.
\end{equation}
\end{definition}
The split between even and odd subspaces of $C\ell(V,Q)$ makes the space $C\ell(V,Q)$ into a \emph{superalgebra}. Though, one should note that the space of odd grade multivectors, $C\ell(V,Q)^-$, is not an algebra in its own right, it is a $C\ell(V,Q)^+$-module. We can then take the even part of a multivector $A$ by $\proj{+}{A}$ and the odd part by $\proj{-}{A}$ and note
\begin{equation}
A = \proj{+}{A} + \proj{-}{A}.
\end{equation}
In the same vein, we will denote an even multivector by $A_+$ and an odd multivector by $A_-$. The even subalgebra is an extremely important entity that arises throughout physics due to its encapsulation of spinors which we touch on next. 

\subsection{Multivector operations and the Clifford and spin groups}
\label{subsubsec:reverse_inverse_clifford_spin_groups}
For the remainder of this paper, let us focus solely on geometric algebras $\G$. Given access to an (pseudo) inner product we have a natural isomorphism between $V$ and $V^*$ by the Riesz representation.  Namely, given an arbitrary basis $\blade{v}_i$ for $V$ there exists the corresponding dual basis $f_i$ for $V^*$ such that $f_i(\blade{v}_j)=\delta_{ij}$. In geometric algebra, this notion is somewhat superfluous as we can realize the dual basis inside $V$ itself in the following manner. Note that there is a unique map $\sharp \colon V^* \to V$ for which $f\mapsto \blade{f^\sharp}$ such that
\begin{equation}
\blade{f_i^\sharp} \cdot \blade{v}_j = \delta_{ij}.
\end{equation}
Hence, if we simply put $\blade{v}^i \coloneqq \blade{f^\sharp}_i$ we can note that $\blade{v}^i$ is simply a vector in the geometric algebra.
\begin{definition}
Let $\blade{v}_1,\blade{v}_2,\dots,\blade{v}_n$ be an arbitrary basis of $V$ generating $\G$. Then we have the \emph{reciprocal basis} $\blade{v}^1,\blade{v}^2,\dots,\blade{v}^n$ satisfying
\begin{equation}
    \blade{v}^i\cdot \blade{v}_j = \delta^i_j,
\end{equation}
and we refer to each $\blade{v}^i$ as a \emph{reciprocal vector}.
\end{definition}
In terms of the inner product $g$, we have that the coefficients are given by $g_{ij}=\blade{v}_i\cdot \blade{v}_j$ and thus we have an explicit definition for the reciprocal vectors by putting $\blade{v}^i = g^{ij}\blade{v}_j$ where $g^{ij}$ is the coefficients to the matrix inverse $(g_{ij})^{-1}$ and we assume the Einstein summation convention. 

The inverse to this isomorphism is $\flat \colon V \to V^*$ which is given by $\blade{v} \mapsto v^\flat$ satisfying
\begin{equation}
v_i^\flat (\blade{v}_j)= \delta_{ij}.
\end{equation}
Given these identifications, there is no need to distinguish between the vector space $V$ and its dual $V^*$ as it suffices to consider $V$ itself with reciprocal vectors $\blade{v^i}$ with the application of the scalar product. For reference, the maps $\sharp$ and $\flat$ are the \emph{musical isomorphisms} \todo{sources}.

For a geometric algebra with a positive definite inner product, all blades have an inverse and hence form a group. With a pseudo inner product, the invertible elements are not quite as broad\todo{give an example later}. To this end, we can construct a group of all invertible elements referred to as the \emph{Clifford group} $\Gamma(\G)$ for an arbitrary geometric algebra $\G$ by
\begin{equation}
\Gamma(\G) \coloneqq \left\{ \prod_{j=1}^k \blade{v}_j ~\vert~ k\in \mathbb{Z}^+,~ \forall j~\colon1\leq j \leq k~\colon~\blade{v}_i \in V ~\textrm{such that}~ g(\blade{v}_i,\blade{v}_i)\neq 0\right\}.
\end{equation}
We refer to elements of the Clifford group as \emph{Clifford multivectors}. Note that Clifford multivectors are not necessarily blades since the product used in the construction is not the exterior product $\wedge$. For any Clifford multivectors $A = \blade{v}_1 \cdots \blade{v}_k$ in the group $\Gamma$, we have that multiplicative inverse $A^{-1}$ is given by $A^{-1} = \blade{v}^k \dots \blade{v}^1$ as we can see that $A^{-1}A=AA^{-1} = 1$ by construction.  Another note is that all scalars, vectors, pseudovectors, and pseudoscalars are always in the Clifford group and have multiplicative inverses. The inverse of a vector $\blade{v}$ is given by $\frac{\blade{v}}{\blade{v}\cdot\blade{v}}$. The form of the inverse motivates the utility of the \emph{reverse} operator $\dagger$ defined so that $A^\dagger = \blade{v}_k \cdots \blade{v}_1$. For a $r$-blade $A_r$, the reverse also satisfies the relationship
\begin{equation}
\label{eq:reverse_sign}
A_r^\dagger = (-1)^{r(r-1)/2} A_r.
\end{equation}
One can see that the multiplicative inverse of an element of the Clifford group $A$ is the reverse of the corresponding product of reciprocal vectors since $A_r^{-1} = (\blade{v}^1 \cdots \blade{v}^k)^\dagger$. When we take $V=\R^n$ with the Euclidean inner product, we can note that elements $s \in \Gamma^+(\G_n)$ act as rotations on multivectors $A\in \G_n$ through a conjugate action
\begin{equation}
A \mapsto s A s^{-1}.
\end{equation}
In fact, all nonzero vectors $\blade{v}\in\Gamma(\G_n)$ define a reflection in the hyperplane perpendicular to $\blade{v}$ via the same conjugation action above. This allows one can realize that all rotations are even products of reflections.

Following these realizations, one can see that the Clifford group $\Gamma(\G_n)$ contains important subgroups such as the orthogonal and special orthogonal groups as quotients
\begin{equation}
\operatorname{O}(n) \cong \Gamma(\G_n)/(\R\setminus 0) \qquad \textrm{and} \qquad \operatorname{SO}(n) \cong \Gamma^+(\G_n) /(\R\setminus 0),
\end{equation}
where $\R\setminus 0$ is the multiplicative group of real numbers. This motivates the following definition.
\begin{definition}
    The \emph{Clifford norm} $\| \cdot \|$ for $s \in \Gamma(\G)$ is given by
    \begin{equation}
    \|s\|^2 \coloneqq ss^\dagger.
    \end{equation}
\end{definition}
Note that when the inner product is positive definite the Clifford norm is indeed a norm \todo{source} but can fail to be a norm in spaces with mixed signature (see \cref{eq:spacetime_inner_product}). Also, note that for vectors the Clifford norm corresponds with the norm induced from the inner product in that with a vector $\blade{v}$ we have $\|\blade{v}\|=\blade{v}\blade{v}^\dagger = \blade{v}\cdot \blade{v}$. We also give the name \emph{unit} to $r$-blades $\blade{A_r}$ with unit Clifford norm $1=\|\blade{A_r}\|$. Finally, this allows us to arrive at a definition for the classical pin and spin groups.
\begin{definition}
\begin{subequations}
The \emph{pin} and \emph{spin} groups $\operatorname{Pin}(V)$ and $\operatorname{Spin}(V)$ are defined to be
\begin{align}
    \operatorname{Pin}(V) &\coloneqq \{s\in \Gamma(\G) ~\vert~ \|s\|=1\}.\\
    \operatorname{Spin}(V) &\coloneqq \{s\in \Gamma^+(\G) ~\vert~ \|s\|=1\}.
\end{align}
\end{subequations}
\end{definition}

Our focus will be the case where we take $\G=\G_n$ for which we put $\spingroup$, but these statements can often be more broadly generalized. Moreover, we can realize this group as a quotient of the Clifford group $\Gamma(\G_n)$ by
\begin{equation}
\operatorname{Spin}(n) \cong \Gamma^+(\G_n)/\R_+,
\end{equation}
where $\R_+$ is the multiplicative group of positive real numbers. The spin group $\operatorname{Spin}(V)$ is a Lie group usually derived via a short exact sequence of groups
\begin{equation}
1 \to \mathbb{Z}/2\mathbb{Z} \to \operatorname{Spin}(V) \to \operatorname{SO}(V) \to 1.
\end{equation}
Here, we have given a more concrete realization of the spin group as special elements inside a geometric algebra. The Lie algebra of the spin group is denoted by $\mathfrak{spin}(V)$ and $\mathfrak{spin}(n)$ when referencing $\spingroup$. This algebra typically characterized as the tangent space of $\operatorname{Spin}(V)$ at the identity. However, through this approach, we realize that $\mathfrak{spin}(V)$ is isomorphic to the algebra of bivectors with the antisymmetric product $\times$\todo{provide a citation.}.  Then, for any bivector $B$, we can generate an element in the spin group given via the exponential
\begin{equation}
e^{B} = \sum_{j=0}^\infty \frac{B^n}{n!}.
\end{equation}
Fundamentally, the even subalgebra $\G^+$ is invariant under the action of $\operatorname{Spin}(V)$ since all elements in both sets are of even grade. This definition follows.
\begin{definition}
Let $\G$ be a geometric algebra with an inner product of arbitrary signature, then we define a \emph{spinor} to be an element of $\G^+$.
\end{definition}
Morally, this definition is telling us $\psi \in \G^+$ is an element that transforms under a left action of an element of $\operatorname{Spin}(V)$ to produce another spinor which leaves us with a convenient definition in that a spinor is simply an even multivector. For more on the topic, see \cite{janssens_special_nodate}.

\todo{spinors are really a module. The odd subspace is also a similar module. Maybe reference superalgebra and physics again a little bit.}

\subsection{Pseudoscalars and duality}
\label{subsubsec:duality_and_pseudoscalars}

Pseudoscalars are a deeply useful aspect of geometric algebra and we will now cover some of their utility. First and foremost, these pseudoscalars grant us a means of determining volumes. This will be a necessary notion in order to define integration in \cref{subsubsec:integration_on_submanifolds}.
\begin{definition}
Let $\G$ be a geometric algebra, then the \emph{volume element} in the arbitrary basis $\blade{v}_1,\dots,\blade{v}_n$ is 
\begin{equation}
\blade{\mu}=\blade{v}_1 \wedge \blade{v}_2 \wedge \cdots \wedge \blade{v}_n.
\end{equation}
\end{definition}
It is worth noting that all volume elements and pseudoscalars are invertible in any geometric algebra. 

We also want to note that the volume element here fits our intuition and indeed we find
\begin{equation}
\label{eq:pseudoscalar_norm}
\|\blade{\mu}\| = \sqrt{\det(g)}.
\end{equation}
Since pseudoscalars are generated by a single element (recall there are ${n \choose n}$ independent grade-$n$ elements), we should realize that the volume element is simply a scalar copy of a pseudoscalar that is unital.
\begin{definition}
Let $\blade{\mu}$ be the volume element, then we have the \emph{unit pseudoscalar}
\begin{equation}
\blade{I} \coloneqq \frac{1}{\|\blade{\mu}\|} \blade{\mu}.
\end{equation}
\end{definition}
As is clear by the definition above, we must have that
\begin{equation}
\|\blade{I}\| = 1.
\end{equation}
The unit pseudoscalar satisfies a useful relationship when swapping the left for right multiplication with an $r$-vector by
\begin{equation}
\blade{I} A_r = (-1)^{r(n-1)} A_r \blade{I}.
\end{equation}
Thus, $\blade{I}$ commutes with the even subalgebra, and anticommutes with the odd subalgebra. We can also see that the inverse for unit pseudoscalar is $\blade{I}^{-1}=\blade{I}^\dagger$ which is an identification that we will often use. Formulas throughout are usually given in their most general context and substitution is done only when working with specialized algebras.  

Note that for a homogeneous $r$-rector $A_r$ we have that $A_r^\perp$ is an $n-r$-vector. Indeed, if we take an invertible $r$-blade $\blade{A_r}$, then we can find the \emph{$\blade{A_r}$-subspace dual} of a multivector $B$ by
\[
B \rfloor \blade{A_r}^{-1}.
\]
The notions of duality here give us geometrical insight. Taking an $s$-blade $\blade{B_s}$ we can note:
\begin{itemize}
    \item If $s>r$, the $\blade{A_r}$-subspace dual of $\blade{B_s}$ vanishes.
    \item If $s=r$, the $\blade{A_r}$-subspace dual of $\blade{B_s}$ is a scalar and is zero if $\blade{B_s}$ contains a vector orthogonal to $\blade{A_r}$.
    \item If $s<r$, the $\blade{A_r}$-subspace dual of $\blade{B_s}$ represents the orthogonal complement of the subspace corresponding to $\blade{B_s}$ in the subspace corresponding to $\blade{A_r}$.
\end{itemize}  
Since the pseudoscalar is a blade representing the entire vector space, this allows one to create dual elements within the entire vector space. 
\begin{definition}
Given a multivector $B$, we define the \emph{dual} of $B$ to be
\begin{equation}
B^\perp \coloneqq B \rfloor \blade{I}^{-1} \equiv B\blade{I}^{-1}.
\end{equation}
\end{definition}
The dual allows one to exchange interior and exterior products in the following way.
\begin{equation}
\label{eq:wedge_to_dot}
 (A \wedge B)^\perp  = A\rfloor B^\perp
\end{equation}
\begin{equation}
\label{eq:dot_to_wedge}
    (A\rfloor B)^\perp = A \wedge B^\perp
\end{equation}
This shows the natural duality between the inner and exterior products and their interpretations as subspace operations. The duality extends further to provide an isomorphism between the spaces of $r$-vectors and $n-r$-vectors since for any $r$-vector $A_r$ we have $A_r^\perp$ is an $n-r$-vector. It is under this isomorphism one can realize that all pseudovectors are $n-1$-blades. Furthermore, for multivectors $A$ and $B$,
\begin{equation}
(AB)^\perp = AB^\perp
\end{equation}
For those familiar with the Hodge star operator, $\star$, this should feel familiar. This is discussed in \cref{subsec:differential_forms}.

\begin{remark}
\label{rem:cross_product}
If we consider $\spacealg$, we can realize the cross product of two vectors $\blade{u}$ and $\blade{v}$ by
\begin{equation}
\blade{u} \cross \blade{v} \coloneqq (\blade{u}\wedge \blade{v})^\perp
\equiv (\blade{u}^\perp)\times (\blade{v}^\perp), 
\end{equation}
where we use the bold notation for $\cross$ to distinguish between the bivector commutator product $\times$ defined in \cref{eq:commutator_product}. The special fact of $\spacealg$ that is abused in a standard multivariate calculus course is that vectors and bivectors are dual to one another. In fact, the first equality is exactly this pedagogical reasoning; the cross product returns a vector perpendicular to the subspace spanned by the two input vectors and is zero when the two inputs are linearly dependent. One can also note that the vector $\blade{w}=\blade{u}\cross \blade{v}$ is sometimes refered to as axial and in other cases the pseudovector $\blade{u}\wedge \blade{v}$ is referred to as axial. The similar product notation of $\times$ and $\cross$ now becomes transparent. 
\end{remark}


\subsection{Blades and subspaces}
\label{subsubsec:blades_and_subspaces}

Each invertible unit $r$-blade $\blade{U_r}$ ($\|\blade{U_r}\|=1$) corresponds to a $r$-dimensional subspace and can be identified with a point in the Grassmannian of $r$-dimensional subspaces in an $n$-dimensional vector space, $\Grassmannian{r}{n}$. We will often allude to this identification directly by referring to a subspace via a reference to a unit blade, e.g., the subspace $\blade{U_r}$. Extending the dual to act on the unit $r$-blades that make up $\Grassmannian{r}{n}$, one realizes that $\Grassmannian{r}{n}^\perp = \Grassmannian{n-r}{n}$ shows the spaces are in bijection. Moreover, given a subspace $\blade{U_r}$, we can complete the vector space by
\begin{equation}
\blade{U_r}\wedge \blade{U_r}^\perp = \blade{I}.
\end{equation} 
We can also note that any invertible blade $\blade{A_r}$ is simply a scaling of some unit blade so that $\blade{A_r} = \alpha \blade{U_r}$. This interpretation also proves to be a wonderfully geometrical perspective on the products defined in \cref{eq:dot,eq:wedge,eq:left_contraction,eq:right_contraction}. For example, we see that there are a handful of reasons to adopt the additional multiplication symbols $\rfloor$ and $\lfloor$ \cite{chisolm_geometric_2012}. 
\begin{itemize}
    \item The products $\rfloor$ and $\lfloor$ allow us to avoid needing to pay special attention to the specific grade of each multivector in a product. The product $\cdot$ on $A_r$ and $B_s$ depends on $k$ and $s$ and as such given by either $\rfloor$ or $\lfloor$ but one must know $k$ and $s$ in order to define this product exactly. 
    \item We gain geometrical insight on the structure of $r$-blades in terms of their corresponding subspaces. Let $\blade{A_r}$ and $\blade{B_s}$ be nonzero blades with $r,s\geq 1$ then
    \begin{itemize}
        \item $\blade{A_r} \rfloor \blade{B_s} =0$ iff $\blade{A_r}$ contains a nonzero vector orthogonal to $\blade{B_s}$.
        \item If $r<s$ then if $\blade{A_r}\rfloor \blade{B_s} \neq 0$ then the result is a $s-r$-blade representing the orthogonal complement of $\blade{A_r}$ in $\blade{B_s}$.
        \item If $\blade{A_r}$ is a subspace of $\blade{B_s}$ then $\blade{A_r}\blade{B_s} = \blade{A_r}\rfloor \blade{B_s}$.
        \item If $\blade{A_r}$ and $\blade{B_s}$ are orthogonal, then $\blade{A_r}\blade{B_s} = \blade{A_r} \wedge \blade{B_s}$.
    \end{itemize}
\end{itemize}

We also have the equivalences
\begin{equation}
\label{eq:left_contraction_dot}
A_r \cdot B_s \equiv A_r \rfloor B_s \qquad \textrm{if $k\leq s$}
\end{equation}
\begin{equation}
\label{eq:right_contraction_dot}
A_r \cdot B_s \equiv A_r \lfloor B_s \qquad \textrm{if $k\geq s$}.
\end{equation}
For homogeneous $r$-vectors $A_r$ and $B_r$, the products above simplify to 
\begin{equation}
\label{dot_equivalent_contraction}
    A_r \cdot B_r \equiv A_r \lfloor B_r \equiv A_r \rfloor B_r.
\end{equation}
In fact, if we are given two $r$-blades $\blade{A_r} = \blade{a_1} \wedge \cdots \wedge \blade{a_r}$ and $\blade{B_r} = \blade{b_1} \wedge \cdots \wedge \blade{b_r}$ we have the 
\begin{equation}
\label{eq:dot_product}
\blade{A_r} \cdot \blade{B_r}^\dagger = \det(\blade{a_i} \cdot \blade{b_j} )_{i,j=1}^r = \blade{A_r}^\dagger \cdot \blade{B_r}^\dagger,
\end{equation}
which is the typical extension of the inner product $g$ to an inner product on $\bigwedge^r (V)$ through linearity.

Given the direct relationship between unit $r$-blades and $r$-dimensional subspaces we can also form a compact way of projecting multivectors into subspaces in a manner closely related to the subspace dual.  \begin{definition}
Given an multivector $B$, the \emph{projection} onto the subspace $\blade{A_r}$ is
\begin{equation}
\label{eq:projection}
\projection_{\blade{A_r}}(B) \coloneqq B\rfloor \blade{A_r} \blade{A_r}^{-1} \equiv (B\rfloor \blade{A_r})\rfloor \blade{A_r}^{-1}
\end{equation}
\end{definition}
Following this definition, one can see that
\begin{equation}
\projection_{\blade{A_r}}(B) \in \bigoplus_{j=0}^r \G^j = \G^{0+\cdots + r},
\end{equation}
since the subspace $\blade{A_r}$ is $r$-dimensional and moreover the operation preserves grades since
\begin{equation}
\projection_{\blade{A_r}}(\proj{j}{B}) \in \G^j.
\end{equation}
For example, given vectors $\blade{u}$ and $\blade{v}$ we retrieve the familiar statement 
\begin{equation}
\projection_{\blade{u}} (\blade{v}) = (\blade{v} \cdot \blade{u}) \frac{\blade{u}}{\|\blade{u}\|^2}.
\end{equation}

A dual notion exists as well; we can project onto the subspace perpendicular to $\blade{A_r}$.
\begin{definition}
Given a multivector $B$, the \emph{rejection} from the subspace $\blade{A_r}$ is
\begin{equation}
\label{eq:rejection}
\rejection_{\blade{A_r}}(B) \coloneqq B \wedge \blade{A_r} \blade{A_r}^{-1} \equiv (B\wedge \blade{A_r})\lfloor \blade{A_r}^{-1}.
\end{equation}
\end{definition}
Note that this operation is also grade preserving. In the case we have a blade $\blade{C_k}$ with $k<r$ and $k<n-r$, we can note
\begin{equation}
\label{eq:projection+rejection_blade}
\blade{C_k}=\projection_{\blade{A_r}}(\blade{C_k}) + \rejection_{\blade{A_r}}(\blade{C_k}).
\end{equation}
This provides us a way to revisit the geometric notions of the interior and exterior products. In particular, we note that
\begin{align}
    B \rfloor \blade{A_r} &= \projection_{\blade{A_r}} (B) \blade{A_r}\\
    B \wedge \blade{A_r} &= \rejection_{\blade{A_r}}(B) \blade{A_r}.
\end{align}
Both the notion of projection and rejection prove to be useful and behave nicely with the dual by
\begin{equation}
\label{eq:projection_rejection_duality}
\projection_{\blade{A_r}^\perp}(B) = \rejection_{\blade{A_r}}(B),
\end{equation}
and
\begin{equation}
\projection_{\blade{A_r}}(B^\perp) = \rejection_{\blade{A_r}}(B)^\perp.
\end{equation}
\todo{prove both of these statements?}
Finally, the exterior product of orthogonal blades gives us a direct sum of subspaces in the following sense. Let $\blade{A_r}$ and $\blade{B_s}$ be orthogonal so that $\blade{A_r}\wedge \blade{B_s}=\blade{A_r}\blade{B_s}$, then we can note that if $k<r$ and $k<s$ we have
\begin{equation}
\label{eq:projection_sum_of_subspaces}
    \projection_{\blade{A_r}\wedge \blade{B_s}} (\blade{C_k}) = \projection_{\blade{A_r}}(\blade{C_k}) + \projection_{\blade{B_s}}(\blade{C_k}).
\end{equation}
Perhaps it is most enlightening for the reader to revisit \cref{eq:projection+rejection_blade,eq:projection_sum_of_subspaces} replacing $\blade{C_k}$ with a vector $\blade{v}$ since a vector will always prove to be a representative for a ``small enough" subspace.


\subsection{Motivating example}
\label{subsubsec:motivating_example}

Rather than a sequence of multiple examples, it will prove to be far more illuminating to construct one large example for which most of the preliminaries to this point can be used in a meaningful way. As such, we shall not rule out the utility of geometric algebras with pseudo inner products. The classical example is the \emph{spacetime algebra} defined by taking $V=\R^4$ with a vector basis $\blade{\gamma}_0,\blade{\gamma}_1,\blade{\gamma}_2,\blade{\gamma}_3$ satisfying
\begin{subequations}
\begin{align}
\blade{\gamma}_0 \cdot \blade{\gamma}_0 &= -1\\
\blade{\gamma}_0 \cdot \blade{\gamma}_i &= 0  &i=1,2,3\\
\blade{\gamma}_i \cdot \blade{\gamma}_j &= \delta_{ij}, &i,j=1,2,3.
\end{align}
\end{subequations}
We refer to $\blade{\gamma}_0$ as \emph{temporal} since its square is negative and $\blade{\gamma}_i$ for $i=1,2,3$ as \emph{spatial} since its square is positive. For this basis, we can denote the matrix for this inner product $\eta =\operatorname{diag}(-+++)$ (often called the \emph{Minkowski metric}) and define $Q$ from $\eta$. Then, we have for a spacetime vector $\blade{v} = v_0 \blade{\gamma}_0 +v_1 \blade{\gamma}_1 + v_2 \blade{\gamma}_2 + v_3 \blade{\gamma}_3$,
\begin{equation}
\label{eq:spacetime_inner_product}
\|\blade{v}\| = \blade{v} \cdot \blade{v} = -v_0^2 + \sum_{i=1}^3 v_i^2.
\end{equation}
Then we put $\G_{1,3}$ to represent the spacetime algebra and, in broader generality, we put $\G_{p,q}$ for a geometric algebra with $p$ temporal vectors and $q$ spatial vectors. The factor $p$ will return in various different calculations. The reader may wish to, for example, revisit \cref{subsubsec:reverse_inverse_clifford_spin_groups} with $\G_{p,q}$ in mind in order to see a realization of the groups $\operatorname{SO}(p,q)$, $\operatorname{Spin}(p,q)$, and the spacetime spinors. 

As the naming above suggests, the geometric algebra of Euclidean space, $\G_3$, should naturally inside of the spacetime algebra. Note that we have the \emph{spatial pseudoscalar} $\blade{I}_S \coloneqq \blade{\gamma}_1 \wedge \blade{\gamma}_2 \wedge \blade{\gamma}_3$, which, allowing for an extension of our notion of projection to the whole algebra, allows us to put
\begin{equation}
\projection_{\blade{I}_S}(\G_{1,3}) \equiv \rejection_{\blade{\gamma}_0} (\G_{1,3}) = \G_3.
\end{equation}
Perhaps one could refer to this mapping as the \emph{static map} as we project only onto the spatial subspace and, via duality, reject the temporal subspace. It is also worth noting that this static map is not just producing an isomorphic copy of $\G_3$, but a a copy of $\G_3$ directly. Now, in $\G_3$, we can specify an arbitrary multivector $A$ by
\begin{equation}
A= a_0 + a_1 \blade{\gamma}_1 + a_2 \blade{\gamma}_2 + a_3 \blade{\gamma}_3 + a_{12} \blade{B}_{12} + a_{13} \blade{B}_{13} + a_{23} \blade{B}_{23} + a_{123} \blade{\gamma}_1 \wedge \blade{\gamma}_2 \wedge \blade{\gamma}_3,
\end{equation}
and so the grade projections read
\begin{subequations}
\begin{align}
\proj{0}{A}&=a_0\\
\proj{1}{A}&=a_1 \blade{\gamma}_1 + a_2 \blade{\gamma}_2 + a_3 \blade{\gamma}_3\\
\proj{2}{A}&=a_{12} \blade{B}_{12} + a_{13} \blade{B}_{13} + a_{23} \blade{B}_{23}\\
\proj{3}{A}&= a_{123} \blade{\gamma}_1 \wedge \blade{\gamma}_2 \wedge \blade{\gamma}_3.
\end{align}
\end{subequations}
Then, we can write a even multivector as
\begin{equation}
q = q_0 + q_{23}\blade{B}_{23} + q_{31} \blade{B}_{31} + q_{12} \blade{B}_{12}.
\end{equation}
Note as well that
\begin{subequations}
\begin{align}
\blade{B}_{23}^2 = \blade{B}_{31}^2 = \blade{B}_{12}^2 &= -1\\
\blade{B}_{23}\blade{B}_{31}\blade{B}_{12} &= +1,
\end{align}
\end{subequations}
which is typical for spatial bivectors. In this case, one may notice that this even subalgebra is extremely close to being a copy of the quaternion algebra $\quat$. One can arrive at a representation of the quaternions by taking
\begin{equation}
\boldsymbol{i} \leftrightarrow \blade{B}_{23}, \quad \boldsymbol{j} \leftrightarrow -\blade{B}_{31}=\blade{B}_{13}, \quad \boldsymbol{k} \leftrightarrow \blade{B}_{12},
\end{equation}
and noting that we then have $\boldsymbol{ijk}=-1$ as well as $\boldsymbol{i}^2=\boldsymbol{j}^2=\boldsymbol{k}^2=-1$. A more in depth explanation is provided in \cite{doran_geometric_2003}. Thus, a purely we realize a quaternion as a parabivector $q$ and a purely imaginary quaternion is simply the grade-2 portion of the parabivector $q$. We also realize $\quat$ as scalar copies of elements of $\operatorname{Spin}(3) \cong \operatorname{Sp}(1)$. That is to say that $\quat \cong \R \times \operatorname{Spin}(3)$. Indeed, since elements of $\G_3^+$ are simply parabivectors, the parabivectors admit a natural spin representation.

But we are not done here, and we can project down one dimension further by
\begin{equation}
    \projection_{\blade{\gamma}_1 \wedge \blade{\gamma}_2} (\G_3) = \G_2.
\end{equation}
To see this in action, we let $\blade{v}=v_1 \blade{\gamma}_1 + v_2 \blade{\gamma}_2 + v_3 \blade{\gamma}_3$
\begin{subequations}
\begin{align}
    \projection_{\blade{\gamma}_1 \wedge \blade{\gamma}_2} = \projection_{\blade{B}_{12}} (\blade{v}) &= (v_1 \blade{\gamma}_1 + v_2 \blade{\gamma}_2 + v_3 \blade{\gamma}_3)\rfloor \blade{B}_{12}\blade{B}_{12}^{-1}\\
    &= (v_1 \blade{\gamma}_2 - v_2 \blade{\gamma}_1)\blade{B}_{12}^{-1} \\
    &= v_1 \blade{\gamma}_1 + v_2 \blade{\gamma}_2.
\end{align}
\end{subequations}
Then, arbitrary multivectors $A$ and $B$ can be specified by
\[
A = a_0 + a_1 \blade{\gamma}_1 + a_2 \blade{\gamma}_2 + a_{12} \blade{B}_{12}, \qquad B = b_0 +b_1 \blade{\gamma}_1 + b_2 \blade{\gamma}_2 + b_{12}\blade{B}_{12}.
\]
We can then take the product $AB$ to yield
\begin{subequations}
\begin{align}
\proj{0}{AB} &= a_0b_0 + a_1 b_1 + a_2 b_2 - a_{12}b_{12}\\
\proj{1}{AB} &= (a_0 b_1 + a_1 b_0 - a_2 b_{12} + a_{12} b_2) \blade{\gamma}_1 + (a_0 b_2 + a_2 b_0 + a_1b_{12} - a_{12} b_1) \blade{\gamma}_2\\
\proj{2}{AB} &= (a_1b_2 - a_2 b_1)\blade{B}_{12}.
\end{align}
\end{subequations}
Most notably, we see that $\blade{B}_{12}^2=-1$ and this allows us to consider a parabivector
\begin{equation}
z = x + y \blade{B}_{12}
\end{equation}
which is exactly a representation of the complex number $\zeta = x+ iy$ in $\G_2^{0+2}=\G_2^+$.  Thus, the even subalgebra of this geometric algebra is indeed isomorphic to the complex numbers $\C$. Indeed, there is one unit 2-blade $\blade{B}_{12}$ in $\G_2$ to form the spin algebra $\mathfrak{spin}(2) \cong \R$ and as a consequence all unit norm elements in $\G_2^+$ can be written as
\begin{equation}
   e^{\theta \blade{B}_{12}} = \sum_{n=0}^\infty \frac{\theta \blade{B}_{12}}{n!} = \cos(\theta)+\blade{B}_{12}\sin(\theta),
\end{equation}
where $\theta \blade{B}_{12}$ is a general bivector in $\G_2$ when $\theta \in \R$ is arbitrary. Hence, we arrive at $\operatorname{Spin}(2)\cong \operatorname{U}(1)$. Any element in $\C$ is also a scaled version of an element of the spin group $\operatorname{Spin}(2)$. Hence, we can use a spin representation for an element in $\C$ via $z=re^{\theta \blade{B}_{12}} \in \R\times \operatorname{Spin}(2)$.  This special case shows that parabivectors in $\G_2$ have a unique spin representation and they are spinors as well since the whole of the even subalgebra consists of parabivectors.

But, the above work is not necessary special to the starting point of $\G_{1,3}$ or $\G_3$. In fact, if we take $\G_n$ for $n\geq 2$, then there are natural copies of $\C$ contained inside of $\G_n$. In particular, we have the isomorphism
\begin{equation}
    \C \cong \{\lambda + \beta \blade{B} ~\vert~ \lambda,\beta \in \G_n^0,~ \blade{B} \in \Grassmannian{2}{n}. \},
\end{equation}
which shows that complex numbers arise as parabivectors via the representation
\begin{equation}
        \zeta = x + y\blade{B},
\end{equation}
since $\blade{B}^2=-1$. Given the standard basis $\blade{e_1},\dots,\blade{e_n}$ we have copies of $\C$ for each of the ${ n \choose 2}$ unit bivectors $\blade{B}_{jk}$ with $k=2,\dots,n$ and $j<k$.



\todo{Talk about bivectors, spinors, and rotors. Rotations and what not. Euler angles. Would all be good to put in here. Rotations in the complex plane.}