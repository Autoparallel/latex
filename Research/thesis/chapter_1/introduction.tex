In 1980, Alberto Calder\'on proposed an inverse problem in his paper \emph{On an inverse boundary value problem} \cite{calderon_inverse_2006} where he asks if one can determine the electrical conductivity matrix of some Ohmic medium from knowledge of voltage and current measurements on the boundary of the given domain. This problem goes under the name of Electrical Impedance Tomography (EIT). Physically, the EIT problem is a static 3-dimensional boundary value inverse problem wherein the practicioner has access to voltage to a discrete subset of the boundary of a body and make noisy measurements of the outgoing current flux along the same discrete subset. In other words, a partial and noisy version of the voltage-to-current map is known.

One notes that the voltage-to-current map inputs a scalar potential (the Dirichlet data) on the boundary which, if the conductivity matrix was known, would allow one to determine the potential on the interior by solving a second order elliptic partial differential equation with coefficients defined by the conductivity. In an Ohmic material, the current field is induced by the electric field (the gradient of the potential) and the conductivity, and hence for a fixed conductivity, utilizing different Dirichlet data can induce different current fields in the body. The practicioner has the ability to measure the outgoing current flux along the boundary (the Neumann data) and thus, they have access to the voltage-to-current map.   

This problem can be generalized naturally into dimensions $\geq 2$ as a geometric inverse problem. One replaces the medium with a manifold and the conductivity becomes the Riemannian metric. Thus, the second order elliptic equation from before amounts to finding scalar fields in the kernel of the Laplace-Beltrami operator. This leads to a small caveat that in dimension 2 since the Laplace-Beltrami operator is conformally invariant. At any rate, the EIT problem is equivalent to determining an unknown Riemannian manifold up to isometry from the classical Dirichlet-to-Neumann (DN) map which inputs a scalar field and outputs the outward normal component of the derivative of the solution \cite{feldman_calderproblem_nodate, salo_calderon_nodate, uhlmann_inverse_2014}. 

There are a handful of approaches to solving this problem, but it remains unsolved. In order to make progress, theorists have allowed themselves access to larger sets of data, for example, complete knowledge of a generalized DN map on differential forms \cite{krupchyk_inverse_2011,sharafutdinov_complete_2013,belishev_dirichlet_2008,joshi_inverse_nodate}. In dimension 2, the smooth problem has been solved up to conformal invariance and in dimension $\geq 3$, the problem has been solved for analytic manifolds \cite{lassas_determining_2001}. Another approach that is unique to a manifold of dimension 2 appears in \cite{belishev_calderon_2003}. In this paper, Belishev determines the algebra of holomorphic functions from the DN map and realizes the spectrum of this algebra homeomorphic to the underlying manifold by Gelfand. The metric $g$ is then recovered up to conformal class by extracting the complex structure from this algebra as well. An attempt to generalize this approach to dimension $n=3$ can be found in by replacing the complex structure with a quaternionic structure but this has not lead to a complete solution \cite{belishev_algebras_2017, belishev_algebraic_2019}. It has been shown that the 3-dimensional round ball can be determined up to homeomorphism from a quaternionic spectrum. Belishev and Vakulenko ask whether this can be extended to higher dimensions and to other spaces. An answer to this question is provided by \cref{thm:gelfand}.

In this work, I first introduce the geometric algebras $\G$ as special cases of more general Clifford algebras in \cref{subsec:clifford_and_geometric_algebras}. Following this, I take a smooth, oriented, Riemannian manifold $M$ and construct a Clifford algebra bundle whose sections lie in the space $\G(M)$ and are referred to as multivector fields in \cref{sec:geometric_manifolds}. The graded algebraic structure of $\G(M)$ expands upon the exterior algebra of forms $\Omega(M)$, and moreover, there exists a natural differential structure via the gradient operator $\grad$, which finds similarities to the Hodge-Dirac operator $d+\delta$. The space $\G(M)$ proves to be more rich than $\Omega(M)$ since one can realize $\Omega(M)$ as a trivial case of some $\G(M)$. Moreover, it is quite natural to look at multivector fields that consist of many differently graded elements at once. For example, multivector fields that lie in the kernel of $\grad$ are called monogenic and these fields share many of the same properties as holomorphic functions on $\C$ including, but not limited to, a Cauchy integral formula \cref{eq:cauchy_integral}. However, unlike holomorphic functions, the space of monogenic fields $\monogenicfields{}$ is not, in general, commutative or an algebra.

A useful version of a Green's formula is shown in \cref{thm:multivector_greens_formula}, and this allows us to prove a multivector version of the Hodge-Morrey decomposition that we realize in the following theorem.
\begin{customthm}{3.1.1}[Monogenic Hodge Decomposition]
The space of multivector fields $\G(M)$ has the $L^2$-orthogonal decomposition
\begin{equation}
\G(M) = \monogenicfields{} \oplus \pseudoscalar \grad \G(M).
\end{equation}
\end{customthm}
\todo{I think I can get rid of the $I$ here since it is an isomorphism.}

The space $\monogenicfields{}$ is a right module over the constant multivectors $\G$. For the special case of the Euclidean geometric algebra $\G_n$, I define a space of module homomorphisms from $\monogenicfields{}$ to $\G_n$ and refer to these morphisms $\G_n$-functionals. Inside $\monogenicfields{}$ lie commutative subalgebras $\algebra{\bivector}(M)$ that are analogs of $\C$ and on these algebras we can define $\G_n$-characters as the $\G_n$-functionals that are also algebra morphisms on each $\algebra{\bivector}(M)$ into $\G_n$. The space of $\G_n$-characters, $\characters(M)$, with the weak-$\ast$ topology, is shown to be homeomorphic to $M$ in the special case where $M$ is a region of $\R^n$ and inherits the Euclidean metric. This is summarized in the following theorem.
\begin{customthm}{3.3.1}
For any $\delta \in \characters(M)$, there is a point $x^\delta \in M$ such that $\delta(f) = f(x^\delta)$ for any $f\in \monogenics(M)$ a monogenic field. Given the weak-$\ast$ topology on $\dualmonogenics(M)$, the map
\[
\gamma \colon \characters(M) \to M, \quad \delta \mapsto x^\delta
\]
is a homeomorphism. 
%The Gelfand transform 
%\[
%\widehat{~} \colon \monogenics(M) \to C(\characters(M); \G_n), \quad \widehat{f}(\delta) \coloneqq \delta(f), \quad \delta \in \characters(M),
%\]
%is an isometry onto its image, so that $\characters(M)$ is isomorphic to $\widehat{\monogenics(M)}$ as algebras.
\end{customthm}

Owing to the original intention of this work, I consider physical and geometric inverse boundary value problems related to the Calder\'on problem. For example, I discuss the electric and magnetic impedance tomography problems and their statements in terms of multivector fields \cref{sec:tomography}. Relationships between the two problems are established, and one finds that the Ohmic property of a medium couples together the scalar potential $u$ and the magnetic bivector field $b$ into a single monogenic field. Given knowledge of the electrostatic and magnetostatic version of the DN map alongside this new relationship, can one determine the underlying conductivity? Likewise, in higher dimensions, do \cref{thm:gelfand,thm:monogenic_hodge} provide new tools for solving the Calder\'on or other related inverse problems? Finally, there also exists a Hilbert transform in two guises via \cite{belishev_dirichlet_2008,brackx_hilbert_2008}. Are these two notions equivalent? Does either add any more useful information for solving boundary inverse problems?