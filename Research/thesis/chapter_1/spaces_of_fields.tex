\subsection{Hodge-type decompositions}
Let us define the following spaces of $r$-vectors. We write this in terms of the multivector equivalents to forms.
\begin{itemize}
    \item The \emph{exact $r$-vectors},
    \begin{equation}
        \exactvectors{r}\coloneqq \{\grad \wedge A_{r-1} ~\vert~ A_{r-1} \in \G^{r-1}(M), ~ \projection_{\blade{\nu}}(A_{r-1}) = 0\};
    \end{equation}
    \item The \emph{co-exact $r$-vectors},
    \begin{equation}
        \coexactvectors{r}\coloneqq \{\grad \rfloor A_{r+1} ~\vert~ A_{r+1} \in \G^{r+1}(M), ~ \rejection_{\blade{\nu}}(A_{r+1}) = 0\};
    \end{equation}
    \item The \emph{harmonic fields},
    \begin{equation}
        \harmonicfields{r}\coloneqq \{A_r \in \G^{r}(M) ~\vert~ \grad \wedge A_r = 0, ~ \grad \rfloor A_r = 0\}.
    \end{equation}
    \item The \emph{Dirichlet harmonic fields},
    \begin{equation}
        \harmonicdirichlet{r}\coloneqq \{A_r \in \harmonicfields{r} ~\vert~ \projection_{\blade{\nu}}(A_r) = 0\}.
    \end{equation}
    \item The \emph{Neumann harmonic fields},
    \begin{equation}
        \harmonicneumann{r}\coloneqq \{A_r \in \harmonicfields{r} ~\vert~ \rejection_{\blade{\nu}}(A_r) = 0\}.
    \end{equation}
\end{itemize}
Notice that the exact and coexact forms satisfy not only a differential condition, but a boundary condition as well. Then, under the $r$-form inner product, we find the orthogonal direct sum decomposition
\begin{equation}
\Omega^r(M) = \exactvectors{r} \oplus \coexactvectors{r} \oplus \harmonicfields{r},
\end{equation}
known as the Hodge-Morrey decomposition. Within the space of harmonic fields we have
\begin{align}
    \harmonicex{r} &\coloneqq \{A_r \in \harmonicfields{r} ~\vert~ A_r = \grad \wedge B_{r-1} \},\\
    \harmonicex{r} &\coloneqq \{A_r \in \harmonicfields{r} ~\vert~ A_r = \grad \rfloor B_{r+1} \}.
\end{align}
Further, we have two decompositions of the space of harmonic fields 
\begin{align}
    \harmonicfields{r} &= \harmonicdirichlet{r} \oplus \harmonicco{r}, \\
    \harmonicfields{r} &= \harmonicneumann{r} \oplus \harmonicex{r},
\end{align}
which are the Friedrichs decompositions.

\begin{proposition}
    The harmonic fields $\harmonicfields{r}$ are monogenic $r$-vectors so that $\monogenicfields{r} = \harmonicfields{r}$.
\end{proposition}
\begin{proof}
Trivial
\end{proof}

We can utilize 
Monogenics of a single grade are already studied. but now we can study monogenics of mixed grades!
\subsection{Integral transforms}
