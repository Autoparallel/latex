\documentclass[doctor]{thesis}
\usepackage{import}
\usepackage{preamble}
\usepackage{environments}
% package for todo 
\setlength{\marginparwidth}{2cm}
%\usepackage[colorinlistoftodos]{todonotes}
\usepackage[disable]{todonotes}
\setuptodonotes{size=\scriptsize,backgroundcolor=red!15!white} 

%\usepackage{showkeys} %show cites and refs




% Title Page
%%%%%%%%%%%%%%%%%%%%%%%%%%%%%%%%%%%%%%%%%%%%%%%%%%%%%%%%%%%%%%%%

% title of your thesis
\title{Approaching Geometric Inverse Problems using Analysis of Multivector Fields}

% author's name
\author{Colin Roberts}

% author's email address
\email{colin.roberts@rams.colostate.edu}

% department name
\department{Department of Mathematics}

% semester of completion
\semester{Spring 2021}

% committee member names
\advisor{Clayton Shonkwiler}
\committee{Henry Adams} % as many committee entries as you need
\committee{Wolfgang Bangerth}
\committee{Jacob Roberts}

% Copyright Page
%%%%%%%%%%%%%%%%%%%%%%%%%%%%%%%%%%%%%%%%%%%%%%%%%%%%%%%%%%%%%%%%

% here is an example of student copyright declaration
% note that the \copyright command prints the copyright symbol,
% so we use the name \mycopyright instead
\mycopyright{%
Copyright by Colin Roberts 2020 \\
All Rights Reserved
}

%\tableofcontents

\abstract{
The geometrical inverse problem of determining an unknown Riemannian manifold $M$ from the Dirichlet-to-Neumann (DN) map on the boundary is known as the Calder\'on problem and, in dimension two, this problem is solvable up to conformal deformation. One technique is to us the DN map to the algebra of holomorphic functions on the manifold which, by Gelfand, this set is then homeomorphic to the manifold. In higher dimensions, one can replace the complex line bundle with differential forms, but another natural choice is to consider a geometric (or Clifford) algebra bundle whose sections are multivector fields. The graded algebra of multivector fields comes with a natural differential operator known as the gradient (or Dirac) operator that replaces the Hodge-Dirac operator in the exterior algebra of forms. Fields in the kernel of the gradient are monogenic, and these monogenic fields are a natural generalization of holomorphic functions. In this work, we prove a version of the Hodge-Morrey decomposition which splits the space of multivector fields into an orthogonal sum of monogenic fields and gradients. We define a suitable notion of a spectrum of the space of monogenic fields and prove the spectrum to be homeomorphic to the underlying manifold with the weak-$\ast$ topology.
}

% Acknowledgments 
%%%%%%%%%%%%%%%%%%%%%%%%%%%%%%%%%%%%%%%%%%%%%%%%%%%%%%%%%%%%%%%%

\acknowledgements{ Fill in acknowledgements here.
}

\begin{document}

\todo{Start working on a literature review type thing so I can cite sources more easily later.}
\todo{The space $\G_n^+$ is a module over $\spingroup$ and in the case of the complex numbers we can realize this is just a phase. Similar thing for quaternions probably.}




%%%%%%%%%%%%%%%%%%%%%%%%%%%%%%%%%%%%%%%%%%%%%%%%%%%%%%%%%%%%%%%%

\frontmatter % starts preliminary pages
%%%%%%%%%%%%%%%%%%%%%%%%%%%%%%%%%%%%%%%%%%%%%%%%%%%%%%%%%%%%%%%%

\maketitle              % insert title page
\makemycopyright        % insert copyright page
\makeabstract           % insert abstract page
\makeacknowledgements   % insert acknowledgements page

% any extra preliminary pages can be added here
% below is an example of a dedication page
% the dedication page is optional
\prelimtocentry{Dedication} % add table of contents entry
\begin{flatcenter} % center without extra space

    % page title
    DEDICATION

    %\vspace{3em} % place at top
    \vfill % or center on page

    \noindent \textit{I would like to dedicate this thesis to my dog fluffy.}
    \vfill % fill extra space at bottom
\end{flatcenter}
\newpage

\tableofcontents    % insert table of contents
\listoftables       % insert list of tables (optional)
\listoffigures      % insert list of figures (optional)

\mainmatter % starts thesis body

%%%%%%%%%%%%%%%%%%%%%%%%%%%%%%%%%%%%%%%%%%%%%%%%%%%%%%%%%%%%%%%%%%%%%%%%%%%%%%%%%%%%%%%%%%%%%%%%%%%%%%%%%%%%%%%%%%%%%%%%%%
% Chapter 1
%%%%%%%%%%%%%%%%%%%%%%%%%%%%%%%%%%%%%%%%%%%%%%%%%%%%%%%%%%%%%%%%%%%%%%%%%%%%%%%%%%%%%%%%%%%%%%%%%%%%%%%%%%%%%%%%%%%%%%%%%%

\chapter{Introduction}

\section{Introduction}
In 1980, Alberto Calder\'on proposed an inverse problem in his paper \emph{On an inverse boundary value problem} \cite{calderon_inverse_2006} where he asks if one can determine the conductivity matrix of a medium from Cauchy data supplied on the boundary.  In dimensions $n>2$, this is equivalent to determining a Riemannian manfiold up to isometry from the scalar Dirichlet-to-Neumann (DN) operator \cite{feldman_calderproblem_nodate, salo_calderon_nodate, uhlmann_inverse_2014}. The DN operator takes any given Dirichlet boundary values and outputs the corresponding Neumann data of a solution to Laplace's equation in order to generate the relevant Cauchy data.  

One approach to reconstructing the Riemannian metric in dimension $n=2$ appears in \cite{belishev_calderon_2003}, where the author uses the Boundary--Control (BC) method to determine the manifold up to conformal class. \textcolor{red}{Add in a bunch of other citations to the BC method.} The BC method takes an algebraic approach. Specifically, the DN operator determines the algebra of holomorphic functions on $M$ and realizes $M$ as homeomorphic to the Gelfand spectrum of this commutative algebra. The metric $g$ is then recovered after providing $M$ with a complex structure. In dimension $n=2$, the Laplace-Beltrami operator is conformally invariant, and this result cannot be improved.  An attempt to generalize this approach to dimension $n=3$ can be found in by replacing the complex structure with a quaternionic structure but this has not lead to a complete solution \cite{belishev_algebras_2017, belishev_algebraic_2019}.  It has been shown that when $M$ is the 3-dimensional ball in $\R^3$, there is an associated space of harmonic quaternion fields that has a quaternion spectrum homeomorphic to the ball. But, a connection to the DN operator has not been made, and this method has also not been generalized to higher dimensions.

In this paper, I show that there exists a space of spin characters $\characters$ acting on a $\spingroup$ invariant space of monogenic multivector fields on the $n$-dimensional ball that is homeomorphic to the ball.  We then observe that this space of monogenics is determined from the DN map, and thus recover the ball up to homeomorphism from the boundary data.  This is summarized in two main theorems.
\begin{theorem*}
The set of multiplicative $\spinalgebra$-linear functionals on the $\spingroup$ invariant space of monogenic fields $\monogenics$ on the $n$-dimensional ball $\ball$ is homeomorphic to $\ball$ with the Gelfand topology.
\end{theorem*}
\begin{theorem*}
The scalar DN operator determines the $\spingroup$ invariant space of monogenic fields on regions in $\R^n$.
\end{theorem*}
The second theorem can be extended to Riemannian manifolds quite readily.

We first introduce the Clifford algebra setting. Given a vector space with an inner product, we can create the graded Clifford algebra.  In particular, we extend these Clifford algebras to Clifford algebra valued functions (or multivector fields) on regions $M \subset \R^n$.  Inside the multivector fields sit the even graded multivectors consisting of scalars, bivectors, and other $2k$-vectors. In $\R^2$ with the Euclidean inner product, this space is isomorphic to the $\C$-algebra and so the functions valued in this even sub-Clifford algebra can be thought of as complex valued functions.  Clifford analysis generalizes the notion of holomorphicity to monogenicity and we find that monogenic functions lie in the kernel of the Dirac operator $\grad$ just as $\C$-holomorphic functions lie in the kernel of the Wirtinger derivative $\frac{\partial}{\partial \overline{z}}$. Moreover, one has that $\grad$ is the square root Laplace-Beltrami operator $\Delta = \grad^2$. Even monogenic multivector fields are $\spingroup$ invariant and each grade is harmonic (in the kernel of $\Delta$). 

When $M$ is the $n$-ball, we have that space of even monogenics $\monogenics$ which can be generated by the algebras of even graded $B$-planar monogenic biparavector fields (each field constant on translations of the $B$-plane in $\R^n$). Those generating subalgebras are individually isomorphic to the algebra of holomorphic functions on the complex unit disk $\disk$. On these spaces, one can define $\spinalgebra$-linear multiplicative functionals $\characters$, referred to as spin characters. Each spin character is equivalent to a Dirac measure on the $n$-ball which, with the Gelfand topology, provide a homeomorphic copy of the $n$-ball.

The space of $(0+2)$-vector monogenics is found from the DN operator in the following sense.  The DN operator determines a Hilbert transform on multivector fields that allows one to determine the monogenic conjugate bivector field $b$ corresponding to a scalar solution $u$ to the Laplace equation $\Delta u = 0$ so that $f=u+b$ is monogenic. \textcolor{red}{Haven't actually done this yet} Considering all smooth boundary conditions generates the relevant space of monogenics, from which we determine the space of spin characters. Thus, the DN operator provides a means of constructing a homeomorphic of the $n$-ball.
\todo{reword introduction}

%%%%%%%%%%%%%%%%%%%%%%%%%%%%%%%%%%%%%%%%%%%%%%%%%%%%%%%%%%%%%%%%%%%%%%%%%%%%%%%%%%%%%%%%%%%%%%%%%%%%%%%%%%%%%%%%%%%%%%%%%%
% Chapter 2
%%%%%%%%%%%%%%%%%%%%%%%%%%%%%%%%%%%%%%%%%%%%%%%%%%%%%%%%%%%%%%%%%%%%%%%%%%%%%%%%%%%%%%%%%%%%%%%%%%%%%%%%%%%%%%%%%%%%%%%%%%

\chapter{Preliminaries}

\section{Clifford and geometric algebras}
\label{subsec:clifford_and_geometric_algebras}

The complex algebra $\C$ can be generalized in a handful of ways.  Some of which can be found through the use of Clifford algebras and, more specifically, in geometric algebras.  We define the more general Clifford algebras first and realize geometric algebras as particularly nice Clifford algebras with a quadratic form arising from an inner product. Elements of a geometric algebra are known as multivectors and these multivectors carry a wealth of geometric information in their algebraic structure. $\C$ itself can be realized as a special subalgebra of parabivectors in the geometric algebra on $\R^2$ with the Euclidean inner product and the quaternions $\quat$ are realized as an analogous algebra on $\R^3$. In particular, both $\C$ and $\quat$ arise as the 2- and 3-dimensional even Clifford groups $\Gamma^+$ respectively. \todo{reword this paragraph}

First, we present a review of Clifford algebras and the relevant notions needed for this work. Those who feel they are familiar with both Clifford and geometric algebras may wish to skim through this subsection and visit \cref{subsubsec:motivating_example} to review the notation used throughout this manuscript. 

Formally, we let $(V,Q)$ be an $n$-dimensional vector space $V$ over some field $K$ with an arbitrary quadratic form $Q$.  The tensor algebra is given by
\begin{equation}
\mathcal{T}(V) \coloneqq \bigoplus_{j=0}^\infty V^{\otimes j} = K \oplus V \oplus (V\otimes V) \oplus (V\otimes V \otimes V) \oplus \cdots,
\end{equation}
where the elements (tensors) inherit a multiplication $\otimes$ (the tensor product). From the tensor algebra $\mathcal{T}(V)$, we can quotient by the ideal generated by $\blade{v}\otimes \blade{v} - Q(\blade{v})$ to create a new algebra.
\begin{definition}
The \emph{Clifford algebra} $C\ell(V,Q)$ is the quotient algebra
\begin{equation}
C\ell(V,Q) = \mathcal{T}(V) ~ / ~ \langle \blade{v} \otimes \blade{v} - Q(\blade{v}) \rangle.
\end{equation}
\end{definition}
To see how the tensor product descends to the quotient, we let $\blade{v}_1, \dots, \blade{v}_n$ be an arbitrary basis for $V$, then we can consider the tensor product of basis elements $\blade{v}_i \otimes \blade{v}_j$ which induces a product in the quotient $C\ell(V,Q)$ which we refer to as the \emph{Clifford multiplication}. In this basis, we write this product as concatenation $\blade{v_i}\blade{v_j}$ and define the multiplication by
\begin{equation}
\label{eq:clifford_multiplication}
\blade{v}_i \blade{v}_j = \begin{cases} Q(\blade{v}_i) & \textrm{if $i=j$}, \\ \blade{v}_i \wedge \blade{v}_j & \textrm{if $i\neq j$},\end{cases}
\end{equation}
where $\wedge$ is the typical exterior product satisfying $\blade{v}\wedge \blade{w} = - \blade{w}\wedge \blade{v}$ for all $\blade{v},\blade{w}\in V$.  As a consequence, the exterior algebra $\bigwedge(V)$ can be realized as a subalgebra of any Clifford algebra over $V$ or as a Clifford algebra with a trivial quadratic form $Q=0$.  

In the case where $V$ has a (pseudo) inner product $g$, we can induce a quadratic form $Q$ by $Q(\blade{v})=g(\blade{v},\blade{v})$ and give rise to a special type of Clifford algebra which motivates the following definition.
\begin{definition}
Let $V$ be a vector space with an (pseudo) inner product $g(\cdot,\cdot)$. Then taking $Q(\cdot) = g(\cdot,\cdot)$, the Clifford algebra $C \ell(V,Q)$ is called a \emph{geometric algebra}.
\end{definition}
In general, we put $\G$ and assume the inner product and vector space will be arbitrary, given alongside, or will be clear from context.  For example, when $V=\R^n$ we have the standard orthonormal basis $\blade{e}_1,\dots,\blade{e}_n$ which allows us to neatly define the quadratic form $Q$ from the Euclidean inner product which has coefficients $\delta_{ij}$ with respect to this basis. Since we frequently utilize this geometric algebra, we put $\mathcal{G}_n \coloneqq C\ell(\R^n, |\cdot|)$ to simplify notation.  In broader generality, we do not need to have a definite inner product. For example, we can take an inner product where $p$ vectors square to negative values and $q$ vectors square to positive values which is of interest for those studying curved spacetime. Vectors whose square is negative are \emph{temporal} and those whose square is positive are \emph{spatial}. We put $\G_{p,q}$ for a geometric algebra with $p$ temporal vectors and $q$ spatial vectors where, in particular, $p$ vectors square to $-1$ and $q$ vectors square to 1.. The factor $p$ will return in various different calculations.

Geometric algebras are an old and widely studied topic. For more information, see the classical text \cite{hestenes_clifford_1986} or the more modern text \cite{doran_geometric_2003} which also provides a wide range of applications to physics problems. Both these sources include much of the other necessary preliminaries I cover in the remainder of this section. Finally, the paper \cite{chisolm_geometric_2012} proves many of the useful identities and notation used throughout this paper.

\subsection{Multivectors and grading}
\todo{fix all vector indices to be not bold}
Note that $C\ell(V,Q)$ is a $\mathbb{Z}$-graded algebra with elements of grade-0 up to elements of grade-$n$. We refer to grade-0 elements as scalars, grade-1 elements as vectors, grade-2 elements as \emph{bivectors}, grade-$r$ elements as \emph{$r$-vectors}, and grade-$n$ elements as \emph{pseudoscalars}. For example, the pseudoscalar $\blade{\mu} = \blade{v}_1 \wedge \blade{v}_2 \wedge \cdots \wedge \blade{v}_n$ is an $n$-vector we will frequently return to. We denote the space of $r$-vectors by $C\ell(V,Q)^r$. For each grade there is a basis of ${n\choose r}$ \emph{$r$-blades} which are $r$-vectors of the form
\begin{equation}
\label{eq:blade}
\blade{A_r} = \bigwedge_{j=1}^r \blade{v}_j, ~\textrm{for linearly independent}~ \blade{v}_j \in V,
\end{equation}
and we use a boldface of both the character and its subscript to specify that a $r$-vector is a $r$-blade and we note that vectors (since they are $1$-blades) will not use this subscript. Instead, a vector $\blade{v}$ may use a non-boldfaced subscript to reference an index. Briefly, take for example the case where $\dim(V)=3$, then there are ${3\choose 2}=3$ 2-blades that form a basis for the bivectors and one particular choice of a bivector basis would be the following list of 2-blades
\begin{equation}
\label{eq:3_dim_basis}
\blade{B}_{12} = \blade{v}_1 \wedge \blade{v}_2, \quad \blade{B}_{13} = \blade{v}_1 \wedge \blade{v}_3, \quad \blade{B}_{23} = \blade{v}_2 \wedge \blade{v}_3.
\end{equation}
We will repeatedly use the notation $\blade{B}_{ij} \coloneqq \blade{v}_i\wedge \blade{v}_j$ and the underlying basis will be clear from context. We refer to an $n-1$-vector as a \emph{pseudovector} and it should be noted that every $n-1$-vector is a blade (see \cref{subsubsec:duality_and_pseudoscalars}). In other literature, some will refer to a $r$-blade as a \emph{simple} or a \emph{decomposable} $r$-vector\todo{citations}. 

In general, an element $A \in C\ell(V,Q)$ is written as a linear combination of basis elements of all possible grades and we refer to $A$ as a \emph{multivector}.  To extract the grade-$r$ components of $A$, we use the \emph{grade projection} for which we have the notation
\begin{equation}
\proj{r}{A} \in C\ell(V,Q)^r
\end{equation}
to denote the grade-$r$ components of the multivector $A$ (i.e., $\proj{r}{A} \in C\ell(V,Q)^r$). For the scalar component we put $\proj{}{A}$ and we can note we have the cyclic property
\begin{equation}
\label{eq:cyclic_property}
\proj{}{AB\cdots CD} = \proj{}{DAB\cdots C}
\end{equation} 
Any multivector $A$ can then be given by
\begin{equation}
A = \sum_{r=0}^n \proj{r}{A}
\end{equation}
which shows the decomposition via the $\mathbb{Z}$-grading
\begin{equation}
C\ell(V,Q) = \bigoplus_{j=0}^n C\ell(V,Q)^j.
\end{equation}
If $A$ contains only components of a single grade, then we say that $A$ is \emph{homogeneous} and if the components are grade-$r$ we put $A_r$ and refer to $A_r$ as a \emph{homogeneous $r$-vector} or simply an \emph{$r$-vector}.  For example, when we refer to vectors we realize them as 1-vectors and likewise we realize bivectors as 2-vectors. Also of interest will be the elements in
\begin{equation}
 C\ell(V,Q)^{0+2} = C\ell(V,Q)\oplus C\ell(V,Q)^2
\end{equation}
which we refer to as \emph{surface spinors}. \todo{Maybe just define these later with the spinors and call these surface spinors}

The Clifford multiplication of vectors defined in \ref{eq:clifford_multiplication} can be extended to multiplication of vectors with homogeneous $r$-vectors.  In particular, given a vector $\blade{v} \in C\ell(V,Q)$ and a homogeneous $r$-vector $A_r \in C\ell(V,Q)$, we have
\begin{equation}
\label{eq:vector_multiplication}
\blade{v}A_r = \proj{r-1}{\blade{v}A_r} + \proj{r+1}{\blade{v}A_r},
\end{equation}
which decomposes the multiplication into a grade lowering \emph{interior product} and a grade raising \emph{exterior product}.  This allows us to extend the Clifford multiplication further. Given an $s$-vector $B_s$, we have
\begin{equation}
\label{eq:general_clifford_multiplication}
A_r B_s = \proj{|r-s|}{A_rB_s} + \proj{|r-s|+2}{A_rB_s} + \cdots + \proj{r+s}{A_rB_s}.
\end{equation}
This rule for multiplication then allows for the multiplication of two general multivectors in $C\ell(V,Q)$. For this multiplication, specific grades of the product are worth noting.
\begin{equation}
\label{eq:dot}
    A_r \cdot B_s \coloneqq \proj{|r-s|}{A_r B_s}
\end{equation}
\begin{equation}
\label{eq:wedge}
    A_r \wedge B_s \coloneqq \proj{r+s}{A_r B_s}
\end{equation}
\begin{equation}
\label{eq:left_contraction}
    A_r \rfloor B_s \coloneqq \proj{s-r}{A_r B_s}
\end{equation}
\begin{equation}
\label{eq:right_contraction}
    A_r \lfloor B_s \coloneqq \proj{r-s}{A_r B_s}.
\end{equation}
Finally, we have a special product for bivectors called the \emph{commutator product} given by
\begin{equation}
\label{eq:commutator_product}
    A_2 \times B_2 \coloneqq \proj{2}{A_2 B_2} \equiv \frac{1}{2} (A_2 B_2 - B_2 A_2).
\end{equation}
These products are particularly emphasized as many helpful identities used in this paper are phrased using these notions. For example,
\begin{equation}
\label{eq:contraction_swap}
A_r \rfloor B_s = (-1)^{r(s-1)}B_s \lfloor A_r
\end{equation}
\begin{equation}
\label{eq:wedge_swap}
A_r \wedge B_s = (-1)^{rs}B_s\wedge A_r.
\end{equation}
Proofs for the identities used throughout can be found in \cite{chisolm_geometric_2012}.  Taking \cref{eq:vector_multiplication,eq:wedge,eq:left_contraction} into mind, we see that the grade lowering interior product can be written as
\begin{equation}
    \proj{r-1}{\blade{v}A_r} \equiv \blade{v}\rfloor A_r \equiv \blade{v} \cdot A_r
\end{equation}
and the grade raising exterior product can be written as
\begin{equation}
    \proj{r+1}{\blade{v}A_r} \equiv \blade{v} \wedge A_r.
\end{equation}
Finally, to suppress needless additional parentheses later on, we assert that the above products take precedence over the geometrical product in order of operation. For example, for multivectors $A$, $B$, and $C$, we must take
\begin{equation}
A\cdot B C \equiv (A \cdot B)C,
\end{equation}
and extend this to the other products defined in \cref{eq:wedge,eq:left_contraction,eq:right_contraction,eq:commutator_product} as well.

We can also define an inner product on multivector fields that captures that mimics Euclidean inner product on structure of $\R^{2^n}$, i.e., treating each of the basis blades as independent vectors in $\R^{2^n}$. 
\begin{definition}
Let $A,B \in \G$, then the \emph{multivector inner product} is given by
\begin{equation}
(A,B) \coloneqq \proj{}{A^\dagger B}.
\end{equation}
\end{definition}
This product is bilinear, symmetric, positive definite, and satisfies
\begin{equation}
(A,B)=(A^\dagger,B^\dagger)
\end{equation}
so long as $g$ is positive definite. The product $(\cdot,\cdot)$ is a natural extension of the inner product $g$ on the $n$-dimensional space $V$ to the $2^n$-dimensional $\G$. To see this, take an orthonormal basis $\blade{e}_1,\dots,\blade{e}n$ and construct the basis of blades by defining an index set $J=\{j_1,j_2,\dots,j_r\}$ for $0<j_1<j_2<\cdots< j_r\leq n$. The set of all such $J$ for all $0\leq r \leq n$ allows us to define the basis blades $\blade{E}_J = \blade{e}_{j_1}\cdots \blade{e}_{j_r}$ for which we can define any multivector $A$ by $A = \sum_J A_J \blade{E}_J$ where $A_J$ are scalar coefficients. The inner product then returns
\begin{equation}
(A,B) = \sum_{J} A_JB_J.
\end{equation}
The symmetry, definiteness, and bilinearity become apparent. However, if $g$ is not positive definite, then there are vectors called \emph{null vectors} such that $(\blade{v},\blade{v})=0$. This can be realized in the spacetime algebra (see \cref{subsec:motivating_example}).

With this inner product, we have a notion of an adjoint.
\begin{proposition}
\label{prop:adjoint}
Take $A,B,C \in \G$ then
\begin{align}
(CA,B) &= (A,C^\dagger B)\\
(AC,B) &= (A,BC^\dagger)
\end{align}
\end{proposition}
\begin{proof}
First,
\begin{align}
(CA,B) = \proj{}{(CA)^\dagger B} = \proj{}{ A^\dagger C^\dagger B } = (A,C^\dagger B),
\end{align}
and
\begin{align}
(AC,B) = \proj{}{(AC)^\dagger B} = \proj{}{ C^\dagger A^\dagger B } = (A,BC^\dagger),
\end{align}
both by \cref{eq:cyclic_property}
\end{proof}
Also, we have the induced norm.
\begin{definition}
    The \emph{multivector norm} $| \cdot |$ for $A \in \G$ is given by
    \begin{equation}
    |A| \coloneqq \sqrt{(A,A)}.
    \end{equation}
\end{definition}

As discussed, $C\ell(V,Q)$ is naturally a $\mathbb{Z}$-graded algebra but we also find that it carries a $\mathbb{Z}/2\mathbb{Z}$-grading as well. Some would then refer to $C\ell(V,Q)$ as an \emph{superalgebra} \todo{source}. This additional grading can be realized by sorting $r$-vectors in $C\ell(V,Q)$ into the sets where $r$ is even or odd.  We say a $r$-vector is \emph{even} (resp. \emph{odd}) if $r$ is even (resp. odd) and in general if a multivector $A$ is a sum of only even (resp. odd) grade elements we also refer to $A$ as even (resp. odd).  Taking note of the multiplication defined in \ref{eq:general_clifford_multiplication}, one can see that the multiplication of even multivectors with another even multivectors outputs an even multivector and that motivates the following.
\begin{definition}
The \emph{even subalgebra} $C\ell(V,Q)^+ \subset C\ell(V,Q)$ is the subalgebra of even grade multivectors
\begin{equation}
    C\ell(V,Q)^+ \coloneqq C\ell(V,Q)^0 \oplus C\ell(V,Q)^2 \oplus C\ell(V,Q)^4 \oplus \cdots.
\end{equation}
\end{definition}
The split between even and odd subspaces of $C\ell(V,Q)$ makes the space $C\ell(V,Q)$ into a \emph{superalgebra}. Though, one should note that the space of odd grade multivectors, $C\ell(V,Q)^-$, is not an algebra in its own right, it is a $C\ell(V,Q)^+$-module. We can then take the even part of a multivector $A$ by $\proj{+}{A}$ and the odd part by $\proj{-}{A}$ and note
\begin{equation}
A = \proj{+}{A} + \proj{-}{A}.
\end{equation}
In the same vein, we will denote an even multivector by $A_+$ and an odd multivector by $A_-$. The even subalgebra is an extremely important entity that arises throughout physics due to its encapsulation of spinors which we touch on next. 

\subsection{Multivector operations and the Clifford and spin groups}
\label{subsubsec:reverse_inverse_clifford_spin_groups}
For the remainder of this paper, let us focus solely on geometric algebras $\G$. Given access to an (pseudo) inner product we have a natural isomorphism between $V$ and $V^*$ by the Riesz representation.  Namely, given an arbitrary basis $\blade{v}_i$ for $V$ there exists the corresponding dual basis $f_i$ for $V^*$ such that $f_i(\blade{v}_j)=\delta_{ij}$. In geometric algebra, this notion is somewhat superfluous as we can realize the dual basis inside $V$ itself in the following manner. Note that there is a unique map $\sharp \colon V^* \to V$ for which $f\mapsto \blade{f^\sharp}$ such that
\begin{equation}
\blade{f_i^\sharp} \cdot \blade{v}_j = \delta_{ij}.
\end{equation}
Hence, if we simply put $\blade{v}^i \coloneqq \blade{f^\sharp}_i$ we can note that $\blade{v}^i$ is simply a vector in the geometric algebra.
\begin{definition}
Let $\blade{v}_1,\blade{v}_2,\dots,\blade{v}_n$ be an arbitrary basis of $V$ generating $\G$. Then we have the \emph{reciprocal basis} $\blade{v}^1,\blade{v}^2,\dots,\blade{v}^n$ satisfying
\begin{equation}
    \blade{v}^i\cdot \blade{v}_j = \delta^i_j,
\end{equation}
and we refer to each $\blade{v}^i$ as a \emph{reciprocal vector}.
\end{definition}
In terms of the inner product $g$, we have that the coefficients are given by $g_{ij}=\blade{v}_i\cdot \blade{v}_j$ and thus we have an explicit definition for the reciprocal vectors by putting $\blade{v}^i = g^{ij}\blade{v}_j$ where $g^{ij}$ is the coefficients to the matrix inverse $(g_{ij})^{-1}$ and we assume the Einstein summation convention. 

The inverse to this isomorphism is $\flat \colon V \to V^*$ which is given by $\blade{v} \mapsto v^\flat$ satisfying
\begin{equation}
v_i^\flat (\blade{v}_j)= \delta_{ij}.
\end{equation}
Given these identifications, there is no need to distinguish between the vector space $V$ and its dual $V^*$ as it suffices to consider $V$ itself with reciprocal vectors $\blade{v^i}$ with the application of the scalar product. For reference, the maps $\sharp$ and $\flat$ are the \emph{musical isomorphisms} \todo{sources}.

For a geometric algebra with a positive definite inner product, all blades have an inverse and hence form a group. With a pseudo inner product, the invertible elements are not quite as broad\todo{give an example later}. To this end, we can construct a group of all invertible elements referred to as the \emph{Clifford group} $\Gamma(\G)$ for an arbitrary geometric algebra $\G$ by
\begin{equation}
\Gamma(\G) \coloneqq \left\{ \prod_{j=1}^k \blade{v}_j ~\vert~ k\in \mathbb{Z}^+,~ \forall j~\colon1\leq j \leq k~\colon~\blade{v}_i \in V ~\textrm{such that}~ g(\blade{v}_i,\blade{v}_i)\neq 0\right\}.
\end{equation}
We refer to elements of the Clifford group as \emph{Clifford multivectors}. Note that Clifford multivectors are not necessarily blades since the product used in the construction is not the exterior product $\wedge$. For any Clifford multivectors $A = \blade{v}_1 \cdots \blade{v}_k$ in the group $\Gamma$, we have that multiplicative inverse $A^{-1}$ is given by $A^{-1} = \blade{v}^k \dots \blade{v}^1$ as we can see that $A^{-1}A=AA^{-1} = 1$ by construction.  Another note is that all scalars, vectors, pseudovectors, and pseudoscalars are always in the Clifford group and have multiplicative inverses. The inverse of a vector $\blade{v}$ is given by $\frac{\blade{v}}{\blade{v}\cdot\blade{v}}$. The form of the inverse motivates the utility of the \emph{reverse} operator $\dagger$ defined so that $A^\dagger = \blade{v}_k \cdots \blade{v}_1$. For a $r$-blade $A_r$, the reverse also satisfies the relationship
\begin{equation}
\label{eq:reverse_sign}
A_r^\dagger = (-1)^{r(r-1)/2} A_r
\end{equation}
as well as
\begin{equation}
\label{eq:dagger_distribution}
(AB)^\dagger = B^\dagger A^\dagger.
\end{equation}
One can see that the multiplicative inverse of an element of the Clifford group $A$ is the reverse of the corresponding product of reciprocal vectors since $A_r^{-1} = (\blade{v}^1 \cdots \blade{v}^k)^\dagger$. When we take $V=\R^n$ with the Euclidean inner product, we can note that elements $s \in \Gamma^+(\G_n)$ act as rotations on multivectors $A\in \G_n$ through a conjugate action
\begin{equation}
A \mapsto s A s^{-1}.
\end{equation}
In fact, all nonzero vectors $\blade{v}\in\Gamma(\G_n)$ define a reflection in the hyperplane perpendicular to $\blade{v}$ via the same conjugation action above. This allows one can realize that all rotations are even products of reflections.

Following these realizations, one can see that the Clifford group $\Gamma(\G)$ contains important subgroups such as the orthogonal and special orthogonal groups as quotients \todo{Change these to $O(V)$ and stuff for example}
\begin{equation}
\operatorname{O}(V) \cong \Gamma(\G)/(\R\setminus 0) \qquad \textrm{and} \qquad \operatorname{SO}(V) \cong \Gamma^+(\G) /(\R\setminus 0),
\end{equation}
where $\R\setminus 0$ is the multiplicative group of real numbers. We give the name \emph{unit} to $r$-blades $\blade{A_r}$ with unit Clifford norm $1=|\blade{A_r}|$. Finally, this allows us to arrive at a definition for the classical pin and spin groups.
\begin{definition}
\begin{subequations}
The \emph{pin} and \emph{spin} groups $\operatorname{Pin}(V)$ and $\operatorname{Spin}(V)$ are defined to be
\begin{align}
    \operatorname{Pin}(V) &\coloneqq \{s\in \Gamma(\G) ~\vert~ |s|=1\}.\\
    \operatorname{Spin}(V) &\coloneqq \{s\in \Gamma^+(\G) ~\vert~ |s|=1\}.
\end{align}
\end{subequations}
\end{definition}

Our focus will be the case where we take $\G=\G_n$ for which we put $\spingroup$, but these statements can often be more broadly generalized. Moreover, we can realize this group as a quotient of the Clifford group $\Gamma(\G_n)$ by
\begin{equation}
\operatorname{Spin}(V) \cong \Gamma^+(\G)/\R_+,
\end{equation}
where $\R_+$ is the multiplicative group of positive real numbers. The spin group $\operatorname{Spin}(V)$ is a Lie group usually derived via a short exact sequence of groups
\begin{equation}
1 \to \mathbb{Z}/2\mathbb{Z} \to \operatorname{Spin}(V) \to \operatorname{SO}(V) \to 1.
\end{equation}
Here, we have given a more concrete realization of the spin group as special elements inside a geometric algebra. The Lie algebra of the spin group is denoted by $\mathfrak{spin}(V)$ and $\mathfrak{spin}(n)$ when referencing $\spingroup$. This algebra typically characterized as the tangent space of $\operatorname{Spin}(V)$ at the identity. However, through this approach, we realize that $\mathfrak{spin}(V)$ is isomorphic to the algebra of bivectors with the antisymmetric product $\times$\todo{provide a citation.}.  Then, for any bivector $B$, we can generate an element in the spin group given via the exponential
\begin{equation}
e^{B} = \sum_{j=0}^\infty \frac{B^n}{n!}.
\end{equation}
Fundamentally, the even subalgebra $\G^+$ is invariant under the action of $\operatorname{Spin}(V)$ since all elements in both sets are of even grade. This definition follows.
\begin{definition}
Let $\G$ be a geometric algebra with an inner product of arbitrary signature, then we define a \emph{spinor} to be an element of $\G^+$.
\end{definition}
Morally, this definition is telling us $\psi \in \G^+$ is an element that transforms under a left action of an element of $\operatorname{Spin}(V)$ to produce another spinor which leaves us with a convenient definition in that a spinor is simply an even multivector. Or, in other words, we realize that $\G^+$ is really a left $\operatorname{Spin}(V)$ module. Likewise, it motivates the name of surface spinor for the multivectors consisting of only grade-0 and grade-2 elements. For more on the topic, see \cite{janssens_special_nodate}.

\todo{spinors are really a module. The odd subspace is also a similar module. Maybe reference superalgebra and physics again a little bit.}

\subsection{Pseudoscalars and duality}
\label{subsubsec:duality_and_pseudoscalars}

Pseudoscalars are a deeply useful aspect of geometric algebra and we will now cover some of their utility. First and foremost, these pseudoscalars grant us a means of determining volumes. This will be a necessary notion in order to define integration in \cref{subsec:integration_on_submanifolds}.
\begin{definition}
Let $\G$ be a geometric algebra, then the \emph{volume element} in the arbitrary basis $\blade{v}_1,\dots,\blade{v}_n$ is 
\begin{equation}
\blade{\mu}=\blade{v}_1 \wedge \blade{v}_2 \wedge \cdots \wedge \blade{v}_n.
\end{equation}
\end{definition}
It is worth noting that all volume elements and pseudoscalars are invertible in any geometric algebra. 

We also want to note that the volume element here fits our intuition and indeed we find
\begin{equation}
\label{eq:pseudoscalar_norm}
|\blade{\mu}| = \sqrt{\det(g)}.
\end{equation}
Since pseudoscalars are generated by a single element (recall there are ${n \choose n}$ independent grade-$n$ elements), we should realize that the volume element is simply a scalar copy of a pseudoscalar that is unital.
\begin{definition}
Let $\blade{\mu}$ be the volume element, then we have the \emph{unit pseudoscalar}
\begin{equation}
\blade{I} \coloneqq \frac{1}{|\blade{\mu}|} \blade{\mu}.
\end{equation}
\end{definition}
As is clear by the definition above, we must have that
\begin{equation}
|\blade{I}| = 1.
\end{equation}
The unit pseudoscalar satisfies a useful relationship when swapping the left for right multiplication with an $r$-vector by
\begin{equation}
\blade{I} A_r = (-1)^{r(n-1)} A_r \blade{I}.
\end{equation}
Thus, $\blade{I}$ always commutes with the even subalgebra and the commutation property with the odd subalgebra depends on the dimension. Then, we can note
\begin{equation}
\blade{I}^2 = (-1)^{n(n-1)/2+p},
\end{equation}
which lets us see that the inverse is given by
\begin{equation}
\blade{I}^{-1} = (-1)^{n(n-1)/2+p} \blade{I},
\end{equation}
which is an identification that we will often use. Formulas throughout are usually given in their most general context and substitution is done only when working with specialized algebras. From here, one notices that when $g$ is positive definite we have no temporal vectors and $p=0$ which means $\blade{I}^\dagger = \blade{I}^{-1}$.

Note that for a homogeneous $r$-rector $A_r$ we have that $A_r^\perp$ is an $n-r$-vector. Indeed, if we take an invertible $r$-blade $\blade{A_r}$, then we can find the \emph{$\blade{A_r}$-subspace dual} of a multivector $B$ by
\[
B \rfloor \blade{A_r}^{-1}.
\]
The notions of duality here give us geometrical insight. Taking an $s$-blade $\blade{B_s}$ we can note:
\begin{itemize}
    \item If $s>r$, the $\blade{A_r}$-subspace dual of $\blade{B_s}$ vanishes.
    \item If $s=r$, the $\blade{A_r}$-subspace dual of $\blade{B_s}$ is a scalar and is zero if $\blade{B_s}$ contains a vector orthogonal to $\blade{A_r}$.
    \item If $s<r$, the $\blade{A_r}$-subspace dual of $\blade{B_s}$ represents the orthogonal complement of the subspace corresponding to $\blade{B_s}$ in the subspace corresponding to $\blade{A_r}$.
\end{itemize}  
Since the pseudoscalar is a blade representing the entire vector space, this allows one to create dual elements within the entire vector space. 
\begin{definition}
Given a multivector $B$, we define the \emph{dual} of $B$ to be
\begin{equation}
B^\perp \coloneqq B \rfloor \blade{I}^{-1} \equiv B\blade{I}^{-1}.
\end{equation}
\end{definition}
The dual allows one to exchange interior and exterior products in the following way.
\begin{equation}
\label{eq:wedge_to_dot}
 (A \wedge B)^\perp  = A\rfloor B^\perp
\end{equation}
\begin{equation}
\label{eq:dot_to_wedge}
    (A\rfloor B)^\perp = A \wedge B^\perp
\end{equation}
This shows the natural duality between the inner and exterior products and their interpretations as subspace operations. The duality extends further to provide an isomorphism between the spaces of $r$-vectors and $n-r$-vectors since for any $r$-vector $A_r$ we have $A_r^\perp$ is an $n-r$-vector. It is under this isomorphism one can realize that all pseudovectors are $n-1$-blades. Furthermore, for multivectors $A$ and $B$,
\begin{equation}
(AB)^\perp = AB^\perp
\end{equation}
For those familiar with the Hodge star operator, $\star$, this should feel familiar. This is discussed in \cref{subsec:differential_forms}.

\begin{remark}
\label{rem:cross_product}
If we consider $\spacealg$, we can realize the cross product of two vectors $\blade{u}$ and $\blade{v}$ by
\begin{equation}
\label{eq:cross_product}
\blade{u} \cross \blade{v} \coloneqq (\blade{u}\wedge \blade{v})^\perp \equiv \blade{u} \rfloor \blade{v}^\perp
\equiv (\blade{u}^\perp)\times (\blade{v}^\perp) , 
\end{equation}
where we use the bold notation for $\cross$ to distinguish between the bivector commutator product $\times$ defined in \cref{eq:commutator_product}. The special fact of $\spacealg$ that is abused in a standard multivariate calculus course is that vectors and bivectors are dual to one another. In fact, the first equality is exactly this pedagogical reasoning; the cross product returns a vector perpendicular to the subspace spanned by the two input vectors and is zero when the two inputs are linearly dependent. One can also note that the vector $\blade{w}=\blade{u}\cross \blade{v}$ is sometimes refered to as axial and in other cases the pseudovector $\blade{u}\wedge \blade{v}$ is referred to as axial. The similar product notation of $\times$ and $\cross$ now becomes transparent. 
\end{remark}


\subsection{Blades and subspaces}
\label{subsubsec:blades_and_subspaces}

Each invertible unit $r$-blade $\blade{U_r}$ ($|\blade{U_r}|=1$) corresponds to a $r$-dimensional subspace and can be identified with a point in the Grassmannian of $r$-dimensional subspaces in an $n$-dimensional vector space, $\Grassmannian{r}{n}$. We will often allude to this identification directly by referring to a subspace via a reference to a unit blade, e.g., the subspace $\blade{U_r}$. Extending the dual to act on the unit $r$-blades that make up $\Grassmannian{r}{n}$, one realizes that $\Grassmannian{r}{n}^\perp = \Grassmannian{n-r}{n}$ shows the spaces are in bijection. Moreover, given a subspace $\blade{U_r}$, we can complete the vector space by
\begin{equation}
\blade{U_r}\wedge \blade{U_r}^\perp = \blade{I}.
\end{equation} 
We can also note that any invertible blade $\blade{A_r}$ is simply a scaling of some unit blade so that $\blade{A_r} = \alpha \blade{U_r}$. This interpretation also proves to be a wonderfully geometrical perspective on the products defined in \cref{eq:dot,eq:wedge,eq:left_contraction,eq:right_contraction}. For example, we see that there are a handful of reasons to adopt the additional multiplication symbols $\rfloor$ and $\lfloor$. 
\begin{itemize}
    \item The products $\rfloor$ and $\lfloor$ allow us to avoid needing to pay special attention to the specific grade of each multivector in a product. The product $\cdot$ on $A_r$ and $B_s$ depends on $k$ and $s$ and as such given by either $\rfloor$ or $\lfloor$ but one must know $k$ and $s$ in order to define this product exactly. 
    \item We gain geometrical insight on the structure of $r$-blades in terms of their corresponding subspaces. Let $\blade{A_r}$ and $\blade{B_s}$ be nonzero blades with $r,s\geq 1$ then
    \begin{itemize}
        \item $\blade{A_r} \rfloor \blade{B_s} =0$ iff $\blade{A_r}$ contains a nonzero vector orthogonal to $\blade{B_s}$.
        \item If $r<s$ then if $\blade{A_r}\rfloor \blade{B_s} \neq 0$ then the result is a $s-r$-blade representing the orthogonal complement of $\blade{A_r}$ in $\blade{B_s}$.
        \item If $\blade{A_r}$ is a subspace of $\blade{B_s}$ then $\blade{A_r}\blade{B_s} = \blade{A_r}\rfloor \blade{B_s}$.
        \item If $\blade{A_r}$ and $\blade{B_s}$ are orthogonal, then $\blade{A_r}\blade{B_s} = \blade{A_r} \wedge \blade{B_s}$.
    \end{itemize}
\end{itemize}

We also have the equivalences
\begin{equation}
\label{eq:left_contraction_dot}
A_r \cdot B_s \equiv A_r \rfloor B_s \qquad \textrm{if $k\leq s$}
\end{equation}
\begin{equation}
\label{eq:right_contraction_dot}
A_r \cdot B_s \equiv A_r \lfloor B_s \qquad \textrm{if $k\geq s$}.
\end{equation}
For homogeneous $r$-vectors $A_r$ and $B_r$, the products above simplify to 
\begin{equation}
\label{dot_equivalent_contraction}
    A_r \cdot B_r \equiv A_r \lfloor B_r \equiv A_r \rfloor B_r.
\end{equation}
In fact, if we are given two $r$-blades $\blade{A_r} = \blade{a_1} \wedge \cdots \wedge \blade{a_r}$ and $\blade{B_r} = \blade{b_1} \wedge \cdots \wedge \blade{b_r}$ we have the 
\begin{equation}
\label{eq:dot_product}
\blade{A_r} \cdot \blade{B_r}^\dagger = \det(\blade{a_i} \cdot \blade{b_j} )_{i,j=1}^r = \blade{A_r}^\dagger \cdot \blade{B_r},
\end{equation}
which is the typical extension of the inner product $g$ to an inner product on $\bigwedge^r (V)$ through linearity.

Given the direct relationship between unit $r$-blades and $r$-dimensional subspaces we can also form a compact way of projecting multivectors into subspaces in a manner closely related to the subspace dual.  \begin{definition}
Given an multivector $B$, the \emph{projection} onto the subspace $\blade{A_r}$ is
\begin{equation}
\label{eq:projection}
\projection_{\blade{A_r}}(B) \coloneqq B\rfloor \blade{A_r} \blade{A_r}^{-1} \equiv (B\rfloor \blade{A_r})\rfloor \blade{A_r}^{-1}
\end{equation}
\end{definition}
Following this definition, one can see that
\begin{equation}
\projection_{\blade{A_r}}(B) \in \bigoplus_{j=0}^r \G^j = \G^{0+\cdots + r},
\end{equation}
since the subspace $\blade{A_r}$ is $r$-dimensional and moreover the operation preserves grades since
\begin{equation}
\projection_{\blade{A_r}}(\proj{j}{B}) \in \G^j.
\end{equation}
For example, given vectors $\blade{u}$ and $\blade{v}$ we retrieve the familiar statement 
\begin{equation}
\projection_{\blade{u}} (\blade{v}) = (\blade{v} \cdot \blade{u}) \frac{\blade{u}}{|\blade{u}|^2}.
\end{equation}

A dual notion exists as well; we can project onto the subspace perpendicular to $\blade{A_r}$.
\begin{definition}
Given a multivector $B$, the \emph{rejection} from the subspace $\blade{A_r}$ is
\begin{equation}
\label{eq:rejection}
\rejection_{\blade{A_r}}(B) \coloneqq B \wedge \blade{A_r} \blade{A_r}^{-1} \equiv (B\wedge \blade{A_r})\lfloor \blade{A_r}^{-1}.
\end{equation}
\end{definition}
Note that this operation is also grade preserving. In the case we have a blade $\blade{C_k}$ with $k<r$ and $k<n-r$, we can note
\begin{equation}
\label{eq:projection+rejection_blade}
\blade{C_k}=\projection_{\blade{A_r}}(\blade{C_k}) + \rejection_{\blade{A_r}}(\blade{C_k}).
\end{equation}
Another useful result follows.
\begin{proposition}
Let $\G$ come with a positive definite $g$, let $\blade{A_{n-1}}$ unit pseudovector, and let $\blade{B_r}$ be an $r$-vector for $r\leq n-1$. Then,
\begin{equation}
|\blade{B_r}| = \left|\projection_{\blade{A_{n-1}}} (\blade{B_r})\right| + \left|\projection_{\blade{A_{n-1}}}(\blade{B_r}^\perp)^\perp\right|.
\end{equation}
\end{proposition}
\begin{proof}
Take an orthonormal basis $\blade{e}_1,\dots,\blade{e}_n$ for $\G$. Let $J$ be an increasing index set of length $r$, i.e., $J=\{j_1,j_2,\dots,j_r\}$ with $1\leq j_1 < j_2 <\cdots < j_r < n$ and define $\blade{E}_J = \blade{e}_{j_1}\blade{e}_{j_2}\cdots \blade{e}_{j_r}$ with $r\leq n-1$. Then note that any unit pseudovector $\blade{A_{n-1}}$ is a blade and so we can put $\blade{A_{n-1}} = \normal^\perp$ for some unit vector $\normal$.
Note that if $J$ contains $k$ then $\projection_{\blade{A_{n-1}}}(\blade{E}_J)$ is zero and otherwise this product is the identity since $\blade{E}_J$ lies in the subspace of $\blade{A_{n-1}}$. Next,
\begin{align}
\projection_{\blade{A_{n-1}}}(\blade{E}_J^\perp)^\perp &= \left( \blade{E}_J^\perp \rfloor \blade{A_{n-1}} \blade{A_{n-1}}^{-1}\right)^\perp\\
&= \blade{E}_J^\perp \rfloor \blade{A_{n-1}} (\normal \pseudoscalar^{-1})^{-1} \pseudoscalar^{-1}\\
&= (-1)^{n-1} \blade{E}_J^\perp \rfloor (\normal^\perp) \normal\\
&= (-1)^{n-1} (\blade{E}_J^\perp \wedge \normal )^\perp \normal\\
&= (-1)^{(r+1)(n-1)}(\normal \wedge \blade{E}_J^\perp)^\perp \normal\\
&= (-1)^{(n-1)(n+2r+2)/2} (\normal \rfloor \blade{E}_J) \normal.
\end{align}
Note that if $J$ does not contain $k$, then $\normal \rfloor \blade{E}_J=0$ and otherwise the product is the identity up to sign. 

Since this holds for $\blade{E}_J$, we note that we can write $B = \sum_J B_J \blade{E}_J$ so that $B$ is a sum of these basis blades. By linearity of $\projection$, we have proven the proposition.
\end{proof}
\todo{This proof may be true for multivectors not just $r$-vectors. Also, it may be better to say $|\projection_{\blade{A_{n-1}}}(B^\perp)|$ instead and use chisolm 187.}
The projection and rejection provide us a way to revisit the geometric notions of the interior and exterior products. In particular, we note that
\begin{align}
\label{eq:projection_inner_product}
    B \rfloor \blade{A_r} &= \projection_{\blade{A_r}} (B) \blade{A_r}\\
    B \wedge \blade{A_r} &= \rejection_{\blade{A_r}}(B) \blade{A_r}.
\end{align}
Both the notion of projection and rejection prove to be useful and behave nicely with the dual by
\begin{equation}
\label{eq:projection_rejection_duality}
\projection_{\blade{A_r}^\perp}(B) = \rejection_{\blade{A_r}}(B).
\end{equation}
\todo{prove this.}
Finally, the exterior product of orthogonal blades gives us a direct sum of subspaces in the following sense. Let $\blade{A_r}$ and $\blade{B_s}$ be orthogonal so that $\blade{A_r}\wedge \blade{B_s}=\blade{A_r}\blade{B_s}$, then we can note that if $k<r$ and $k<s$ we have
\begin{equation}
\label{eq:projection_sum_of_subspaces}
    \projection_{\blade{A_r}\wedge \blade{B_s}} (\blade{C_k}) = \projection_{\blade{A_r}}(\blade{C_k}) + \projection_{\blade{B_s}}(\blade{C_k}).
\end{equation}
Perhaps it is most enlightening for the reader to revisit \cref{eq:projection+rejection_blade,eq:projection_sum_of_subspaces} replacing $\blade{C_k}$ with a vector $\blade{v}$ since a vector will always prove to be a representative for a ``small enough" subspace.


\subsection{Motivating example}
\label{subsec:motivating_example}

Rather than a sequence of multiple examples, it will prove to be far more illuminating to construct one large example for which most of the preliminaries to this point can be used in a meaningful way. As such, we shall not rule out the utility of geometric algebras with pseudo inner products. The classical example is the \emph{spacetime algebra} defined by taking $V=\R^4$ with a vector basis $\blade{\gamma}_0,\blade{\gamma}_1,\blade{\gamma}_2,\blade{\gamma}_3$ satisfying
\begin{subequations}
\begin{align}
\blade{\gamma}_0 \cdot \blade{\gamma}_0 &= -1\\
\blade{\gamma}_0 \cdot \blade{\gamma}_i &= 0  &i=1,2,3\\
\blade{\gamma}_i \cdot \blade{\gamma}_j &= \delta_{ij}, &i,j=1,2,3.
\end{align}
\end{subequations}
Where $\blade{\gamma}_0$ is temporal since its square is negative and $\blade{\gamma}_i$ for $i=1,2,3$ are all spatial since their squares are positive. For this basis, we can denote the matrix for this inner product $\eta =\operatorname{diag}(-+++)$ (often called the \emph{Minkowski metric}) and define $Q$ from $\eta$. Then, we have for a spacetime vector $\blade{v} = v_0 \blade{\gamma}_0 +v_1 \blade{\gamma}_1 + v_2 \blade{\gamma}_2 + v_3 \blade{\gamma}_3$,
\begin{equation}
\label{eq:spacetime_inner_product}
|\blade{v}| = \blade{v} \cdot \blade{v} = -v_0^2 + \sum_{i=1}^3 v_i^2,
\end{equation}
which defines the algebra $\G_{1,3}$ as the spacetime algebra. The reader may now wish to, for example, revisit \cref{subsubsec:reverse_inverse_clifford_spin_groups} with $\G_{p,q}$ in mind in order to see a realization of the groups $\operatorname{SO}(p,q)$, $\operatorname{Spin}(p,q)$, and the spacetime spinors. 

As the naming above suggests, the geometric algebra of Euclidean space, $\G_3$, should naturally inside of the spacetime algebra. Note that we have the \emph{spatial pseudoscalar} $\blade{I}_S \coloneqq \blade{\gamma}_1 \wedge \blade{\gamma}_2 \wedge \blade{\gamma}_3$, which, allowing for an extension of our notion of projection to the whole algebra, allows us to put
\begin{equation}
\projection_{\blade{I}_S}(\G_{1,3}) \equiv \rejection_{\blade{\gamma}_0} (\G_{1,3}) = \G_3.
\end{equation}
Perhaps one could refer to this mapping as the \emph{static map} as we project only onto the spatial subspace and, via duality, reject the temporal subspace. It is also worth noting that this static map is not just producing an isomorphic copy of $\G_3$, but a a copy of $\G_3$ directly. Now, in $\G_3$, we can specify an arbitrary multivector $A$ by
\begin{equation}
A= a_0 + a_1 \blade{\gamma}_1 + a_2 \blade{\gamma}_2 + a_3 \blade{\gamma}_3 + a_{12} \blade{B}_{12} + a_{13} \blade{B}_{13} + a_{23} \blade{B}_{23} + a_{123} \blade{\gamma}_1 \wedge \blade{\gamma}_2 \wedge \blade{\gamma}_3,
\end{equation}
and so the grade projections read
\begin{subequations}
\begin{align}
\proj{}{A}&=a_0\\
\proj{1}{A}&=a_1 \blade{\gamma}_1 + a_2 \blade{\gamma}_2 + a_3 \blade{\gamma}_3\\
\proj{2}{A}&=a_{12} \blade{B}_{12} + a_{13} \blade{B}_{13} + a_{23} \blade{B}_{23}\\
\proj{3}{A}&= a_{123} \blade{\gamma}_1 \wedge \blade{\gamma}_2 \wedge \blade{\gamma}_3.
\end{align}
\end{subequations}
Then, we can write a even multivector as
\begin{equation}
q = q_0 + q_{23}\blade{B}_{23} + q_{31} \blade{B}_{31} + q_{12} \blade{B}_{12}.
\end{equation}
Note as well that
\begin{subequations}
\begin{align}
\blade{B}_{23}^2 = \blade{B}_{31}^2 = \blade{B}_{12}^2 &= -1\\
\blade{B}_{23}\blade{B}_{31}\blade{B}_{12} &= +1,
\end{align}
\end{subequations}
which is typical for spatial bivectors. In this case, one may notice that this even subalgebra is extremely close to being a copy of the quaternion algebra $\quat$. One can arrive at a representation of the quaternions by taking
\begin{equation}
\boldsymbol{i} \leftrightarrow \blade{B}_{23}, \quad \boldsymbol{j} \leftrightarrow -\blade{B}_{31}=\blade{B}_{13}, \quad \boldsymbol{k} \leftrightarrow \blade{B}_{12},
\end{equation}
and noting that we then have $\boldsymbol{ijk}=-1$ as well as $\boldsymbol{i}^2=\boldsymbol{j}^2=\boldsymbol{k}^2=-1$. A more in depth explanation is provided in \cite{doran_geometric_2003}. Thus, we realize a quaternion as a spinor $q$ and a purely imaginary quaternion is simply the grade-2 portion of the spinor $\proj{2}{q}$. We also realize $\quat$ as scalar copies of elements of $\operatorname{Spin}(3) \cong \operatorname{Sp}(1)$. That is to say that $\quat \cong \R \times \operatorname{Spin}(3)$. Indeed, since elements of $\G_3^+$ are simply surface spinors, the surface spinors admit a natural spin representation.

But we are not done here, and we can project down one dimension further by
\begin{equation}
    \projection_{\blade{\gamma}_1 \wedge \blade{\gamma}_2} (\G_3) = \G_2.
\end{equation}
To see this in action, we let $\blade{v}=v_1 \blade{\gamma}_1 + v_2 \blade{\gamma}_2 + v_3 \blade{\gamma}_3$
\begin{subequations}
\begin{align}
    \projection_{\blade{\gamma}_1 \wedge \blade{\gamma}_2} = \projection_{\blade{B}_{12}} (\blade{v}) &= (v_1 \blade{\gamma}_1 + v_2 \blade{\gamma}_2 + v_3 \blade{\gamma}_3)\rfloor \blade{B}_{12}\blade{B}_{12}^{-1}\\
    &= (v_1 \blade{\gamma}_2 - v_2 \blade{\gamma}_1)\blade{B}_{12}^{-1} \\
    &= v_1 \blade{\gamma}_1 + v_2 \blade{\gamma}_2.
\end{align}
\end{subequations}
Then, arbitrary multivectors $A$ and $B$ can be specified by
\begin{equation}
A = a_0 + a_1 \blade{\gamma}_1 + a_2 \blade{\gamma}_2 + a_{12} \blade{B}_{12}, \qquad B = b_0 +b_1 \blade{\gamma}_1 + b_2 \blade{\gamma}_2 + b_{12}\blade{B}_{12}.
\end{equation}
We can then take the product $AB$ to yield
\begin{subequations}
\begin{align}
\proj{0}{AB} &= a_0b_0 + a_1 b_1 + a_2 b_2 - a_{12}b_{12}\\
\proj{1}{AB} &= (a_0 b_1 + a_1 b_0 - a_2 b_{12} + a_{12} b_2) \blade{\gamma}_1 + (a_0 b_2 + a_2 b_0 + a_1b_{12} - a_{12} b_1) \blade{\gamma}_2\\
\proj{2}{AB} &= (a_1b_2 - a_2 b_1)\blade{B}_{12}.
\end{align}
\end{subequations}
Most notably, we see that $\blade{B}_{12}^2=-1$ and this allows us to consider a spinor
\begin{equation}
z = x + y \blade{B}_{12}
\end{equation}
which is exactly a representation of the complex number $\zeta = x+ iy$ in $\G_2^{0+2}=\G_2^+$.  Thus, the even subalgebra of this geometric algebra is indeed isomorphic to the complex numbers $\C$. Indeed, there is one unit 2-blade $\blade{B}_{12}$ in $\G_2$ to form the spin algebra $\mathfrak{spin}(2) \cong \R$ and as a consequence all unit norm elements in $\G_2^+$ can be written as
\begin{equation}
   e^{\theta \blade{B}_{12}} = \sum_{n=0}^\infty \frac{\theta \blade{B}_{12}}{n!} = \cos(\theta)+\blade{B}_{12}\sin(\theta),
\end{equation}
where $\theta \blade{B}_{12}$ is a general bivector in $\G_2$ when $\theta \in \R$ is arbitrary. Hence, we arrive at $\operatorname{Spin}(2)\cong \operatorname{U}(1)$. Any element in $\C$ is also a scaled version of an element of the spin group $\operatorname{Spin}(2)$. Hence, we can use a spin representation for an element in $\C$ via $z=re^{\theta \blade{B}_{12}} \in \R\times \operatorname{Spin}(2)$.  This special case shows that spinors in $\G_2$ have a unique spin representation.

But, the above work is not necessary special to the starting point of $\G_{1,3}$ or $\G_3$. In fact, if we take $\G_n$ for $n\geq 2$, then there are natural copies of $\C$ contained inside of $\G_n$. In particular, we have the isomorphism
\begin{equation}
\label{eq:c_isomorphisms}
    \C \cong \{x + y \blade{B} ~\vert~ x,y \in \G_n^0,~ \blade{B} \in \Grassmannian{2}{n}. \},
\end{equation}
which shows that complex numbers arise as surface spinors via the representation
\begin{equation}
        \zeta = x + y\blade{B},
\end{equation}
since $\blade{B}^2=-1$. Given the standard basis $\blade{e_1},\dots,\blade{e_n}$ we have copies of $\C$ for each of the ${ n \choose 2}$ unit bivectors $\blade{B}_{jk}$ with $k=2,\dots,n$ and $j<k$.



\todo{Talk about bivectors, spinors, and rotors. Rotations and what not. Euler angles. Would all be good to put in here. Rotations in the complex plane.}

\section{Geometric manifolds}
\label{sec:geometric_manifolds}
We want to generalize the setting of geometric algebra to include a smooth structure. For instance, we can consider a manifold $M$ (likely with boundary $\partial M$) with a metric structure and develop a geometric algebra at each tangent space to this manifold (e.g., following \cite{schindler_geometric_2020}). We refer to this as the \emph{geometric tangent space} and put $C\ell(T_xM,g_x)$.
\begin{definition}
A manifold $M$ with a pseudo-Riemannian metric $g$ is a \emph{geometric manifold} if each tangent space is a geometric tangent space.
\end{definition}
On geometric manifolds we will be able to attach multivector fields and compute their derivatives as well as integrate. This leads us to the realm of geometric calculus and Cifford analysis. Geometric calculus is intimately related to both the vector calculus in $\R^3$ and differential forms. It has the added advantage of notational convenience and clarity as we have seen with geometric algebra and its subspace operations. In the beginning of \cref{subsec:clifford_and_geometric_algebras} we realize as well that the exterior algebra is contained inside any Clifford algebra and, to this end, geometric calculus will contain the calculus of differential forms. 

Forms are a useful language for proving general theorems about boundary value problems \cite{schwarz_hodge_1995}, and so we will retrieve all of these theorems for our own utility. Given that we have increased geometrical intuition on different graded elements of a geometric algebra, we can realize that we can work with multivector equivalents of forms instead of concentrating on forms of a specific grade. For example, in \todo{reference electromagnetic stuff later on} we see that one can think of the electromagnetic field as a multivector consisting of elements of various degree as opposed to the usual field strength 2-form \todo{cite a typical electromagnetic paper}. In fact, under certain other restrictions such as those present in Ohmic materials, we find there are parabivectors that fall into the kernel of a Dirac-type operator.

This Dirac-type operator, $\grad$, is the grade-1 derivative operator studied in Clifford analysis. Fundamentally, this operator generalizes the Wirtinger derivative for complex functions to multivectors and, as such, generalizes the notion of a $\C$-holomorphic function to that of a monogenic function (see \todo{refence later}). Happily, we even retain a Taylor series representation (see \todo{reference later}) for functions in the kernel of $\grad$ due to a generalized form of the Cauchy integral formula. This Cauchy integral formula has been applied elsewhere (see \cite{brackx_hilbert_2008}). The Cauchy integral also acts as an isomorphism between smooth functions defined on the boundary $\partial M$ of a manifold $M$. 

\subsection{Multivector fields}

In order to develop fields on a geometric manifold we must first create the relevant bundle structure. There is a natural bundle associated to a  geometric manifold given by by gluing together each of the tangent geometric algebras. The \emph{geometric algebra bundle} of a geometric manifold $(M,g)$ is the space
\begin{equation}
\bigsqcup_{x \in M} C\ell(T_xM,g_x).
\end{equation}
Given this bundle, the fields follow.
\begin{definition}
A \emph{(smooth) multivector field} is a ($C^{\infty}$-smooth) section of the geometric algebra bundle. We put $\G(M)$ as the \emph{space of multivector fields on $M$}.
\end{definition}
Note that the we will assume that all multivector fields are $C^\infty$-smooth and drop this additional modifier when speaking of any type of multivector field. The above definition above is very general and we may not find ourselves working over arbitrary geometric manifolds. For example, we highlight a specific use case by letting $M$ be a connected region of $\R^n$. For brevity, we will put $\mathcal{G}_n(M)$ to denote we are working over a region $M\subseteq \R^n$. In this case, the multivectors themselves are realized as constant multivector fields which allows us to say $\G_n \subset \G_n(M)$. This smooth setting simply makes the coefficients of the global basis blades given by $C^\infty$ functions as opposed to $\R$ scalars.  Hence, $\G_n(M)$ is simply the $C^{\infty}$-module equivalent of $\mathcal{G}_n$.

Perhaps the $C^\infty$-module structure obfuscates the point slightly, but the notion of a smooth section does not.  One should think of the fields in $\G_n(M)$ as multivector valued functions on $M \subset \R^n$. Taking this identification allows for an extended toolbox at our disposal. In particular, points in $M$ are uniquely identified with constant vector fields in $\G_n^1$ and one can consider endomorphisms living in $\G_n$ (acting on $\G_n^1$) as acting on the input of fields in $\G_n(M)$ as well (see \cref{rem:input_projection}).  Thus, there is not only an algebraic structure on the fields themselves, but on the point in which the field is evaluated.  This is perhaps the key insight on why authors developed the so-called vector manifolds widely used in the geometric algebra landscape. Fundamentally, this is true in all local coordinates for an arbitrary manifold $M$, but it is not a global phenomenon since not all manifolds admit everywhere nonzero constant vector fields. 

\begin{remark}
\label{rem:input_projection}
    If we consider a multivector field $f \in \G_n(\R^n)$. With $x\in \R^n$ being identified with the vector $\blade{x} \in \G_n^1$, we can safely put $f(\blade{x})$.  One may be interested in the restriction of the input of $f$ to a subspace $\blade{U_r}$ which yields $f(\projection_{\blade{U_r}}(\blade{x}))$.  
\end{remark}

As noted throughout \cref{subsec:clifford_and_geometric_algebras}, there are spaces of multivectors inside $\G$ of interest and each of these extends to their field counterpart. Construction of each is done pointwise and made global through the relevant bundle. Let us list the relevant spaces of fields.
\begin{itemize}
    \item The \emph{$r$-vector fields},
    \begin{equation}
        \G^r(M) \coloneqq \left\{\textrm{smooth sections of } \bigsqcup_{x \in M} C\ell(T_xM,g_x)^r\right\}; 
    \end{equation}
    \item The \emph{spinor fields},
    \begin{equation}
        \G^+(M) \coloneqq \left\{\textrm{smooth sections of } \bigsqcup_{x \in M} C\ell(T_xM,g_x)^+\right\};
    \end{equation}
    \item The \emph{parabivector fields},
    \begin{equation}
        \G^{0+2}(M) \coloneqq \left\{\textrm{smooth sections of } \bigsqcup_{x \in M}  C\ell(T_xM,g_x)^{0+2}\right\};
    \end{equation}
%    \item The \emph{Clifford fields},
%    \begin{equation}
%        \G^r(M) \coloneqq \left\{\textrm{smooth sections of } \bigsqcup_{p \in M} \Gamma(C\ell(T_pM,g_p))\right\};
%    \end{equation}
%    \item The \emph{spin group fields},
%    \begin{equation}
%        \G^r(M) \coloneqq \left\{\textrm{smooth sections of } \bigsqcup_{p \in M} \operatorname{Spin}(T_pM)\right\};
%    \end{equation}
%    \item The \emph{spin algebra fields},
%    \begin{equation}
%        \G^r(M) \coloneqq \left\{\textrm{smooth sections of } \bigsqcup_{p \in M} \mathfrak{spin}(T_pM)\right\};
%    \end{equation}
\end{itemize}
Our operations from \cref{subsec:clifford_and_geometric_algebras} must carry over. To that end, we simply define all the products seen in \cref{eq:dot,eq:wedge,eq:left_contraction,eq:right_contraction} to act pointwise in each geometric tangent space. Previously we referred to $r$-blades as special $r$-vectors. Thus, we realize an $r$-blade field $\blade{A_r} \in \G^r(M)$ assumes the same form of \cref{eq:blade} where the vectors $\blade{v}_j$ are to be understood as vector fields for which all $\blade{v}_j(x)$ are linearly independent in $T_xM$ at the point $x$.

Given local coordinates $x^i$ on $M$ containing the point $p$, the tangent vectors in a neighborhood about $p$ are induced from the coordinates by $\frac{\partial}{\partial x^i}$. However, this choice of basis may be canonical, but it is not arbitrary. Instead, at each point we can simply choose an arbitrary local vector basis $\blade{v}_i$ and let the components of the metric be given in this basis by $g_{{ij}(x)} = \blade{v}_i(x)\cdot \blade{v}_j(x)$. From here, we can suppress the pointwise notion and instead just put $g_{ij}=\blade{v}_i\cdot \blade{v}_j$ locally. This allows us to work notationally with bases in a global manner without any reference to coordinates, so long as we assume the understanding is clear -- these vector bases do only exist locally. If explicit computations are to be carried out, one can just take the canonical basis so that $\blade{v}_i=\frac{\partial}{\partial x^i}$. Thus, locally we have the reciprocal basis $\blade{v}^i=g^{ij}\blade{v}_j$, the reverse $\dagger$, dual $\perp$, projection $\projection$, and rejection $\rejection$ that act on multivector fields pointwise in $C\ell(T_xM,g_x)$. 

\subsection{Geometric calculus}

On $M$ we have the unique torsion free Levi-Civita connection $\nabla$ for which we can define the covariant derivative $\nabla_{\blade{u}}$ for a vector field $\blade{u}$. The covariant derivative is extended to act on multivector fields following \cite{schindler_geometric_2020}. We can note that $\nabla_{\blade{u}}$ is a grade preserving differential operator so that
\begin{align}
    \nabla_{\blade{u}} \proj{r}{A_r} = \proj{r}{\nabla_{\blade{u}} \proj{r}{A_r}},
\end{align}
and it is a dot-compatible and wedge-compatible operator since
\begin{align}
    \nabla_{\blade{u}} (A\cdot B) &= (\nabla_{\blade{u}} A) \cdot B + A \cdot (\nabla_{\blade{u}} B)\\
    \nabla_{\blade{u}} (A\wedge B) &= (\nabla_{\blade{u}} A) \wedge B + A \wedge (\nabla_{\blade{u}} B)
\end{align}
\begin{definition}
    Let $\blade{v}_i$ be an arbitrary basis, then the \emph{gradient} (or \emph{Dirac operator}) $\grad$ is defined by
\begin{equation} 
\grad = \sum_{i} \blade{v}^i \nabla_{\blade{v}_i}.
\end{equation}
\end{definition}
The space of multivector fields $\G(M)$ along with $\grad$ is usually referred to as geometric calculus. One should note that $\grad$ is acts as a grade-1 element. Thus, the gradient splits into two operators, 
\begin{align}
\grad \rfloor &\colon \G_n^r(M) \to \G_n^{r-1}(M), \\
\grad \wedge &\colon \G_n^r(M) \to \G_n^{r+1}(M),
\end{align}
which satisfy the properties
\begin{align}
\label{eq:differential_properties}
(\grad \wedge)^2=0,\\
(\grad \rfloor)^2 = 0,
\end{align}
when acting on a homogeneous $r$-vector. Since \ref{eq:differential_properties} holds, the gradient operator gives rise to the grade preserving \emph{Laplace-Beltrami operator}
\[
\Delta = \grad^2 = \grad \rfloor \circ \grad \wedge + \grad \wedge \circ \grad \rfloor,
\]
which is manifestly coordinate invariant by definition.  It also motivates the use of the physicist notation $\grad^2=\Delta$, but we do not adopt this here.  We refer to multivector fields $f$ in the kernel of the Laplace-Beltrami operator \emph{harmonic multivector fields} or simply as \emph{harmonic}.

Note that since Euclidean space $\R^n$ has global orthonormal coordinates $\blade{e}_i$ we can choose a global constant vector field basis since we identified $\G_n^1$ with $\G(\R^n)^1$. With respect to these fields, we have the that $\nabla_{\blade{u}}$ reduces to the directional derivative. Note then that $\blade{u} \cdot \grad = \nabla_{\blade{u}}$ defines the directional derivative via the gradient. In fact, given a subspace $\blade{U_r}$, one could even describe a derivative in $\blade{U_r}$ by $\projection_{\blade{U_r}}(\grad)$.

\subsection{Differential forms}
\label{subsubsec:differential_forms}

The language of differential forms rests neatly inside geometric calculus. We will develop the relationship between multivectors and forms which will serve as a link between the two notions so that researchers with interest in Clifford analysis can communicate with those who study forms. In order to do so, we appeal to the language of differential forms and build a relationship between multivector fields and forms through measures. Forms have their appeal in global understanding via their properties through integration (e.g., Stokes' and Green's theorems) and the exterior calculus along with de Rham cohomology will provide us a larger toolbox.

Given coordinates $x^i$ on $M$ we have the local basis tangent vector fields $\blade{v}_i=\frac{\partial}{\partial x^i}$ with the corresponding 1-forms $dx^i$ that are each local sections of $T^*M$ and are the exterior derivatives (or gradients) of the coordinate functions.  Typically, 1-forms are viewed as linear functionals on tangent vectors and in these coordinates we have $dx^i  (\partial_j) = \delta^i_j$ and one can thus take a pairing of 1-form fields and vector fields and integrate over 1-dimensional submanifolds. The benefit of this definition is that the 1-forms $dx^i$ carry a natural measure and we can form product measures via the exterior product $\wedge$.

Let $M$ be an $n$-dimensional pseudo Riemannian manifold with metric $g$, let $\Omega(M)$ be the exterior algebra of smooth form fields on $M$, and let $\Omega^r(M)$ be the space of smooth $r$-form fields on $M$. Then we have the Riemannian volume measure $\omega \in \Omega^n(M)$ given in local coordinates by
\begin{equation}
\omega = \sqrt{|g|} dx^1\dots dx^n.
\end{equation}
\begin{definition}
The \emph{$r$-dimensional directed measure} $dX_r$ is given in local coordinates by
\begin{equation}
    dX_r \coloneqq \blade{v}_{i_1} \wedge \cdots \wedge \blade{v}_{i_r} dx^{i_1} \cdots dx^{i_r}. 
\end{equation}
\end{definition}
For example, along a 2-dimensional submanifold we have the 2-dimensional directed measure 
\begin{equation}
    dX_2 = \blade{v}_i \wedge \blade{v}_j dx^i dx^j
\end{equation}
and we can note that 
\begin{equation}
(\blade{v}^i \wedge \blade{v}^j)\cdot dX_2^\dagger = dx^idx^j - dx^j dx^i
\end{equation}
is completely antisymmetric and provides us a surface measure we can integrate; this is a differential 2-form. We then find that
\begin{equation}
\label{eq:volume_form}
\omega = \blade{I}^{-1} \cdot dX_n = \blade{I}^{-1 \dagger} \cdot dX_n^\dagger = 1^\perp \cdot dX_n,
\end{equation}
where $\blade{I}$ is the unit pseudoscalar field defined on $M$ with respect to $g$. The last of the equalities above is quite important. It seeks to tell us that, morally, we will tend integrate duals.

We can now write a $r$-form $\alpha_r = \alpha_{i_1 \cdots i_r} dx^{i_1}\wedge \cdots dx^{i_r}$ as 
\begin{equation}
\alpha_r = A_r \cdot dX_k^\dagger,
\end{equation}
where
\begin{equation}
A_r = \frac{1}{r!} \alpha_{i_1 \cdots i_r} \blade{v}^{i_`} \wedge \cdots \blade{v}^{i_r}.
\end{equation}
We refer to $A_r$ as the \emph{multivector equivalent} of $\alpha_r$ and note that by \cref{eq:volume_form} that the multivector equivalent to $\omega$ is $I^{-1 \dagger}$. This provides an isomorphism between $r$-forms and $r$-vectors via a contraction with the $r$-dimensional volume directed measure. In this sense, a differential form is made up of two essential components namely the multivector field and the $r$-dimensional directed measure. Hence, we can see now how a differential form simply appends the measure attached to the underlying space. We can also see how this generalizes the musical isomorphism $\flat$ by taking a vector field $\blade{v}$ and noting
\begin{equation}
\label{eq:line_element}
\blade{v} \cdot dX_1 = v_i  \blade{v}_i \cdot \blade{v}^j dx^j = v_i dx^i.
\end{equation}

The exterior algebra of differential forms comes with an addition $+$ and exterior multiplication $\wedge$.  We note that the sum of two $r$-forms $\alpha_r$ and $\beta_r$ is also a $r$-form which we can see reduces to addition on the multivector equivalents $A_r$ and $B_r$ by
\begin{equation}
\alpha_r + \beta_r = (A_r \cdot dX_r^\dagger)+(B_r \cdot dX_r^\dagger) = (A_r + B_r) \cdot dX_r^\dagger,
\end{equation}
due to the linearity of $\cdot$.  If instead had an $s$ form $\beta_s$ then we have the exterior product
\begin{equation}
\alpha_r \wedge \beta_s = (A_r \wedge B_s) \cdot dX_{r+s}^\dagger,
\end{equation}
where $dX_{r+s}=0$ if $r+s>n$.  

With differential forms one also has the exterior derivative $d$ giving rise to the exterior calculus. On the multivector equivalents we have
\begin{equation}
d \alpha_r = (\grad \wedge A_r) \cdot dX_{r+1}^\dagger,
\end{equation}
which realizes the exterior derivative as the grade raising component of the gradient $\grad$. Of course, for scalar fields, this returns the gradient as desired. We will find $\grad \rfloor$ can be identified with the codifferential $\delta$ up to a sign. 

\subsection{Integration}
\label{subsubsec:integration_on_submanifolds}

Given a $r$-dimensional submanifold $R \subset M$ with a $r$-form $\alpha_r$ defined on $R$, we can integrate this $r$-form. However, we want to phrase this in terms of the the multivector equivalents.  First, let $\omega_R$ be the volume measure for the submanifold $R$.  Given $R$ is a submanifold of $M$, for any $x \in R$ we have tangent space $T_x R$ which is a subspace of $T_x M$. Hence, we can put $\blade{I}_R(x)^{-1 \dagger}$ to be the multivector equivalent of $\omega_R$ by
\begin{equation}
\omega_R = \blade{I}_R^{-1 \dagger} \cdot dX_r^\dagger = \blade{I}_R^{-1} \cdot dX_r.
\end{equation}
We should think of $\blade{I}_R^{-1 \dagger}$ as representing the subspace $T_x R \subset T_x M$ and note that we think of $\blade{I}_R^{-1 \dagger}$ as a unit pseudoscalar field defined on $R$. 

An $s$-vector field $A_s$ on $R$ is said to be \emph{tangent to $R$} if
\begin{equation}
A_s = \projection_{\blade{I}_R}(A_s)
\end{equation} 
so that for any $x \in R$ that $A_s = \operatorname{P}_{\blade{I}_R(x)}(A_s(x))$. Immediately we can conclude that we must have $s\leq r$ or this projection is zero (see \cref{subsubsec:blades_and_subspaces}). We may, for example, wish to integrate scalar fields $A_0$ over $R$ and in this case we can put $A_r = A_0 \blade{I}_R^{-1}$ and contract with $dX_r$ to create a tangent $r$-form on $R$ by 
\begin{equation}
\alpha_r = A_r \cdot dX_r^\dagger = A \omega_R
\end{equation}
which can be integrated as
\begin{equation}
\int_K \alpha = \int_K A_0 \omega_R.
\end{equation}
This of course applies to scalar fields on $M$ itself, for which we can take $A_n = A_0 \blade{I}^{-1}$. Then this form can be integrated by
\begin{equation}
\int_M \alpha_n = \int_M A_0 \omega.
\end{equation}

There is also the normal space $N_x R$ that is everywhere orthogonal to $T_x R$ with respect to $g$ on $M$. This yields the normal $n-r$-blade field $\blade{\nu}_R = \blade{I}_R^\perp$. Since $R$ is a submanifold of $M$, we have the inclusion $\iota \colon R \to M$ and the induced pullback on forms $\iota^* \colon \Omega(M) \to \Omega(R)$. 
\begin{proposition}
Let $\alpha_s$ be an $s$-form defined on $M$ and let $\iota \colon R \to M$ be the inclusion of the submanifold $R$ into $M$. Then the pullback $\iota^*$ on the multivector equivalent $A_s$ is given by
\begin{equation}
\iota^* \alpha_s = \projection_{\blade{I}_R}(A_s) \cdot dX_s.
\end{equation}
\end{proposition}
\begin{proof}
Note that by definition we have
\[
(\iota^* \alpha_s)_x (\blade{v}_1,\dots,\blade{v}_r) = (\alpha_s)_x(d\iota_x\blade{v}_1,\dots, d\iota_x\blade{v}_r ),
\]
for arbitrary vector fields $\blade{v}_1,\dots,\blade{v}_s$ and at all $x\in R$. Then, since $\iota$ is inclusion, we have
\[
d\iota_x = \projection_{\blade{I}_R(x)},
\]
at each point $x \in R$ and hence
\[
\iota^* \alpha_s = \alpha_s \circ \projection_{\blade{I}_R}.
\]
For all $\blade{v}_i$ we can put
\[
\blade{v}_i = \projection_{\blade{I}_R}(\blade{v}_i) + \rejection_{\blade{I}_R}(\blade{v}_i),
\]
and note for the multivector equivalent
\begin{align}
\label{eq:previous_1}
(\projection_{\blade{I}_R}(A_s) \cdot dX_s)(\blade{v}_1,\dots,\blade{v}_s) &= (\projection_{\blade{I}_R}(A_s) \cdot dX_s)(\projection_{\blade{I}_R}(\blade{v}_1) + \rejection_{\blade{I}_R}(\blade{v}_1),\dots,\projection_{\blade{I}_R}(\blade{v}_s) + \rejection_{\blade{I}_R}(\blade{v}_s))\\
&= (\projection_{\blade{I}_R}(A_s) \cdot dX_s)(\projection_{\blade{I}_R}(\blade{v}_1),\dots,\projection_{\blade{I}_R}(\blade{v}_s)),
\end{align}
since $\projection_{\blade{I}_R}(A_s)$ is supported only on $R$. Then, if $s\leq r$,
\begin{align*}
\iota^*\alpha_s &= (A_s \cdot dX_s)(\projection_{\blade{I}_R}(\blade{v}_1),\dots,\projection_{\blade{I}_R}(\blade{v}_s))\\
&= ((\projection_{\blade{I}_R}(A_s) + \rejection_{\blade{I}_R}(A_s)) \cdot dX_s)(\projection_{\blade{I}_R}(\blade{v}_1),\dots,\projection_{\blade{I}_R}(\blade{v}_s))\\
&= (\projection_{\blade{I}_R}(A_s) \cdot dX_s)(\projection_{\blade{I}_R}(\blade{v}_1),\dots,\projection_{\blade{I}_R}(\blade{v}_s)),
\end{align*}
and by \cref{eq:previous_1} we have our intended result. If $s>r$, then 
\[
\iota^* \alpha_s = 0 = \projection_{\blade{I}_R}(A_s)
\]
which proves the proposition.
\end{proof}

The above seems to motivate the choice of \cite{schwarz_hodge_1995} to put $\tangent_R = \iota^*$ to refer to the tangential part of a differential form. The normal part of a form is $\normal_R \alpha_s = \alpha_s - \tangent_R \alpha_s$. The following corollary is immediate given \cref{eq:projection_rejection_duality,eq:projection+rejection_blade}. 
\begin{corollary}
Let $\alpha_s$ be an $s$-form with normal part $\normal_R \alpha_s$, then on the multivector equivalent $A_s$ we have
\begin{equation}
\normal_R \alpha_p = \operatorname{P}_{\blade{\nu}_R}(A_s)\cdot dX_s^\dagger = \rejection_{I_R}(A_s)\cdot dX_s^\dagger.
\end{equation}
\end{corollary}

This is pertinent when we take $M$ to be a manifold with boundary $\partial M$. In this case we let $\blade{I}_\partial$ denote the tangent $n-1$-blade and build boundary measure via
\begin{equation}
\omega_\partial \coloneqq \blade{I}_\partial^{-1} \cdot dX_{n-1}.
\end{equation}
The normal space is 1-dimensional and we put $\blade{\nu}$ to refer to the boundary normal space. It is common to compute the flux of a vector field $\blade{v}$ through $\partial M$ by integrating $\projection_{\blade{\nu}}(\blade{v})$ over the boundary. However, the the vector field $\projection_{\blade{\nu}}(\blade{v})$ is the multivector equivalent of a 1-form. Hence, what we should have is a pseudovector $\projection_{\blade{I}_\partial}(\blade{v}^\perp)$ which is the equivalent to the $n-1$-form
\begin{equation}
\projection_{\blade{I}_\partial}(\blade{v}^\perp) \cdot dX_{n-1}^\dagger = (-1)^p \blade{v}\cdot \blade{\nu} \omega_\partial.
\end{equation}
This tells us that the flux is determined both by the vector field $\blade{v}$ and the local geometry of $\partial M$ captured by $\omega_\partial$. A proof follows. 
\begin{proposition}
\label{prop:flux}
Then the flux of a vector field $\blade{v}$ through $\partial M$ is
\begin{equation}
\int_{\partial M} \projection_{\blade{I}_\partial}(\blade{v}^\perp) \cdot dX_{n-1}^\dagger = (-1)^p\int_{\partial M} \blade{v} \cdot \blade{\nu} \omega_\partial,
\end{equation}
where $p$ is the number of temporal vectors in $\G(M)$.
\end{proposition}
\begin{proof}
Take
\begin{align*}
\projection_{\blade{I}_\partial}(\blade{v}^\perp) &= \blade{v}^\perp \rfloor \blade{I}_\partial \blade{I}_\partial^{-1}\\
    &= (\blade{v}^\perp \wedge \blade{\nu})^\perp \blade{I}_\partial^{-1}\\
    &= (-1)^{n-1} (\blade{\nu} \wedge \blade{v}^\perp)^\perp \blade{I}_\partial^{-1}\\
    &= (-1)^{n-1} (\blade{\nu} \rfloor \blade{v})^{\perp \perp} \blade{I}_\partial^{-1}\\
    &= (-1)^{\frac{1}{2}(n+2)(n-1)+p} \blade{v}\cdot \blade{\nu} \blade{I}_\partial^{-1}\\
    &= (-1)^p \blade{v} \cdot \blade{\nu} \blade{I}_\partial^{-1 \dagger}.
\end{align*}
Hence 
\[
\projection_{\blade{I}_\partial}(\blade{v}^\perp) \cdot dX_{n-1}^\dagger =(-1)^s \blade{v} \cdot \blade{\nu} \omega_\partial.
\]
\end{proof}

For smooth $r$-forms $\alpha_r$ and $\beta_r$, we have an $L^2$-inner product 
\begin{equation}
\int_M \alpha_r \wedge \star \beta_r 
\end{equation}
where $\star$ is the Hodge star. By definition, the Hodge star acts on $r$-forms by returning a Hodge dual $n-r$-form so that on the multivector equivalents we have
\begin{equation}
\alpha_r \wedge \star \beta_r  = (A_r\cdot B_r^\dagger)\omega
\end{equation}
as well as
\begin{equation}
    \alpha_r \wedge \star \alpha_r = \|A_r\|\omega,
\end{equation}
where $\|A_r\|$ is the pointwise Clifford norm. For the action of $\star$ on the multivector equivalents we will put $B_r^\star$. 

\begin{proposition}
We have that $B_r^\star$ is given by
\begin{equation}
B_r^\star = (-1)^{r(n-r)}(B_r^\perp)^\dagger.
\end{equation}
\label{prop:multivector_hodge_star}
\end{proposition}
\begin{proof}
Indeed, 
\begin{align*}
    \alpha_r \wedge \star \beta_r &= (A_r \wedge B_r^\star) \cdot dX_n^\dagger\\
    &=(-1)^{r(n-r)} (A_r \wedge (B_r^\perp)^\dagger ) \cdot dX_n^\dagger\\
    &=(-1)^{r(n-r)} (B_r^\perp \wedge A_r^\dagger)^\dagger \cdot dX_n^\dagger\\
    &=(A_r^\dagger \wedge B_r^\perp)^\dagger \cdot dX_n^\dagger\\
    &=(A_r^\dagger \cdot B_r)^\perp \cdot dX_n\\
    &= A_k \cdot B_k^\dagger \omega,
\end{align*}
with the final equality by \cref{eq:dot_product}.
\end{proof}

\subsection{Multivector valued integrals}

The integrals defined before are all scalar valued, but geometric calculus allows for an extension to multivector valued integrals. 
\todo{more work to define and show some stuff here}

Given our definition of the Hodge star on multivector equivalents, we can now define a multivector valued $L^2$-inner product on multivector fields.

\begin{definition}
Let $A_r$ and $B_s$ be a $r$- and $s$-vector fields. Then the \emph{Clifford inner product} is defined by
\begin{equation}
\multivecinnerproduct{A_r}{B_s} \coloneqq \int_M A_r B_s^\dagger \omega.
\end{equation}
If $\multivecinnerproduct{A_r}{B_s} = 0$, then we say $A_r$ and $B_s$ are \emph{Clifford orthogonal}.
\end{definition}
In particular, one can consider the graded components of this inner product by
\begin{equation}
\multivecinnerproduct{A_r}{B_s}_j \coloneqq \proj{j}{\multivecinnerproduct{A_r}{B_s}} = \proj{j}{\int_M A_r B_s^\dagger \omega} \equiv \int_M \proj{j}{A_r B_s^\dagger} \omega,
\end{equation}
which we refer to as the \emph{$j$-vector valued Clifford inner product}. The Clifford inner product extends the notion of the $r$-form inner product.  One should view this extension as the same extension we find in the containment between $\bigwedge(V)$ and the more general $C\ell(V,Q)$. The perspective is that the Clifford algebras always contain at least as much information as the alternating algebras and geometric algebras will strictly resolve finer details. For example, the following proposition is immediate.
\begin{proposition}
Given two $r$-forms, the $r$-form inner product is equal to the scalar valued Clifford inner product on their corresponding multivector equivalents.
\end{proposition}
\begin{proof}
Let $\alpha_r$ and $\beta_r$ be $r$-forms with multivector equivalents $A_r$ and $B_r$ respectively. Then
\begin{align*}
    \int_M \alpha_r \wedge \star \beta = \int_M A_r \cdot B_r^\dagger \omega = \int_M \proj{0}{A_rB_r^\dagger} \omega = \multivecinnerproduct{A_r}{B_r}_0,
\end{align*}
by the proof of \cref{prop:multivector_hodge_star}. 
\end{proof}
Note that when $s\neq r$, this scalar valued Clifford inner product is zero. Hence, the orthogonal direct sum with respect to the $L^2$ multivector inner product extends the grade based direct sum. It will suffice to use the symbol $\oplus$ for both. 

Two subsets of fields may be orthogonal with respect to the scalar valued Clifford inner product, but they may fail to be Clifford orthogonal. In fact, the grade based decomposition does not hold over the multivector field inner product. For example, consider $M$ a region of $\R^n$ then $\G_n(M)$ and take a constant subspace (unit blade) $\blade{B}_{12}$ and a blade inside this subspace $\blade{e}_2$ then
\begin{equation}
\multivecinnerproduct{\blade{B}_{12}}{\blade{e}_2} = \int_M  \blade{e}_1 \omega = \blade{e}_1 \textrm{vol}(M),
\end{equation}
which shows the fields $\blade{B}_{12}$ and $\blade{e}_2$ are not Clifford orthogonal.
\todo{this is kind of interesting. It seems to say that the inner product value is almost describing how the vectors differ and by how much of $M$ they differ by...?}
We extend the notion of Clifford orthogonal to spaces. That is, if we have a space of multivector fields $X$ and another space of multivector fields $Y$ such that for any $A\in X$ and $B\in Y$ we have $\multivecinnerproduct{A}{B}=0$ then we say $X$ and $Y$ are Clifford orthogonal and we put $X \boxplus Y$ to refer to the orthogonal direct sum with respect to the Clifford inner product.



\subsection{Stokes' and Green's formula}
\todo{do stokes' for $r$-dimensional submanifolds $R$}
On forms, we have a compact form of Stokes' theorem
\[
\int_M d \alpha_{n-1} = \int_{\partial \Omega} \iota^* \alpha_{n-1},
\]
for sufficiently smooth $n-1$-forms $\alpha_{n-1}$. Then, in terms of the multivector equivalents, Stokes' theorem reads
\begin{equation}
\label{eq:stokes_theorem}
\int_M (\grad \wedge A_{n-1})\cdot dX_n = \int_{\partial M} \projection_{\blade{I}_\partial}(A_{n-1}) \cdot dX_{n-1}.
\end{equation}
But this has another, more physical, interpretation, in particular with the dual relationship. If we take a vector field $\blade{v}$, then
\begin{equation}
\label{eq:stokes_theorem_dual}
\int_M (\grad \wedge \blade{v}^\perp) \cdot dX_n = \int_{\partial M} \projection_{\blade{I}_\partial}(\blade{v}^\perp) \cdot dX_{n-1},
\end{equation}
which implies
\begin{equation}
\int_M \grad \cdot \blade{v} \omega = \int_{\partial M} \blade{v}\cdot \blade{\nu} \omega_\partial
\end{equation}

And, in broader generality we have Green's formula which is given by
\begin{equation}
\int_M d\alpha_{r-1} \wedge \star \beta_r = \int_M \alpha_{r-1} \wedge \star \delta \beta_r + \int_{\boundary} \iota^* (\alpha_{r-1} \wedge \star \beta_r)
\end{equation}
This equation motivates the definition of $\delta$ as the adjoint to $d$ under the $r$-form inner product. In the case of a closed manifold $M$, $\boundary = \emptyset$ and the boundary integral vanishes. 

\begin{definition}
We define the adjoint operator $\grad \wedge^*$ to $\grad \wedge$ by
\[
\grad \wedge^* = (-1)^{\frac{1}{2}(3n^2-4nr-n+r^2+3r+2)} \grad \rfloor.
\]
\end{definition}
Classically, the Hodge-Dirac operator $d+\delta$ is constructed. One can compare this operator to $\grad$ and notice the subtle differences in the dependence on the manifold dimension and degree of the multivector.

\begin{proposition}
On multivector equivalents $A_{r-1}$ and $B_r$, we have Green's formula
\begin{equation}
\multivecinnerproduct{\grad \wedge A_{r-1}}{B_r}_0 = \multivecinnerproduct{A_{r-1}}{\grad \wedge^* B_r}_0 + (-1)^{r(n-r)+\frac{(n-r)(n-r-1)}{2}+p} \int_{\partial M} (A_{r-1}\rfloor B_r) \cdot \blade{\nu} \omega_\partial.
\end{equation}
\end{proposition}
\begin{proof}
First, we have
\begin{equation}
\int_M d(\alpha_{r-1} \wedge \star \beta_r) = \underbrace{\int_M d\alpha_{r-1} \wedge \star \beta_r}_{1} + \underbrace{(-1)^{r-1} \int_M \alpha_{r-1} \wedge d \star \beta_r}_{2},
\end{equation} 
by the Leibniz rule. By Stokes' theorem,
\begin{equation}
\int_M d(\alpha_{r-1} \wedge \star \beta_r) = \underbrace{\int_\boundary \iota^*(\alpha_{r-1} \wedge \star \beta_r)}_{3}.
\end{equation}
For underbrace 1,
\begin{equation}
\int_M d\alpha_{r-1} \wedge \star \beta_r = \int_M (\grad \wedge A_{r-1}) \cdot B_r^\dagger \omega = \multivecinnerproduct{\grad \wedge A_{r-1}}{B_r}_0.
\end{equation}
For underbrace 2,
\begin{align}
    (-1)^{r-1}\int_M \alpha_{r-1} \wedge d \star \beta_r  &= \int_M A_{r-1} \wedge (\grad \wedge B_r^\star) \cdot dX_n^\dagger\\
    &= (-1)^{r-1+n(n-r)} \int_M [A_{r-1} \wedge (\grad \wedge (B_r^\perp)^\dagger)] \cdot dX_n^\dagger\\
    &= (-1)^{r-1 + n(n-r)+\frac{(n-r)(n-r-1)}{2}} \int_M A_{r-1} \wedge (\grad \rfloor B_r)^\perp \cdot dX_n^\dagger\\
    &= (-1)^{\frac{1}{2}(3n^2-4nr-n+r^2+3r+2)} \int_M A_{r-1} \rfloor (\grad \rfloor B_r) \omega\\
    &= \multivecinnerproduct{A_{r-1}}{\grad \wedge^* B_r}_0.
\end{align}
For underbrace 3,
\begin{align}
\int_\boundary \iota^*(\alpha_{r-1} \wedge \star \beta_r) &= \int_\boundary \projection_{\blade{I}_\partial} (A_{r-1} \wedge B_r^\star) \cdot dX_{n-1}^\dagger\\
&= (-1)^{r(n-r)+\frac{(n-r)(n-r-1)}{2}} \int_\boundary \projection_{\blade{I}_\partial} (A_{r-1} \wedge B_r^\perp) \cdot dX_{n-1}^\dagger\\
&= (-1)^{r(n-r)+\frac{(n-r)(n-r-1)}{2}} \int_\boundary \projection_{\blade{I}_\partial}( (A_{r-1} \rfloor B_r)^\perp) \cdot dX_{n-1}^\dagger\\
&= (-1)^{r(n-r)+\frac{(n-r)(n-r-1)}{2}+p} \int_\boundary (A_{r-1}\rfloor B_r) \cdot \blade{\nu} \omega_\partial 
\end{align}
with the final equality by \cref{prop:flux}.
\end{proof}

\subsection{Fundamental theorem of geometric calculus}








%%%%%%%%%%%%%%%%%%%%%%%%%%%%%%%%%%%%%%%%%%%%%%%%%%%%%%%%%%%%%%%%%%%%%%%%%%%%%%%%%%%%%%%%%%%%%%%%%%%%%%%%%%%%%%%%%%%%%%%%%%
% Chapter 3
%%%%%%%%%%%%%%%%%%%%%%%%%%%%%%%%%%%%%%%%%%%%%%%%%%%%%%%%%%%%%%%%%%%%%%%%%%%%%%%%%%%%%%%%%%%%%%%%%%%%%%%%%%%%%%%%%%%%%%%%%%

\chapter{Analysis of multivector fields}

\section{Spaces of fields}
\label{sec:spaces_of_fields}
\subsection{Hodge-type decompositions}
Let us define the following spaces of $r$-vectors. We write this in terms of the multivector equivalents to forms.
\begin{itemize}
    \item The \emph{exact $r$-vectors},
    \begin{equation}
        \exactvectors{r}\coloneqq \{\grad \wedge A_{r-1} ~\vert~ A_{r-1} \in \G^{r-1}(M), ~ \projection_{\blade{\nu}}(A_{r-1}) = 0\};
    \end{equation}
    \item The \emph{co-exact $r$-vectors},
    \begin{equation}
        \coexactvectors{r}\coloneqq \{\grad \rfloor A_{r+1} ~\vert~ A_{r+1} \in \G^{r+1}(M), ~ \rejection_{\blade{\nu}}(A_{r+1}) = 0\};
    \end{equation}
    \item The \emph{harmonic fields},
    \begin{equation}
        \harmonicfields{r}\coloneqq \{A_r \in \G^{r}(M) ~\vert~ \grad \wedge A_r = 0, ~ \grad \rfloor A_r = 0\}.
    \end{equation}
    \item The \emph{Dirichlet harmonic fields},
    \begin{equation}
        \harmonicdirichlet{r}\coloneqq \{A_r \in \harmonicfields{r} ~\vert~ \projection_{\blade{\nu}}(A_r) = 0\}.
    \end{equation}
    \item The \emph{Neumann harmonic fields},
    \begin{equation}
        \harmonicneumann{r}\coloneqq \{A_r \in \harmonicfields{r} ~\vert~ \rejection_{\blade{\nu}}(A_r) = 0\}.
    \end{equation}
\end{itemize}
Notice that the exact and coexact forms satisfy not only a differential condition, but a boundary condition as well. Then, under the $r$-form inner product, we find the orthogonal direct sum decomposition
\begin{equation}
\Omega^r(M) = \exactvectors{r} \oplus \coexactvectors{r} \oplus \harmonicfields{r},
\end{equation}
known as the Hodge-Morrey decomposition. Within the space of harmonic fields we have
\begin{align}
    \harmonicex{r} &\coloneqq \{A_r \in \harmonicfields{r} ~\vert~ A_r = \grad \wedge B_{r-1} \},\\
    \harmonicex{r} &\coloneqq \{A_r \in \harmonicfields{r} ~\vert~ A_r = \grad \rfloor B_{r+1} \}.
\end{align}
Further, we have two decompositions of the space of harmonic fields 
\begin{align}
    \harmonicfields{r} &= \harmonicdirichlet{r} \oplus \harmonicco{r}, \\
    \harmonicfields{r} &= \harmonicneumann{r} \oplus \harmonicex{r},
\end{align}
which are the Friedrichs decompositions.

\begin{proposition}
    The harmonic fields $\harmonicfields{r}$ are monogenic $r$-vectors so that $\monogenicfields{r} = \harmonicfields{r}$.
\end{proposition}
\begin{proof}
Trivial
\end{proof}

We can utilize 
Monogenics of a single grade are already studied. but now we can study monogenics of mixed grades!
\subsection{Integral transforms}


\section{Algebras of fields}
\label{sec:algebras_of_fields}
\subsection{Banach algebras of Clifford fields}

The space $\monogenics{}$ is a vector space due but it is not, in general, an algebra. For instance, if $M$ is dimension $n=2$, then $\monogenics{+}$ is an algebra due to the commutivity of $\monogenics{+}$. Yet, the $\monogenics{}$ does contain algebras that are commutative Banach algebras. 

\subsubsection{Planar monogenic fields}

Generically, if I take some multivector $A$ times a monogenic field $f$, $Af$ need not be monogenic which is a reason why $\monogenics(\Omega)$ fails to be an algebra. But, there are certain types of monogenic fields in which this property is true. We describe a set of parabivectors that operate entirely on a plane given by a unit bivector $B$. These specific fields will be of great utility for the remainder of this paper.
\begin{definition}
    Let $f$ be a parabivector and $B$ a unit $2$-blade. Then $f$ is a \emph{$B$-planar field} if $f = \operatorname{P}_B \circ f \circ \operatorname{P}_B$.
\end{definition} 
We then refer to the \emph{$B$-planar monogenic fields} $f$ when $f$ is both $B$-planar and monogenic. Planar monogenic fields will serve as a realization of complex valued functions since they carry over some additional nice properties and admit a nice representation.
\begin{lemma}
    Let $f$ be a $B$-planar monogenic field, then:
\begin{itemize}
    \item The directional derivatives in all directions other than in the $B$ plane are zero;
    \item We have the representation $f=u+\beta B$ for a $u,\beta \in G_n^0(\Omega)$ and $B$ the given unit bivector.
\end{itemize}
\end{lemma}
\begin{proof}
~
    \begin{itemize}
    \item Let $v$ be a unit vector not in the $B$ plane so that $\projection{B}{v}=0$. Since $f$ is $B$-planar, we know $f=f \circ \operatorname{P}_B$ which shows that $f(x+\epsilon v)= f(x)$.  It follows that $\nabla_v f=0$.
    \item Let $f=u+b$ for $u\in \G_n^0$ and $b\in G_n^2$. Then $f=\projection{B}{v} \circ f$ and so $\projection{B}{u+b}=u+b$. In particular, $\operatorname{P}_B=b$ and thus $b=\beta B$ for a scalar $\beta \in \G_n^0$.
\end{itemize}
\end{proof}
To get a geometric interpretation of $B$-planar fields we can note that they are constant on translations of the $B$-plane.  It follows that 
\begin{equation}
\label{eq:exterior_b_derivative}
(\grad \wedge B)f = 0.
\end{equation}
In $\R^3$, for example, this amounts to fields constant along an axis $\omega=IB^{-1}$ perpendicular to $B$ as
\begin{equation}
\label{eq:omega_axial_equivalence}
\grad \wedge B = \grad \wedge \omega I =\grad \cdot \omega = \nabla_\omega.
\end{equation}

\todo[inline]{Rephrase this with rejection?}

Recall from Example \ref{ex:complex_representation} that multivectors in the form $\zeta=x+yB$ mimic the complex number $\zeta$ when $B$ is a unit $2$-blade since $B^2=-1$.  Planar monogenic fields are thus a direct analog of $\C$-holomorphic functions.  Indeed, for simplicity take the orthonormal basis $e_i$ and the blade $B=B_{12}$ and for scalar fields $u$ and $\beta$ put
\[
f=u+\beta B_{12}
\]
and note
\[
\grad f = 0 
\]
yields the Cauchy-Riemann equations
\[
\nabla_{e_1} u = \nabla_{e_2} \beta \qquad \textrm{and} \qquad \nabla_{e_2}u = -\nabla_{e_1} \beta.
\]
Holomorphic functions form an algebra and we shall show the $B$-planar monogenic fields do as well. 

We let 
\[
\algebra{B}(\Omega) = \{f ~\vert~ \textrm{$f$ is $B$-planar and monogenic}\}
\]
be the space of $B$-planar monogenic fields. For any $2$-blade $B$ in $\Grassmannian{2}{n}$, we have a space $\algebra{B}(\Omega)$. Multiplication of two $B$-planar fields $f=u_f+\beta_f B$ and $g=u_g+\beta_g B$ is given by
\begin{equation}
\label{eq:axial_multiplication}
fg = u_f u_g - \beta_f \beta_g + B (u_f b_g + u_g b_f) = gf.
\end{equation}

Another property mimics $\C$-holomorphicity.  Namely, scaling a holomorphic function by constant complex numbers remains holomorphic. We realize this for $B$-planar fields as $\operatorname{Spin}(2)$ invariance (really $\R \times \operatorname{Spin}(2)$ invariant).  The following corollary follows from Lemma \ref{lem:clifford_invariant} since $\R \times \operatorname{Spin}(2)$ is a subgroup of $\Gamma^+$ 
\begin{corollary}
    \label{cor:mult_by_i_monogenic}
    Let $f=u+\beta B$ be an $B$-planar monogenic field and let $\zeta=x+yB$ for constant scalars $x$ and $y$. Then $\zeta f$ is a $B$-planar monogenic.
\end{corollary}
\begin{proof}
    Note that $\zeta$ admits the representation $\zeta = re^{\theta B}$ as seen in Example \ref{ex:exponential_of_bivector} for some $r,\theta \in \R$ with $r= \|\zeta\|$. If $\|\zeta\|=1$, then this corollary follows immediately from Lemma \ref{lem:clifford_invariant} as $\zeta \in \spingroup$. If $r\neq 1$, we note that the the corollary remains true given the $\R$-linearity of $\grad$.
\end{proof}
The point here is that we have now effectively found functions that can be scaled by $B$-planar constants $\zeta$ and remain monogenic. 
 
With the above, we show the space $\algebra{B}(\Omega)$ is closed under multiplication and is in fact abelian.
\begin{lemma}
\label{lem:product_of_monogenics}
    Let $f$ and $g$ be monogenic and $B$-planar. Then $fg=gf$, and $fg$ is a $B$-planar monogenic.
\end{lemma}
\begin{proof}~
    \begin{itemize}
        \item First, it is clear that $fg=gf$ by Equation \ref{eq:axial_multiplication}.
        \item The product $fg$ is $B$-planar since $u_f,u_g,\beta_f$, and $\beta_g$ are all constant on translations of the $B$-plane, i.e. that $fg = fg \circ \operatorname{P}_B$.  Due again to Equation \ref{eq:axial_multiplication} we have $fg = \operatorname{P}_B \circ fg$ as well.  
    \item To see that the product is monogenic, we have
    \[
        \grad(fg) = \grad(u_fu_g - b_f b_g +  B(u_f b_g + u_g b_f)).
    \]
    Then the grade-1 components are
    \[
        \proj{1}{\grad(fg)}=\grad \wedge (u_f u_g - b_f b_g) + \grad \cdot B(u_f b_g + u_g b_f),
    \]
    and note that we have
    \begin{align*}
        \grad(u_f u_g - b_f b_g) &= (\grad u_f) u_g + u_f (\grad u_g) - (\grad b_f) b_g - b_f (\grad b_g)\\
        \grad \cdot B(u_f b_g + u_g b_f) &= (\grad \cdot B u_f) b_g + u_f (\grad \cdot B b_g) + b_f(\grad \cdot B u_g) + (\grad \cdot B b_f) u_g,
    \end{align*}
    and since $f$ and $g$ are both monogenic we have
    \begin{align*}
        \proj{1}{\grad(fg)} &= (\grad \cdot B u_f - \grad  b_f)b_g + (\grad \cdot B u_g - \grad  b_g)b_f.
    \end{align*}
    \[
        0=\proj{1}{\grad Bf} = \grad \cdot B u_f - \grad b_f
    \]
    by Corollary \ref{cor:mult_by_i_monogenic} and likewise for $\proj{1}{\grad Bg}$. Thus,
    \[
        \proj{1}{\grad(fg)}=0.
    \]
    
    The grade-3 components for the gradient are
    \[
        \proj{3}{\grad(fg)} = \grad \wedge B (u_f b_g + u_g b_f),
    \]
    and we can note that $\grad \wedge B=0$ since $u_f,b_g,u_g,$ and $b_f$ are all $B$-planar.
\end{itemize}
\end{proof}

From the above work, we realize that for each $\algebra{B}(\Omega)$ we have a well defined multiplicative structure. This realizes that $\algebra{B}(\Omega)$ sits inside of the space of monogenic spinors $\monogenics^+(\Omega)$. We arrive at the following corollary.
\begin{corollary}
The space $\algebra{B}(\Omega)$ is a commutative unital Banach algebra.
\end{corollary}
\begin{proof}
Let $f$ and $g$ be $B$-planar monogenic fields. It is clear that the sum $f+g$ is a $B$-planar monogenic by the linearity of $\grad$ and the projection. Since $fg=gf$ is $B$-planar and monogenic we find that each $\algebra{B}(\Omega)$ is an algebra. Since $\algebra{B}(\Omega)$ is a commutative subalgebra of $\G_n^+(\Omega)$, it is also a commutative Banach algebra. \textcolor{red}{Shorten a lot}
\end{proof}


\subsubsection{$\omega$-axial fields}
The authors in \cite{belishev_algebraic_2019,belishev_algebras_2019} give a thorough treatment of an analogous story but with quaternion fields.  We show the relationship between the two stories in this section and we find them to be entirely equivalent. As in Example \ref{ex:quaternions}, we can see these quaternion fields as parabivector fields.  The authors work exclusively in 3-dimensions and quickly specialize to the fields which are $\omega$-axial due to their rich algebraic structure. There, $\omega$ is a purely imaginary unit quaternion. Their harmonic $\omega$-axial fields are equivalent to monogenic $B$-planar fields if we take the axis $\omega = BI^{-1}$. First, note we define $\omega$-axial in the same way.
\begin{definition}
    Let $A \in \G_3(\Omega)$ be a multivector field then $A$ is \emph{$\omega$-axial} if $A(x+t\omega) = A(x+t\omega)$.  
\end{definition}

This definition allows us to perfectly coincide the notions of $B$-planar monogenic fields with $\omega$-axial harmonic quaternion fields.
\begin{proposition}
    In $\R^3$, every $B$-planar monogenic field is in correspondence with an $\omega$-axial harmonic quaternion field $h = \varphi + \psi \omega$. 
\end{proposition}
\begin{proof}
    Let $f$ be a $B$-planar monogenic field with $\tilde{\omega}=BI^{-1}$ and note that $f(x+t\tilde{\omega)}=f(x)$ since $\projection{B}{t\omega}=0$. Thus, $f$ is $\tilde{\omega}$-axial.
    
    Given the quaternion multiplication is a left handed bivector multiplication (see Example \ref{ex:quaternions}, we can replace the purely imaginary quaternion $\omega$ and get a vector in $\G_3^1$ by using the correspondence $\boldsymbol{i} \leftrightarrow e_1$, $\boldsymbol{j}\leftrightarrow e_2$, and $\boldsymbol{k}\leftrightarrow e_3$ we generate $\tilde{\omega} \in \G_3^1$. We then have the $2$-blade $B=\tilde{\omega} I$ such that
    \[
        \tilde{h} = \varphi + \psi B,
    \]
    is the corresponding parabivector in $\G_3$. It's clear that $\operatorname{P}_B \circ \tilde{h} = \tilde{h}$. Likewise, since $\varphi$ and $\psi$ were constant on the axis given by $\omega$, then by the previous work $\varphi \circ \operatorname{P}_B$ and $\psi \circ \operatorname{P}_B$ implies that $\tilde{h} \circ \operatorname{P}_B$ and so $\tilde{h}$ is a $B$-planar. Hence, setting $\varphi = u$ and $\psi=\beta$, we recover a unique $f$ from a given $h$.

Then, if $h=\varphi + \psi \omega$ is harmonic, we know
\[
\grad \psi = \omega \times \grad \varphi,
\]
where we take the vector cross product $\times$.  Based on Example \ref{ex:cross_product}, we can see that corresponding $B$-planar field $f=u+\beta B$ yields the analogous equation
\[
\grad u = \grad \cdot \beta B = (\grad \wedge \tilde{\omega})I = \tilde{\omega } \times \grad \beta.
\]
Thus, the notions of an $\omega$-axial harmonic quaternion field coincides with $B$-planar monogenic fields in $\R^3$ so long as $B=\tilde{\omega}I$.
\end{proof}

The $\omega$-axial fields do not generalize properly and this definition is solely a happy circumstance seen in $\R^3$ given the duality between vectors and bivectors.  In higher dimensions, the notion of $B$-planar retains all the desired properties that let us define a notion of a Gelfand spectrum.



\subsubsection{Spinor spectrum}

This story no longer continues in higher dimensions and one can find the two and three dimensional cases to be happy accidents.  Instead, now we must deal fully with the situation at hand to dissect the relevant algebras. At our disposal are the algebras $\algebra{B}(\Omega)$ of $B$-planar monogenic fields. Take the case where the domain $\ball \subset \R^n$ is the unit $n$-ball and moreover let $\disk$ be the unit disk in $\C \cong \R^2$.  By Gelfand, the maximal ideal space of the commutative Banach algebra $\algebra{B}(\ball)$ is homeomorphic to the disk given the isomorphism mapping the blade $B \leftrightarrow i$ in the complex plane. Since the space $\monogenics$ is no longer commutative let alone an algebra, we are at a loss to determine maximal ideals.  Instead, one can note that maximal ideals of a commutative Banach algebra correspond to the multiplicative linear functionals.  Using this identification, we carry on and describe functionals on the monogenic fields.

\textcolor{red}{It's probably worth phrasing this as some kind of algebra morphism}

\begin{definition}
    Define the \emph{spinor dual} $\dualmonogenics(\Omega)$ as
    \[
        \dualmonogenics(\Omega) \coloneqq \{ l \in \mathcal{L}(\monogenics^+(\Omega); \G_n^+) ~\vert~ l(sf) = sl(f), ~\forall f \in \monogenics(\Omega), ~s \in \G_n^+ \},
    \]
    and refer to elements of $\dualmonogenics(\Omega)$ are \emph{spinor functionals}. \textcolor{red}{Maybe have $s\in \mathfrak{spin}(n)$ instead?}
\end{definition}
Similarly, we will now define the spinor functionals that are multiplicative on the $B$-planar monogenics. In other words, spin characters are simply algebra homomorphisms from $\algebra{B}(\Omega)$ to $\G_n^+$.
\begin{definition}
    The \emph{spinor spectrum} $\characters(\Omega)$ is the set
    \[
        \characters(\Omega) \coloneqq \{ \mu \in \dualmonogenics(\Omega) ~\vert~ \mu(fg) = \mu(f)\mu(g),~ \forall f \in \algebra{B}(\Omega) \textrm{ and } \forall g \in \algebra{B'}, ~ B,B' \in \Grassmannian{2}{n}\},
    \]
    and we refer to the elements as \emph{spinor characters}.
\end{definition}
\textcolor{red}{Maybe we don't need multiplicative on different algebras somehow?}
\begin{example}
\textcolor{red}{Edit this}
In the case where $\Omega$ itself is 2-dimensional and compact, we realize $\G_n^+$ is isomorphic to $\C$ and we find that these match the typical definition for characters $\mu\in \characters(\Omega)$.  These spin characters each amount to function evalation. Take $f\in \monogenics(\Omega)$ and note that $f \in \algebra{B}(\Omega)$ as well.  $f$ is then a holomorphic function when we identify $B \leftrightarrow i$ and as such the spin character $\mu$ acts by $\mu(f)=f(x_\mu)$ for some point $x_\mu \in \Omega$ showing the correspondence of points in $\Omega$ with spin characters in $\characters(\Omega)$. Hence, with the weak-$\ast$ topology, the space $\characters(\Omega)$ is homeomorphic to $\Omega$. 
\end{example}

\todo[inline]{There is the question now on what is the homeomorphism type of $\algebra{B}(\Omega)$ for an arbitrary $\Omega$ and for a given $B$. Use 2d Belishev somehow? Describe the weak-$\ast$ topology here to use later.} 


\section{Gelfand theory}
\label{sec:gelfand_theory}
\subsection{Topology from monogenics}

We seek to determine that the space $\characters(\Omega)$ is homeomorphic to $\Omega$.  Thinking of the Calder\'on problem, we may only have access to functions defined on $\Omega$ and not the whole of $\Omega$ itself.  If one can recover the spinor characters $\characters(\Omega)$, we can utilize the following result.

\begin{theorem}
For any $\mu \in \characters(\Omega)$, there is a point $x^\mu \in \Omega$ such that $\mu(f) = f(x_\mu)$ for any $f\in \monogenics(\Omega)$ a monogenic spinor field. Given the weak-$\ast$ topology on $\characters(\Omega)$, the map
\[
\gamma \colon \characters(\Omega) \to \Omega, \quad \mu \mapsto x^\mu
\]
is a homeomorphism. The Gelfand transform 
\[
\widehat{~} \colon \monogenics(\Omega) \to C(\characters(\Omega); \G_n), \quad \widehat{f}(\mu) \coloneqq \mu(f), \quad \mu \in \characters(\Omega),
\]
is an isometry onto its image, so that $\characters(\Omega)$ is isomorphic to $\widehat{\monogenics(\Omega)}$ as algebras.
\end{theorem}

We prove this theorem in two main parts and discuss the result in this section. First, we can realize a power series representation for elements in a ball $\ball$ and denote this sit as $\monogenics(\ball)$. This power series is constructed using specific $B$-planar monogenic fields. Finally, we constructively show a correspondence between $\mu \in \characters(\ball)$ with $x^\mu \in \ball$. \textcolor{red}{Then we can use these to cover $\Omega$ or something?}

\subsubsection{Power series}

\todo[inline]{This really is a honest to god Taylor series so I should call it that.}

One beautiful result in Clifford analysis is the celebrated generalization of the Cauchy integral formula for $\C$-holomorphic functions. Details of the Cauchy integral formula and Hilbert transform for multivector fields can be found in \cite{brackx_hilbert_2008}. We have the fundamental solution to $\grad$ is a vector field given by
\[
E(x) = \frac{1}{a_m} \frac{x}{\|x\|^m},
\]
for $x\in \R^n$. That is to say that $\grad E(x) = \delta(x)$. For any region $\Omega \subset \R^n$ with boundary $\Sigma$, we define the \emph{Cauchy kernel} for $x\in \R^n$ and $y \in \Sigma$ using the fundamental solution $E$ as
\[
C(y, x) = -\frac{1}{a_n} \nu(x_0) E(x-y),
\]
where $a_n$ is the surface area of the $n$-ball and $\nu(x_0)$ is the outward normal at $x_0$. The \emph{Cauchy integral} for $\phi \in L_2(\Sigma)$ is then
\[
\cauchy[\phi](x) = \frac{1}{a_n} \int_{\Sigma} \frac{y-x}{\|x-y\|^n} \nu(y) \phi(y) d\Sigma(y).
\]
The Cauchy integral is indeed a monogenic function and note that for a scalar $\phi$ we have $\cauchy[\phi] \in \monogenics(\Omega)$ since it must be a parabivector as well.

Fix a basis $e_1,\dots,e_n$ in $\R^n$ and we can define the functions $z_j^i = x^j - x^i e^i e_j$. Recall that for an orthonormal basis the reciprocal basis elements $e^i=e_i$ satisfy $e^i \cdot e_j = 1$. \textcolor{red}{Ryan uses $e_i^{-1}$ actually. Are the reciprocal basis elements the inverses? Yes see \url{https://math.stackexchange.com/questions/811248/wedge-product-between-nonorthogonal-basis-and-its-reciprocal-basis-in-geometric}} To further condense notation, we let $B_{ij}=e_i e_j$ be the 2-blade acting as the pseudoscalar for the $e_i e_j$-plane and likewise put $B_j^i = e^ie_j$ and $B^{ij}=e^i e^j$ as necessary. In the same vein, the functions $z_j^i$ are very analogous to $z$ in $\C$ but rather in the $B_j^i$ plane.  We then note
\[
z_j^i = x^j - x^i B_j^i = e_j\projection{B_j^i}{x}.
\]
One can quickly confirm that the $z_j^i$ are monogenic and are indeed $B_j^i$-planar by construction. These functions find their use in a power series representation for monogenic fields $f$.
\begin{itemize}
    \item Consider the function $z_{\sigma(j)}^1(x)=x^{\sigma(j)} - x^1 B_{\sigma(j)}^i$ for $\sigma \in \{2,\dots,n\}$ a permutation.  
    \item Let $f \in \monogenics^+(\Omega)$.  Then by Theorem 4 in \cite{ryan_clifford_2004}, we can center a ball of radius $R$ at $w$ to get the monogenic polynomials
    \[
        P_{j_2 \dots j_n}(x) = \frac{1}{j!} \sum_{\textrm{permutations}}z_{\sigma(1)}^1(x-w) \cdots z_{\sigma(j)}^1(x-w).
    \]
    Each polynomial in the collection
    \[
    \mathcal{P}(\Omega) = \{P_{j_2 \cdots j_n} ~\vert~ j_2+\cdots+j_n = j, ~0\leq j < \infty\}
    \]
    is monogenic and linearly independent.
    These polynomials generate $f$ as a power (Taylor) series as
    \[
        f(x) = \sum_{j=0}^\infty \left(\sum_{{j_2 \cdots j_n}_{j_2 + \cdots j_n = j}} P_{j_2 \cdots j_n} (x-w) a_{j_2 \cdots j_n}(w) \right),
    \]
    where the coefficients are found using the Cauchy integral
    \[
        a_{j_2 \cdots j_n} = \frac{1}{a_n} \int_{\partial B(w,R)} \frac{\partial^j G(x-w)}{\partial x_2^{j_2} \cdots \partial x_n^{j_n}} \nu(x) f(x) d\Sigma(x).
    \]
    Each coefficient $a_{j_2 \cdots j_n} \in \G_n^+$. \textcolor{red}{Yes but these are coming in as a right module multiplication. So this should be noted and checked}
    \item This series converges uniformly to $f$ for points $x\in \ball$.
\end{itemize}


We have now found that all monogenic fields are generated as power series of homogeneous polynomials in the variables $z_j^i$. Thus, we have a direct route between the algebras $\algebra{B_j^i}(\ball)$ and the monogenic spinor fields $\monogenics(\ball)$.  In each algebra $\algebra{B_j^i}(\ball)$ the $z_j^i$ act much like a realization of $z\in \C$.  We will find that the action of the spin characters on $z_j^i$ can be understood and extended through the power series to all monogenic spinors. The power series representation seen here is one of the strong reasons to utilize geometric calculus and study the results of Clifford analysis. 



\subsubsection{Correspondence}

The functions $z_j^i$ play a crucial role in the above power series representation but they also play a key part in determining the behavior of the spin characters $\mu \in \characters$.  If we are able to deduce the action $\mu(z_j^i)$, then we can extend this to any monogenic $f$ via the power series representation. Note that $\mu(1)=1$ since it is an algebra homomorphish and so for any $2$-blade $B$ and $\mu \in \characters(\ball)$ that the image of the axial algebras $\mathbb{A}_B=\mu(\algebra{B}(\ball))$ are all commutative subalgebras of $\G_n^+$.  In particular, for a constant $\alpha+\beta B \in \algebra{B}(\ball)$, $\mu(\alpha+\beta B)=\alpha+\beta B$ by definition and so we retrieve $\mathbb{A}_B$ must be generated by linear combinations of the scalar $1$ and the bivector $B$.  Thus, $\mathbb{A}_B$ is an isomorphic copy of $\G_2^+ \cong \C$ as the even subalgebra of the $B$-plane.

Working in terms of an arbitrary basis and applying $\mu$ yields
\[
\mu(z_j^i) = \alpha_j^i + \beta_j^i B_j^i,
\]
for some constants $\alpha_j^i$ and $\alpha_j^i$.  The $z_j^i$ are not independent from one another.  In fact, we have two key relationships in that
\begin{equation}
\label{eq:z_reciprocal_relationship}
z_j^i B_i^j  = -z_i^j.
\end{equation}
Similarly, we have
\begin{equation}
\label{eq:z_relationship}
z_j^i = z_j^k + z_k^i B_j^k.
\end{equation}

Thus, we can take $\mu$ of Equations \ref{eq:z_reciprocal_relationship} and \ref{eq:z_relationship} and determine a relationship on the constants $\alpha_j^i$ and $\beta_j^i$. First, using Equation \ref{eq:z_reciprocal_relationship}
\[
\mu(z_j^i B_i^j) = \mu(z_j^i) B_i^j = -\mu(z_i^j)
\]
yields
\[
(\alpha_j^i + \beta_j^iB_j^i)B_i^j = \beta_j^i + \alpha_j^i B_i^j = - \alpha_i^j - \beta_i^j B_i^j 
\]
and so $\alpha_i^j = -\beta_j^i$ for all $i \neq j$. Next, using Equation \ref{eq:z_relationship}
\[
\mu(z_j^i) = \mu(z_j^k + z_k^i B_j^k) = \mu(z_j^k)+\mu(z_k^i)B_j^k
\]
and so
\[
a_j^i + b_j^i B_j^i = \alpha_j^k + \beta_j^kB_j^k + (\alpha_k^i + \beta_k^i B_k^i)B_j^k = \alpha_j^k + \beta_k^i B_j^i + (\alpha_k^i + \beta_j^k)B_j^k
\]
yields the relationships $\alpha_j^i = \alpha_j^k$, $\beta_j^i = \beta_k^i$, and $\alpha_k^i=-\beta_j^k$. 

Briefly, picture $\alpha_j^i$ and $\beta_j^i$ as components of the $n \times n$ matrices $\alpha$ and $\beta$.  We can index rows by the superscript and columns by the subscript and see that $\alpha$ and $\beta$ both have zero diagonal (since we do not have functions $z_i^i$). The relationship $\alpha_i^j = -\beta_j^i$ for $i\neq j$ then shows that $\alpha = -\beta^\top$.  Then we have $\alpha_j^i = \alpha_j^k$ for $i\neq j \neq k$ shows that $\alpha$ is constant along rows and hence $\beta$ is constant along columns (which shows $\alpha = -\beta^\top$ is consistent with the additional relationship $\beta_j^i = \beta_k^i$). The final relationship $\alpha_k^i = -\beta_j^k$ is consistent as well. The matrices $\alpha$ and $\beta$ are thus uniquely determined by $n$ numbers.  Moreover, treating $\mu(z_j^i)=z_j^i(x_\mu)$ for some $x_\mu \in \R^n$ satisfies the relationships granted above. Thus, we simply find the $x_\mu$ such that we retrieve the desired components for $\alpha$ and $\beta$.  

Using the power series representation for a monogenic spinor $f$ we can extend $\mu$ to act on $\monogenics^+(\ball)$ by the multiplicative and $\G_n^+$ linear nature of $\mu$ since we also note again that the coefficients $a_{j_2 \cdots j_n} \in \G_n^+$. Using the correspondence, we then realize $\mu(f)=f(x_\mu)$ for the corresponding $x_\mu \in \R^n$. To see that this point $x_\mu \in \ball$, we take a field defined on $\G_n(\R^n)$ and monogenic in $\G_n(\Omega)$. For any $x_0 \in \R^n \setminus \ball$ we have the field $E(x_0 - x)$ is monogenic for $x\in \ball$. Then for a spin character $\mu$ we have a sequence of functions $E_n \to E(x_\mu - x)$ such that $\mu(E_n)$ is bounded for all $n$ but diverges in the limit.   \textcolor{red}{Can we actually just argue that we can determine all $x_0$ such that $E(x_0-x)$ is monogenic on $\Omega$ therefore we can determine $\R^n \setminus \Omega$?}

\todo[inline]{Make thie more explicit and do an example or something in 3D. Show that $x\mu$ is in the ball.Finish this and note that this proves the theorem.}

\todo[inline]{I'm not even sure we need to do this with $\Omega=\ball$ other than for part of the proof with the power series. But if $\Omega$ is compact, it fits inside a ball of some radius $r$ and so we should still be able to represent all the monogenics on $\Omega$ with this. The trick is we have a function that is monogenic except at a point. }

\todo[inline]{If work with weak monogenic functions then we can probably use mollifiers and stitch together monogenics on $\Omega$ from various open balls in $\Omega$ that are monogenic except at some set of measure zero. Then this should allow us to probably speak more accurately about the delta function and $E$ and probably suup this all up to determine the homeomorphism type of any embedded manifold.}

\subsection{Discussion}

Perhaps the above result should not be so surprising.  One could venture to the Atiyah-Singer index theorem which relates the topological information of a manifold with the elliptic operators.  In particular, the Dirac operator (the gradient $\grad$) is indeed elliptic. Indeed, this seemingly sparks the motivation for the Calder\'on problem.  There, the elliptic operator is the Laplace-Beltrami operator $\Delta$.  However, this is an inverse problem in which we do not know the space (or the metric) and are asked to, in a sense, determine the Laplace-Beltrami operator from information on the boundary of a Riemannian manifold.  With this boundary data, one would hopefully be able to decipher $\Delta$ and as such, construct a copy of the desired Riemannian manifold.




%%%%%%%%%%%%%%%%%%%%%%%%%%%%%%%%%%%%%%%%%%%%%%%%%%%%%%%%%%%%%%%%%%%%%%%%%%%%%%%%%%%%%%%%%%%%%%%%%%%%%%%%%%%%%%%%%%%%%%%%%%
% Chapter 4
%%%%%%%%%%%%%%%%%%%%%%%%%%%%%%%%%%%%%%%%%%%%%%%%%%%%%%%%%%%%%%%%%%%%%%%%%%%%%%%%%%%%%%%%%%%%%%%%%%%%%%%%%%%%%%%%%%%%%%%%%%

\chapter{Inverse problems}

\section{Tomography}
\label{sec:tomography}
There is an application in mind with the toolbox we have developed. This is the Calder\'on problem. This physical inverse problem is due to  Alberto Calder\'on who asked how much information of a domain can we determine from measurements along the boundary of the domain. To conduct this experiment physically, one applies a voltage along subsets of the boundary of a given domain and the user measures the outgoing current flux. It is this set of information, the boundary $\partial M$, the input voltage $\phi$, and the measured flux $j$ that is accessible to the user. From this information, can one determine the conductivity of the interior $M$? This is the Electrical Impedance Tomography (EIT) problem. \todo{citations}

Other forms of this problem exist. For example, magnetic impedance tomography \todo{citation}, ultrasound tomography \todo{citation}, and magnetic resonance imaging are all examples of tomography.  Fundamentally, these problems exist to determine the interior structure of materials that we do not wish to, or, cannot destroy to determine more. To make an approach to these problems in general, we can consider geometrical analogs. For example, in EIT (at least in dimensions $n>2$, one can do away with the notion of the conductivity by replacing the matrix with an intrinsic Riemannian metric. 

Tomography is useful, yet, challenging practice for which there are unanswered questions. For example, it has yet to been proved that the smooth EIT problem with complete boundary measurements even has a solution. \todo{add citations and other results here} One may ask just how much information is necessary to solve the EIT, or related tomography problems. This line of thought has lead researchers to consider generalizations using differential forms \todo{sources}. Using forms, there is less restriction on the types of functions we use to perform tomography and, moreover, what information we allow ourselves to know along the boundary. 

\subsection{Forward problem}

\subsubsection{Electrical Impedance Tomography}
Let $M$ be a smooth, compact, oriented, Euclidean, 3-dimensional region in $\R^3$ with boundary $\partial M$; $M$ plays the role of the domain we wish to perform EIT on. Take $\sigma$ to be a symmetric positive definite matrix to play the role of a conductivity. If $\sigma$ can be diagonalized as an scalar field times an identity matrix, we say that $M$ is constructed of \emph{isotropic} material, otherwise $M$ is made of \emph{anisotropic} material. We have access to the boundary $\boundary$ and we to this end, we make choices of a static scalar potential (voltage) $\phi$ to apply along $\boundary$. This applied voltage induces the potential $u$ in the interior of $M$. Since $M$ is Euclidean, we have the freedom to choose a global basis for which the metric coefficients satisfy $g_{ij}=\delta_{ij}$. Thus, we construct the $M$ as a geometric manifold, where each geometric tangent space is Euclidean $C\ell(T_xM, |\cdot|)$, so that we are working with multivector fields in $\G_n(M)$. Finally, we posit that $M$ is built from an electrically conductive Ohmic material. Succinctly, the scalar potential $u$ and the current $\blade{j}$ satisfy Ohm's law 
\begin{equation}
\label{eq:ohms_law}
-\sigma \grad \wedge u= \blade{j}
\end{equation}
on the entirety of $M$. We also put $\blade{E}\coloneqq \grad \wedge u$ as the electric field. 

Inside $M$ there must be no free charges that can accumulate and we arrive at the following conservation law
\begin{equation}
\label{eq:conservation_law}
\int_{\boundary} \blade{j} \cdot \blade{\nu} \mu_\partial = \int_{\boundary} \projection_{\blade{I}_\partial}(\blade{j}^\perp) \cdot dX_{n-1} = 0
\end{equation}
due to \cref{prop:flux}. Via Stokes' theorem through \cref{eq:stokes_theorem,eq:stokes_theorem_dual} we arrive at the conclusion that 
\begin{equation}
\grad \cdot \blade{j}= 0.
\end{equation}
Thus, for the scalar potential we have
\begin{equation}
\grad \cdot (\sigma \grad \wedge u) = 0,
\end{equation} 
as an equivalent condition to \cref{eq:conservation_law}. A more thorough analysis can be found in \cite{feldman_calderproblem_nodate}. 

\todo{left off here}

Taking some arbitary basis, conductivity matrix assumes the components $\sigma_{ij}$ for $i,j=1,2,3$.  Via \cite{uhlmann_inverse_2014} in dimension $n>2$, we can realize that the conductivity matrix can be replaced with an intrinsic Riemannian metric with the components in this basis given by
\begin{equation}
\label{eq:conductivity_metric}
    g_{ij} = (\det \sigma^{k\ell} )^{\frac{1}{n-2}} (\sigma^{ij})^{-1}, \quad \sigma^{ij} = (\det g_{k\ell})^{\frac{1}{2}} (g_{ij})^{-1}.
\end{equation}
It is worth noting that these cannot hold in dimension $n=2$. Due to \cref{eq:conductivity_metric}, we can remove the extrinsic need of $\sigma$ with an intrinsic $g$ on the Clifford bundle structure. That is, we are working with $\G(M)$ where each geometric tangent space is given by $C\ell(T_xM,g_x)$. Hence, Ohm's law is given as
\begin{equation}
-\grad \wedge u = \blade{j}.
\end{equation}
Then by \cref{eq:conservation_law}, we find the scalar potential is harmonic
\begin{equation}
\Delta u = 0 \quad \textrm{in $M$}.
\end{equation}

In the realm of EIT, the Dirichlet data $\phi$ amounts to an input voltage along the boundary and by Ohm's law $\blade{j}=\grad \wedge u$ provides us the current. For any given solution to the boundary value problem, there is the corresponding Neumann data is the outward normal derivative of the solution $u$, $\nabla_{\blade{\nu}} \phi$. In this case, all vectors are spatial and since $\blade{\nu}$ is unital, $\blade{\nu}=\blade{\nu}^{-1}$ which allows us to note
\begin{equation}
\nabla_{\blade{\nu}} \phi = \blade{\nu}\rfloor (\grad \wedge \phi) = (\grad \wedge \phi) \cdot \blade{\nu} = \projection_{\blade{\nu}} (\grad \wedge \phi) \blade{\nu},
\end{equation}
with the last equality by \cref{eq:projection_inner_product}. The sole difference in interpration lies in the fact that the projection $\projection_{\blade{\nu}}(\grad \wedge u)$ is vector valued whereas $\nabla_{\blade{\nu}}\phi$ is scalar valued. Since the span of $\blade{\nu}$ is one dimensional, the difference is only in taking the whole outward component of $\grad \wedge \phi$ itself or the coefficient thereof. This motivates the so called Voltage-to-Current (VC) operator or  \emph{Dirichlet-to-Neumann (DN) map}
\begin{equation}
\label{eq:classical_dn_map}
\Lambda_\textrm{Cl} \phi = \projection_{\blade{\nu}}(\grad \wedge u),
\end{equation}
and we put $\Lambda_{\textrm{Cl}} \phi = \normalcurrent$ as the normal component of the boundary current $\blade{j}\vert_{\boundary}$. he inverse problem is to determine $g$ from complete knowledge of $\Lambda_{\textrm{Cl}}$.


\subsubsection{Magnetic Impedance Tomography}

Tomography can be performed using magnetic fields as well. In this case, we consider the boundary value problem for the magnetic vector field $\blade{h}$ by
\begin{equation}
\label{eq:magnetic_forward_problem}
\begin{cases}
\Delta \blade{h} = 0, ~\grad \rfloor \blade{h}=0 & \textrm{in $M$}\\
\blade{\nu}\cross \blade{h} = \tangentialcurrent,
\end{cases}
\end{equation}
\todo{this is right, but it will be rotated from what we do with the electric. So we can always multiply by $I_{\partial}^{-1}$.}
where $\tangentialcurrent$ is the tangential component of the boundary current $\blade{j}\vert_{\boundary}$. It becomes quite clear there is a direct relationship between the electric and magnetic impedance tomography problems. We shall examine this further.

Via Maxwell's equations, we note Ampere's law
\begin{equation}
\grad \cross \blade{h} = \current.
\end{equation}
Via \cref{rem:cross_product}, we see
\begin{equation}
\grad \rfloor \blade{h}^\perp,
\end{equation}
is equivalent and this leads us to define $\magneticbivector\coloneqq \blade{h}^\perp$ as the \emph{magnetic bivector field}. In \cref{eq:magnetic_forward_problem}, we can note that
\begin{equation}
\grad \rfloor \blade{h} = \grad \wedge \magneticbivector = 0
\end{equation}
and moreover
\begin{equation}
\Delta \blade{h} = \grad \rfloor (\grad \wedge \blade{h}) = \grad \rfloor (\grad \rfloor \magneticbivector)\blade{I} = (-1)^{3n(n-1)/2+p} \left(\grad \wedge (\grad \rfloor \magneticbivector)\right)^\perp.
\end{equation}
Finally, with another application of \cref{rem:cross_product}, we find \cref{eq:magnetic_forward_problem} can be written equivalently as
\begin{equation}
\begin{cases}
\Delta \magneticbivector = 0, ~\grad \wedge \magneticbivector=0 & \textrm{in $M$}\\
\blade{\nu}\rfloor \magneticbivector = \tangentialcurrent,
\end{cases}
\end{equation}
in terms of the magnetic bivector field $\magneticbivector$.

it follows that 
\begin{equation}
-\sigma \grad \wedge u^\phi = \grad \cdot b,
\end{equation}
where $b$ is the magnetic bivector field.


\subsubsection{Generalization to forms}

This problem can be cast in a new light by considering harmonic $r$-forms instead of a harmonic 0-form $u$. 
Given some $\varphi \in \Omega^r(\boundary)$, we have the boundary value problem
\begin{equation}
\label{eq:bvp_forms}
\begin{cases} 
\Delta \alpha_r = 0, & \textrm{in $M$}\\
\iota^* \alpha_r = \varphi, \quad \iota^*(\delta \alpha_r) = 0 & \textrm{on $\boundary$}.
\end{cases}
\end{equation}
As stated in \cite{belishev_dirichlet_2008}, there exists a solution $\alpha_r$ to this problem up to a monogenic Dirichlet field $\lambda_D$.

\todo{belishev sharafutdinov and Shonkwiler sharafutdinov definitions}
Note that the operator $\Lambda$ is often referred to as the \emph{scalar} DN map since the input is the scalar field $\phi$ whereas a more general operator on differential $r$-forms has been described in \cite{belishev_dirichlet_2008,sharafutdinov_complete_2013}. There, we begin with equation \cref{eq:bvp_forms}. The DN map is extended to $r$-forms by
\begin{equation}
\Lambda \varphi = \iota^* (\star d \alpha_r).
\end{equation}
In terms of the multivector equivalent $A_r$, we find
\begin{equation}
\iota^*(\star d \alpha_r) = \projection_{\blade{I}_\partial} ((\grad \wedge A_r)^\star )\cdot dX_{n-r-1}^\dagger = 
\end{equation}
\todo{this should also be some kind of rotated version of $P_{I_\partial}(\grad \rfloor A_r^\perp$} One should note that in the case of a scalar potential 
\begin{equation}
\Lambda_\textrm{Cl} \phi = \Lambda \phi
\end{equation}

\vspace*{5pt}
\noindent\textbf{Calder\'on problem.} Let $\Omega$ be an unknown Riemannian manifold with unknown metric $g$ and with known boundary $\Sigma$ and known DN operator $\Lambda$. Can one recover $\Omega$ and the spatial inner product $g$ from knowledge of $\Sigma$ and $\Lambda$?
\vspace*{5pt}

\subsection{Multivector tomography}

There are two notable related questions that can be stated in terms of multivectors. First, the most natural boundary value problems are 
\begin{equation}
\label{eq:multivector_harmonic_bvp}
\begin{cases}
\Delta A = 0 & \textrm{in $M$},\\
A\vert_{\partial M} = B\vert_{\partial M} & \textrm{on $\partial M$},
\end{cases}
\end{equation}
and
\begin{equation}
\label{eq:multivector_monogenic_bvp}
\begin{cases}
\grad A = 0 & \textrm{in $M$},\\
A\vert_{\partial M} = B\vert_{\partial M} & \textrm{on $\partial M$}.
\end{cases}
\end{equation}
It should be noted that we have
\begin{equation}
A\vert_{\partial M} = \projection_{\blade{I}_\partial}(A) + \rejection_{\blade{I}_\partial}(A) = \projection_{\blade{I}_\partial}(A) + \projection_{\blade{\nu}}(A)
\end{equation}
in order to consider all boundary values for a multivector. 

\todo{Show that a multivector DN map is well defined. There are sort of 4 options here.}

\subsection{Recovery}

With the DN operator, we can reconstruct the boundary four current $J$.  On $\Sigma$, we have the gradient $\grad_\Sigma$ inherited from $\grad$ on $\Omega$.  In particular, we have the relationship
\[
\grad_\Sigma \phi = \projection{I_\Sigma}{\grad \phi},
\]
which is accessible with our knowledge of $\phi$ and $\Sigma$. The boundary current is then
\[
\current\vert_{\Sigma} = \grad_\Sigma \phi + \Lambda(\phi).
\]
Though we do not have access to $u^\phi$ directly, we do know that $\Delta u^\phi = \rho$ and as such we have the boundary four current by
\[
J\vert_\Sigma = \Delta u^\phi\vert_\Sigma \gamma_0 + \current\vert_\Sigma
\]
as well as the interior four current $J = \current$ since the interior is free of charges.  Defining the the four vector potential as before, we arrive at the extra equation $\Delta \vectorpotential = \current$ in $\Omega$. Once again define the magnetic bivector field $b=\grad \wedge \vectorpotential$ and we note that Ohm's law implies $\grad \cdot b = -\grad \wedge u^\phi$ in $\Omega$ and so the parabivector field $f=u^\phi + b$ is spatially monogenic since we also have $\grad \wedge b = 0$.  This all holds assuming that we can solve the electromagnetic Neumann boundary value problem
\[
\begin{cases} \Delta A = \current & \textrm{in $\Omega$}\\ A = A_\Sigma & \textrm{on $\Sigma$} \end{cases}
\]
\todo[inline]{Show that we can determine the magnetic potential $A_\Sigma$ on the boundary. This may also show that the two notions of the DN operator are equivalent. That'd be nice.}

\todo[inline]{If we show there is always a unique monogenic conjugate $b$ for any harmonic $u$ then this must be what we are doing here. Is this gauranteed by the Cauchy integral?}

\subsubsection{Ohm's law}
and we arrive at $\Delta u = 0$ for the scalar potential and $\Delta \vectorpotential = \current$ for the magnetic vector potential. In terms of the magnetic field bivector, we have $\grad \cdot b = \current$ and once again by Ohm's law we have $-\grad \wedge u^\phi = \grad \cdot b$. This leads us to consider the parabivector field $f=u+b$. We can note that $f$ is (spatially) monogenic since 
\[
\grad f = 0 ~\iff~ -\grad \wedge u^\phi =  \grad \cdot b ~\textrm{and}~ \grad \wedge b = 0,
\]
is satisfied. We see now that the fact that the body $\Omega$ is ohmic gives us a necessary coupling between the scalar potential and the magnetic field.
The classical forward problem in terms of geometric calculus is given by the following scenario. We have an ohmic $M$ and we find the electrostatic potential $u$ satisfying the Dirichlet problem
\begin{equation}
\label{eq:dirichlet_problem}
\begin{cases} \Delta u^\phi = 0 & \textrm{ in $M$} \\  u^\phi \vert_{\boundary} = \phi & \textrm{ on $\boundary$}. \end{cases}.
\end{equation}



Though briefly we mentioned $\Omega$ as a Riemannian manifold, we now take $\Omega$ to be a region in $\R^n$ for brevity. Using the DN operator, one can define a \emph{Hilbert transform} by
\[
T \phi  = d\Lambda^{-1} \phi,
\]
as in \cite{belishev_dirichlet_2008}. It has yet to be shown that this definition coincides with the definition in \cite{brackx_hilbert_2008}, but there is reason to believe they are related. The classical Hilbert transform on $\C$ inputs a harmonic function and outputs another harmonic function $v$ such that $u+iv$ is holomorphic. Essentially, this translates into finding a conjugate bivector field $b$ to $u^\phi$ such that $u^\phi +b$ is monogenic. First, we require $\phi$ satisfies
\todo[inline]{This statement should come from the lagrangian perspective hopefully.}
\begin{equation}
\label{eq:conjugate_requirement}
\left( \Lambda + (-1)^{n}d\Lambda^{-1}d\right)\phi = 0,
\end{equation}
where $d$ is the exterior derivative on forms. \textcolor{red}{They show how to find the image of this, perhaps I can show what the kernel is.} As shown earlier in Section \ref{subsec:diff_forms}, $d$ amounts to $\grad \wedge$ on the multivector field constituent of a form.  When condition \ref{eq:conjugate_requirement} is met, there exists a \emph{conjugate form} $\epsilon \in \Omega^{n-2}(M)$. As well, $\epsilon$ is also coclosed in that $\delta \epsilon=0$. To retrieve the constituent $(n-2)$-vector $E$, we just note $\epsilon = E \cdot dX_k$. Given Hodge duality, we have a 2-form $\beta$ such that $\star\beta = \epsilon$ and the corresponding bivector $b^\star=E$.  Combining the fields $u^\phi$ and $b$ into the parabivector $f=u^\phi+b \in \G_n^{0+2}(\Omega)$. We then note that $f$ is monogenic if and only if
\[
\grad \wedge u = -\grad \cdot b \qquad \textrm{and} \qquad \grad \wedge b = 0.
\]

\begin{lemma}
Given the fields $u^\phi$ and $b$ as above, the corresponding parabivector field
\[
f=u^\phi +b
\]
is monogenic.
\end{lemma}
\begin{proof}
Let $\star \beta^\psi = \epsilon$ as before and note that 
\begin{equation}
\label{eq:conjugate_belishev}
d u^\phi = \star d \epsilon = \star d \star \beta^\psi,  
\end{equation}
as shown in Theorem 5.1 in \cite{belishev_dirichlet_2008}. The multivector equivalent of the right hand side of Equation \cite{eq:conjugate_belishev} yields
\begin{align*}
(\grad \wedge b^\star )^\star &= [(\grad \cdot b^\dagger) I]^\star\\
    &= [I^{-1} ((\grad \cdot b^\dagger) I)]^\dagger\\
    &= ((\grad \cdot b^\dagger)I)^\dagger I\\
    &= \grad \cdot b^\dagger && \textrm{since $\dagger$ of a vector is trivial}\\
    &= -\grad \cdot b. && \textrm{since $\dagger$ of a bivector is -1}
\end{align*}
\textcolor{red}{Perhaps I should just show this property in the differntial forms section.} Thus, we have $\grad \wedge u + \grad \cdot b = 0$. Since $\epsilon$ is coclosed we have
\begin{align*}
0=\grad \cdot b^\star &= \grad \cdot (I^{-1} b)^\dagger \\
    &= \grad \cdot (b^\dagger I)\\
    &= (\grad \wedge b^\dagger) I\\
  \implies ~0  &= \grad \wedge b.
\end{align*}
\textcolor{red}{Perhaps I should just show this property in the differntial forms section.} Thus $\grad f =0$ and $F$ is monogenic.
\end{proof}

We have shown that conjugate forms give rise to monogenic fields.  We now seek to determine for what boundary conditions $\phi$ we have at our disposal. Let $E^\parallel \coloneqq \projection{I_\Sigma}{E}$, with $I_\Sigma$ the boundary pseudoscalar satisfying $\nu I_\Sigma = I$. Hence by Equation \ref{eq:projection_rejection_vectors} we have $E^\parallel = \rejection_{\nu}(E)$ then in investigating the requirement from Equation \ref{eq:conjugate_requirement} we find the multivector equivalent
\begin{align*}
    (\Lambda + (-1)^n (\grad \wedge) \Lambda^{-1} (\grad \wedge))\phi &= E^\perp + (-1)^n T E^\parallel
\end{align*}
so we arrive at the fact that we must have
\[
E^\perp = (-1)^{n-1} T E^\parallel.
\]
In other words,
\[
T  \rejection_\nu(E)= (-1)^{n-1}\projection{\nu}{E}.
\]
Thus, the Hilbert transform maps tangential components of $\grad u^\phi = E$ to nontangential boundary components on the boundary.




%%%%%%%%%%%%%%%%%%%%%%%%%%%%%%%%%%%%%%%%%%%%%%%%%%%%%%%%%%%%%%%%%%%%%%%%%%%%%%%%%%%%%%%%%%%%%%%%%%%%%%%%%%%%%%%%%%%%%%%%%%
%%%%%%%%%%%%%%%%%%%%%%%%%%%%%%%%%%%%%%%%%%%%%%%%%%%%%%%%%%%%%%%%%%%%%%%%%%%%%%%%%%%%%%%%%%%%%%%%%%%%%%%%%%%%%%%%%%%%%%%%%%


%
%\section{Conclusion}
%\todo[inline]{Write a conclusion.}
%
%\section{Extras}
%
%Journals:
%\begin{itemize}
%    \item Advances in Applies Clifford Algebras.
%\end{itemize}

% EXTRA STUFF BELOW
%\section{Further questions}
%\subsection{Generating axial monogenics}

The following questions remain for a domain in $\R^3$.

\begin{question}
    For what boundary values $\varphi \in C_\infty(\Sigma)$ can we generate axial monogenics?
\end{question}

\begin{question}
    Do these boundary values exhaust the whole axial algebra $\algebra{\omega}$?
\end{question}

Fix an axis $\omega$ which defines the blade $B = \omega I$ and thus defines the $B$-plane in $\R^3$.  Then, let $f=u+\beta B$ be an $\omega$-axial monogenic.  We can then determine the boundary values for $f$ on $\Sigma$ by orthogonal projection onto the $B$-plane.  That is, we care only about the components of $f$ perpendicular to the axis $\omega$ and hence we take for $\zeta \in \Sigma$
\[
\zeta^\perp = \omega \omega \wedge \zeta = (x\cdot B)B^{-1}.
\]
showing the relationship between projection onto a plane and being orthogonal to an axis in $\R^3$. Specifically, this means that the relationship $f(x)=f(x+t\omega)$ can be written as
\[
f(x)=f((x\cdot B)B^{-1}),
\]
in that we only care about the portion of $x$ along the plane given by $B$.  Thus, for $\xi \in \Sigma$ we have
\[
f(\xi) = f((\xi \cdot B)B^{-1}).
\]

\begin{figure}[H]
	\centering
	%\def\svgwidth{\columnwidth}
	\resizebox{\columnwidth}{!}{\input{omega_axial.pdf_tex}}
\end{figure}

\textcolor{red}{So boundary values of axial monogenics are axial and...?.}

\begin{example}
    Consider the 3-dimensional example with $M=B_3$ and $\Sigma=S^2$.  Let $e_1,e_2,e_3$ be a global orthonormal basis and let $g_{ij}=\delta_{ij}$.  Then let $B=e_1 \wedge e_2$.  Then the paravector field $f(x^1,x^2,x^3)=x^1+x^2B$ is $e_3$-axial. Clearly we can see that $f(x^1,x^2,x^3+t)=f(x^1,x^2,x^3)$ for any $t$.  $f$ is also monogenic as one can show
    \[
        \grad f = e_1 + (e_2 \wedge e_3)I = e_1 - e_1 = 0.
    \]
    Indeed, this $f$ is none other than the complex function $f(z)=z$ with $B$ taking the role of the imaginary unit $i$. 

    Let $x=x^1e_1 + x^2e_2 + x^3e_3$.  Then, 
    \[
        B (x\cdot B) = (e_1e_2)( x^1e_2 -x^2 e_1 ) = x^1 e_1 + x^2 e_2.
    \] 
    Thus, for $\xi \in S^2$, we have $f(\xi)=\xi^1 +\xi^2 B$.
\end{example}

\todo[inline]{If we consider now every $\omega$-axial monogenic can be written as a power series, if we can construct $z$ we should be done...?}

It is clear that we can define a monogenic field $f=u+b$ via the Cauchy integral, but we then require $\nabla_\omega f = 0$.  Let $f=\cauchy[\varphi](x)$, then we must have
\[
\nabla_\omega \proj{0}{\cauchy[\varphi](x)} = 0 \qquad \textrm{and} \qquad \nabla_\omega \proj{2}{\cauchy[\varphi](x)}=0.
\]
The first condition yields
\[
0 = \int_\Sigma \frac{(\nu(\zeta)\cdot x) (\omega \cdot x)}{|x-\zeta|^2} \phi(\zeta) d\Sigma(\zeta).
\]


\begin{theorem}
    For any $\omega \in Gr(1,3)$ we have that $\algebra{\omega}\subset \monogenics$. 
\end{theorem}
\begin{proof}
    \textcolor{red}{This seems to be saying that we need boundary values in some hardy space or something. They defined this conjugacy thing as $G$.}
    Fix a unit vector $\omega$.  We want to show that for any $f=u+b\in \algebra{\omega}$ that $\iota^* u=\phi$ satisfies \ref{eq:conjugacy_requirement}.  That is,
    \[
        G\phi = (\Lambda - d\Lambda^{-1}d) \phi = 0.
    \]
    Note that $\phi$ is the trace of a harmonic function, so this operator is well defined.  Note that the equation
    \[
        \Lambda \psi = d \phi
    \]
    has a solution
\end{proof}

\section{Radon transform and integral geometry}

I feel like there is some way to go from projection onto subspaces as a map to grassmannians and reconstructing the manifold.  It's like a morse function type of thing.  Radon transforms also come to mind.

\section{Relation to the BC Method}

\textcolor{red}{Describe how this process can lead to the BC method in dimension $n=2$}


\section{Conclusion}


\appendix
\section{Appendix}

\todo[inline]{Put axial condition for cauchy integral and some other quick proofs in here.}

\subsection{Spin fibration}
maybe pose this as a question in relation to using the 2d belishev stuff.

\textcolor{red}{The inner product for characters is what you use for fourier theory, maybe we can do something here with characters as maps to the grassmannian? Do these form some kind of orthogonal basis? Also, the Dirac operator and Laplacian are spin invariant! This is what they use the $\mathbb{H}$ module structure for!}

A main question to answer now is how the $B$-planar algebras $\algebra{B}$ relate to the space of monogenic functions $\monogenics$.  In particular, this question seems analogous to the invertibility of a $2$-plane x-ray transform.  Let $f$ be a monogenic, can $f$ be generated by $B$-planar monogenics? Noting that each unit 2-blade corresponds to a unique 2-plane in $\R^n$, we can realize every $B$ as a point in $\Grassmannian{2}{n}$.  Letting $f_B$ be some $B$-planar axial monogenic, is
\[
f = \int_{B \in \Grassmannian{2}{n}} a(B) f_B d \lambda,
\]
where $a(B)$ is a scalar function on $\Grassmannian{2}{n}$ and $d\lambda$ is the Haar measure on $\Grassmannian{2}{n}$ monogenic? Moreover, can any monogenic $f$ be constructed in this manner? First, we start with a lemma describing the form of $f_B$.


\begin{lemma}
    Let $f$ be a monogenic (0+2)-vector and define $f_B \coloneqq \projection{B}{f(\projection{B}{x})}$. Then $f_B$ is $B$-planar and monogenic.  
\end{lemma}
\begin{proof}
    It is clear by definition that $f_B$ is constant along translations of the $B$-plane and can be written as $u_B+\beta b_B$ and so $f_B$ is $B$-planar.  To see $f_B$ is monogenic, let $e_1,\dots,e_n$ be a basis such that $B=e_1e_2$ and $e_i \cdot B = 0$ for $i\neq 1,2$. Then note $\nabla_{e_i} f_B =0$ when $i\neq 1,2$ as well leading to
    \[
        \grad f_B = e^1 \nabla_{e_1}f_B + e^2 \nabla_{e_2}f_B
    \]
    Recall that $f=u+b$ must satisfy
    \[
        \grad \wedge u = \grad \cdot b \qquad \textrm{and} \qquad \grad \wedge b = 0.
    \]
    Specifically,
    \[
        e^1 \wedge \nabla_{e_1} u + e^2 \wedge \nabla_{e_2}u + \cdots + e^n \wedge \nabla_{e_n} = e^1 \cdot \nabla_{e_1} b + e^2 \cdot \nabla_{e_2}b + \cdots + e^n \cdot \nabla_{e_n} b
    \]
    Clearly, $\grad \wedge b_B = 0$, thus we need only show
    \[
        \grad \wedge u_B = \grad \cdot b_B.
    \]
    In particular
\end{proof}


We can note that the $B$-planar monogenics are given by a power series $\sum_{n=0}^\infty a_n (x+yB)^n$ due to the isomorphism of algebras $\mathfrak{spin}(2)\cong \C$ \textcolor{red}{This shouldn't be hard to show without appealing to this isomorphism.} In particular, any $B$-planar monogenic is approximated arbitrarily closely by a homogeneous polynomial of degree $n$ in the variables $x$ and $y$. Moreover, $1$ and $x+yB$ generate the $B$-planar monogenics. $\spingroup$ then acts on $B$. \textcolor{red}{Okay, well maybe there's some nice way to talk about characters as mappings to the grassmannian instead of the circle? Should read more about characters and maybe they are really maps to spin group? They are for the 2d case. Structure space and stuff. Should probably rename some of these things I have.s}


\textcolor{red}{Countable basis for $\monogenics$ ?}


\bibliographystyle{siam}
\bibliography{calderon_problem}





\end{document}
