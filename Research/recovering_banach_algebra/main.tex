\documentclass[12pt]{article}
\usepackage{import}
\usepackage{preamble}
\usepackage{environments}
% package for todo 
\setlength{\marginparwidth}{2cm}
\usepackage[colorinlistoftodos]{todonotes}
\setuptodonotes{size=\scriptsize,backgroundcolor=red!15!white} 

\usepackage{showkeys} %show cites and refs

\title{Commutative Banach Algebras of Multivectors from the Scalar Dirichlet-to-Neumann Operator}
\author{Colin Roberts}



\begin{document}

 \begin{titlingpage}
     \maketitle
     \vfill
     \begin{abstract}
        The problem of determining an unknown Riemannian manifold given the Dirichlet-to-Neumann (DN) operator is known as the Calder\'on problem.  One method of solving this problem in the two dimensional case is through the Boundary Control method.  There, one uses the DN operator to construct a Banach algebra of holomorphic functions on the manifold. The Gelfand transform of this algebra is then homeomorphic to the manifold. In higher dimensions, we replace the complex field with a Clifford algebra and use the DN operator to determine a $\spingroup$ invariant space of monogenic multivector fields. Using a power series representation for monogenic fields, one decomposes the space of monogenics into products of commutative algebras of $(0+2)$-vector fields constant on translations of planes and monogenic in $\R^n$. Using this decomposition, we define spinor characters on the space of monogenic fields that correspond to Dirac measures on the manifold.  The set of these Dirac measures is then homeomorphic to the underlying manifold with the Gelfand topology.
     \end{abstract}
 \end{titlingpage}


\todo[inline]{Replace vector space $V$ with $\vectorspace$.}

\section{Introduction}
In 1980, Alberto Calder\'on proposed an inverse problem in his paper \emph{On an inverse boundary value problem} \cite{calderon_inverse_2006} where he asks if one can determine the conductivity matrix of a medium from Cauchy data supplied on the boundary.  In dimensions $n>2$, this is equivalent to determining a Riemannian manfiold up to isometry from the scalar Dirichlet-to-Neumann (DN) operator \cite{feldman_calderproblem_nodate, salo_calderon_nodate, uhlmann_inverse_2014}. The DN operator takes any given Dirichlet boundary values and outputs the corresponding Neumann data of a solution to Laplace's equation in order to generate the relevant Cauchy data.  

One approach to reconstructing the Riemannian metric in dimension $n=2$ appears in \cite{belishev_calderon_2003}, where the author uses the Boundary--Control (BC) method to determine the manifold up to conformal class. \textcolor{red}{Add in a bunch of other citations to the BC method.} The BC method takes an algebraic approach. Specifically, the DN operator determines the algebra of holomorphic functions on $M$ and realizes $M$ as homeomorphic to the Gelfand spectrum of this commutative algebra. The metric $g$ is then recovered after providing $M$ with a complex structure. In dimension $n=2$, the Laplace-Beltrami operator is conformally invariant, and this result cannot be improved.  An attempt to generalize this approach to dimension $n=3$ can be found in by replacing the complex structure with a quaternionic structure but this has not lead to a complete solution \cite{belishev_algebras_2017, belishev_algebraic_2019}.  It has been shown that when $M$ is the 3-dimensional ball in $\R^3$, there is an associated space of harmonic quaternion fields that has a quaternion spectrum homeomorphic to the ball. But, a connection to the DN operator has not been made, and this method has also not been generalized to higher dimensions.

In this paper, I show that there exists a space of spin characters $\characters$ acting on a $\spingroup$ invariant space of monogenic multivector fields on the $n$-dimensional ball that is homeomorphic to the ball.  We then observe that this space of monogenics is determined from the DN map, and thus recover the ball up to homeomorphism from the boundary data.  This is summarized in two main theorems.
\begin{theorem*}
The set of multiplicative $\spinalgebra$-linear functionals on the $\spingroup$ invariant space of monogenic fields $\monogenics$ on the $n$-dimensional ball $\ball$ is homeomorphic to $\ball$ with the Gelfand topology.
\end{theorem*}
\begin{theorem*}
The scalar DN operator determines the $\spingroup$ invariant space of monogenic fields on regions in $\R^n$.
\end{theorem*}
The second theorem can be extended to Riemannian manifolds quite readily.

We first introduce the Clifford algebra setting. Given a vector space with an inner product, we can create the graded Clifford algebra.  In particular, we extend these Clifford algebras to Clifford algebra valued functions (or multivector fields) on regions $M \subset \R^n$.  Inside the multivector fields sit the even graded multivectors consisting of scalars, bivectors, and other $2k$-vectors. In $\R^2$ with the Euclidean inner product, this space is isomorphic to the $\C$-algebra and so the functions valued in this even sub-Clifford algebra can be thought of as complex valued functions.  Clifford analysis generalizes the notion of holomorphicity to monogenicity and we find that monogenic functions lie in the kernel of the Dirac operator $\grad$ just as $\C$-holomorphic functions lie in the kernel of the Wirtinger derivative $\frac{\partial}{\partial \overline{z}}$. Moreover, one has that $\grad$ is the square root Laplace-Beltrami operator $\Delta = \grad^2$. Even monogenic multivector fields are $\spingroup$ invariant and each grade is harmonic (in the kernel of $\Delta$). 

When $M$ is the $n$-ball, we have that space of even monogenics $\monogenics$ which can be generated by the algebras of even graded $B$-planar monogenic biparavector fields (each field constant on translations of the $B$-plane in $\R^n$). Those generating subalgebras are individually isomorphic to the algebra of holomorphic functions on the complex unit disk $\disk$. On these spaces, one can define $\spinalgebra$-linear multiplicative functionals $\characters$, referred to as spin characters. Each spin character is equivalent to a Dirac measure on the $n$-ball which, with the Gelfand topology, provide a homeomorphic copy of the $n$-ball.

The space of $(0+2)$-vector monogenics is found from the DN operator in the following sense.  The DN operator determines a Hilbert transform on multivector fields that allows one to determine the monogenic conjugate bivector field $b$ corresponding to a scalar solution $u$ to the Laplace equation $\Delta u = 0$ so that $f=u+b$ is monogenic. \textcolor{red}{Haven't actually done this yet} Considering all smooth boundary conditions generates the relevant space of monogenics, from which we determine the space of spin characters. Thus, the DN operator provides a means of constructing a homeomorphic of the $n$-ball.
\todo{reword introduction}

\section{Preliminaries}
The complex algebra $\C$ can be generalized in a handful of ways.  Some of which can be found through the use of Clifford algebras and, more specifically, in geometric algebras.  We define the more general Clifford algebras first and realize geometric algebras as particularly nice Clifford algebras with a quadratic form arising from an inner product. Elements of a geometric algebra are known as multivectors and these multivectors carry a wealth of geometric information in their algebraic structure. $\C$ itself can be realized as a special subalgebra of biparavectors in the geometric algebra on $\R^2$ with the Euclidean inner product and the quaternions $\quat$ are realized as an analogous algebra on $\R^3$. In particular, both $\C$ and $\quat$ arise as the 2- and 3-dimensional even Clifford groups $\Gamma^+$ respectively.

\subsection{Clifford and geometric algebras}

Formally, we let $(V,Q)$ be an $n$-dimensional vector space $V$ over some field $K$ with an arbitrary quadratic form $Q$.  The tensor algebra is given by
\[
\mathcal{T}(V) \coloneqq \bigoplus_{j=0}^\infty V^{\otimes j} = K \bigoplus V \oplus (V\otimes V) \oplus (V\otimes V \otimes V) \oplus \cdots,
\]
where the elements (tensors) inherit a multiplication $\otimes$ (the tensor product). From the tensor algebra $\mathcal{T}(V)$, we can quotient by the ideal generated by $v\otimes v - Q(v)$ to create a new algebra.
\begin{definition}
The \emph{Clifford algebra} $C\ell(V,Q)$ is the quotient algebra
\begin{equation}
C\ell(V,Q) = \mathcal{T}(V) ~ / ~ \langle v \otimes v - Q(v) \rangle.
\end{equation}
\end{definition}
To see how the tensor product descends to the quotient, we let $e_1, \dots, e_n$ be an arbitrary basis for $V$, then we can consider the tensor product of basis elements $e_i \otimes e_j$ which induces a product in the quotient $C\ell(V,Q)$ which we refer to as the \emph{Clifford multiplication}. In this basis, we write this product as concatenation $e_ie_j$ and define the multiplication by
\begin{equation}
\label{eq:clifford_multiplication}
e_i e_j = \begin{cases} Q(e_i) & \textrm{if $i=j$}, \\ e_i \wedge e_j & \textrm{if $i\neq j$},\end{cases}
\end{equation}
where $\wedge$ is the typical exterior product satisfying $v\wedge w = - w\wedge v$ for all $v,w\in V$.  As a consequence, the exterior algebra $\bigwedge(V)$ can be realized as a subalgebra of any Clifford algebra over $V$ or as a Clifford algebra with a trivial quadratic form $Q=0$.  

In the case where $V$ has a (pseudo) inner product $g$, we can induce a quadratic form $Q$ by $Q(v)=g(v,v)$ and give rise to a special type of Clifford algebra which motivates the following definition.
\begin{definition}
Let $V$ be a vector space with an (pseudo) inner product $g(\cdot,\cdot)$. Then taking $Q(\cdot) = g(\cdot,\cdot)$, the Clifford algebra $C \ell(V,Q)$ is called a \emph{geometric algebra}.
\end{definition}
In general, we put $\geometricalg$ and assume the inner product will be given alongside or will be clear from context.  For example, when $V=\R^n$ and we define $Q$ from the Euclidean inner product $|\cdot|$, we have $C\ell(V,Q)=\mathcal{G}(\R^n)$ and moreover we let $\mathcal{G}_n \coloneqq \mathcal{G}(\R^n)$. Geometric algebras are an old and widely studied topic. For more information, see the classical text \cite{hestenes_clifford_1986} or the more modern text \cite{doran_geometric_2003} which also provides a wide range of applications to physics problems. Both these sources include much of the other necessary preliminaries I cover in the remainder of this section. Finally, the paper \cite{chisolm_geometric_2012} proves many of the useful identities and notation used throughout this paper.

\subsubsection{Grading and multivectors}
Note that $C\ell(V,Q)$ is a $\mathbb{Z}$-graded algebra with elements of grade-0 up to elements of grade-$n$. We refer to grade-0 elements as scalars, grade-1 elements as vectors, grade-2 elements as \emph{bivectors}, grade-$r$ elements as \emph{$r$-vectors}, and grade-$n$ elements as \emph{pseudoscalars}. We denote the space of $r$-vectors by $C\ell(V,Q)^r$. For each grade there is a basis of ${n\choose r}$ \emph{$r$-blades} which are $r$-vectors of the form
\begin{equation}
\blade{A_r} = \prod_{j=1}^k v_j, ~\textrm{for linearly independent}~ v_j \in V,
\end{equation}
and we use a boldface to specify that a $r$-vector is a $r$-blade (except for the special case of vectors which can be thought of as $1$-blades). For example, if $\dim(V)=3$, then there are ${3\choose 2}=3$ 2-blades that form a basis for the bivectors and one particular choice of a bivector basis would be the following list of 2-blades
\begin{equation}
\label{eq:3_dim_basis}
\blade{B}_{12} = e_1 \wedge e_2, \quad \blade{B}_{13} = e_1 \wedge e_3, \quad \blade{B}_{23} = e_e \wedge e_3.
\end{equation}
We refer to an $(n-1)$-blade as a \emph{pseudovector} and it should be noted that every $(n-1)$-vector is a pseudovector. In other literature, some will refer to a $r$-blade as a \emph{simple} or a \emph{decomposable} $r$-vector\todo{citations}. 

In general, an element $A \in C\ell(V,Q)$ is written as a linear combination of basis elements of all possible grades and we refer to $A$ as a \emph{multivector}.  To extract the grade-$r$ components of $A$, we use the notation
\begin{equation}
\proj{r}{A}
\end{equation}
to denote the grade-$r$ components of the multivector $A$. Any multivector $A$ can then be given by
\begin{equation}
A = \sum_{r=0}^n \proj{r}{A}
\end{equation}
which shows the decomposition
\begin{equation}
C\ell(V,Q) = \bigoplus_{j=0}^n C\ell(V,Q)^j.
\end{equation}
If $A$ contains only components of a single grade, then we say that $A$ is \emph{homogeneous} and if the components are grade-$r$ we write $A_r$ and refer to $A_r$ as a \emph{homogeneous $r$-vector} or simply a \emph{$r$-vector}.  For example, when we refer to vectors we realize them as 1-vectors and likewise we realize bivectors as 2-vectors. Also of interest will be the elements in
\begin{equation}
 C\ell(V,Q)^{0+2} = C\ell(V,Q)\oplus C\ell(V,Q)^2
\end{equation}
which we refer to as \emph{biparavectors}.

The Clifford multiplication of vectors defined in \ref{eq:clifford_multiplication} can be extended to multiplication of vectors with homogeneous $r$-vectors.  In particular, given a vector $v \in C\ell(V,Q)$ and a homogeneous $r$-vector $A_r \in C\ell(V,Q)$, we have
\begin{equation}
\label{eq:vector_multiplication}
aA_r = \proj{r-1}{aA_r} + \proj{r+1}{aA_r},
\end{equation}
which decomposes the multiplication into a grade lowering \emph{interior product} and a grade raising \emph{exterior product}.  This allows us to extend the Clifford multiplication further. Given a homogeneous grade-$s$ multivector $B_s$, we have
\begin{equation}
\label{eq:general_clifford_multiplication}
A_k B_s = \proj{|r-s|}{A_rB_s} + \proj{|r-s|+2}{A_rB_s} + \cdots + \proj{r+s}{A_rB_s}.
\end{equation}
This rule for multiplication then allows for the multiplication of two general multivectors in $C\ell(V,Q)$. For this multiplication, specific grades of the product are worth noting.
\begin{equation}
\label{eq:dot}
    A_r \cdot B_s \coloneqq \proj{|r-s|}{A_r B_s}
\end{equation}
\begin{equation}
\label{eq:wedge}
    A_r \wedge B_s \coloneqq \proj{r+s}{A_r B_s}
\end{equation}
\begin{equation}
\label{eq:left_contraction}
    A_r \rfloor B_s \coloneqq \proj{s-r}{A_r B_s}
\end{equation}
\begin{equation}
\label{eq:right_contraction}
    A_r \lfloor B_s \coloneqq \proj{r-s}{A_r B_s}.
\end{equation}
These products are particularly emphasized as many helpful identities used in this paper are phrased using these notions. Taking \cref{eq:vector_multiplication,eq:wedge,eq:left_contraction} into mind, we see that we have the grade lowering interior product can be written as
\begin{equation}
    \proj{r-1}{aA_r} = a\rfloor A_r = a \cdot A_r
\end{equation}
and the grade raising exterior product can be written as
\begin{equation}
    \proj{r+1}{aA_r} = a \wedge A_r.
\end{equation}

As discussed, $C\ell(V,Q)$ is naturally a $\mathbb{Z}$-graded algebra but we also find that it carries a $\mathbb{Z}/2\mathbb{Z}$-grading as well. This additional grading can be realized by sorting $k$-vectors in $C\ell(V,Q)$ into the sets where $k$ is even or odd.  We say a $k$-vector is \emph{even} (resp. \emph{odd}) $k$ is even (resp. odd) and in general if a multivector $A$ is a sum of only even (resp. odd) grade elements we also refer to $A$ as even (resp. odd).  Taking note of the multiplication defined in \ref{eq:general_clifford_multiplication}, one can see that the multiplication of even multivectors with another even multivectors outputs an even multivector.  Thus, the even multivectors form closed subalgebra of $C\ell(V,Q)$ which we denote by $C\ell(V,Q)^+$. We end this subsection with a few examples.

\begin{example}
\label{ex:complex_representation}~
Consider $\G_2$ with $e_1$ and $e_2$ the standard vector basis and note that we have $1$ as the basis scalar, and $\blade{B_{12}} = e_1\wedge e_2 = e_1e_2$ as the basis pseudoscalar.  Then, an arbitrary multivector $A$ and $B$ can be specified by
\[
A = a_0 + a_1 e_1 + a_2 e_2 + a_{12} B_{12}, \qquad B = b_0 +b_1 e_1 + b_2 e_2 + b_{12}\blade{B_{12}},
\]
and the graded elements of $A$, for example, can be extracted as
\begin{subequations}
\begin{align}
\proj{0}{A}&=a_0\\
\proj{1}{A}&=a_1 e_1 + a_2 e_2\\
\proj{2}{A}&=a_{12} \blade{B_{12}}.
\end{align}
\end{subequations}
We can then take the product $AB$ to yield
\begin{subequations}
\begin{align}
\proj{0}{AB} = a_0b_0 + a_1 b_1 + a_2 b_2 - a_{12}b_{12}\\
\proj{1}{AB} = (a_0 b_1 + a_1 b_0 - a_2 b_{12} + a_{12} b_2) e_1 + (a_0 b_2 + a_2 b_0 + a_1b_{12} - a_{12} b_1) e_2\\
\proj{2}{AB} = (a_1b_2 - a_2 b_1)\blade{B_{12}}.
\end{align}
\end{subequations}
Most notably, we see that $\blade{B_{12}}^2=-1$ and this allows us to consider a biparavector
\begin{equation}
z = x + y \blade{B_{12}}
\end{equation}
as a representation of the complex number $\zeta = x+ iy$ in $\G_n^{0+2}$.  Thus, the even subalgebra of this Clifford algebra is indeed isomorphic to the complex numbers $\C$. 
\end{example}

\begin{example}
\label{ex:quaternions}
Next, take $\G_3$ with the standard vector basis $e_1,e_2,e_3$, then an arbitrary multivector $A$ is specified by
\begin{equation}
A= a + \alpha_1 e_1 + \alpha_2 e_2 + \alpha_3 e_3 + \beta_{12} \blade{B_{12}} + \beta_{13} \blade{B_{13}} + \beta_{23} \blade{B_{23}} + \mu e_1 \wedge e_2 \wedge e_3
\end{equation}
in general, and we have
\begin{subequations}
\begin{align}
\proj{0}{A}&=a\\
\proj{1}{A}&=\alpha_1 e_1 + \alpha_2 e_2 + \alpha_3 e_3\\
\proj{2}{A}&=\beta_{12} \blade{B_{12}} + \beta_{13} \blade{B_{13}} + \beta_{23} \blade{B_{23}}\\
\proj{3}{A}&= \mu e_1 \wedge e_2 \wedge e_3.
\end{align}
\end{subequations}
Then, let
\begin{equation}
\blade{B_{23}} = e_2 e_3, \quad \blade{B_{31}} = e_3 e_1, \quad \blade{B_{12}} = e_1 e_2,
\end{equation}
and note that we can write a even multivector as
\begin{equation}
q = a + \beta_{23}\blade{B_{23}} + \beta_{31} \blade{B_{31}} + \beta_{12} \blade{B_{12}}.
\end{equation}
Note as well that
\begin{equation}
\blade{B_{23}}^2 = \blade{B_{31}}^2 = \blade{B_{12}}^2 = -1,
\end{equation}
and
\begin{equation}
\blade{B_{23}}\blade{B_{31}}\blade{B_{12}} = +1.
\end{equation}
In this case, this even subalgebra is extremely close to being a copy of the quaternion algebra $\quat$. Indeed, one can arrive at a representation of the quaternions by taking
\begin{equation}
\boldsymbol{i} \leftrightarrow \blade{B_{23}}, \quad \boldsymbol{j} \leftrightarrow -\blade{B_{31}}=\blade{B_{13}}, \quad \boldsymbol{k} \leftrightarrow \blade{B_{12}},
\end{equation}
and noting that we then have $ijk=-1$ as well as $i^2=j^2=k^2=-1$. A more in depth explanation is provided in \cite{doran_geometric_2003}.

Once again, quaternions arise naturally as parabivectors since we can put
\begin{equation}
q= \alpha + u_1 \blade{B_{23}} - u_2 \blade{B_{13}} + u_3 \blade{B_{12}},
\end{equation}
and recover the necessary arithmetic seen in $\quat$.
\end{example}

\begin{remark}
If we take $\G_n$ for $n\geq 2$, then there are natural copies of $\C$ contained inside of $C\ell(V,Q)$. In particular, we have the isomorphism
    \[
        \C \cong \{\lambda + \beta \blade{B} ~\vert~ \lambda,\beta \in C\ell(V,Q)^0,~ \blade{B} \in C\ell(V,Q)^2,~ \blade{B}^2=-1. \},
    \]
   which shows that complex numbers arise as biparavectors. Given the standard basis $e_1,\dots,e_n$ we have copies of $\C$ for each of the ${ n \choose 2}$ unit bivectors $B_{jk}$ with $k=2,\dots,n$ and $j<k$. Note that $\blade{B_{jk}}\blade{B_{jk}}=-1$ and we have the representation of $\C$ since
    \[
        \zeta = x + y\blade{B},
    \]
    behaves as a complex number $z=x+iy$.
\end{remark}

\begin{example}
\label{ex:spacetime_algebra}
We shall not rule out the utility of geometric algebras with pseudo inner products. The classical example is the \emph{spacetime algebra} defined by taking $V=\R^4$ with a vector basis $\gamma_0,\gamma_1,\gamma_2,\gamma_3$ satisfying
\begin{subequations}
\begin{align}
\gamma_0 \cdot \gamma_0 &= -1\\
\gamma_0 \cdot \gamma_i &= 0  &i=1,2,3\\
\gamma_i \cdot \gamma_j &= \delta_{ij}, &i,j=1,2,3.
\end{align}
\end{subequations}
We refer to $\gamma_0$ as \emph{temporal} and $\gamma_i$ for $i=1,2,3$ as \emph{spatial}. For this basis, we can denote the matrix for this inner product $\eta =\operatorname{diag}(-+++)$ and define $Q$ from $\eta$. Then, we have for a vector $A = a_0 \gamma_0 +a_1 \gamma_1 + a_2 \gamma_2 + a_3 \gamma_3$ we have
\[
A\cdot A = -a_0^2 + \sum_{i=1}^3 a_i^2.
\]
\end{example}

\begin{remark}
For the cases with pseudo inner products with $p$ vectors satisfying $e_i^2 = -1$ for $i=1,\dots, p$ and $q$ vectors satisfying $e_j^2=1$ for $q=p+1,\dots,p+q$, we will denote the algebras by $\G_{p,q}$. The spacetime algebra is thus $\G_{1,3}$. 
\end{remark}

\subsubsection{Duality and pseudoscalars}
\label{subsection:duality_and_pseudoscalars}

For the remainder of this paper we will be mostly working with geometric algebras with a positive definite inner product $g$. Given access to an inner product we have a natural isomorphism between $V$ and $V^*$ by the Riesz representation.  Namely, given an arbitrary basis $e_i$ for $V$ there exists the dual basis $f_i$ for $V^*$ such that $f_i(e_j)=\delta_{ij}$.  This dual basis resides inside $V$ itself in the following manner. There is then a unique map $\sharp \colon V^* \to V$ with $f\mapsto f^\sharp$ such that
\[
f_i^\sharp \cdot e_j = \delta_{ij},
\]
where $\delta_{ij}$ is the Kronecker delta symbol. In terms of the geometric algebra, we put $e^i \coloneqq f_i^\sharp$ and can note that $e^i$ is simply a vector in the geometric algebra. For an arbitrary basis $e_1,\dots,e_n$ for $V$, the coefficients for the inner product $g$ are given by $g_{ij}=e_i\cdot e_j$ and we can put $e^i = g^{ij}e_j$ where $g^{ij}$ is the coefficients to matrix inverse of $g_{ij}$.  There is inverse isomorphism $\flat \colon V \to V^*$ given by $e \mapsto e^\flat$ satisfying
\[
e_i^\flat (e_j)= \delta_{ij}.
\]
Given these identifications, there is no need to distinguish between the vector space $V$ and its dual $V^*$ as it suffices to consider $V$ itself with reciprocal basis elements $e^i$ with the application of the scalar product.

A volume element can be defined by $\mu=e_1 \wedge e_2 \wedge \cdots \wedge e_n = \sqrt{|g|} I$ where $\sqrt{|g|}$ is the square root of the determinant of the matrix $g_{ij}$ and $I$ is the unit pseudoscalar. It follows that the unit pseudoscalar is given by $I=\frac{1}{\sqrt{|g|}} e_1 \wedge e_2 \wedge \cdots e_n$. We can define $\mu^{-1}$ such that $\mu^{-1}\mu = 1 = \mu \mu^{-1}$ and analogously $I^{-1}$.  One can equivalently put $e^j = (-1)^{j-1} e_1 \wedge e_2 \wedge \cdots \wedge \breve{e_j} \wedge \cdots \wedge e_n \mu^{-1}$ and note that this gives $\mu^{-1} = e^n \wedge \cdots \wedge e^1$.  Conveniently, the unit pseudoscalar satisfies the relation
\[
IA_k = (-1)^{k(n-1)} A_k I.
\]
Thus, $I$ commutes with the even subalgebra, and anticommutes with the odd subalgebra.  

Note that for a homogeneous $r$-rector $A_r$ we have that $A_r^\perp$ is an $n-r$-vector. Indeed, if we take an invertible $r$-blade $\blade{A_r}$, then we can find the \emph{$\blade{A_r}$-subspace dual} of a multivector $B$ by
\[
B \rfloor \blade{A_r}^{-1}.
\]
The notions of duality here give us geometrical insight. Taking an $s$-blade $\blade{B_s}$ we can note:
\begin{itemize}
    \item If $s>r$, the $\blade{A_r}$-subspace dual of $\blade{B_s}$ vanishes.
    \item If $s=r$, the $\blade{A_r}$-subspace dual of $\blade{B_s}$ is a scalar and is zero if $\blade{B_s}$ contains a vector orthogonal to $\blade{A_r}$.
    \item If $s<r$, the $\blade{A_r}$-subspace dual of $\blade{B_s}$ represents the orthogonal complement of the subspace corresponding to $\blade{B_s}$ in the subspace corresponding to $\blade{A_r}$.
\end{itemize}  
Since the pseudoscalar is a blade representing the entire vector space, this allows one to create dual elements within the entire vector space. Given a multivector $B$, we define the \emph{dual} of $B$ to be
\[
B^\perp \coloneqq B \rfloor I^{-1} = AI^{-1}.
\]
The dual allows one to exchange interior and exterior products in the following way.
\begin{equation}
\label{eq:wedge_to_dot}
 (A \wedge B)^\perp  = A\rfloor B^\perp
\end{equation}
\begin{equation}
\label{eq:dot_to_wedge}
    (A\rfloor B)^\perp = A \wedge B^\perp
\end{equation}
This shows the natural duality between the inner and exterior products and their interpretations as subspace operations. The duality extends further to provide an isomorphism between the spaces of $r$-vectors and $n-r$-vectors since for any $r$-vector $A_r$ we have $A_r^\perp$ is an $n-r$-vector. It is under this isomorphism one can realize that all pseudovectors are $n-1$-blades. 

\begin{example}
\label{ex:cross_product}
Consider $\spacealg$ with the standard orthonormal vector basis $e_1,\dots,e_n$ and Euclidean inner product.  Then, we can define the \emph{cross product} of two vectors $u$ and $v$ by
\[
u \cross v = (u\wedge v)I^{-1},
\]
where we use the bold notation for $\cross$ to distinguish between the bivector commutator product. The special fact of $\spacealg$ is that vectors and bivectors (pseudoscalars in 3-dimensions) are dual to one another. One can also note that the vector $w=u\times v$ is sometimes refered to as axial and in other cases the pseudovector $u\wedge v$ is referred to as axial. 

Referring back to Example \ref{ex:quaternions}, we can realize the cross product of vectors as the bivector commutator product
\[
u \cross v = (u^\perp)\times (v^\perp), 
\]
for which the similar product notation of $\times$ and $\cross$ now becomes transparent. The necessary relationships for the cross product are seen clearly on the products of the basis blades $\blade{B_{23}}, \blade{B_{31}}$, and $\blade{B_{12}}$. In particular, $e_1 = \blade{B_{23}}^\perp$, $e_2 = \blade{B_{31}}^\perp$, and $e_3 = \blade{B_{12}}^\perp$. This also shows that every bivector in a 3-dimensional space is a 2-blade.
\end{example}


\subsubsection{Blades and subspaces}

Each unit $r$-blade $\blade{A_r}$ ($\|\blade{A_r}\|=1$) corresponds to a $r$-dimensional subspace and can be identified with a point in $\Grassmannian{r}{n}$.
There are a handful of reasons to adopt the additional multiplication symbols $\rfloor$ and $\lfloor$. 
\begin{itemize}
    \item The products $\rfloor$ and $\lfloor$ allow us to avoid needing to pay special attention to the specific grade of each multivector in a product. The product $\cdot$ on $A_r$ and $B_s$ depends on $k$ and $s$ and as such given by either $\rfloor$ or $\lfloor$ but one must know $k$ and $s$ in order to define this product exactly. 
    \item We gain geometrical insight on the structure of $r$-blades in terms of their corresponding subspaces. Let $\blade{A_k}$ and $\blade{B_s}$ be nonzero blades with $r,s\geq 1$ then
    \begin{itemize}
        \item $\blade{A_r} \rfloor \blade{B_s} =0$ iff $\blade{A_r}$ contains a nonzero vector orthogonal to $\blade{B_s}$.
        \item If $r<s$ then if $\blade{A_r}\rfloor \blade{B_s} \neq 0$ then the result is a $s-r$-blade representing the orthogonal complement of $\blade{A_r}$ in $\blade{B_s}$.
        \item If $\blade{A_r}$ is a subspace of $\blade{B_s}$ then $\blade{A_r}\blade{B_s} = \blade{A_r}\rfloor \blade{B_s}$.
        \item If $\blade{A_r}$ and $\blade{B_s}$ are orthogonal, then $\blade{A_r}\blade{B_s} = \blade{A_r} \wedge \blade{B_s}$.
    \end{itemize}
\end{itemize}
See \cite{chisolm_geometric_2012} theorem 16.


We also have the identities
\begin{equation}
\label{eq:left_contraction_dot}
A_r \cdot B_s = A_r \rfloor B_s \qquad \textrm{if $k\leq s$}
\end{equation}
\begin{equation}
\label{eq:right_contraction_dot}
A_r \cdot B_s = A_r \lfloor B_s \qquad \textrm{if $k\geq s$}.
\end{equation}
For homogeneous $r$-vectors $A_r$ and $B_r$, the products above simplify to 
\begin{equation}
\label{dot_equivalent_contraction}
    A_r \lfloor B_r = A_r \rfloor B_r = A_r \cdot B_r.
\end{equation}
Using this notation, for a vector $\alpha$ we have
\begin{equation}
\alpha A_k = \alpha \rfloor A_k + \alpha \wedge A_k,
\end{equation}
so the $\cdot$ and $\lfloor$ notation coincide for left multiplication by vectors. If we are given two $k$-blades $A_k = \alpha_1 \wedge \cdots \wedge \alpha_k$ and $B_k = \beta_1 \wedge \cdots \wedge \beta_k$ we have 
\begin{equation}
\label{eq:dot_product}
A_k \cdot B_k^\dagger = \det(\alpha_i \cdot \beta_j )_{i,j=1}^k,
\end{equation}
which is equivalent to $A_k \rfloor B_k$ and $A_k \lfloor B_k$ through \ref{dot_equivalent_contraction} and this is extended to all $k$-vectors as is typically seen when constructing the inner product of $k$-vectors (see \cite{hestenes_clifford_1986}. If we are given two bivectors $B$ and $B'$, then we have another special multiplication
\begin{equation}
\label{eq:bivector_product}
B\times B' \coloneqq \proj{2}{BB'} = \frac{1}{2} (BB' - B'B),
\end{equation}
which is the grade preserving anti-symmetric portion of the product $BB'$ which we refer to as the \emph{bivector commutator product}.


\subsubsection{Projection and rejection}

Given the direct relationship between unit $r$-blades and $r$-dimensional subspaces we can also form a compact way of projecting multivectors into subspaces in a manner closely related to the subspace dual.  In general, given an multivector $B$ the \emph{projection} onto the subspace corresponding to the invertible $\blade{A_r}$ is
\begin{equation}
\label{eq:projection}
\projection_{B}(\blade{A_r}) \coloneqq B\rfloor \blade{A_r} \blade{A_r}^{-1} = (B\rfloor \blade{A_r})\rfloor \blade{A_r}^{-1}
\end{equation}
By definition, we have
\[
\projection_{\blade{A_r}}(B) \in \bigoplus_{j=0}^r \G_n^j = \G_n^{0+\cdots + r},
\]
since the subspace corresponding to $\blade{A_r}$ is $r$-dimensional and moreover the operation preserves grades since
\[
\projection_{\blade{A_r}}(B) \in G_n^j,
\]
shows the projection preserves grades.

Given vectors $u$ and $v$ we retrieve the familiar statement 
\[
\projection_u (v) = (v \cdot u) \frac{u}{\|u\|^2}.
\]

If instead we wish to project onto the subspace perpendicular to $\blade{A_r}$ we can use the \emph{rejection} operation which we define by
\begin{equation}
\label{eq:rejection}
\rejection_{\blade{A_r}}(B) \coloneqq B \wedge \blade{A_r} \blade{A_r}^{-1} = (B\wedge \blade{A_r})\lfloor \blade{A_r}^{-1}.
\end{equation}
Note that this operation is also grade preserving. In the case we have a vector $v$, we can note
\begin{equation}
\label{eq:projection+rejection_vector}
\projection_{\blade{A_r}}(v) + \rejection_{\blade{A_r}}(v) = v.
\end{equation}

To rehash the geometric notions of the interior and exterior products, note that
\begin{align}
    B \rfloor \blade{A_r} &= \projection_{\blade{A_r}} (B) \blade{A_r}\\
    B \wedge \blade{A_r} &= \rejection_{\blade{A_r}}(B) \blade{A_r}.
\end{align}

To see this in action, we let $v=v^1 e_1 + v^2 e_2 + v^3 e_3$ and let $\blade{B_{12}}=e_1 e_2$ and note
\begin{align*}
    \rejection_{\blade{B_{12}}}(v) &= [(v^1 e_1 + v^2 e_2 + v^3 e_3)\wedge (e_1 e_2)]\blade{B_{12}}^{-1}\\
    &= v^3 e_3 e_1 e_2 e^2 e^1 \\
    &= v^3 e_3.
\end{align*}
Both the notion of projection and rejection prove to be useful and behave nicely with the dual. Take, for example, vectors $u$ and $v$ and note
\begin{equation}
\label{eq:projection_rejection_vector}
\projection_{u^\perp}(v) = \rejection_u(v).
\end{equation}

Finally, the exterior product of orthogonal blades gives us a direct sum of subspaces. Let $\blade{A_r}$ and $\blade{B_s}$ and be orthogonal so that $\blade{A_r}\wedge \blade{B_s}=\blade{A_r}\blade{B_s}$, then we can note
\begin{equation}
    \projection_{\blade{A_r}\wedge \blade{B_s}} (v) = \projection_{\blade{A_r}}(v) + \projection_{\blade{B_s}}(v).
\end{equation}


\subsection{Multivector fields}

We want to generalize the setting of geometric algebra to include a smooth structure. One can take the work above for $\mathcal{G}_n$ and consider a $C^{\infty}$-module structure as opposed to the $\R$-algebra structure in the proceeding section. For brevity, we put $\mathcal{G}_n(\R^n)$ for the $C^\infty$-module and $\G_n$ for the $\R$-algebra. The multivectors themselves can be realized as constant multivector fields so that $\G_n \subset \G_n(\R^n)$. This smooth setting simply makes the coefficients of the global basis blades given by $C^\infty$ functions as opposed to $\R$ scalars.  In this case, we refer to a generic element in the $C^{\infty}$-module $\mathcal{G}_n$ as a \emph{multivector field}. We take $\Omega \subset \R^n$ as a connected region in $\R^n$ for the entirety of this paper and we put
\[
\G_n(\Omega) \coloneqq \{ f \colon \Omega \to \G_n ~\vert~ \textrm{$f$ is $C^\infty$-smooth}\},
\]
where smoothness is meant in terms of the $C^\infty$-module structure.

Perhaps the $C^\infty$-module structure obfuscates the point slightly.  Instead, one should think of the fields in $\G_n(\Omega)$ as multivector valued functions on $\Omega \subset \R^n$.  Taking this identification allows for an extended toolbox at our disposal.  In particular, points in $\Omega$ are uniquely identified with constant vector fields in $\G_n^1$ and one can consider endomorphisms living in $\G_n$ (acting on $\G_n^1$) as acting on the input of fields in $\G_n(\Omega)$ as well.  Thus, there is not only an algebraic structure on the fields themselves, but on the point in which the field is evaluated.  This is perhaps the key insight on why authors developed the so-called vector manifolds widely used in the geometric algebra landscape.

\begin{example}
    Consider a multivector field $f$ valued in $\G_n(\R^n)$.  With $x\in \R^n$ being identified with the vector in $\G_n^1$, we output a multivector $f(x) \in \G_n$ at each point $x$.  One may be interested in the restriction of $f$ to a vector subspace of $\R^n$ which amounts to using projection on the input.  For example, perhaps we wish to know how $f$ behaves on the subspace corresponding to some $r$-blade $\blade{A_r}$.  As such, it suffices to then study $f(\projection_{\blade{A_r}}(x))$.  
\end{example}

We refer to smooth fields valued in $\G_n^+$ as \emph{spinor fields} and put $\G_n^+(\Omega)$ to refer to the $C^\infty$-module counterpart. These fields will be shown to carry a Banach algebra structure. 


\subsubsection{Directional derivative and gradient}

Note that $\R^n$ has global coordinates and thus we can choose a global constant vector field basis $e_1,\dots,e_n$ and we generate $\G_n$ from this basis. Note that we will adopt the Einstein summation convention when needed. With respect to these fields, we have the \emph{directional derivative} $\nabla_\omega$ with $\omega = \omega^i e_i$. The \emph{gradient} (or \emph{Dirac operator}) is defined as $\grad = \sum_{i} e^i \nabla_{e_i}$ and it acts a grade-1 element in the algebra.   Note then that $\omega \cdot \grad = \nabla_\omega$ defines the directional derivative via the gradient. The directional derivative is also grade preserving in that for a multivector $A$
\begin{equation}
\nabla_\omega \proj{r}{A} = \proj{r}{\nabla_\omega A}.
\end{equation}

This structure defined above is typically referred to as \emph{geometric calculus}.  The setting for geometric calculus extends the setting of differential forms and reduces some of the complexity with tensor computations.  Since $\grad$ is a grade-1 object, it acts on a homogeneous $r$-vector $A_r$ by
\begin{equation}
\grad A_r = \proj{r-1}{\grad A_r} + \proj{r+1}{\grad A_r} \coloneqq \grad \rfloor A_r + \grad \wedge A_r.
\end{equation}
Thus, the gradient splits into two operators, 
\begin{align}
\grad \rfloor &\colon \G_n^r(\Omega) \to \G_n^{r-1}(\Omega), \\
\grad \wedge &\colon \G_n^r(\Omega) \to \G_n^{r+1}(\Omega),
\end{align}
which satisfy the properties
\begin{align}
\label{eq:differential_properties}
(\grad \wedge)^2=0,\\
(\grad \rfloor)^2 = 0,
\end{align}
when acting on a homogeneous $r$-vector. Since \ref{eq:differential_properties} holds, the gradient operator gives rise to the grade preserving \emph{Laplace-Beltrami operator}
\[
\Delta = \grad^2 = \grad \rfloor \circ \grad \wedge + \grad \wedge \circ \grad \rfloor,
\]
which is manifestly coordinate invariant by definition.  It also motivates the use of the physicist notation $\grad^2=\Delta$, but we do not adopt this here.  We refer to multivector fields $f$ in the kernel of the Laplace-Beltrami operator \emph{harmonic}.

\subsubsection{Monogenic fields}

Geometric calculus includes another definition for multivectors that is a big motivation for those who study Clifford analysis. 
\begin{definition}
 Let $f \in \G_n(\Omega)$. Then we say that $f$ is \emph{monogenic} if $f \in \ker(\grad)$.
\end{definition}

Monogenic fields are of utmost importance as they have many beautiful properties. One should find them as a suitable generalization of the notion of complex holomorphicity. For example, in regions of Euclidean spaces, a monogenic field $f$ can be completely determined by its Dirichlet boundary values through a generalized Cauchy integral formula. For any even monogenic field, the each of the graded components of $f$ are harmonic.  

We put 
\[
\monogenics(\Omega) \coloneqq \{f \in \G_n(\Omega) ~\vert~ \grad f =0\}
\]
to refer to elements of this set as \emph{monogenic fields} on $\Omega$. As a subset, we also have the \emph{monogenic spinors} $\monogenics^+(\Omega)$, which are simply the even monogenic fields and the \emph{monogenic parabivectors} $\monogenics^{0+2}(\Omega)$. Though these spaces do not form algebras in their own right, they do indeed form a vector space as sums of monogenic functions are monogenic due to the linearity of the gradient.  Moreover, the monogenic spinors are invariant under multiplication from the Clifford group $\Gamma^+$.

\begin{lemma}
\label{lem:clifford_invariant}
Let $s\in \spingroup$ then $\grad \circ s = s \circ \grad$.  In particular, the space of monogenic spinors $\monogenics^+(\Omega)$ is $\spingroup$ invariant.
\end{lemma}
This lemma is classical in the theory of the Dirac operator, Clifford analysis, and harmonic analysis so we omit a proof.  One can see \cite{janssens_special_nodate}, for example.

\subsection{Differential forms}
\label{subsec:diff_forms}

It has become clear that geometric algebra and geometric calculus combine into a single toolbox of multivector field analysis that is useful for vector space algebra and the calculus of $\R^n$. Conveniently, the language of differential forms rests neatly inside this toolbox as well. As such, we will also develop a means of integrating multivector fields. In this subsection we connect the two together into a single framework and note the additional benefits geometric algebra and calculus provide over forms. In order to do so, we appeal to the language of differential forms and build a relationship between multivector fields and forms through measures. Forms have their appeal in global understanding via their properties through integration (e.g., Stokes' and Green's theorems) and their utility extends to boundary value problems \cite{schwarz_hodge_1995}.  

Given that there exists a global coordinate system $x^i$ on $\R^n$, we can place this set of coordinates on any region $\Omega$. Then, we form the basis of tangent vector fields $\partial_i = \frac{\partial}{\partial x^i}$ with the reciprocal 1-forms $dx^i$ that are each global sections of $T^*\Omega$ and are the exterior derivatives (or gradients) of the coordinate functions.  Typically, 1-forms are viewed as linear functionals on tangent vector fields and in these coordinates we have $dx^i  (\partial_j) = \delta^i_j$.  The benefit of this definition is that the 1-forms $dx^i$ carry a natural measure and we can form product measures via the exterior product $\wedge$.  For example, for a 2-dimensional surface $\Sigma$ we have the \emph{directed measure} $d\Sigma = e_i \wedge e_j dx^i dx^j$ and we can note that $(e^i \wedge e^j)\cdot d\Sigma^\dagger = dx^idx^j - dx^j dx^i$ is completely antisymmetric and provides us a surface measure we can integrate; this is a differential 2-form.

In an $n$-dimensional space with a position dependent inner product $g$, we have the $n$-dimensional volume directed measure $d\Omega = \sqrt{|g|} dx^1\dots dx^n$. If we then define $dX_n = e^n \wedge \cdots \wedge e^1 dx^1 \dots dx^n$ we then find that
\[
d\Omega = I^\dagger \cdot dX_n.
\]
Here $I$ is the pseudoscalar field defined on $\Omega$ with respect to $g$. Similarly, for $k<n$, we can define the $k$-dimensional volume measure as 
\[
dX_k = \frac{1}{k!}(e^{i_k}\wedge \cdots \wedge e^{i_1}) dx^{i_1} \cdots dx^{i_k}.
\]
We can now write a $k$-form $\alpha_k$ as $\alpha_k = A_k \cdot dX_k$. In this sense, a differential form is made up of two essential components namely the multivector field and the $k$-dimensional volume directed measure. For example, if we wish to write a 2-form $\alpha_2$ we take $dX_2 = \frac{1}{2!} e^j \wedge e^i dx^i dx^j$ and $A_2 = a_{ij} e_i \wedge e_j$ to yield
\[
\alpha_2 = A_2 \cdot dX_2 = \frac{a_{ij}}{2!} (e_i \wedge e_j) \cdot (e^j \wedge e^i) dx^i dx^j = \frac{a_{ij}}{2!} (dx^i dx^j - dx^j dx^i)
\]
Thus, we arrive at an isomorphism between $k$-forms and $k$-vectors as a contraction with the $k$-dimensional volume directed measure $dX_k$ since
\[
\alpha_k = A_k \cdot dX_k.
\]
Hence, we can see now how a differential form simply appends the measure attached to the underlying space. We can also see how this generalizes the musical isomorphisms $\sharp$ and $\flat$ by taking a vector field $a$ and noting
\begin{equation}
\label{eq:line_element}
a \cdot dX_1 = a^i e_i \cdot e^j dx^j = a^i dx^i,
\end{equation}
corresponds to the usual $\flat$ map on vector fields.

\subsubsection{Exterior algebra and calculus}
The exterior algebra of differential forms comes with an addition $+$ and exterior multiplication $\wedge$.  We note that the sum of two $k$-forms $\alpha_k$ and $\beta_k$ that $\alpha_k+\beta_k$ is also a $k$-form which we can see by letting $\alpha_k = A_k \cdot dX_k$ and $\beta_k = B_k \cdot dX_k$ and putting
\[
\alpha_k + \beta_k = (A_k \cdot dX_k)+(B_k \cdot dX_k) = (A_k + B_k) \cdot dX_k,
\]
due to the linearity of $\cdot$.  If instead had an $s$ form $\beta_s$ then we have the exterior product
\[
\alpha_k \wedge \beta_s = (A_k \wedge B_k) \cdot dX_{k+s},
\]
where $dX_{k+s}=0$ if $k+s>n$.  

With differential forms one also has the exterior derivative $d$ giving rise to the calculus of forms.  Given we can write a differential $k$-form as $\alpha_k = A_k \wedge dX_k$,  In particular, we have
\[
d \alpha_k = (\grad \wedge A_k) \cdot dX_{k+1},
\]
which realizes the exterior derivative as the grade raising component of the gradient $\grad$. Of course, for scalar fields, this returns the gradient as desired. 

Here, $\grad \wedge$ can be identified with the exterior derivative $d$ and $\grad \rfloor$ can be identified with the codifferential $\delta$ on differential forms up to a sign \cite{schindler_geometric_2020}. This of course means the standard properties that apply to $d$ and $\delta$ apply to $\grad \wedge$ and $\grad \rfloor$.


\subsubsection{Integration on submanifolds}

Given a $k$-dimensional submanifold of $K \subset \Omega$ with a $k$-form $\alpha_k$ defined on $K$, we can integrate the $k$-form. Using the multivector equivalents leads us to the $k$-dimensional directed measure $dK$ for the submanifold $K$.  Given $K$ is a submanifold of $\Omega$, for any $x \in K$ we have tangent space $T_x K$ which corresponds to a tangent $k$-blade $I_K(x)$.  We put $I_K$ as the smooth $k$-blade field everywhere tangent to $K$. Then we have the directed volume measure on $K$ given by
\[
dK = I_K^\dagger \cdot dX_k.
\]
For a tangent $k$-vector field $A_k$ on $K$, we must have for any $x \in K$ that $f = \operatorname{P}_{I_K} \circ f$ so that these fields lie purely tangent to $K$. In particular, we can always put $A_k = A I_k^\dagger$ for a scalar field $A$. These fields can contract with the directed measure $dX_k$ to create a $k$-form on $K$ by $\alpha_k = A_k \cdot dX_k = A dK$ which can be integrated as
\[
\int_K \alpha = \int_K A dK.
\]
Hence, on $\Omega$ itself, we can decompose top grade forms by taking
\[
\alpha_n = A_n \cdot dX_n = A I^\dagger \cdot dX_n
\]
for a scalar field $A$ satisfying $A_n = AI^\dagger$. Then this form can be integrated by
\[
\int_\Omega \alpha_n = \int_\Omega A d\Omega.
\]

There is also the normal space $N_x K$ that is everywhere orthogonal (with respect to $g$ on $\Omega$) to $T_x K$.  In particular, we have the normal $(n-k)$-blade field $\nu = I_K^\dagger I$. Note that for a unit $k$-blade $I_K$ we have $I_K^{-1}=I_K^\dagger$ and we see $I_K \nu = I$. Since $K$ is a submanifold of $\Omega$ we have the inclusion $\iota \colon K \to \Omega$ and the induced pullback on forms $\iota^*$ which is equivalent to the tangent projection operator $\tangent_K$ seen in \cite{schwarz_hodge_1995}. Given a $p$-form $\alpha_p$ defined on $\Omega$, we have that $\tangent_K \alpha_p = \alpha_p \circ \operatorname{P}_{I_K}$. Specifically, $\alpha_p = A_p \cdot dX_p$ we have \todo[inline]{I should probably work through this to show it's true}
\[
\tangent_K \alpha_p  = A_p \cdot (dX_p \circ \operatorname{P}_{I_K}) = \operatorname{P}_{I_K}(A_p) \cdot dX_p = \rejection_{\nu}(A_p)\cdot dX_p.
\]
The normal projection $\normal_K$ is then $\normal_K \alpha_p = \alpha_p - \tangent_K \alpha_p$ and moreover
\[
\normal_K \alpha_p = \rejection_{I_K}(A_p)\cdot dX_p = \operatorname{P}_{\nu}(A_p)\cdot dX_p.
\]
\todo[inline]{Not sure this is true. but \url{https://en.wikipedia.org/wiki/Geometric_algebra} talks about projection and rejection.}

This is pertinent when we take the submanifold $\Sigma = \partial \Omega$. There, $I_\Sigma$ yields the directed measure
\[
d\Sigma \coloneqq I_\Sigma^\dagger \cdot dX_{n-1}.
\]
The normal space is 1-dimensional and $\nu$ is the unit normal vector to the boundary. The pullback coincides with projection into the tangent space given by $I_\Sigma$.  Then, for 1-forms $\alpha = \cdot dX_1$ it is apparent that $\tangent_\Sigma \alpha = \projection{I_\Sigma}{a} \cdot dX_1$ and $\normal_\Sigma \alpha = \rejection_{I_\Sigma}(a) \cdot dX_1$ by Equations \ref{eq:projection+rejection_vector} and \ref{eq:projection_rejection_vector}. One can then find the flux of a vector field through $\Sigma$ arises as an $(n-1)$-form $\projection{\nu}{a} I^{-1} \cdot dX_{n-1}$. Once again we see that the flux is determined both by the vector field $a$ and the local geometry of $\Sigma$ captured by $d\Sigma$ in the following way. Note that  $\nu^{-1}=\nu$ since $\|\nu\|=1$ everywhere on $\Sigma$ and so $\projection{\nu}{a} I^{-1}= a \cdot \nu \nu I^{-1} = a \cdot \nu I_\Sigma^\dagger$ which gives us the corresponding form $a \cdot \nu d\Sigma$ and the total flux of $a$ through $\Sigma$ is then
\[
\int_\Sigma (\projection{\nu}{a} I^{-1}) \cdot dX_{n-1} = \int_\Sigma a \cdot \nu d\Sigma.
\]

\subsubsection{$k$-form inner product}
\todo[inline]{Show that this relates back to the spinor norm.}
For smooth $k$-forms $\alpha_k = A_k \cdot dX_k$ and $\beta_k = B_k \cdot dX_k$, we have an inner product 
\[
\langle \alpha_k, \beta_k \rangle = \int_\Omega \alpha_k \wedge \star \beta_k 
\]
where $\star$ is the Hodge star. The Hodge star on $k$-forms inputs a $k$-form and outputs a a specific dual $(n-k)$-form so that we always have $\alpha_k \wedge \star \beta_k  = (A_k\cdot B_k^\dagger)d\Omega$ as we note Equation \ref{eq:dot_product}. Thus, we can realize how $\star$ acts on multivector representative. We let $\star \beta_k = B_k^\star \cdot dX_{n-k}$ by $B_k^\star = (I^{-1} B_k)^\dagger$.  Indeed, we have
\begin{align*}
    \alpha_k \wedge \star \beta_k &= (A_k \wedge B_k^\star) \cdot dX_n\\
    &= A_k \cdot B_k^\dagger d\Omega.
\end{align*}

\subsubsection{Stokes' and Green's theorem}

For regions $\Omega$ with boundary $\Sigma$, we have a compact form of Stokes' theorem
\[
\int_\Omega d \alpha_{n-1} = \int_\Sigma \tangent_\Sigma \alpha_{n-1},
\]
for sufficiently smooth $(n-1)$-forms $\alpha_{n-1}$. Taking the multivector equivalent $\alpha_{n-1} = A_{n-1} \cdot dX_{n-1}$ we retrieve \emph{Stokes' theorem} as
\[
\int_\Omega (\grad \wedge A_{n-1}) \cdot dX_n = \int_\Sigma \projection{I_\Sigma}{A_{n-1}} \cdot dX_{n-1}  = \int_\Sigma \rejection_{\nu}(A_{n-1}) \cdot dX_{n-1}
\]
Let $v = A_{n-1} I$ denote the vector field dual to the pseudovector $A_{n-1}$, then we have the more recognizable \textcolor{red}{(Helmholtzian?)} form of Stokes' theorem
\[
\int_\Omega \grad \cdot v d\Omega = \int_\Sigma v \cdot \nu d\Sigma.
\]
\todo[inline]{Show this work?} This provides a compact relationship for those who choose to work with vector fields and those who choose to work with forms. \textcolor{red}{Pause and reflect here about what Green's theorem and Stokes' theorem are really saying. Use DUAL notation!}

Finally,




\todo[inline]{More on integration and do Ohms law as an example of some of this stuff. Do all of hodge decomposition and stuff?}

\section{Algebras of multivector fields}
\subsection{Banach algebras of Clifford fields}

\todo[inline]{Finish this section. I'm saying this here but it should go later on, but this should lead to the weak formulation for the laplace equation??? Does there exist an inner product instead of just a norm? }

Letting $\Omega$ be a region in $\R^n$, $\Gamma(\Omega)$ has a norm induced from the spinor norm in the $L_2$ sense by
\[
\spinnorm{s} = \int_\Omega ss^\dagger d\Omega
\]
This gives us a normed algebra of Clifford fields. One can see that we have the unit $1$ in this algebra. We also have for multivectors $s,r \in \Gamma(\Omega)$ (constant Clifford fields)
\[
\|sr\| = \|s\|\|r\|
\]
since
\[
\|sr\|^2 = (sr)(sr)^\dagger = srr^\dagger s^\dagger = s\|rr^\dagger\|^2 s^\dagger = \|s\|^2 \|r\|^2.
\]
It follows for non-constant $C^\infty$-fields $s$ and $r$
\[
\spinnorm{sr} \leq \spinnorm{s}\spinnorm{r}.
\]
This shows the algebra is uniform. Identifying the constant fields in the algebra $\Gamma(\Omega)$ with $\R^{2^n}$ we see that the algebra is also complete.


\subsection{Axial monogenic fields}
\todo[inline]{What fields do we care about that are Clifford fields (invertible).  Biparavectors? Axial biparavectors? Rephrase in terms of hilbert transform? Copy stuff from other paper here.}

For this section, let $n=\dim(M)=3$. Supposing that $\phi$ satisfies \ref{eq:conjugate_requirement} (\textcolor{red}{I dropped this requirement for now}) one can generate paravectors $f=u+b$ and define the space of \emph{monogenic paravectors}
\begin{align*}
\monogenics &= \{ f ~ \vert ~ ~\grad f=0\}\\
\end{align*}
The original requirement that $\Delta u^\phi =0$ is obtained since $f$ is monogenic. We can then generate an algebra from this set by
\[
\algebra{} = \{ fg ~\vert~ f,g \in \monogenics\},
\]
but, as mentioned in \cite{belishev_algebras_2019}, this algebra generated by these monogenic fields in $\monogenics$ produce fields that are not monogenic.  Indeed, this is a well known fact in Clifford analysis mentioned in \cite{schepper_introductory_nodate}.  Fundamentally, however, this fact that the product of monogenics is no longer monogenics makes the direct approach in \cite{belishev_calderon_2003} intractable. This issue comes down to the lack of commutivity of paravectors in dimensions higher than $2$.  However, for certain so-called axial fields, commutivity is regained. In fact, the construction of these fields was done in \cite{belishev_algebras_2017} in order to create a closed commutative algebra of monogenic fields. These axial fields will relate directly to complex holomorphic functions.

In \cite{belishev_algebras_2017, belishev_algebras_2019}, the definition of axial is defined for quaternion fields and the properties are discussed.  It is evident from the Example \ref{ex:quaternions} that quaternion fields are analogous to paravector fields via the given identification.  This identification is key in connecting the relevant algebras to the DN map. So we proceed by following the definitions in place.  

\begin{definition}
    Let $F=U+B$ be a paravector and let $\omega$ be a unit vector.  We then say that $F$ is \emph{$\omega$-axial} if $\nabla_\omega F = 0$.  
\end{definition}

\todo[inline]{Make sure I define the covariant derivative and stuff}

From the grade preserving nature of $\nabla$, we see that the requirement $\nabla_\omega f=0$ reduces to a grade-wise requirement
\[
\nabla_\omega U = 0 \qquad \textrm{and} \qquad \nabla_\omega B = 0.
\]
Thus, we can write $B=\beta \omega I = \beta B$ for a smooth scalar field $\beta$ satisfying $\nabla_\omega \beta =0$. So long as $\omega$-axial monogenics are closed under multiplication, we can recover a sub-algebra of holomorphic functions inside of the larger algebra $\monogenics$ generated by monogenic paravectors. If we take two $\omega$-axial monogenic fields $f=u_f + \beta_f B$ and $g=u_g + \beta_g B$, then we have
\begin{equation}
\label{eq:axial_multiplication}
fg = u_f u_g - \beta_f \beta_g + B (u_f b_g + u_g b_f).
\end{equation}
Namely, this follows from the fact that
\[
B^2 = (\omega I)^2 = -1.
\]
This fact is essential. In essence, we now have a direct representation of a holomorphic function if we let $i=B$.  One should then realize that an $\omega$-axial monogenic $f$ is built by translating a holomorphic function along the direction defined by $\omega$ since $f$ has no dependence on this direction. Moreover, it is clear that $B$ is a 2-blade.  Note that for some unit vectors $r$ and $p$, we have $\omega = r \times p$.  Thus, $B =  (r \times p)I^{-1}$.  Indeed, this fits with the interpretation above in that $B$ is acting as a pseudoscalar in some manner.  To say this fully, $B$ is the pseudoscalar for the plane spanned by $r$ and $p$. Another way of rephrasing $f$ being $\omega$-axial is then to say that $f$ is constant on all translations of the $r p$-plane. In this case, $f$ depends solely on two variables and is exactly a holomorphic function. This is simply dual to the notion of being constant along straight lines in a 3-dimensional space.  One can think of $\omega$ as a member of the Grassmanian $Gr(1,3)$ whereas its dual $B=\omega I$ lies in $Gr(2,3)$ which is isomorphic. Indeed, $I$ gives a natural isomorphism between $Gr(1,3)$ and $Gr(2,3)$.

If $f$ is an $\omega$-axial monogenic, then we can recall the Cauchy-Riemann equations yield
\begin{equation}
\label{eq:axial_cauchy_riemann}
\grad u = (\omega \wedge \grad \beta)I \qquad \textrm{and} \qquad - B \wedge \grad \beta B = 0.
\end{equation}

On this plane given by the blade $B$, we want to realize $B$ acting as $i$ for a holomorphic function. In particular, this means we need the Dirac operator to respect multiplication by constant paravectors (which is analogous to scaling complex functions by a complex number). If one has an $\omega$-axial monogenic $f$, we wish that for a constant paravector $k=k_1 + k_2 B$ that $\grad (kf)=0$ as well. $\grad$ is clearly $\R$-linear, so it sufficies to show the following.

\begin{lemma}
    \label{lem:mult_by_i_monogenic}
    Let $f=u+\beta B$ be an $\omega$-axial monogenic paravector, then $B f$ is $\omega$-axial and monogenic.
\end{lemma}
\begin{proof}~
    
\todo[inline]{I can use equations 82 from Chisolm to avoid the use of the cross product}
    It is clear that $B f$ is $\omega$-axial due to the grade preserving linearity of the covariant derivative.
    
    To see that $B f$ is monogenic, we take $B  f = B  u - \beta$.  Then,
    \begin{align*}
    \grad (Bf) = \grad (B u) - \grad \beta,
    \end{align*}
    where we have the graded components
    \begin{align*}
        \proj{1}{\grad (B f)} &= (\grad \cdot Bu)  - \grad \beta\\
        \proj{3}{\grad (Bf)} &= (\grad \wedge B u).
    \end{align*}
    Note that
    \begin{align*}
    \grad \cdot (Bu) =  -\omega \times (\grad \wedge u)  =- \omega \times (\omega \times \grad\beta) = -\omega (\nabla_\omega \beta)+\grad \beta = \grad \beta
    \end{align*}
    by \ref{eq:axial_cauchy_riemann} and thus $\proj{1}{\grad (Bf)}=0$. 
    
    For the grade-3 component,
    \begin{align*}
        \grad \wedge (B u) &= \omega \cdot  (\grad \wedge B)II^{-1}u = I^{-1} \nabla_\omega u=0
    \end{align*}
    since $u$ is $\omega$-axial. Thus we have $\grad(B f)=0$ is monogenic.
\end{proof}

The point here is that we have now effectively found functions that can be scaled by $\alpha + \beta B$ and remain monogenic.  This is the constant multiple rule for the Wirtinger derivative for complex functions. Generically, if I take some multivector $A$ times a monogenic field $f$, $Af$ need not be monogenic.

\begin{proposition}
    Let $f$ and $g$ be monogenic and $\omega$-axial. Then $fg=gf$, $fg$ is $\omega$-axial, and $fg$ is monogenic.
\end{proposition}
\begin{proof}
\todo[inline]{Clean this up with better notation}
    \begin{itemize}
        \item First, it is clear that $fg=gf$ by Equation \ref{eq:axial_multiplication}.
        \item The product $fg$ is $\omega$-axial simply by the product rule of the multivector covariant derivative. That is,
        \[
            \nabla_\omega (fg) = (\nabla_\omega f)g + f(\nabla_\omega g) =0.
        \]

    \item 



To see that the product is monogenic, we have
    \[
        \grad(fg) = \grad(u_fu_g - b_f b_g +  B(u_f b_g + u_g b_f)).
    \]
    Then the grade-1 components are
    \[
        \proj{1}{\grad(fg)}=\grad \wedge (u_f u_g - b_f b_g) + \grad \cdot B(u_f b_g + u_g b_f),
    \]
    and the grade-3 components are
    \[
        \proj{3}{\grad(fg)} = \grad \wedge B (u_f b_g + u_g b_f).
    \]
    For the grade-1 components, we have
    \begin{align*}
        \grad(u_f u_g - b_f b_g) &= (\grad u_f) u_g + u_f (\grad u_g) - (\grad b_f) b_g - b_f (\grad b_g)\\
        \grad \cdot I\omega(u_f b_g + u_g b_f) &= (\grad \cdot I\omega u_f) b_g + u_f (\grad \cdot B b_g) + b_f(\grad \cdot B u_g) + (\grad \cdot B b_f) u_g,
    \end{align*}
    and since $f$ and $g$ are both monogenic we have
    \begin{align*}
        \proj{1}{\grad(fg)} &= (\grad \cdot B u_f - \grad  b_f)b_g + (\grad \cdot B) u_g - \grad  b_g)b_f.
    \end{align*}
    Then, note that 
    \[
        \proj{1}{\grad Bf} = \grad \cdot B u_f - \grad b_f=0
    \]
    by Lemma \ref{lem:mult_by_i_monogenic} and likewise for $\proj{1}{\grad Bg}$. Thus,
    \[
        \proj{1}{\grad(fg)}=0.
    \]
    Likewise, for the grade-3 component of the gradient 
    \begin{align*}
        \proj{3}{\grad(fg)} &= I^{-1} \nabla_\omega (u_f b_g + u_g b_f)=0,
    \end{align*}
    by the product rule for the covariant derivative and the fact that $f$ and $g$ are $\omega$-axial.
\end{itemize}
\end{proof}

\todo[inline]{Add in power series stuff here.  We can write $f=u+ib$ as a power series of $x+yB$?}


\textcolor{red}{As we move through the different axial vectors, it's as if we're doing some tomography on 2d slices of the domain.}

\todo[inline]{Now describe how to do the rest of the algebra stuff here.} 


\begin{theorem}
(2D Gelfand) For any $\mu \in \mathcal{M}$ there is a point $z^\mu \in D$ such that $\mu = \delta_{z^\mu}$. The map
\[
\gamma \colon \mathcal{M} \to D, \quad \mu \mapsto z^\mu
\]
is a homemorphism so that $\mathcal{M} \cong D$. The Gelfand transform
\[
\Gamma \colon \mathcal{A}(D) \to C^\C (\mathcal{M}), \quad (\Gamma f)(\mu) = \mu(f), \quad \mu \in \mathcal{M}
\]
is an isometric isomorphism onto its image, so that $\mathcal{A}(D)\cong \Gamma(\mathcal{A}(D))$.
\end{theorem}


\textcolor{red}{In local coordinates the following definition works...}

\begin{definition}
    Let $B$ be a unit 2-blade then we say that a (0+2)-vector $f_B$ is $B$-planar if $f_B = \projection{B}{} \circ f_B \circ \projection{B}{}$ for all $x$.
\end{definition} 


\todo[inline]{I need to mention that an $\omega$-axial field is a scalar + a scalar times $\omega$ as well. Rewrite this proof.}
\begin{proposition}
    In $\R^3$, if $\omega I = B$, then $B$-planar is in correspondence with a $\omega$-axial quaternion field $h = \alpha + \psi \omega$. 
\end{proposition}
\begin{proof}
    Let $f$ be $\omega$-axial so that $\nabla_\omega f =0$ for some unit vector $\omega$. In particular,
    \[
        \nabla_\omega f = 0 \quad \iff \quad f(x + t\omega) = f(x),
    \]
    for any $t\in \R$. Letting $B=\omega I$, we have
    \begin{align*}
        x+t\omega=(x+t\omega)BB^{-1}&=x\rfloor B B^{-1} + x\wedge BB^{-1} + t\omega \rfloor BB^{-1} + t\omega \wedge B B^{-1}\\
        &= (x\rfloor B) \rfloor B^{-1} + (x\cdot \omega)\omega + (t\omega \cdot \omega)\omega\\
        &= (x\rfloor B)\rfloor B^{-1} + 
(x+t\omega)\cdot \omega \omega.
    \end{align*}
    Since $f$ is $\omega$-axial
    \[
        f(x)=f(x+t\omega)=f((x\rfloor B)\rfloor B^{-1} + (x+t\omega)\cdot \omega \omega)=f((x\rfloor B)\rfloor B^{-1}),
    \]
    and so $f$ is also $B$-planar and the proof is complete.
\end{proof}

\textcolor{red}{Discuss why we need $B$-planar in higher dimensions and also mention that we need $B$ to be an invertible bivector. All blades are invertible?}



\subsection{Spinor spectrum}

This story no longer continues in higher dimensions and one can find the two and three dimensional cases to be happy accidents.  Instead, now we must deal fully with the situation at hand to dissect the relevant algebras. In this vein, we can generate a special algebra $\algebra{B}$ of $B$-planar monogenic spinors from the $B$-planar monogenic $(0+2)$-vectors.  The question is then for all does
\[
\overline{\bigoplus_{B \in \Grassmannian{2}{n}} \algebra{B} } = \monogenics.
\]

Letting $\ball$ be the unit ball in $\R^n$ and $\disk$ be the unit disk in $\C \cong \R^2$.  By Gelfand, the maximal ideal space of $\algebra{B}$ is homeomorphic to the disk given the isomorphism mapping the blade $B \mapsto i$ in the complex plane.  The space $\monogenics$ is no longer an algebra, so we are at a loss to determine maximal ideals.  However, we can describe functionals on the monogenics.

\begin{definition}
    Define the \emph{spinor dual} $\dualmonogenics$ as
    \[
        \dualmonogenics \coloneqq \{ l \in \mathcal{L}(\monogenics; \spinalgebra) ~\vert~ l(sf) = sl(f), ~\forall f \in \monogenics, ~s \in \spinalgebra \}
    \]
\end{definition}
$\dualmonogenics$ are the spinor valued functionals or \emph{spin functionals}. Similarly, we have the definition for the spinor functionals that are multiplicative on the $B$-planar monogenics.
\begin{definition}
    The \emph{spinor spectrum} is the set
    \[
        \characters \coloneqq \{ \mu \in \dualmonogenics ~\vert~ \mu(fg) = \mu(f)\mu(g),~ \forall f,g \in \algebra{B}, ~ B \in \Grassmannian{2}{n}\},
    \]
    and we refer to the elements as \emph{spin characters}.
\end{definition}
The elements in the spinor spectrum are simply algebra homomorphisms from $\algebra{B}$ to $\spinalgebra$. In the 2-dimensional case, there is only one unique choice of $B$ and $\mathfrak{spin}(2)$ is isomorphic to $\C$.  We realize this as only a special case of a more general notion of a spin character.

\textcolor{red}{Describe the weak-$\ast$ topology here too.} 


\section{Gelfand theory}
\subsection{Topology from monogenics}

We seek to determine that the space $\characters(\Omega)$ is homeomorphic to $\Omega$.  Thinking of the Calder\'on problem, we may only have access to functions defined on $\Omega$ and not the whole of $\Omega$ itself.  If one can recover the spinor characters $\characters(\Omega)$, we can utilize the following result.

\begin{theorem}
For any $\mu \in \characters(\Omega)$, there is a point $x^\mu \in \Omega$ such that $\mu(f) = f(x_\mu)$ for any $f\in \monogenics(\Omega)$ a monogenic spinor field. Given the weak-$\ast$ topology on $\characters(\Omega)$, the map
\[
\gamma \colon \characters(\Omega) \to \Omega, \quad \mu \mapsto x^\mu
\]
is a homeomorphism. The Gelfand transform 
\[
\widehat{~} \colon \monogenics(\Omega) \to C(\characters(\Omega); \G_n), \quad \widehat{f}(\mu) \coloneqq \mu(f), \quad \mu \in \characters(\Omega),
\]
is an isometry onto its image, so that $\characters(\Omega)$ is isomorphic to $\widehat{\monogenics(\Omega)}$ as algebras.
\end{theorem}

We prove this theorem in two main parts and discuss the result in this section. First, we can realize a power series representation for elements in a ball $\ball$ and denote this sit as $\monogenics(\ball)$. This power series is constructed using specific $B$-planar monogenic fields. Finally, we constructively show a correspondence between $\mu \in \characters(\ball)$ with $x^\mu \in \ball$. \textcolor{red}{Then we can use these to cover $\Omega$ or something?}

\subsubsection{Power series}

\todo[inline]{This really is a honest to god Taylor series so I should call it that.}

One beautiful result in Clifford analysis is the celebrated generalization of the Cauchy integral formula for $\C$-holomorphic functions. Details of the Cauchy integral formula and Hilbert transform for multivector fields can be found in \cite{brackx_hilbert_2008}. We have the fundamental solution to $\grad$ is a vector field given by
\[
E(x) = \frac{1}{a_m} \frac{x}{\|x\|^m},
\]
for $x\in \R^n$. That is to say that $\grad E(x) = \delta(x)$. For any region $\Omega \subset \R^n$ with boundary $\Sigma$, we define the \emph{Cauchy kernel} for $x\in \R^n$ and $y \in \Sigma$ using the fundamental solution $E$ as
\[
C(y, x) = -\frac{1}{a_n} \nu(x_0) E(x-y),
\]
where $a_n$ is the surface area of the $n$-ball and $\nu(x_0)$ is the outward normal at $x_0$. The \emph{Cauchy integral} for $\phi \in L_2(\Sigma)$ is then
\[
\cauchy[\phi](x) = \frac{1}{a_n} \int_{\Sigma} \frac{y-x}{\|x-y\|^n} \nu(y) \phi(y) d\Sigma(y).
\]
The Cauchy integral is indeed a monogenic function and note that for a scalar $\phi$ we have $\cauchy[\phi] \in \monogenics(\Omega)$ since it must be a parabivector as well.

Fix a basis $e_1,\dots,e_n$ in $\R^n$ and we can define the functions $z_j^i = x^j - x^i e^i e_j$. Recall that for an orthonormal basis the reciprocal basis elements $e^i=e_i$ satisfy $e^i \cdot e_j = 1$. \textcolor{red}{Ryan uses $e_i^{-1}$ actually. Are the reciprocal basis elements the inverses? Yes see \url{https://math.stackexchange.com/questions/811248/wedge-product-between-nonorthogonal-basis-and-its-reciprocal-basis-in-geometric}} To further condense notation, we let $B_{ij}=e_i e_j$ be the 2-blade acting as the pseudoscalar for the $e_i e_j$-plane and likewise put $B_j^i = e^ie_j$ and $B^{ij}=e^i e^j$ as necessary. In the same vein, the functions $z_j^i$ are very analogous to $z$ in $\C$ but rather in the $B_j^i$ plane.  We then note
\[
z_j^i = x^j - x^i B_j^i = e_j\projection{B_j^i}{x}.
\]
One can quickly confirm that the $z_j^i$ are monogenic and are indeed $B_j^i$-planar by construction. These functions find their use in a power series representation for monogenic fields $f$.
\begin{itemize}
    \item Consider the function $z_{\sigma(j)}^1(x)=x^{\sigma(j)} - x^1 B_{\sigma(j)}^i$ for $\sigma \in \{2,\dots,n\}$ a permutation.  
    \item Let $f \in \monogenics^+(\Omega)$.  Then by Theorem 4 in \cite{ryan_clifford_2004}, we can center a ball of radius $R$ at $w$ to get the monogenic polynomials
    \[
        P_{j_2 \dots j_n}(x) = \frac{1}{j!} \sum_{\textrm{permutations}}z_{\sigma(1)}^1(x-w) \cdots z_{\sigma(j)}^1(x-w).
    \]
    Each polynomial in the collection
    \[
    \mathcal{P}(\Omega) = \{P_{j_2 \cdots j_n} ~\vert~ j_2+\cdots+j_n = j, ~0\leq j < \infty\}
    \]
    is monogenic and linearly independent.
    These polynomials generate $f$ as a power (Taylor) series as
    \[
        f(x) = \sum_{j=0}^\infty \left(\sum_{{j_2 \cdots j_n}_{j_2 + \cdots j_n = j}} P_{j_2 \cdots j_n} (x-w) a_{j_2 \cdots j_n}(w) \right),
    \]
    where the coefficients are found using the Cauchy integral
    \[
        a_{j_2 \cdots j_n} = \frac{1}{a_n} \int_{\partial B(w,R)} \frac{\partial^j G(x-w)}{\partial x_2^{j_2} \cdots \partial x_n^{j_n}} \nu(x) f(x) d\Sigma(x).
    \]
    Each coefficient $a_{j_2 \cdots j_n} \in \G_n^+$. \textcolor{red}{Yes but these are coming in as a right module multiplication. So this should be noted and checked}
    \item This series converges uniformly to $f$ for points $x\in \ball$.
\end{itemize}


We have now found that all monogenic fields are generated as power series of homogeneous polynomials in the variables $z_j^i$. Thus, we have a direct route between the algebras $\algebra{B_j^i}(\ball)$ and the monogenic spinor fields $\monogenics(\ball)$.  In each algebra $\algebra{B_j^i}(\ball)$ the $z_j^i$ act much like a realization of $z\in \C$.  We will find that the action of the spin characters on $z_j^i$ can be understood and extended through the power series to all monogenic spinors. The power series representation seen here is one of the strong reasons to utilize geometric calculus and study the results of Clifford analysis. 



\subsubsection{Correspondence}

The functions $z_j^i$ play a crucial role in the above power series representation but they also play a key part in determining the behavior of the spin characters $\mu \in \characters$.  If we are able to deduce the action $\mu(z_j^i)$, then we can extend this to any monogenic $f$ via the power series representation. Note that $\mu(1)=1$ since it is an algebra homomorphish and so for any $2$-blade $B$ and $\mu \in \characters(\ball)$ that the image of the axial algebras $\mathbb{A}_B=\mu(\algebra{B}(\ball))$ are all commutative subalgebras of $\G_n^+$.  In particular, for a constant $\alpha+\beta B \in \algebra{B}(\ball)$, $\mu(\alpha+\beta B)=\alpha+\beta B$ by definition and so we retrieve $\mathbb{A}_B$ must be generated by linear combinations of the scalar $1$ and the bivector $B$.  Thus, $\mathbb{A}_B$ is an isomorphic copy of $\G_2^+ \cong \C$ as the even subalgebra of the $B$-plane.

Working in terms of an arbitrary basis and applying $\mu$ yields
\[
\mu(z_j^i) = \alpha_j^i + \beta_j^i B_j^i,
\]
for some constants $\alpha_j^i$ and $\alpha_j^i$.  The $z_j^i$ are not independent from one another.  In fact, we have two key relationships in that
\begin{equation}
\label{eq:z_reciprocal_relationship}
z_j^i B_i^j  = -z_i^j.
\end{equation}
Similarly, we have
\begin{equation}
\label{eq:z_relationship}
z_j^i = z_j^k + z_k^i B_j^k.
\end{equation}

Thus, we can take $\mu$ of Equations \ref{eq:z_reciprocal_relationship} and \ref{eq:z_relationship} and determine a relationship on the constants $\alpha_j^i$ and $\beta_j^i$. First, using Equation \ref{eq:z_reciprocal_relationship}
\[
\mu(z_j^i B_i^j) = \mu(z_j^i) B_i^j = -\mu(z_i^j)
\]
yields
\[
(\alpha_j^i + \beta_j^iB_j^i)B_i^j = \beta_j^i + \alpha_j^i B_i^j = - \alpha_i^j - \beta_i^j B_i^j 
\]
and so $\alpha_i^j = -\beta_j^i$ for all $i \neq j$. Next, using Equation \ref{eq:z_relationship}
\[
\mu(z_j^i) = \mu(z_j^k + z_k^i B_j^k) = \mu(z_j^k)+\mu(z_k^i)B_j^k
\]
and so
\[
a_j^i + b_j^i B_j^i = \alpha_j^k + \beta_j^kB_j^k + (\alpha_k^i + \beta_k^i B_k^i)B_j^k = \alpha_j^k + \beta_k^i B_j^i + (\alpha_k^i + \beta_j^k)B_j^k
\]
yields the relationships $\alpha_j^i = \alpha_j^k$, $\beta_j^i = \beta_k^i$, and $\alpha_k^i=-\beta_j^k$. 

Briefly, picture $\alpha_j^i$ and $\beta_j^i$ as components of the $n \times n$ matrices $\alpha$ and $\beta$.  We can index rows by the superscript and columns by the subscript and see that $\alpha$ and $\beta$ both have zero diagonal (since we do not have functions $z_i^i$). The relationship $\alpha_i^j = -\beta_j^i$ for $i\neq j$ then shows that $\alpha = -\beta^\top$.  Then we have $\alpha_j^i = \alpha_j^k$ for $i\neq j \neq k$ shows that $\alpha$ is constant along rows and hence $\beta$ is constant along columns (which shows $\alpha = -\beta^\top$ is consistent with the additional relationship $\beta_j^i = \beta_k^i$). The final relationship $\alpha_k^i = -\beta_j^k$ is consistent as well. The matrices $\alpha$ and $\beta$ are thus uniquely determined by $n$ numbers.  Moreover, treating $\mu(z_j^i)=z_j^i(x_\mu)$ for some $x_\mu \in \R^n$ satisfies the relationships granted above. Thus, we simply find the $x_\mu$ such that we retrieve the desired components for $\alpha$ and $\beta$.  

Using the power series representation for a monogenic spinor $f$ we can extend $\mu$ to act on $\monogenics^+(\ball)$ by the multiplicative and $\G_n^+$ linear nature of $\mu$ since we also note again that the coefficients $a_{j_2 \cdots j_n} \in \G_n^+$. Using the correspondence, we then realize $\mu(f)=f(x_\mu)$ for the corresponding $x_\mu \in \R^n$. To see that this point $x_\mu \in \ball$, we take a field defined on $\G_n(\R^n)$ and monogenic in $\G_n(\Omega)$. For any $x_0 \in \R^n \setminus \ball$ we have the field $E(x_0 - x)$ is monogenic for $x\in \ball$. Then for a spin character $\mu$ we have a sequence of functions $E_n \to E(x_\mu - x)$ such that $\mu(E_n)$ is bounded for all $n$ but diverges in the limit.   \textcolor{red}{Can we actually just argue that we can determine all $x_0$ such that $E(x_0-x)$ is monogenic on $\Omega$ therefore we can determine $\R^n \setminus \Omega$?}

\todo[inline]{Make thie more explicit and do an example or something in 3D. Show that $x\mu$ is in the ball.Finish this and note that this proves the theorem.}

\todo[inline]{I'm not even sure we need to do this with $\Omega=\ball$ other than for part of the proof with the power series. But if $\Omega$ is compact, it fits inside a ball of some radius $r$ and so we should still be able to represent all the monogenics on $\Omega$ with this. The trick is we have a function that is monogenic except at a point. }

\todo[inline]{If work with weak monogenic functions then we can probably use mollifiers and stitch together monogenics on $\Omega$ from various open balls in $\Omega$ that are monogenic except at some set of measure zero. Then this should allow us to probably speak more accurately about the delta function and $E$ and probably suup this all up to determine the homeomorphism type of any embedded manifold.}

\subsection{Discussion}

Perhaps the above result should not be so surprising.  One could venture to the Atiyah-Singer index theorem which relates the topological information of a manifold with the elliptic operators.  In particular, the Dirac operator (the gradient $\grad$) is indeed elliptic. Indeed, this seemingly sparks the motivation for the Calder\'on problem.  There, the elliptic operator is the Laplace-Beltrami operator $\Delta$.  However, this is an inverse problem in which we do not know the space (or the metric) and are asked to, in a sense, determine the Laplace-Beltrami operator from information on the boundary of a Riemannian manifold.  With this boundary data, one would hopefully be able to decipher $\Delta$ and as such, construct a copy of the desired Riemannian manifold.



\section{Calder\'on problem}
\todo[inline]{Okay, we can surely recover $\monogenics^{0+2}(\Omega)$ which is $\monogenics^+(\Omega)$ when $\Omega$ is dimension 3 or less.  Is this all we really need? Otherwise, we may be at a loss here.}

\todo[inline]{Go over Ohm's law (or do it in the forms and integration section) but relate it back to the stuff here so that the conjugate field gets some interpretation.}

\todo[inline]{Explain this using the variational approach and explain that $\Omega$ is ohmic where Ohm's law is a linearization of conductivity and such (just like linear elasticity). The electromagnetic potential (or something) is a monogenic spinor?}
\subsection{Electromagnetism}

Consider the spacetime algebra $\G_{1,3}$ seen in Example \ref{ex:spacetime_algebra}. Then the spacetime multivector fields on $\Omega$ are $\G_{1,3}(\Omega)$ with the basis vector fields $e_t,e_1,e_2,e_3$.  A vector field $F$ on $\Omega$ is then given by
\[
A= A_t e_t + A_1 e_1 + A_2 e_2 + A_3 e_3,
\]
where each coefficient is a smooth scalar field. We denote now by $\grad_{st} = e^t \nabla_{e_t} + \sum_{j=1}^3 e^j \nabla_{e_j}$ the spacetime gradient and take $\grad = \sum_{j=1}^3 e^j \nabla_{e_j}$ as the spatial gradient. Let $\vectorpotential = A_1 e_1 + A_2 e_2 + A_3 e_3$ be the spacelike part of the spacetime vector $A$ and let $\phi = A_t$ be the timelike part. If the vector field does not depend on the temporal variable $t$ we have $\nabla_{e_t} A=0$ and thus
\[
\grad_{st} A = \grad u e_t + \grad \wedge \vectorpotential + \grad \cdot \vectorpotential.
\]
If we then take the Lorenz gauge condition $\grad \cdot \vectorpotential =0$, we have
\[
\|\grad_{st}A\|^2 = \| \grad u\|^2 + \|\grad \wedge \vectorpotential\|^2.
\]
The Lagrangian for this field is then
\[
\mathcal{L}(A) = \|\grad_{st}A\|^2 - A \cdot J,
\]
where $J = \rho e_t + J_1 e_1 + J_2 e_2 + J_3 e_3$ is a spacetime vector field \textcolor{red}{A lagrange multiplier?}. The Euler-Lagrange equations with the gauge condition yields
\[
\grad_{st}^2 A = J.
\]
Let $\current = J_1 e_1 + J_2 e_2 + J_3 e_3$ and if we take a static four current $\nabla_{e_t} J=0$ we must have $\nabla_{e_t} A =0$ and we arrive at two equations
\[
\grad \cdot \grad \wedge u e_t = \rho e_t \qquad \textrm{and} \qquad \grad \cdot \grad \wedge \vectorpotential = \current, 
\]
of course one can take $\Delta u = \rho$, but we this equation arises from the spacetime formulation itself. Note that we did not force an inner product on the spatial vectors $e_1,e_2,e_3$ other than they are orthogonal with the temporal vector $e_t$.  These equations we have are the invariant forms of the equations with respect to any (positive definite) spatial inner product. This will be important momentarily.

In this, we have realized the electric and magnetic fields
\[
\grad \wedge u e_t = e \qquad \textrm{and} \qquad \grad \wedge \vectorpotential = b,
\]
and note the electric field $e$ is a spacetime bivector and the magnetic field $b$ is a purely spatial bivector that $\grad \wedge e = 0$ and $\grad \wedge b =0$ are satisfied. The fact that $e$ is a spacetime bivector means it behaves like a spacelike vector when acted on by spatial gradient $\grad$ owing to the static Faraday's law $\grad \times E =0$. Since $b$ is purely spatial, we see $\grad \wedge b = 0$ mimics the Gauss's law for magnetism if we take the unit spacelike trivector $I$ and let $B=bI$ be the magnetic vector field we have $\grad \cdot B = 0$.

In the EIT problem, we begin with a region $\Omega$ with unknown symmetric positive definite conductivity matrix $\gamma$. We apply a static scalar potential $\phi$ on $\Sigma$ which produces the potential $u^\phi$ in the interior. We assume $\Omega$ is an ohmic material in that Ohm's law $-\gamma \grad \wedge u^\phi = \current$ is satisfied.  It follows that 
\[
-\gamma \grad \wedge u^\phi = \grad \cdot b.
\]
The conservation law
\[
\int_\Sigma J \cdot \nu d\Sigma = 0,
\]
implies $\grad \cdot \current= 0$ and we arrive at $\grad \cdot (\gamma \grad \wedge u) = 0$. See, for example, \cite{feldman_calderproblem_nodate}. 

The conductivity matrix was given in an terms of an orthonormal spatial basis under the Euclidean inner product and we can write the components $\gamma^{ij}$ for $i,j=1,2,3$.  In \cite{uhlmann_inverse_2014} we find a relationship between the intrinsic Riemannian metric on a space and the conductivity by
\begin{equation}
\label{eq:conductivity_metric}
    g_{ij} = (\det \gamma^{k\ell} )^{\frac{1}{n-2}} (\gamma^{ij})^{-1}, \quad \gamma^{ij} = (\det g_{k\ell})^{\frac{1}{2}} (g_{ij})^{-1}.
\end{equation}
If we impose the inner product on the spatial components come from $g$ with coefficients in the basis given by the above equations, we can note we have Ohm's law by
\[
-\grad \wedge u^\phi = \current,
\]
In the static case when there are no free charges inside $\Omega$, we have
\[
\Delta_{st} A = \current \quad \textrm{in $\Omega$},
\]
and we arrive at $\Delta u = 0$ for the scalar potential and $\Delta \vectorpotential = \current$ for the magnetic vector potential. In terms of the magnetic field bivector, we have $\grad \cdot b = \current$ and once again by Ohm's law we have $-\grad \wedge u^\phi = \grad \cdot b$. This leads us to consider the parabivector field $f=u+b$. We can note that $f$ is (spatially) monogenic since 
\[
\grad f = 0 ~\iff~ -\grad \wedge u^\phi =  \grad \cdot b ~\textrm{and}~ \grad \wedge b = 0,
\]
is satisfied. We see now that the fact that the body $\Omega$ is ohmic gives us a necessary coupling between the scalar potential and the magnetic field.

\subsection{Biot Savart Law}

Recall the Biot Savart law from electromagnetic theory
\[
\vec{B}(y)=\frac{1}{4\pi} \int_\Omega \current \times \frac{y-x}{|y-x|^3} d\Omega(x),
\]
which satisfies 
\[
\grad \times \vec{B} = \current + \frac{1}{4\pi} \grad \wedge \int_\Omega \frac{\grad \cdot \current}{|y-x|}d\Omega(x) -\frac{1}{4\pi} \grad \wedge \int_\Sigma \frac{\current \cdot \nu}{|y-x|}d\Sigma(x)
\]
In the EIT problem we do not allow charges to accumulate in the interior and so we must have
\[
\grad \cdot \current = 0,
\]
so long as $\grad \cdot \current$ is continuous \cite{feldman_calderproblem_nodate}. Hence we are left with
\[
\grad \times \vec{B} = \current -\frac{1}{4\pi} \grad \wedge \int_\Sigma \frac{\Lambda(\phi)\\}{|y-x|}d\Sigma(x),
\]
where $\Lambda$ is the DN map.

\begin{remark}
    It seems like this now says that $u$ has a conjugate field $B$ if and only if
\[
\frac{1}{4\pi} \grad \wedge \int_\Sigma \frac{\Lambda(\phi)\\}{|y-x|}d\Sigma(x) = 0.
\]
\end{remark}
Assuming we can swap differentiation and integration we have
\begin{align*}
\grad \wedge \int_\Sigma \frac{\Lambda(\phi)}{|y-x|} d\Sigma(x) &= \int_\Sigma \frac{\Lambda(\phi)(y-x)}{|y-x|^3}d\Sigma(x),
\end{align*}
since $\grad \wedge \Lambda = 0$ \textcolor{red}{In B.V. DN-Forms}.

\begin{remark}
Perhaps we can just rearrange to see:
\[
\cauchy [\current] = \frac{1}{4\pi} \int_\Sigma \frac{y-x}{|y-x|^3} (\nu \cdot \current + \nu \wedge \current) d\Sigma(x) 
\]
and we note $\current \cdot \nu = \Lambda(\phi)$ for which we have found
\[
 \frac{1}{4\pi} \int_\Sigma \frac{\Lambda(\phi)(y-x)}{|y-x|^3}=\grad \times \vec{B}-\current,
\]
Hence 
\[
\cauchy[\current] = \grad \times \vec{B} - \current + \frac{1}{4\pi} \int_\Sigma \frac{y-x}{|y-x|^3} \nu \wedge \current d \Sigma(x)
\]
\end{remark}

In terms of geometric algebra, we wish to show the analogous statement for the magnetic bivector field
\[
B(y) = \frac{1}{4\pi} \int_\Omega \current \times \frac{y-x}{|y-x|^3} d\Omega(x),
\] 
in that
\[
\grad \cdot \frac{1}{4\pi} \int_\Omega \current \wedge \frac{y-x}{|y-x|^3} d\Omega(x) = \current.
\]



\subsubsection{Discussion}

The scalar potential in the EIT problem arises inside of a four vector potential for the electromagnetic field.  The electromagnetic potential satisfies Maxwell's equations which can be succinctly stated as $\grad_{st}^2 A = J$, for the four current $J$.  When the four current $J$ does not depend on time, we arrive at the static equations where the electrostatic potential $u$ and magnetic spatial vector potential $\vectorpotential$ are split into separate equations. Removal of time dependence decouples these potentials. We realize the magnetic field as the bivector $b=\grad \wedge \vectorpotential$ and the electric vector field $E=\grad \wedge u$. 

These fields interact with materials which carry an intrinsic inner product related to the conductivity by \ref{eq:conductivity_metric}. If the material is ohmic, we have Ohm's law given by $\grad \wedge u = \grad \cdot b$ which leads to the parabivector field $f=u+b$ to be monogenic. This relationship is important and is not fully realized without the proper treatment of the electromagnetic potential. 

In an electrostatic boundary value problem, one can supply the scalar potential $\phi$ on the boundary of a region. This forced scalar potential induces the scalar potential inside of the region and the scalar potential is harmonic when the interior is free of charges.  This scalar potential drives a current $\current$ via Ohm's law, and this current is related to the magnetic bivector field $b$. One may only have access to the boundary of the region and can make measurements of the resulting current flux $\projection{\nu}{\current}$ that corresponds to a given input scalar potential $\phi$. Is this enough to determine the underlying inner product of the region?


\subsection{Generalization}
\todo[inline]{Explain how we can put $\gamma$ as a spatial metric and incorporate this into the geometric algebra for $\G_{1,n}(\Omega)$ stuff. Contract away time part again and we get the same equations.}
What we have seen for the electromagnetic field is there is a coupling between the electric bivector field and the magnetic bivector field via the four vector potential.  This can be generalized to fields in $\G_{1,n}(\Omega)$ to produce analogous equations.

\subsection{Inverse problem}

A particular application for the work we have done thus far is with the Calder\'on inverse problem. One can work with differential forms, but we have found forms to be rooted in multivectors contracted with a directed measure.  We also note the previous portion on electromagnetism provides a convenient understanding for this problem. The forward problem in terms of geometric calculus is given by the following scenario. We have an ohmic $\Omega$ and we find the electrostatic potential $u$ satisfying the Dirichlet problem
\begin{equation}
\label{eq:dirichlet_problem}
\begin{cases} \Delta u^\phi = 0 & \textrm{ in $\Omega$} \\  u^\phi \vert_\Sigma = \phi & \textrm{ on $\Sigma$}. \end{cases}.
\end{equation}
In the realm of Electrical Impedence Tomography (EIT), the Dirichlet data $\phi$ amounts to an input voltage along the boundary and by Ohm's law $\current=\grad \wedge u^\phi$ provides us the current. For any given solution to the boundary value problem, there is the corresponding Neumann data $\current^\perp=\projection{\nu}{\grad u^\phi}$ where $\nu$ is the normal to the boundary $\Sigma$ defined by $\nu = I_\Sigma I$ for the oriented boundary pseudoscalar $I_\Sigma$. This motivates the so called Voltage-to-Current (VC) operator $\phi \mapsto \current^\perp$. In general, we refer to set of both boundary conditions $(\phi, \current^\perp)$ $\forall \phi$ as the \emph{Cauchy data} and define the \emph{Dirichlet-to-Neumann (DN) operator} $\Lambda$ such that $\Lambda \phi = \current^\perp$. This mimics the VC operator in EIT. With our notation from before we have
\[
\Lambda \phi = \projection{\nu}{\grad u^\phi} = \current^\perp.
\] 
Note that this operator $\Lambda$ is often referred to as the \emph{scalar} DN operator since the input is the scalar field $\phi$ whereas a more general operator on differential $k$-forms has been described in \cite{belishev_dirichlet_2008,sharafutdinov_complete_2013}. The inverse problem follows.

\vspace*{5pt}
\noindent\textbf{Calder\'on problem.} Let $\Omega$ be an unknown Riemannian manifold with unknown metric $g$ and with known boundary $\Sigma$ and known DN operator $\Lambda$. Can one recover $\Omega$ and the spatial inner product $g$ from knowledge of $\Sigma$ and $\Lambda$?
\vspace*{5pt}

\subsection{Recovering monogenic fields from $\Lambda$}

With the DN operator, we can reconstruct the boundary four current $J$.  On $\Sigma$, we have the gradient $\grad_\Sigma$ inherited from $\grad$ on $\Omega$.  In particular, we have the relationship
\[
\grad_\Sigma \phi = \projection{I_\Sigma}{\grad \phi},
\]
which is accessible with our knowledge of $\phi$ and $\Sigma$. The boundary current is then
\[
\current\vert_{\Sigma} = \grad_\Sigma \phi + \Lambda(\phi).
\]
Though we do not have access to $u^\phi$ directly, we do know that $\Delta u^\phi = \rho$ and as such we have the boundary four current by
\[
J\vert_\Sigma = \Delta u^\phi\vert_\Sigma \gamma_0 + \current\vert_\Sigma
\]
as well as the interior four current $J = \current$ since the interior is free of charges.  Defining the the four vector potential as before, we arrive at the extra equation $\Delta \vectorpotential = \current$ in $\Omega$. Once again define the magnetic bivector field $b=\grad \wedge \vectorpotential$ and we note that Ohm's law implies $\grad \cdot b = -\grad \wedge u^\phi$ in $\Omega$ and so the parabivector field $f=u^\phi + b$ is spatially monogenic since we also have $\grad \wedge b = 0$.  This all holds assuming that we can solve the electromagnetic Neumann boundary value problem
\[
\begin{cases} \Delta A = \current & \textrm{in $\Omega$}\\ A = A_\Sigma & \textrm{on $\Sigma$} \end{cases}
\]
\todo[inline]{Show that we can determine the magnetic potential $A_\Sigma$ on the boundary. This may also show that the two notions of the DN operator are equivalent. That'd be nice.}

\todo[inline]{If we show there is always a unique monogenic conjugate $b$ for any harmonic $u$ then this must be what we are doing here. Is this gauranteed by the Cauchy integral?}

Though briefly we mentioned $\Omega$ as a Riemannian manifold, we now take $\Omega$ to be a region in $\R^n$ for brevity. Using the DN operator, one can define a \emph{Hilbert transform} by
\[
T \phi  = d\Lambda^{-1} \phi,
\]
as in \cite{belishev_dirichlet_2008}. It has yet to be shown that this definition coincides with the definition in \cite{brackx_hilbert_2008}, but there is reason to believe they are related. The classical Hilbert transform on $\C$ inputs a harmonic function and outputs another harmonic function $v$ such that $u+iv$ is holomorphic. Essentially, this translates into finding a conjugate bivector field $b$ to $u^\phi$ such that $u^\phi +b$ is monogenic. First, we require $\phi$ satisfies
\todo[inline]{This statement should come from the lagrangian perspective hopefully.}
\begin{equation}
\label{eq:conjugate_requirement}
\left( \Lambda + (-1)^{n}d\Lambda^{-1}d\right)\phi = 0,
\end{equation}
where $d$ is the exterior derivative on forms. \textcolor{red}{They show how to find the image of this, perhaps I can show what the kernel is.} As shown earlier in Section \ref{subsec:diff_forms}, $d$ amounts to $\grad \wedge$ on the multivector field constituent of a form.  When condition \ref{eq:conjugate_requirement} is met, there exists a \emph{conjugate form} $\epsilon \in \Omega^{n-2}(M)$. As well, $\epsilon$ is also coclosed in that $\delta \epsilon=0$. To retrieve the constituent $(n-2)$-vector $E$, we just note $\epsilon = E \cdot dX_k$. Given Hodge duality, we have a 2-form $\beta$ such that $\star\beta = \epsilon$ and the corresponding bivector $b^\star=E$.  Combining the fields $u^\phi$ and $b$ into the parabivector $f=u^\phi+b \in \G_n^{0+2}(\Omega)$. We then note that $f$ is monogenic if and only if
\[
\grad \wedge u = -\grad \cdot b \qquad \textrm{and} \qquad \grad \wedge b = 0.
\]

\begin{lemma}
Given the fields $u^\phi$ and $b$ as above, the corresponding parabivector field
\[
f=u^\phi +b
\]
is monogenic.
\end{lemma}
\begin{proof}
Let $\star \beta^\psi = \epsilon$ as before and note that 
\begin{equation}
\label{eq:conjugate_belishev}
d u^\phi = \star d \epsilon = \star d \star \beta^\psi,  
\end{equation}
as shown in Theorem 5.1 in \cite{belishev_dirichlet_2008}. The multivector equivalent of the right hand side of Equation \cite{eq:conjugate_belishev} yields
\begin{align*}
(\grad \wedge b^\star )^\star &= [(\grad \cdot b^\dagger) I]^\star\\
    &= [I^{-1} ((\grad \cdot b^\dagger) I)]^\dagger\\
    &= ((\grad \cdot b^\dagger)I)^\dagger I\\
    &= \grad \cdot b^\dagger && \textrm{since $\dagger$ of a vector is trivial}\\
    &= -\grad \cdot b. && \textrm{since $\dagger$ of a bivector is -1}
\end{align*}
\textcolor{red}{Perhaps I should just show this property in the differntial forms section.} Thus, we have $\grad \wedge u + \grad \cdot b = 0$. Since $\epsilon$ is coclosed we have
\begin{align*}
0=\grad \cdot b^\star &= \grad \cdot (I^{-1} b)^\dagger \\
    &= \grad \cdot (b^\dagger I)\\
    &= (\grad \wedge b^\dagger) I\\
  \implies ~0  &= \grad \wedge b.
\end{align*}
\textcolor{red}{Perhaps I should just show this property in the differntial forms section.} Thus $\grad f =0$ and $F$ is monogenic.
\end{proof}

We have shown that conjugate forms give rise to monogenic fields.  We now seek to determine for what boundary conditions $\phi$ we have at our disposal. Let $E^\parallel \coloneqq \projection{I_\Sigma}{E}$, with $I_\Sigma$ the boundary pseudoscalar satisfying $\nu I_\Sigma = I$. Hence by Equation \ref{eq:projection_rejection_vectors} we have $E^\parallel = \rejection_{\nu}(E)$ then in investigating the requirement from Equation \ref{eq:conjugate_requirement} we find the multivector equivalent
\begin{align*}
    (\Lambda + (-1)^n (\grad \wedge) \Lambda^{-1} (\grad \wedge))\phi &= E^\perp + (-1)^n T E^\parallel
\end{align*}
so we arrive at the fact that we must have
\[
E^\perp = (-1)^{n-1} T E^\parallel.
\]
In other words,
\[
T  \rejection_\nu(E)= (-1)^{n-1}\projection{\nu}{E}.
\]
Thus, the Hilbert transform maps tangential components of $\grad u^\phi = E$ to nontangential boundary components on the boundary.




% EXTRA STUFF BELOW
\section{Further questions}
\subsection{Generating axial monogenics}

The following questions remain for a domain in $\R^3$.

\begin{question}
    For what boundary values $\varphi \in C_\infty(\Sigma)$ can we generate axial monogenics?
\end{question}

\begin{question}
    Do these boundary values exhaust the whole axial algebra $\algebra{\omega}$?
\end{question}

Fix an axis $\omega$ which defines the blade $B = \omega I$ and thus defines the $B$-plane in $\R^3$.  Then, let $f=u+\beta B$ be an $\omega$-axial monogenic.  We can then determine the boundary values for $f$ on $\Sigma$ by orthogonal projection onto the $B$-plane.  That is, we care only about the components of $f$ perpendicular to the axis $\omega$ and hence we take for $\zeta \in \Sigma$
\[
\zeta^\perp = \omega \omega \wedge \zeta = (x\cdot B)B^{-1}.
\]
showing the relationship between projection onto a plane and being orthogonal to an axis in $\R^3$. Specifically, this means that the relationship $f(x)=f(x+t\omega)$ can be written as
\[
f(x)=f((x\cdot B)B^{-1}),
\]
in that we only care about the portion of $x$ along the plane given by $B$.  Thus, for $\xi \in \Sigma$ we have
\[
f(\xi) = f((\xi \cdot B)B^{-1}).
\]

\begin{figure}[H]
	\centering
	%\def\svgwidth{\columnwidth}
	\resizebox{\columnwidth}{!}{\input{omega_axial.pdf_tex}}
\end{figure}

\textcolor{red}{So boundary values of axial monogenics are axial and...?.}

\begin{example}
    Consider the 3-dimensional example with $M=B_3$ and $\Sigma=S^2$.  Let $e_1,e_2,e_3$ be a global orthonormal basis and let $g_{ij}=\delta_{ij}$.  Then let $B=e_1 \wedge e_2$.  Then the paravector field $f(x^1,x^2,x^3)=x^1+x^2B$ is $e_3$-axial. Clearly we can see that $f(x^1,x^2,x^3+t)=f(x^1,x^2,x^3)$ for any $t$.  $f$ is also monogenic as one can show
    \[
        \grad f = e_1 + (e_2 \wedge e_3)I = e_1 - e_1 = 0.
    \]
    Indeed, this $f$ is none other than the complex function $f(z)=z$ with $B$ taking the role of the imaginary unit $i$. 

    Let $x=x^1e_1 + x^2e_2 + x^3e_3$.  Then, 
    \[
        B (x\cdot B) = (e_1e_2)( x^1e_2 -x^2 e_1 ) = x^1 e_1 + x^2 e_2.
    \] 
    Thus, for $\xi \in S^2$, we have $f(\xi)=\xi^1 +\xi^2 B$.
\end{example}

\todo[inline]{If we consider now every $\omega$-axial monogenic can be written as a power series, if we can construct $z$ we should be done...?}

It is clear that we can define a monogenic field $f=u+b$ via the Cauchy integral, but we then require $\nabla_\omega f = 0$.  Let $f=\cauchy[\varphi](x)$, then we must have
\[
\nabla_\omega \proj{0}{\cauchy[\varphi](x)} = 0 \qquad \textrm{and} \qquad \nabla_\omega \proj{2}{\cauchy[\varphi](x)}=0.
\]
The first condition yields
\[
0 = \int_\Sigma \frac{(\nu(\zeta)\cdot x) (\omega \cdot x)}{|x-\zeta|^2} \phi(\zeta) d\Sigma(\zeta).
\]


\begin{theorem}
    For any $\omega \in Gr(1,3)$ we have that $\algebra{\omega}\subset \monogenics$. 
\end{theorem}
\begin{proof}
    \textcolor{red}{This seems to be saying that we need boundary values in some hardy space or something. They defined this conjugacy thing as $G$.}
    Fix a unit vector $\omega$.  We want to show that for any $f=u+b\in \algebra{\omega}$ that $\iota^* u=\phi$ satisfies \ref{eq:conjugacy_requirement}.  That is,
    \[
        G\phi = (\Lambda - d\Lambda^{-1}d) \phi = 0.
    \]
    Note that $\phi$ is the trace of a harmonic function, so this operator is well defined.  Note that the equation
    \[
        \Lambda \psi = d \phi
    \]
    has a solution
\end{proof}

\section{Radon transform and integral geometry}

I feel like there is some way to go from projection onto subspaces as a map to grassmannians and reconstructing the manifold.  It's like a morse function type of thing.  Radon transforms also come to mind.

\section{Relation to the BC Method}

\textcolor{red}{Describe how this process can lead to the BC method in dimension $n=2$}


\section{Conclusion}


\appendix
\section{Appendix}

\todo[inline]{Put axial condition for cauchy integral and some other quick proofs in here.}

\subsection{Spin fibration}
maybe pose this as a question in relation to using the 2d belishev stuff.

\textcolor{red}{The inner product for characters is what you use for fourier theory, maybe we can do something here with characters as maps to the grassmannian? Do these form some kind of orthogonal basis? Also, the Dirac operator and Laplacian are spin invariant! This is what they use the $\mathbb{H}$ module structure for!}

A main question to answer now is how the $B$-planar algebras $\algebra{B}$ relate to the space of monogenic functions $\monogenics$.  In particular, this question seems analogous to the invertibility of a $2$-plane x-ray transform.  Let $f$ be a monogenic, can $f$ be generated by $B$-planar monogenics? Noting that each unit 2-blade corresponds to a unique 2-plane in $\R^n$, we can realize every $B$ as a point in $\Grassmannian{2}{n}$.  Letting $f_B$ be some $B$-planar axial monogenic, is
\[
f = \int_{B \in \Grassmannian{2}{n}} a(B) f_B d \lambda,
\]
where $a(B)$ is a scalar function on $\Grassmannian{2}{n}$ and $d\lambda$ is the Haar measure on $\Grassmannian{2}{n}$ monogenic? Moreover, can any monogenic $f$ be constructed in this manner? First, we start with a lemma describing the form of $f_B$.


\begin{lemma}
    Let $f$ be a monogenic (0+2)-vector and define $f_B \coloneqq \projection{B}{f(\projection{B}{x})}$. Then $f_B$ is $B$-planar and monogenic.  
\end{lemma}
\begin{proof}
    It is clear by definition that $f_B$ is constant along translations of the $B$-plane and can be written as $u_B+\beta b_B$ and so $f_B$ is $B$-planar.  To see $f_B$ is monogenic, let $e_1,\dots,e_n$ be a basis such that $B=e_1e_2$ and $e_i \cdot B = 0$ for $i\neq 1,2$. Then note $\nabla_{e_i} f_B =0$ when $i\neq 1,2$ as well leading to
    \[
        \grad f_B = e^1 \nabla_{e_1}f_B + e^2 \nabla_{e_2}f_B
    \]
    Recall that $f=u+b$ must satisfy
    \[
        \grad \wedge u = \grad \cdot b \qquad \textrm{and} \qquad \grad \wedge b = 0.
    \]
    Specifically,
    \[
        e^1 \wedge \nabla_{e_1} u + e^2 \wedge \nabla_{e_2}u + \cdots + e^n \wedge \nabla_{e_n} = e^1 \cdot \nabla_{e_1} b + e^2 \cdot \nabla_{e_2}b + \cdots + e^n \cdot \nabla_{e_n} b
    \]
    Clearly, $\grad \wedge b_B = 0$, thus we need only show
    \[
        \grad \wedge u_B = \grad \cdot b_B.
    \]
    In particular
\end{proof}


We can note that the $B$-planar monogenics are given by a power series $\sum_{n=0}^\infty a_n (x+yB)^n$ due to the isomorphism of algebras $\mathfrak{spin}(2)\cong \C$ \textcolor{red}{This shouldn't be hard to show without appealing to this isomorphism.} In particular, any $B$-planar monogenic is approximated arbitrarily closely by a homogeneous polynomial of degree $n$ in the variables $x$ and $y$. Moreover, $1$ and $x+yB$ generate the $B$-planar monogenics. $\spingroup$ then acts on $B$. \textcolor{red}{Okay, well maybe there's some nice way to talk about characters as mappings to the grassmannian instead of the circle? Should read more about characters and maybe they are really maps to spin group? They are for the 2d case. Structure space and stuff. Should probably rename some of these things I have.s}


\textcolor{red}{Countable basis for $\monogenics$ ?}

\section{Tomography on convex regions}


\section{Other}

\subsection{Cauchy and Poisson integrals}

In regions of $\R^n$, one can define a Cauchy integral operator and Hilbert transform for multivector fields.  The details of these integral operators are laid out in \cite{brackx_hilbert_2008}. Note that the authors there take the opposite signature to $\mathcal{G}_n$ and define the gradient operator as $\underline{\partial} = e_j \nabla_{e_j}$.  Thus, we have $\grad = g^{ij} \underline{\partial}$.  Nonetheless, the fundamental solution to $\grad$ is a vector field given by
\[
E(x) = \frac{1}{a_m} \frac{x}{|x|^m},
\]
for $x\in \R^n$.  This is clear to see if we take $e_i$ to be a (local) orthonormal basis
\begin{align*}
\grad \wedge E &= \frac{1}{a_m}  \left( \frac{1}{|x|^n} \nabla_{e_i} x^j + x^j \nabla_{e_i} \frac{1}{|x|^n} \right) e^i \wedge e_j\\
&= \left( \frac{1}{|x|^n} \delta_i^j -\frac{3x^ix^j}{|x|^{n+2}} \right) e^i \wedge e_j\\
&= -\frac{3x^i x^j}{|x|^{n+2}} e^i \wedge e_j &&\textrm{since $e^i \wedge e_i=0$}\\
&= 0 &&\textrm{since $e^j\wedge e_i = -e^i \wedge e_j$ for an orthonormal basis.}
\end{align*}
(see \url{https://math.stackexchange.com/questions/811248/wedge-product-between-nonorthogonal-basis-and-its-reciprocal-basis-in-geometric}. This is also clear since $E$ is a radial field and thus has no curl. Then, let $B_{\epsilon}$ be the $n$-ball of radius $\epsilon$ centered at the origin and we have
\begin{align*}
\int_{B_\epsilon} \grad \cdot Ed \Omega &= \int_{S_\epsilon} E \cdot \nu d\Sigma\\
&=\frac{1}{a_n} \int_{S_\epsilon} \frac{x\cdot \frac{x}{|x|}}{|x|^n} d\Sigma\\
&= \frac{1}{a_n} \int_{S_\epsilon} \frac{1}{\epsilon^{n-1}} d\Sigma\\
&= \frac{1}{a_n} \int_{S_\epsilon} \frac{1}{\epsilon^{n-1}} \epsilon^{n-1} d\phi_1 d\phi_2 \cdots d\phi_{n-1}\\
&= 1.
\end{align*}

Let $\partial M = \Sigma$ and define now the $\G_n$ valued inner product on multivector fields $f,g \in L_2(\Sigma)$
\[
\innerproduct{f}{g} = \int_{\partial M} f(\zeta)g(\zeta)d\Sigma(\zeta).
\]
We can then define the \emph{Cauchy kernel} for $x\in M$ and $\zeta \in \partial M$ using the fundamental solution $E$ as
\[
C(\zeta, x) = -\frac{1}{a_n} \nu(\zeta) E(x-\zeta)
\]
where $\nu(\zeta)$ is the outward normal vector to the hypersurface $\Sigma = \partial M$. Note the inclusion of the minus sign is due to the signature of the inner product $g$. The Cauchy integral for $\phi \in L_2(\partial M)$ is then
\[
\cauchy[\phi](x) = \innerproduct{C(\zeta,x)}{\phi(\zeta)} =\frac{1}{a_n} \int_{\Sigma} \frac{\zeta-x}{|x-\zeta|^n} \nu(\zeta) \phi(\zeta) d\Sigma(\zeta).
\]
The most important properties of the Cauchy integral is that $\cauchy[\phi]$ is monogenic in $M$ and for a scalar function $\phi$, $\cauchy[\phi]$ is a paravector.  Specifically,
\begin{align*}
\cauchy[\phi](x) &= \frac{1}{a_n} \int_{\Sigma} \frac{\zeta-x}{|x-\zeta|^n} \nu(\zeta) \phi(\zeta) d\Sigma(\zeta)\\
&= \frac{1}{a_n} \left(\int_{\Sigma} \phi(\zeta) \frac{\zeta-x}{|x-\zeta|^n} \cdot \nu(\zeta) d\Sigma(\zeta) + \int_{\Sigma} \phi(\zeta) \frac{\zeta-x}{|x-\zeta|^n} \wedge \nu(\zeta) d\Sigma(\zeta)\right)
\end{align*}

Similarly, for the $n$-ball of radius $r$, $B_r \subset \R^n$, we have the \emph{Poisson kernel}
\[
P(\zeta,x) = \frac{1}{a_n}\frac{r^2-|x|^2}{r|x-\zeta|^n}.
\]
Notably, we have the Poisson integral
\[
\poisson[\phi](x) = \innerproduct{P(\zeta,x)}{\phi(\zeta)} = \frac{1}{a_n} \int_\Sigma \phi(\zeta) \frac{r^2-|x|^2}{r|x-\zeta|^n},
\]
which is harmonic on $B_r$ and extends continuously onto $\Sigma$. Briefly letting $g_{ij}=\delta_{ij}$, if we then consider the Cauchy integral over $\Sigma = \partial B_r = S_r$ then it is apparent that the Poisson integral deviates from the scalar part of the Cauchy integral as
\[
\proj{0}{\cauchy[\phi](x)} = \frac{1}{a_n} \int_\Sigma \phi(\zeta) \frac{r^2- x\cdot \xi}{r|x-\xi|^n} d\Sigma (\zeta).
\]
Sadly, this means that we do not have the boundary behavior of the Cauchy integral that we desire.  Namely, the $\iota^* \proj{0}{\cauchy[\phi](x)}\neq \phi$ in general. It is also worth noting that it is an open problem to determine a general form for the Poisson integral for other domains in $\R^n$. However, since $\cauchy[\phi](x)$ is monogenic, we have that the components are harmonic.

\begin{itemize}
    \item Prove that the Hilbert transforms are equivalent on traces of harmonic functions. Specifically, $Td\phi = d\psi$.  
    \item Discuss hardy spaces as closure of $\monogenics$. 
    \item $H^2=1$ on $L_2(\partial M)$ which should show we satisfy the theorem below.
\end{itemize}

Let $M\subset \R^3$ be a 

\subsection{Generating axial monogenics}

The following questions remain for a domain in $\R^3$.

\begin{question}
    For what boundary values $\varphi \in C_\infty(\Sigma)$ can we generate axial monogenics?
\end{question}

\begin{question}
    Do these boundary values exhaust the whole axial algebra $\algebra{\omega}$?
\end{question}

Fix an axis $\omega$ which defines the blade $B = \omega I$ and thus defines the $B$-plane in $\R^3$.  Then, let $f=u+\beta B$ be an $\omega$-axial monogenic.  We can then determine the boundary values for $f$ on $\Sigma$ by orthogonal projection onto the $B$-plane.  That is, we care only about the components of $f$ perpendicular to the axis $\omega$ and hence we take for $\zeta \in \Sigma$
\[
\zeta^\perp = \omega \omega \wedge \zeta = (x\cdot B)B^{-1}.
\]
showing the relationship between projection onto a plane and being orthogonal to an axis in $\R^3$. Specifically, this means that the relationship $f(x)=f(x+t\omega)$ can be written as
\[
f(x)=f((x\cdot B)B^{-1}),
\]
in that we only care about the portion of $x$ along the plane given by $B$.  Thus, for $\xi \in \Sigma$ we have
\[
f(\xi) = f((\xi \cdot B)B^{-1}).
\]

\begin{figure}[H]
	\centering
	%\def\svgwidth{\columnwidth}
	\resizebox{\columnwidth}{!}{\input{omega_axial.pdf_tex}}
\end{figure}

\textcolor{red}{So boundary values of axial monogenics are axial and...?.}

\begin{example}
    Consider the 3-dimensional example with $M=B_3$ and $\Sigma=S^2$.  Let $e_1,e_2,e_3$ be a global orthonormal basis and let $g_{ij}=\delta_{ij}$.  Then let $B=e_1 \wedge e_2$.  Then the paravector field $f(x^1,x^2,x^3)=x^1+x^2B$ is $e_3$-axial. Clearly we can see that $f(x^1,x^2,x^3+t)=f(x^1,x^2,x^3)$ for any $t$.  $f$ is also monogenic as one can show
    \[
        \grad f = e_1 + (e_2 \wedge e_3)I = e_1 - e_1 = 0.
    \]
    Indeed, this $f$ is none other than the complex function $f(z)=z$ with $B$ taking the role of the imaginary unit $i$. 

    Let $x=x^1e_1 + x^2e_2 + x^3e_3$.  Then, 
    \[
        B (x\cdot B) = (e_1e_2)( x^1e_2 -x^2 e_1 ) = x^1 e_1 + x^2 e_2.
    \] 
    Thus, for $\xi \in S^2$, we have $f(\xi)=\xi^1 +\xi^2 B$.
\end{example}

\todo[inline]{If we consider now every $\omega$-axial monogenic can be written as a power series, if we can construct $z$ we should be done...?}

It is clear that we can define a monogenic field $f=u+b$ via the Cauchy integral, but we then require $\nabla_\omega f = 0$.  Let $f=\cauchy[\varphi](x)$, then we must have
\[
\nabla_\omega \proj{0}{\cauchy[\varphi](x)} = 0 \qquad \textrm{and} \qquad \nabla_\omega \proj{2}{\cauchy[\varphi](x)}=0.
\]
The first condition yields
\[
0 = \int_\Sigma \frac{(\nu(\zeta)\cdot x) (\omega \cdot x)}{|x-\zeta|^2} \phi(\zeta) d\Sigma(\zeta).
\]


\begin{theorem}
    For any $\omega \in Gr(1,3)$ we have that $\algebra{\omega}\subset \monogenics$. 
\end{theorem}
\begin{proof}
    \textcolor{red}{This seems to be saying that we need boundary values in some hardy space or something. They defined this conjugacy thing as $G$.}
    Fix a unit vector $\omega$.  We want to show that for any $f=u+b\in \algebra{\omega}$ that $\iota^* u=\phi$ satisfies \ref{eq:conjugacy_requirement}.  That is,
    \[
        G\phi = (\Lambda - d\Lambda^{-1}d) \phi = 0.
    \]
    Note that $\phi$ is the trace of a harmonic function, so this operator is well defined.  Note that the equation
    \[
        \Lambda \psi = d \phi
    \]
    has a solution
\end{proof}

\section{Radon transform and integral geometry}

I feel like there is some way to go from projection onto subspaces as a map to grassmannians and reconstructing the manifold.  It's like a morse function type of thing.  Radon transforms also come to mind.

\section{Relation to the BC Method}

\textcolor{red}{Describe how this process can lead to the BC method in dimension $n=2$}


\section{Conclusion}


\appendix
\section{Appendix}

\todo[inline]{Put axial condition for cauchy integral and some other quick proofs in here.}

\bibliographystyle{siam}
\bibliography{calderon_problem}





\end{document}
