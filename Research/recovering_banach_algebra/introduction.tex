In 1980, Alberto Calder\'on proposed an inverse problem in his paper \emph{On an inverse boundary value problem} \cite{calderon_inverse_2006} where he asks if one can determine the conductivity matrix of a medium from Cauchy data supplied on the boundary.  In dimensions $n>2$, this is equivalent to determining a Riemannian manfiold up to isometry from the scalar Dirichlet-to-Neumann (DN) operator \cite{feldman_calderproblem_nodate, salo_calderon_nodate, uhlmann_inverse_2014}. The DN operator takes any given Dirichlet boundary values and outputs the corresponding Neumann data of a solution to Laplace's equation in order to generate the relevant Cauchy data.  

One approach to reconstructing the Riemannian metric in dimension $n=2$ appears in \cite{belishev_calderon_2003}, where the author uses the Boundary--Control (BC) method to determine the manifold up to conformal class. \textcolor{red}{Add in a bunch of other citations to the BC method.} The BC method takes an algebraic approach. Specifically, the DN operator determines the algebra of holomorphic functions on $M$ and realizes $M$ as homeomorphic to the Gelfand spectrum of this commutative algebra. The metric $g$ is then recovered after providing $M$ with a complex structure. In dimension $n=2$, the Laplace-Beltrami operator is conformally invariant, and this result cannot be improved.  An attempt to generalize this approach to dimension $n=3$ can be found in by replacing the complex structure with a quaternionic structure but this has not lead to a complete solution \cite{belishev_algebras_2017, belishev_algebraic_2019}.  It has been shown that when $M$ is the 3-dimensional ball in $\R^3$, there is an associated space of harmonic quaternion fields that has a quaternion spectrum homeomorphic to the ball. But, a connection to the DN operator has not been made, and this method has also not been generalized to higher dimensions.

In this paper, I show that there exists a space of spin characters $\characters$ acting on a $\spingroup$ invariant space of monogenic multivector fields on the $n$-dimensional ball that is homeomorphic to the ball.  We then observe that this space of monogenics is determined from the DN map, and thus recover the ball up to homeomorphism from the boundary data.  This is summarized in two main theorems.
\begin{theorem*}
The set of multiplicative $\spinalgebra$-linear functionals on the $\spingroup$ invariant space of monogenic fields $\monogenics$ on the $n$-dimensional ball $\ball$ is homeomorphic to $\ball$ with the Gelfand topology.
\end{theorem*}
\begin{theorem*}
The scalar DN operator determines the $\spingroup$ invariant space of monogenic fields on regions in $\R^n$.
\end{theorem*}
The second theorem can be extended to Riemannian manifolds quite readily.

We first introduce the Clifford algebra setting. Given a vector space with an inner product, we can create the graded Clifford algebra.  In particular, we extend these Clifford algebras to Clifford algebra valued functions (or multivector fields) on regions $M \subset \R^n$.  Inside the multivector fields sit the even graded multivectors consisting of scalars, bivectors, and other $2k$-vectors. In $\R^2$ with the Euclidean inner product, this space is isomorphic to the $\C$-algebra and so the functions valued in this even sub-Clifford algebra can be thought of as complex valued functions.  Clifford analysis generalizes the notion of holomorphicity to monogenicity and we find that monogenic functions lie in the kernel of the Dirac operator $\grad$ just as $\C$-holomorphic functions lie in the kernel of the Wirtinger derivative $\frac{\partial}{\partial \overline{z}}$. Moreover, one has that $\grad$ is the square root Laplace-Beltrami operator $\Delta = \grad^2$. Even monogenic multivector fields are $\spingroup$ invariant and each grade is harmonic (in the kernel of $\Delta$). 

When $M$ is the $n$-ball, we have that space of even monogenics $\monogenics$ which can be generated by the algebras of even graded $B$-planar monogenic biparavector fields (each field constant on translations of the $B$-plane in $\R^n$). Those generating subalgebras are individually isomorphic to the algebra of holomorphic functions on the complex unit disk $\disk$. On these spaces, one can define $\spinalgebra$-linear multiplicative functionals $\characters$, referred to as spin characters. Each spin character is equivalent to a Dirac measure on the $n$-ball which, with the Gelfand topology, provide a homeomorphic copy of the $n$-ball.

The space of $(0+2)$-vector monogenics is found from the DN operator in the following sense.  The DN operator determines a Hilbert transform on multivector fields that allows one to determine the monogenic conjugate bivector field $b$ corresponding to a scalar solution $u$ to the Laplace equation $\Delta u = 0$ so that $f=u+b$ is monogenic. \textcolor{red}{Haven't actually done this yet} Considering all smooth boundary conditions generates the relevant space of monogenics, from which we determine the space of spin characters. Thus, the DN operator provides a means of constructing a homeomorphic of the $n$-ball.