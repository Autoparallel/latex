\subsection{Spin fibration}
maybe pose this as a question in relation to using the 2d belishev stuff.

\textcolor{red}{The inner product for characters is what you use for fourier theory, maybe we can do something here with characters as maps to the grassmannian? Do these form some kind of orthogonal basis? Also, the Dirac operator and Laplacian are spin invariant! This is what they use the $\mathbb{H}$ module structure for!}

A main question to answer now is how the $B$-planar algebras $\algebra{B}$ relate to the space of monogenic functions $\monogenics$.  In particular, this question seems analogous to the invertibility of a $2$-plane x-ray transform.  Let $f$ be a monogenic, can $f$ be generated by $B$-planar monogenics? Noting that each unit 2-blade corresponds to a unique 2-plane in $\R^n$, we can realize every $B$ as a point in $\Grassmannian{2}{n}$.  Letting $f_B$ be some $B$-planar axial monogenic, is
\[
f = \int_{B \in \Grassmannian{2}{n}} a(B) f_B d \lambda,
\]
where $a(B)$ is a scalar function on $\Grassmannian{2}{n}$ and $d\lambda$ is the Haar measure on $\Grassmannian{2}{n}$ monogenic? Moreover, can any monogenic $f$ be constructed in this manner? First, we start with a lemma describing the form of $f_B$.


\begin{lemma}
    Let $f$ be a monogenic (0+2)-vector and define $f_B \coloneqq \projection{B}{f(\projection{B}{x})}$. Then $f_B$ is $B$-planar and monogenic.  
\end{lemma}
\begin{proof}
    It is clear by definition that $f_B$ is constant along translations of the $B$-plane and can be written as $u_B+\beta b_B$ and so $f_B$ is $B$-planar.  To see $f_B$ is monogenic, let $e_1,\dots,e_n$ be a basis such that $B=e_1e_2$ and $e_i \cdot B = 0$ for $i\neq 1,2$. Then note $\nabla_{e_i} f_B =0$ when $i\neq 1,2$ as well leading to
    \[
        \grad f_B = e^1 \nabla_{e_1}f_B + e^2 \nabla_{e_2}f_B
    \]
    Recall that $f=u+b$ must satisfy
    \[
        \grad \wedge u = \grad \cdot b \qquad \textrm{and} \qquad \grad \wedge b = 0.
    \]
    Specifically,
    \[
        e^1 \wedge \nabla_{e_1} u + e^2 \wedge \nabla_{e_2}u + \cdots + e^n \wedge \nabla_{e_n} = e^1 \cdot \nabla_{e_1} b + e^2 \cdot \nabla_{e_2}b + \cdots + e^n \cdot \nabla_{e_n} b
    \]
    Clearly, $\grad \wedge b_B = 0$, thus we need only show
    \[
        \grad \wedge u_B = \grad \cdot b_B.
    \]
    In particular
\end{proof}


We can note that the $B$-planar monogenics are given by a power series $\sum_{n=0}^\infty a_n (x+yB)^n$ due to the isomorphism of algebras $\mathfrak{spin}(2)\cong \C$ \textcolor{red}{This shouldn't be hard to show without appealing to this isomorphism.} In particular, any $B$-planar monogenic is approximated arbitrarily closely by a homogeneous polynomial of degree $n$ in the variables $x$ and $y$. Moreover, $1$ and $x+yB$ generate the $B$-planar monogenics. $\spingroup$ then acts on $B$. \textcolor{red}{Okay, well maybe there's some nice way to talk about characters as mappings to the grassmannian instead of the circle? Should read more about characters and maybe they are really maps to spin group? They are for the 2d case. Structure space and stuff. Should probably rename some of these things I have.s}


\textcolor{red}{Countable basis for $\monogenics$ ?}