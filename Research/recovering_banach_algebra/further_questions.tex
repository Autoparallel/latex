\subsection{Generating axial monogenics}

The following questions remain for a domain in $\R^3$.

\begin{question}
    For what boundary values $\varphi \in C_\infty(\Sigma)$ can we generate axial monogenics?
\end{question}

\begin{question}
    Do these boundary values exhaust the whole axial algebra $\algebra{\omega}$?
\end{question}

Fix an axis $\omega$ which defines the blade $B = \omega I$ and thus defines the $B$-plane in $\R^3$.  Then, let $f=u+\beta B$ be an $\omega$-axial monogenic.  We can then determine the boundary values for $f$ on $\Sigma$ by orthogonal projection onto the $B$-plane.  That is, we care only about the components of $f$ perpendicular to the axis $\omega$ and hence we take for $\zeta \in \Sigma$
\[
\zeta^\perp = \omega \omega \wedge \zeta = (x\cdot B)B^{-1}.
\]
showing the relationship between projection onto a plane and being orthogonal to an axis in $\R^3$. Specifically, this means that the relationship $f(x)=f(x+t\omega)$ can be written as
\[
f(x)=f((x\cdot B)B^{-1}),
\]
in that we only care about the portion of $x$ along the plane given by $B$.  Thus, for $\xi \in \Sigma$ we have
\[
f(\xi) = f((\xi \cdot B)B^{-1}).
\]

\begin{figure}[H]
	\centering
	%\def\svgwidth{\columnwidth}
	\resizebox{\columnwidth}{!}{\input{omega_axial.pdf_tex}}
\end{figure}

\textcolor{red}{So boundary values of axial monogenics are axial and...?.}

\begin{example}
    Consider the 3-dimensional example with $M=B_3$ and $\Sigma=S^2$.  Let $e_1,e_2,e_3$ be a global orthonormal basis and let $g_{ij}=\delta_{ij}$.  Then let $B=e_1 \wedge e_2$.  Then the paravector field $f(x^1,x^2,x^3)=x^1+x^2B$ is $e_3$-axial. Clearly we can see that $f(x^1,x^2,x^3+t)=f(x^1,x^2,x^3)$ for any $t$.  $f$ is also monogenic as one can show
    \[
        \grad f = e_1 + (e_2 \wedge e_3)I = e_1 - e_1 = 0.
    \]
    Indeed, this $f$ is none other than the complex function $f(z)=z$ with $B$ taking the role of the imaginary unit $i$. 

    Let $x=x^1e_1 + x^2e_2 + x^3e_3$.  Then, 
    \[
        B (x\cdot B) = (e_1e_2)( x^1e_2 -x^2 e_1 ) = x^1 e_1 + x^2 e_2.
    \] 
    Thus, for $\xi \in S^2$, we have $f(\xi)=\xi^1 +\xi^2 B$.
\end{example}

\todo[inline]{If we consider now every $\omega$-axial monogenic can be written as a power series, if we can construct $z$ we should be done...?}

It is clear that we can define a monogenic field $f=u+b$ via the Cauchy integral, but we then require $\nabla_\omega f = 0$.  Let $f=\cauchy[\varphi](x)$, then we must have
\[
\nabla_\omega \proj{0}{\cauchy[\varphi](x)} = 0 \qquad \textrm{and} \qquad \nabla_\omega \proj{2}{\cauchy[\varphi](x)}=0.
\]
The first condition yields
\[
0 = \int_\Sigma \frac{(\nu(\zeta)\cdot x) (\omega \cdot x)}{|x-\zeta|^2} \phi(\zeta) d\Sigma(\zeta).
\]


\begin{theorem}
    For any $\omega \in Gr(1,3)$ we have that $\algebra{\omega}\subset \monogenics$. 
\end{theorem}
\begin{proof}
    \textcolor{red}{This seems to be saying that we need boundary values in some hardy space or something. They defined this conjugacy thing as $G$.}
    Fix a unit vector $\omega$.  We want to show that for any $f=u+b\in \algebra{\omega}$ that $\iota^* u=\phi$ satisfies \ref{eq:conjugacy_requirement}.  That is,
    \[
        G\phi = (\Lambda - d\Lambda^{-1}d) \phi = 0.
    \]
    Note that $\phi$ is the trace of a harmonic function, so this operator is well defined.  Note that the equation
    \[
        \Lambda \psi = d \phi
    \]
    has a solution
\end{proof}

\section{Radon transform and integral geometry}

I feel like there is some way to go from projection onto subspaces as a map to grassmannians and reconstructing the manifold.  It's like a morse function type of thing.  Radon transforms also come to mind.

\section{Relation to the BC Method}

\textcolor{red}{Describe how this process can lead to the BC method in dimension $n=2$}


\section{Conclusion}


\appendix
\section{Appendix}

\todo[inline]{Put axial condition for cauchy integral and some other quick proofs in here.}

\subsection{Spin fibration}
maybe pose this as a question in relation to using the 2d belishev stuff.

\textcolor{red}{The inner product for characters is what you use for fourier theory, maybe we can do something here with characters as maps to the grassmannian? Do these form some kind of orthogonal basis? Also, the Dirac operator and Laplacian are spin invariant! This is what they use the $\mathbb{H}$ module structure for!}

A main question to answer now is how the $B$-planar algebras $\algebra{B}$ relate to the space of monogenic functions $\monogenics$.  In particular, this question seems analogous to the invertibility of a $2$-plane x-ray transform.  Let $f$ be a monogenic, can $f$ be generated by $B$-planar monogenics? Noting that each unit 2-blade corresponds to a unique 2-plane in $\R^n$, we can realize every $B$ as a point in $\Grassmannian{2}{n}$.  Letting $f_B$ be some $B$-planar axial monogenic, is
\[
f = \int_{B \in \Grassmannian{2}{n}} a(B) f_B d \lambda,
\]
where $a(B)$ is a scalar function on $\Grassmannian{2}{n}$ and $d\lambda$ is the Haar measure on $\Grassmannian{2}{n}$ monogenic? Moreover, can any monogenic $f$ be constructed in this manner? First, we start with a lemma describing the form of $f_B$.


\begin{lemma}
    Let $f$ be a monogenic (0+2)-vector and define $f_B \coloneqq \projection{B}{f(\projection{B}{x})}$. Then $f_B$ is $B$-planar and monogenic.  
\end{lemma}
\begin{proof}
    It is clear by definition that $f_B$ is constant along translations of the $B$-plane and can be written as $u_B+\beta b_B$ and so $f_B$ is $B$-planar.  To see $f_B$ is monogenic, let $e_1,\dots,e_n$ be a basis such that $B=e_1e_2$ and $e_i \cdot B = 0$ for $i\neq 1,2$. Then note $\nabla_{e_i} f_B =0$ when $i\neq 1,2$ as well leading to
    \[
        \grad f_B = e^1 \nabla_{e_1}f_B + e^2 \nabla_{e_2}f_B
    \]
    Recall that $f=u+b$ must satisfy
    \[
        \grad \wedge u = \grad \cdot b \qquad \textrm{and} \qquad \grad \wedge b = 0.
    \]
    Specifically,
    \[
        e^1 \wedge \nabla_{e_1} u + e^2 \wedge \nabla_{e_2}u + \cdots + e^n \wedge \nabla_{e_n} = e^1 \cdot \nabla_{e_1} b + e^2 \cdot \nabla_{e_2}b + \cdots + e^n \cdot \nabla_{e_n} b
    \]
    Clearly, $\grad \wedge b_B = 0$, thus we need only show
    \[
        \grad \wedge u_B = \grad \cdot b_B.
    \]
    In particular
\end{proof}


We can note that the $B$-planar monogenics are given by a power series $\sum_{n=0}^\infty a_n (x+yB)^n$ due to the isomorphism of algebras $\mathfrak{spin}(2)\cong \C$ \textcolor{red}{This shouldn't be hard to show without appealing to this isomorphism.} In particular, any $B$-planar monogenic is approximated arbitrarily closely by a homogeneous polynomial of degree $n$ in the variables $x$ and $y$. Moreover, $1$ and $x+yB$ generate the $B$-planar monogenics. $\spingroup$ then acts on $B$. \textcolor{red}{Okay, well maybe there's some nice way to talk about characters as mappings to the grassmannian instead of the circle? Should read more about characters and maybe they are really maps to spin group? They are for the 2d case. Structure space and stuff. Should probably rename some of these things I have.s}


\textcolor{red}{Countable basis for $\monogenics$ ?}