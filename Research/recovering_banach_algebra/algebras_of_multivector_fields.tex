\subsection{Banach algebras of Clifford fields}

\todo[inline]{Finish this section. I'm saying this here but it should go later on, but this should lead to the weak formulation for the laplace equation??? Does there exist an inner product instead of just a norm? }

Letting $\Omega$ be a region in $\R^n$, $\Gamma(\Omega)$ has a norm induced from the spinor norm in the $L_2$ sense by
\[
\spinnorm{s} = \int_\Omega ss^\dagger d\Omega
\]
This gives us a normed algebra of Clifford fields. One can see that we have the unit $1$ in this algebra. We also have for multivectors $s,r \in \Gamma(\Omega)$ (constant Clifford fields)
\[
\|sr\| = \|s\|\|r\|
\]
since
\[
\|sr\|^2 = (sr)(sr)^\dagger = srr^\dagger s^\dagger = s\|rr^\dagger\|^2 s^\dagger = \|s\|^2 \|r\|^2.
\]
It follows for non-constant $C^\infty$-fields $s$ and $r$
\[
\spinnorm{sr} \leq \spinnorm{s}\spinnorm{r}.
\]
This shows the algebra is uniform. Identifying the constant fields in the algebra $\Gamma(\Omega)$ with $\R^{2^n}$ we see that the algebra is also complete.


\subsection{Axial monogenic fields}
\todo[inline]{What fields do we care about that are Clifford fields (invertible).  Biparavectors? Axial biparavectors? Rephrase in terms of hilbert transform? Copy stuff from other paper here.}

For this section, let $n=\dim(M)=3$. Supposing that $\phi$ satisfies \ref{eq:conjugate_requirement} (\textcolor{red}{I dropped this requirement for now}) one can generate paravectors $f=u+b$ and define the space of \emph{monogenic paravectors}
\begin{align*}
\monogenics &= \{ f ~ \vert ~ ~\grad f=0\}\\
\end{align*}
The original requirement that $\Delta u^\phi =0$ is obtained since $f$ is monogenic. We can then generate an algebra from this set by
\[
\algebra{} = \{ fg ~\vert~ f,g \in \monogenics\},
\]
but, as mentioned in \cite{belishev_algebras_2019}, this algebra generated by these monogenic fields in $\monogenics$ produce fields that are not monogenic.  Indeed, this is a well known fact in Clifford analysis mentioned in \cite{schepper_introductory_nodate}.  Fundamentally, however, this fact that the product of monogenics is no longer monogenics makes the direct approach in \cite{belishev_calderon_2003} intractable. This issue comes down to the lack of commutivity of paravectors in dimensions higher than $2$.  However, for certain so-called axial fields, commutivity is regained. In fact, the construction of these fields was done in \cite{belishev_algebras_2017} in order to create a closed commutative algebra of monogenic fields. These axial fields will relate directly to complex holomorphic functions.

In \cite{belishev_algebras_2017, belishev_algebras_2019}, the definition of axial is defined for quaternion fields and the properties are discussed.  It is evident from the Example \ref{ex:quaternions} that quaternion fields are analogous to paravector fields via the given identification.  This identification is key in connecting the relevant algebras to the DN map. So we proceed by following the definitions in place.  

\begin{definition}
    Let $F=U+B$ be a paravector and let $\omega$ be a unit vector.  We then say that $F$ is \emph{$\omega$-axial} if $\nabla_\omega F = 0$.  
\end{definition}

\todo[inline]{Make sure I define the covariant derivative and stuff}

From the grade preserving nature of $\nabla$, we see that the requirement $\nabla_\omega f=0$ reduces to a grade-wise requirement
\[
\nabla_\omega U = 0 \qquad \textrm{and} \qquad \nabla_\omega B = 0.
\]
Thus, we can write $B=\beta \omega I = \beta B$ for a smooth scalar field $\beta$ satisfying $\nabla_\omega \beta =0$. So long as $\omega$-axial monogenics are closed under multiplication, we can recover a sub-algebra of holomorphic functions inside of the larger algebra $\monogenics$ generated by monogenic paravectors. If we take two $\omega$-axial monogenic fields $f=u_f + \beta_f B$ and $g=u_g + \beta_g B$, then we have
\begin{equation}
\label{eq:axial_multiplication}
fg = u_f u_g - \beta_f \beta_g + B (u_f b_g + u_g b_f).
\end{equation}
Namely, this follows from the fact that
\[
B^2 = (\omega I)^2 = -1.
\]
This fact is essential. In essence, we now have a direct representation of a holomorphic function if we let $i=B$.  One should then realize that an $\omega$-axial monogenic $f$ is built by translating a holomorphic function along the direction defined by $\omega$ since $f$ has no dependence on this direction. Moreover, it is clear that $B$ is a 2-blade.  Note that for some unit vectors $r$ and $p$, we have $\omega = r \times p$.  Thus, $B =  (r \times p)I^{-1}$.  Indeed, this fits with the interpretation above in that $B$ is acting as a pseudoscalar in some manner.  To say this fully, $B$ is the pseudoscalar for the plane spanned by $r$ and $p$. Another way of rephrasing $f$ being $\omega$-axial is then to say that $f$ is constant on all translations of the $r p$-plane. In this case, $f$ depends solely on two variables and is exactly a holomorphic function. This is simply dual to the notion of being constant along straight lines in a 3-dimensional space.  One can think of $\omega$ as a member of the Grassmanian $Gr(1,3)$ whereas its dual $B=\omega I$ lies in $Gr(2,3)$ which is isomorphic. Indeed, $I$ gives a natural isomorphism between $Gr(1,3)$ and $Gr(2,3)$.

If $f$ is an $\omega$-axial monogenic, then we can recall the Cauchy-Riemann equations yield
\begin{equation}
\label{eq:axial_cauchy_riemann}
\grad u = (\omega \wedge \grad \beta)I \qquad \textrm{and} \qquad - B \wedge \grad \beta B = 0.
\end{equation}

On this plane given by the blade $B$, we want to realize $B$ acting as $i$ for a holomorphic function. In particular, this means we need the Dirac operator to respect multiplication by constant paravectors (which is analogous to scaling complex functions by a complex number). If one has an $\omega$-axial monogenic $f$, we wish that for a constant paravector $k=k_1 + k_2 B$ that $\grad (kf)=0$ as well. $\grad$ is clearly $\R$-linear, so it sufficies to show the following.

\begin{lemma}
    \label{lem:mult_by_i_monogenic}
    Let $f=u+\beta B$ be an $\omega$-axial monogenic paravector, then $B f$ is $\omega$-axial and monogenic.
\end{lemma}
\begin{proof}~
    
\todo[inline]{I can use equations 82 from Chisolm to avoid the use of the cross product}
    It is clear that $B f$ is $\omega$-axial due to the grade preserving linearity of the covariant derivative.
    
    To see that $B f$ is monogenic, we take $B  f = B  u - \beta$.  Then,
    \begin{align*}
    \grad (Bf) = \grad (B u) - \grad \beta,
    \end{align*}
    where we have the graded components
    \begin{align*}
        \proj{1}{\grad (B f)} &= (\grad \cdot Bu)  - \grad \beta\\
        \proj{3}{\grad (Bf)} &= (\grad \wedge B u).
    \end{align*}
    Note that
    \begin{align*}
    \grad \cdot (Bu) =  -\omega \times (\grad \wedge u)  =- \omega \times (\omega \times \grad\beta) = -\omega (\nabla_\omega \beta)+\grad \beta = \grad \beta
    \end{align*}
    by \ref{eq:axial_cauchy_riemann} and thus $\proj{1}{\grad (Bf)}=0$. 
    
    For the grade-3 component,
    \begin{align*}
        \grad \wedge (B u) &= \omega \cdot  (\grad \wedge B)II^{-1}u = I^{-1} \nabla_\omega u=0
    \end{align*}
    since $u$ is $\omega$-axial. Thus we have $\grad(B f)=0$ is monogenic.
\end{proof}

The point here is that we have now effectively found functions that can be scaled by $\alpha + \beta B$ and remain monogenic.  This is the constant multiple rule for the Wirtinger derivative for complex functions. Generically, if I take some multivector $A$ times a monogenic field $f$, $Af$ need not be monogenic.

\begin{proposition}
    Let $f$ and $g$ be monogenic and $\omega$-axial. Then $fg=gf$, $fg$ is $\omega$-axial, and $fg$ is monogenic.
\end{proposition}
\begin{proof}
\todo[inline]{Clean this up with better notation}
    \begin{itemize}
        \item First, it is clear that $fg=gf$ by Equation \ref{eq:axial_multiplication}.
        \item The product $fg$ is $\omega$-axial simply by the product rule of the multivector covariant derivative. That is,
        \[
            \nabla_\omega (fg) = (\nabla_\omega f)g + f(\nabla_\omega g) =0.
        \]

    \item 



To see that the product is monogenic, we have
    \[
        \grad(fg) = \grad(u_fu_g - b_f b_g +  B(u_f b_g + u_g b_f)).
    \]
    Then the grade-1 components are
    \[
        \proj{1}{\grad(fg)}=\grad \wedge (u_f u_g - b_f b_g) + \grad \cdot B(u_f b_g + u_g b_f),
    \]
    and the grade-3 components are
    \[
        \proj{3}{\grad(fg)} = \grad \wedge B (u_f b_g + u_g b_f).
    \]
    For the grade-1 components, we have
    \begin{align*}
        \grad(u_f u_g - b_f b_g) &= (\grad u_f) u_g + u_f (\grad u_g) - (\grad b_f) b_g - b_f (\grad b_g)\\
        \grad \cdot I\omega(u_f b_g + u_g b_f) &= (\grad \cdot I\omega u_f) b_g + u_f (\grad \cdot B b_g) + b_f(\grad \cdot B u_g) + (\grad \cdot B b_f) u_g,
    \end{align*}
    and since $f$ and $g$ are both monogenic we have
    \begin{align*}
        \proj{1}{\grad(fg)} &= (\grad \cdot B u_f - \grad  b_f)b_g + (\grad \cdot B) u_g - \grad  b_g)b_f.
    \end{align*}
    Then, note that 
    \[
        \proj{1}{\grad Bf} = \grad \cdot B u_f - \grad b_f=0
    \]
    by Lemma \ref{lem:mult_by_i_monogenic} and likewise for $\proj{1}{\grad Bg}$. Thus,
    \[
        \proj{1}{\grad(fg)}=0.
    \]
    Likewise, for the grade-3 component of the gradient 
    \begin{align*}
        \proj{3}{\grad(fg)} &= I^{-1} \nabla_\omega (u_f b_g + u_g b_f)=0,
    \end{align*}
    by the product rule for the covariant derivative and the fact that $f$ and $g$ are $\omega$-axial.
\end{itemize}
\end{proof}

\todo[inline]{Add in power series stuff here.  We can write $f=u+ib$ as a power series of $x+yB$?}


\textcolor{red}{As we move through the different axial vectors, it's as if we're doing some tomography on 2d slices of the domain.}

\todo[inline]{Now describe how to do the rest of the algebra stuff here.} 


\begin{theorem}
(2D Gelfand) For any $\mu \in \mathcal{M}$ there is a point $z^\mu \in D$ such that $\mu = \delta_{z^\mu}$. The map
\[
\gamma \colon \mathcal{M} \to D, \quad \mu \mapsto z^\mu
\]
is a homemorphism so that $\mathcal{M} \cong D$. The Gelfand transform
\[
\Gamma \colon \mathcal{A}(D) \to C^\C (\mathcal{M}), \quad (\Gamma f)(\mu) = \mu(f), \quad \mu \in \mathcal{M}
\]
is an isometric isomorphism onto its image, so that $\mathcal{A}(D)\cong \Gamma(\mathcal{A}(D))$.
\end{theorem}


\textcolor{red}{In local coordinates the following definition works...}

\begin{definition}
    Let $B$ be a unit 2-blade then we say that a (0+2)-vector $f_B$ is $B$-planar if $f_B = \projection{B}{} \circ f_B \circ \projection{B}{}$ for all $x$.
\end{definition} 


\todo[inline]{I need to mention that an $\omega$-axial field is a scalar + a scalar times $\omega$ as well. Rewrite this proof.}
\begin{proposition}
    In $\R^3$, if $\omega I = B$, then $B$-planar is in correspondence with a $\omega$-axial quaternion field $h = \alpha + \psi \omega$. 
\end{proposition}
\begin{proof}
    Let $f$ be $\omega$-axial so that $\nabla_\omega f =0$ for some unit vector $\omega$. In particular,
    \[
        \nabla_\omega f = 0 \quad \iff \quad f(x + t\omega) = f(x),
    \]
    for any $t\in \R$. Letting $B=\omega I$, we have
    \begin{align*}
        x+t\omega=(x+t\omega)BB^{-1}&=x\rfloor B B^{-1} + x\wedge BB^{-1} + t\omega \rfloor BB^{-1} + t\omega \wedge B B^{-1}\\
        &= (x\rfloor B) \rfloor B^{-1} + (x\cdot \omega)\omega + (t\omega \cdot \omega)\omega\\
        &= (x\rfloor B)\rfloor B^{-1} + 
(x+t\omega)\cdot \omega \omega.
    \end{align*}
    Since $f$ is $\omega$-axial
    \[
        f(x)=f(x+t\omega)=f((x\rfloor B)\rfloor B^{-1} + (x+t\omega)\cdot \omega \omega)=f((x\rfloor B)\rfloor B^{-1}),
    \]
    and so $f$ is also $B$-planar and the proof is complete.
\end{proof}

\textcolor{red}{Discuss why we need $B$-planar in higher dimensions and also mention that we need $B$ to be an invertible bivector. All blades are invertible?}



\subsection{Spinor spectrum}

This story no longer continues in higher dimensions and one can find the two and three dimensional cases to be happy accidents.  Instead, now we must deal fully with the situation at hand to dissect the relevant algebras. In this vein, we can generate a special algebra $\algebra{B}$ of $B$-planar monogenic spinors from the $B$-planar monogenic $(0+2)$-vectors.  The question is then for all does
\[
\overline{\bigoplus_{B \in \Grassmannian{2}{n}} \algebra{B} } = \monogenics.
\]

Letting $\ball$ be the unit ball in $\R^n$ and $\disk$ be the unit disk in $\C \cong \R^2$.  By Gelfand, the maximal ideal space of $\algebra{B}$ is homeomorphic to the disk given the isomorphism mapping the blade $B \mapsto i$ in the complex plane.  The space $\monogenics$ is no longer an algebra, so we are at a loss to determine maximal ideals.  However, we can describe functionals on the monogenics.

\begin{definition}
    Define the \emph{spinor dual} $\dualmonogenics$ as
    \[
        \dualmonogenics \coloneqq \{ l \in \mathcal{L}(\monogenics; \spinalgebra) ~\vert~ l(sf) = sl(f), ~\forall f \in \monogenics, ~s \in \spinalgebra \}
    \]
\end{definition}
$\dualmonogenics$ are the spinor valued functionals or \emph{spin functionals}. Similarly, we have the definition for the spinor functionals that are multiplicative on the $B$-planar monogenics.
\begin{definition}
    The \emph{spinor spectrum} is the set
    \[
        \characters \coloneqq \{ \mu \in \dualmonogenics ~\vert~ \mu(fg) = \mu(f)\mu(g),~ \forall f,g \in \algebra{B}, ~ B \in \Grassmannian{2}{n}\},
    \]
    and we refer to the elements as \emph{spin characters}.
\end{definition}
The elements in the spinor spectrum are simply algebra homomorphisms from $\algebra{B}$ to $\spinalgebra$. In the 2-dimensional case, there is only one unique choice of $B$ and $\mathfrak{spin}(2)$ is isomorphic to $\C$.  We realize this as only a special case of a more general notion of a spin character.

\textcolor{red}{Describe the weak-$\ast$ topology here too.} 
