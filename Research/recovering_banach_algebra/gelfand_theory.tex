\subsection{Topology from monogenics}

For the remainder of this section, we will consider the domain of interest to be the unit $n$-ball, $\ball$. We seek to determine that the space $\characters(\ball)$ is homeomorphic to $\ball$.  Thinking of the Calder\'on problem, we may only have access to functions defined on $\ball$ and not the whole of $\ball$ itself.  If one can recover the spin characters $\characters(\ball)$, we can utilize the following result.

\begin{theorem}
For any $\mu \in \characters(\ball)$, there is a point $x^\mu \in \ball$ such that $\mu(f) = f(x_\mu)$ for any $f\in \monogenics(\ball)$. Given the weak-$\ast$ topology on $\characters(\ball)$, the map
\[
\gamma \colon \characters(\ball) \to \ball, \quad \mu \mapsto x^\mu
\]
is a homeomorphism. The Gelfand transform 
\[
\widehat{~} \colon \monogenics(\ball) \to C(\characters(\ball); \Gamma^+), \quad \widehat{f}(\mu) \coloneqq \mu(f), \quad \mu \in \characters(\ball),
\]
is an isometry onto its image, so that $\characters(\ball)$ is isomorphic to $\widehat{\monogenics(\ball)}$ as algebras.
\end{theorem}

We prove this theorem in two main parts. First, we can realize a power series representation for elements in $\monogenics(\ball)$. This power series is constructed using specific $B$-planar monogenic fields. Finally, we constructively show a correspondence between $\mu \in \characters(\ball)$ with $x^\mu \in \ball$. 

\subsubsection{Power series}

One beautiful result in Clifford analysis is the celebrated generalization of the Cauchy integral formula for $\C$-holomorphic functions. Details of the Cauchy integral formula and Hilbert transform for multivector fields can be found in \cite{brackx_hilbert_2008}. We have the fundamental solution to $\grad$ is a vector field given by
\[
E(x) = \frac{1}{a_m} \frac{x}{|x|^m},
\]
for $x\in \R^n$. That is to say that $\grad E(x) = \delta(x)$. For any region $\Omega \subset \R^n$ with boundary $\Sigma$, we define the \emph{Cauchy kernel} for $x\in \R^n$ and $y \in \Sigma$ using the fundamental solution $E$ as
\[
C(y, x) = -\frac{1}{a_n} \nu(x_0) E(x-y),
\]
where $a_n$ is the surface area of the $n$-ball and $\nu(x_0)$ is the outward normal at $x_0$. The \emph{Cauchy integral} for $\phi \in L_2(\Sigma)$ is then
\[
\cauchy[\phi](x) = \frac{1}{a_n} \int_{\Sigma} \frac{x_0-x}{|x-x_0|^n} \nu(x_0) \phi(x_0) d\Sigma(x_0).
\]
The Cauchy integral is indeed a monogenic function and note that for a scalar $\phi$ we have $\cauchy[\phi] \in \monogenics(\Omega)$ since it must be a parabivector as well.

\todo[inline]{do for an arbitrary basis for the remainder and fix all the notation up}
Fix a basis $e_1,\dots,e_n$ in $\R^n$ and we can define the functions $z_j^i = x^j - x^i e_i e_j$. Recall that for an orthonormal basis the reciprocal basis elements $e^i=e_i$. To further condense notation, we let $B_{ij}=e_i e_j$ be the 2-blade acting as the pseudoscalar for the $e_i e_j$-plane and likewise put $B_j^i = e^ie_j$ and $B^{ij}=e^i e^j$ as necessary. In the same vein, the functions $z_j^i$ are very analogous to $z$ in $\C$ but rather in the $B_j^i$ plane.  One can note
\[
z_j^i = x^j - x^i B_j^i = e_j\projection{B_{ij}}{x}.
\]
It is worth noting that the $z_j^i$ are monogenic and are $B_j^i$-planar. \todo[inline]{Show this, or find where I show it later.}

For sake of simplicity, we let $e_1,\dots, e_n$ be an arbitrary basis for $\R^n$.  
\begin{itemize}
    \item Consider the function $z_{B_\sigma(j)}(x)=x_{\sigma(j)} - x_1 e^1 e_{\sigma(j)}$ for $\sigma \in \{2,\dots,n\}$ a permutation.  Note that $z_{B_\sigma(j)}$ is $B_{\sigma(j)}$-planar with $B_{\sigma(j)}=e^1 e_{\sigma(j)}$.  Moreover, $z_{B_\sigma(j)}$ is monogenic as
    \[
        \grad z_{B_\sigma(j)} = e_{\sigma(j)} - e_1 e^1 e_{\sigma(j)} = 0.
    \]
    We denote as well $B_{\sigma(j)} = e^1 e_{\sigma(j)}$.
    \item Let $f \in \monogenics$.  Then by Theorem 4 in \cite{ryan_left_1986}, we have the monogenic polynomials
    \[
        P_{j_2 \dots j_n}(x) = \frac{1}{j!} \sum_{\textrm{permutations}}z_{B_\sigma(1)}(x) \cdots z_{B_\sigma(j)}(x),
    \]
    which generate $f$ as a power series as
    \[
        f(x) = \sum_{j=0}^\infty \left(\sum_{{j_2 \cdots j_n}_{j_2 + \cdots j_n = j}} P_{j_2 \cdots j_n} (x) a_{j_2 \cdots j_n}\right),
    \]
    where
    \[
        a_{j_2 \cdots j_n} = \frac{1}{\omega_n} \int_{\partial \Sigma} \nabla_{e_2}^{j_2} \cdots \nabla_{e_n}^{j_n} G(y) \nu(y) f(y) d \Sigma(y),
    \]
    where $G(y)$ is the Cauchy kernel. 
\todo[inline]{Relate this to the Cauchy kernel I defined.}
\end{itemize}




\subsubsection{Correspondence}

The functions $z_j^i$ play a crucial role in the above power series representation but they also play a key part in determining the behavior of the spin characters $\mu \in \characters$.  If we are able to deduce the action $\mu(z_j^i)$, then we can extend this to any monogenic $f$ via the power series representation. Note that for any $\mu \in \characters$ that $\mathbb{A}_B=\mu(\algebra{B})$ is a commutative subalgebra of $\mathfrak{spin}(n)$.  In particular, for a constant $c \in \algebra{B}$, $\mu(c)=c$ and so we retrieve $\mathbb{A}_B$ must be generated by scalars and the bivector $B$.  Thus, $\mathbb{A}_B$ is an isomorphic copy of $\mathfrak{spin}(2)$ as the algebra of the $B$-plane. \textcolor{red}{Note that $\mu$ will be constant on $2$-blades.}

Working with the same orthonormal basis and applying $\mu$ yields
\[
\mu(z_j^i) = a_j^i + b_j^i B_{ij},
\]
for some constants $a_j^i$ and $b_j^i$.  The $z_j^i$ are not independent from one another.  In fact, we have two key relationships in that
\begin{equation}
\label{eq:z_reciprocal_relationship}
z_j^i B_{ij}  = z_i^j.
\end{equation}
Similarly, we have
\begin{equation}
\label{eq:z_relationship}
z_j^i = B_{jk} z_j^k B_{kj} - z_i^k B_{ij}.
\end{equation}
We simply compute the above,
\begin{align*}
B_{jk} z_j^k B_{kj} - z_i^k B_{ij} &= B_{jk} (x^j-x^k B_{kj}) B_{kj} - (x^i -x^k B_{ki}) B_{ij}\\
    &= x^j -x^k B_{kj} -x^i B_{ij} +x^k B_{kj}\\
    &= z_j^i.
\end{align*}

Thus, we can take $\mu$ of Equations \ref{eq:z_reciprocal_relationship} and \ref{eq:z_relationship}. First, 
\[
\mu(z_i^j) = \mu(z_j^i B_{ij}) = \mu(z_j^i) B_{ij}
\]
yields
\[
a_i^j - b_i^j B_{ij} = -b_j^i + a_j^i B_{ij}
\]
and so $a_i^j = b_j^i$ for all $i \neq j$. Next,
\[
\mu(z_j^i) = \mu(B_{jk} z_j^k B_{kj} - z_i^k B_{ij}) = B_{jk} \mu{z_j^k} B_{kj} - \mu(z_i^k) B_{ij}
\]
and so
\[
a_j^i + b_j^i B_{ij} = B_{jk} (a_j^k +b_j^k B_{kj}) B_{kj} - (a_i^k + b_i^k B_{ki}) B_{ij} = a_j^k - a_i^k B_{ij}
\]
yields $a_j^i = a_j^k$ and $a_i^k=-b_j^i$. In particular, for all $z_j^i$, we have the relationships
\[
a_i^j =- b_j^i, \quad a_j^i = a_j^k, \quad a_i^k =- b_j^i, \quad \textrm{for $i\neq j \neq k$.}
\]
More simply, we can note 
\[
a_i^\bullet = - b_\bullet^i ~\forall i \qquad \textrm{and} \qquad  a_j^\bullet = a_j^\bullet ~\forall j.
\]
Letting $\mu(z_j^i) = z_j^i(x_\mu)$ satisfies these requirements above since $z_j^i(x_\mu) = x_\mu^j - x_\mu^i B_{ij}$ for all $i\neq j$. 
\textcolor{red}{Make the matrix argument and stuff then show that the point itself must also be in $\ball$.}

\todo[inline]{Finish this and note that this proves the theorem. Motivate the next section.}


