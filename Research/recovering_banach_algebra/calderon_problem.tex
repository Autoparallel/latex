Let $u^\phi \in \Omega^0(M)$ be a smooth 0-form (scalar function) that is a solution to the following Dirichlet boundary value problem
\begin{equation}
\label{eq:dirichlet_problem}
\begin{cases} \Delta u^\phi = 0 & \textrm{ in $M$} \\  \iota^*( u) = \phi. \end{cases},
\end{equation}
where $\Delta$ refers to the Laplace-Beltrami operator on differential forms. For the Calder\'on problem, the manifold $M$ and metric $g$ are unknown and one seeks to determine as much as possible about $(M,g)$ from measurements on the boundary.  Due to the relationship between the EIT and Calder\'on problem, we use the notation $\phi$ for the Dirichlet boundary values since $\phi$ should be thought of as the prescribed voltage along the boundary. 

For any given solution to the boundary value problem, there is the corresponding Neumann data $E=\iota^*(\star d u)$.  As with $\phi$, the notation $E$ is used as the Neumann data measured in the EIT problem corresponds to the electric field flux at the boundary. One attains the current $J$ by multiplying with $E$ by the boundary conductivity matrix. The set of both boundary conditions $(\phi, E)$ is the \emph{Cauchy data} and the \emph{Dirichlet-to-Neumann (DN) map} $\Lambda$ is defined such that $\Lambda \phi = E$ and in particular this yields $\iota^*(\star d u^\phi)= E$. Note that this map $\Lambda$ is often referred to as the \emph{scalar} DN map as $\Lambda \colon \Omega^0(\partial M) \to \Omega^{n-1}(\partial M)$ inputs a scalar Dirichlet condition. An extension of the DN map to forms can be found in \cite{belishev_dirichlet_2008,sharafutdinov_complete_2013}. The Calder\'on problem for Riemannian manifolds is then to recover the pair $(M,g)$ up to isometry from complete knowledge of the DN map $\Lambda$. 

Denote by $\harmonicfunctions = \{u \in \Omega^0(M) ~\vert~ du=0\}$ the space of harmonic 0-forms on $M$.  From the DN map, one can define the \emph{Hilbert transform} $T\colon \iota^* \harmonicfunctions \to \iota^* \harmonicfunctions$.  This function acts on traces of harmonic forms by
\[
T \phi  = d\Lambda^{-1} \phi,
\]
and is defined in \cite{belishev_dirichlet_2008}. The authors show benefit to defining the Hilbert transform as it provides the ability to generate so called conjugate forms.  When the condition
\begin{equation}
\label{eq:conjugate_requirement}
\left( \Lambda + (-1)^{n}d\Lambda^{-1}d\right)\phi = 0, 
\end{equation}
is met, then there exists a \emph{conjugate form} $\epsilon^\psi \in \Omega^{n-2}(M)$ with boundary trace $\psi = \iota^* \epsilon$ satisfying $Td\phi = d \psi$. As well, $\epsilon$ is also coclosed in that $\delta \epsilon=0$. 

Now, there exists a 2-form $b^\psi$ such that $\star b^\psi = \epsilon$.  Using the isomorphism between forms and multivectors, we can let $U$ be the scalar field corresponding to $u^\phi$ and we can let $B$ be the bivector field corresponding to $b^\psi$.  We can add these to yield the paravector $F=U+B \in \multivectorfields$.   Recall that a multivector field is monogenic if $\grad F=0$.  Applying this to the paravector $F$ yields the equations
\[
\grad \wedge U = -\grad \cdot B \qquad \textrm{and} \qquad \grad \wedge B = 0.
\]
The conjugacy relation $du^\phi = \star d \epsilon^\psi$ is equivalent to having the multivector $F$ be monogenic.

\begin{lemma}
Given the forms $u^\phi$ and $b^\psi$ conjugate as above, the corresponding paravector field
\[
F = U + B
\]
is monogenic.
\end{lemma}
\begin{proof}
Let $\star b^\psi = \epsilon$ and note that 
\[
d u = \star d \epsilon = \star d \star b^\psi.  
\]
Now, writing the multivector equivalent of the right hand side yields
\begin{align*}
(\grad \wedge B^\star )^\star &= [(\grad \cdot B^\dagger) I]^\star\\
    &= [I^{-1} ((\grad \cdot B^\dagger) I)]^\dagger\\
    &= ((\grad \cdot B^\dagger)I)^\dagger I\\
    &= \grad \cdot B^\dagger && \textrm{since $\dagger$ of a vector is trivial}\\
    &= -\grad \cdot B. && \textrm{since $\dagger$ of a bivector is -1}
\end{align*}
Thus, we have $\grad \wedge U + \grad \cdot B = 0$. Since $\epsilon$ is coclosed we have
\begin{align*}
0=\grad \cdot B^\star &= \grad \cdot (I^{-1} B)^\dagger \\
    &= \grad \cdot (B^\dagger I)\\
    &= (\grad \wedge B^\dagger) I\\
    &= \grad \wedge B.
\end{align*}
Thus $\grad F =0$ and $F$ is monogenic.
\end{proof}

\subsection{Calderon problem in geometric calculus}

Indeed, the above work invites one to rephrase the problem in terms of geometric calculus.  Instead, the classical problem is given as follows.

\begin{question}
Let $M$ be an unknown Riemannian manifold with unknown metric $g$ and with known boundary $\Sigma$.  Let $u^\phi \in \multivectorfields$ be a scalar field satisfying the Dirichlet problem
\begin{equation}
\label{eq:dirichlet_problem_multivector}
\begin{cases} \Delta u^\phi = 0 & \textrm{ in $M$} \\  u\vert_\Sigma = \phi. \end{cases},
\end{equation}
Define the Dirichlet to Neumann map as
\[
\Lambda u^\phi = \projection{\nu}{\grad u^\phi},
\]
where $\nu$ is the normal to $\Sigma$ given by $I_\Sigma I$.  Can one recover $M$ and $g$ from knowledge of $\Sigma$ and $\Lambda$?
\end{question}

It is a well known fact that the inverse of the DN map is known up to a constant

\subsection{EXTRA STUFF}
\todo[inline]{Add about the 2D problem and generating algebras?}

\todo[inline]{We should start with the boundary algebra and show that we can generate algebras inside. Use the maximum principle. }

For this section, let $n=\dim(M)=3$. Supposing that $\phi$ satisfies \ref{eq:conjugate_requirement} (\textcolor{red}{I dropped this requirement for now}) one can generate paravectors $f=u+b$ and define the space of \emph{monogenic paravectors}
\begin{align*}
\monogenics &= \{ f ~ \vert ~ ~\grad f=0\}\\
\end{align*}
The original requirement that $\Delta u^\phi =0$ is obtained since $f$ is monogenic. We can then generate an algebra from this set by
\[
\algebra{} = \{ fg ~\vert~ f,g \in \monogenics\},
\]
but, as mentioned in \cite{belishev_algebras_2019}, this algebra generated by these monogenic fields in $\monogenics$ produce fields that are not monogenic.  Indeed, this is a well known fact in Clifford analysis mentioned in \cite{schepper_introductory_nodate}.  Fundamentally, however, this fact that the product of monogenics is no longer monogenics makes the direct approach in \cite{belishev_calderon_2003} intractable. This issue comes down to the lack of commutivity of paravectors in dimensions higher than $2$.  However, for certain so-called axial fields, commutivity is regained. In fact, the construction of these fields was done in \cite{belishev_algebras_2017} in order to create a closed commutative algebra of monogenic fields. These axial fields will relate directly to complex holomorphic functions.

In \cite{belishev_algebras_2017, belishev_algebras_2019}, the definition of axial is defined for quaternion fields and the properties are discussed.  It is evident from the Example \ref{ex:quaternions} that quaternion fields are analogous to paravector fields via the given identification.  This identification is key in connecting the relevant algebras to the DN map. So we proceed by following the definitions in place.  

