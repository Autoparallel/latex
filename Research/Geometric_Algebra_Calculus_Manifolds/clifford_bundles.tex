\textcolor{blue}{Can one show that with some $Q$ the Clifford sections on $M$ form a Banach algebra?}

\url{https://en.wikipedia.org/wiki/Clifford_algebra} Look at "basis and dimension" to show that we will work with orthonormal frames.
Let $(M,Q)$ be an $m$-dimensional smooth manifold with a quadratic form $Q$ on the tangent bundle.  By polarisation, this induces 
\[
g(u,v)=\frac{1}{2}(Q(u+v)-Q(u)-Q(v)).
\]
Thus any quadratic manifold is a Riemannian manifold $(M,g)$.  Then we can consider an induced Clifford bundle $C\ell(TM,g)$ on $M$ with the quadratic form $Q(v)=g(v,v)$. Other sources make the choice of $Q(v)=-g(v,v)$ in order to mimic the complex and quaternionic fields.  

Specifically, we define $C\ell(T_pM,g_p)$ to be the Clifford algebra on the tangent space $T_pM$ and let
\[
\clifford(TM,g) \coloneqq \dot{\bigcup}_{p\in M} \clifford (T_pM,g_p)
\]
be the Clifford bundle.

We can then define the space of Clifford sections by noting we have a natural projection
\[
\pi \colon \clifford(TM,g) \to M
\]
that maps a Clifford element to the point at which it is based and putting
\[
\Gamma \clifford(TM,g) \coloneqq \left\{ \sigma \colon M \to C\ell(TM,g) ~\vert~ \pi \circ \sigma = \id_M\right\}.
\]
\textcolor{blue}{How do we define smoothness here? Maybe just consider a section of the tensor algebra.  Then we can show the smoothness and algebra properties from a quotient map from there.}

\begin{definition}
Define a \emph{$k$-blade field} as a smooth section of
\[
\Large{\odot}^k(TM)
\]
\end{definition}

\begin{proposition}
The space $\cliffordsec$ forms an algebra.
\end{proposition}

\begin{proposition}
There exists a norm on $\cliffordsec$ via involution (multiplication by the pseudoscalar).
\end{proposition}

\begin{proposition}
With this norm, $\cliffordsec$ is a Banach algebra.
\end{proposition}

\begin{proposition}
With (blah) we have that $\cliffordsec$ is a Banach *-algebra.
\end{proposition}


We need that the algebra of clifford sections is a Banach *-algebra.
\url{https://en.wikipedia.org/wiki/Gelfand%E2%80%93Naimark_theorem}

\url{https://en.wikipedia.org/wiki/Banach_algebra#Spectral_theory} This ties up the knot for the relationship of different notions of spectra.

Can we define an inner product on the clifford algebra in a natural way? \url{https://math.stackexchange.com/questions/2606319/is-the-natural-norm-on-the-exterior-algebra-submultiplicative} That is, multiply the components of rank $n$ in a meaningful way like in the link.

Relationship to frame bundles and flags?

Two versions of a Clifford norm and mention of $\clifford (p,q,r)$ \url{https://math.stackexchange.com/questions/1128844/about-the-definition-of-norm-in-clifford-algebra}

Also this
\url{https://mathoverflow.net/questions/176140/norms-on-clifford-algebra-c-norm}

I feel like a good Ph.D. project would be to fucking make a single source that has information on clifford algebras on manifolds with all the relations to function algebras and such.

This is cool \url{https://www.jstor.org/stable/pdf/1970397.pdf}

Clifford homology/cohomology/elliptic complexes with dirac operator (derivative)

How do we define this dirac operator in a coordinate free way? Will it be able to give us $\IM \subseteq \KER$? Maybe there is some way to say "yes, up to non-harmonic functions" as $D^2$ is a k-blade laplacian.

This definition is essentially defined throughout \cite{geometric_algebra_physicists}

\begin{definition}
Define
\[
D_L = \sum_{i=1}^n e_i \frac{\partial}{\partial x^i}
\]
and
\[
D_R = \sum_{i=1}^n \frac{\partial}{\partial x^i} e_i
\]
\end{definition}

\begin{definition}
Define the \emph{exterior derivative} $D\wedge$ by
\[
\frac{1}{2}\left(D_L - D_R\right)
\]
and the \emph{interior derivative} $D\cdot$ by
\[
\frac{1}{2}\left(D_L+D_R\right).
\]
\end{definition}

And again, in \cite{geometric_algebra_physicists}, it's shown that
\[
(D\cdot)^2 = 0 = (D\wedge)^2.
\]

\[
\begin{tikzcd}
\dots \arrow[r, bend right] & {\Large\odot}^{k-2}V \arrow[r, bend right] \arrow[l, bend right] & {\Large\odot}^{k-1}V \arrow[r, bend right] \arrow[l, bend right] & {\Large{\odot}}^k V \arrow[r, "D\wedge"', bend right] \arrow[l, "D\cdot"', bend right] \arrow[rr, "(D\wedge)^2"', bend right=60] \arrow[ll, "(D\cdot)^2"', bend right=60] & {\Large\odot}^{k+1}V \arrow[r, bend right] \arrow[l, bend right] & {\Large\odot}^{k+2}V \arrow[r, bend right] \arrow[l, bend right] & \dots \arrow[l, bend right]
\end{tikzcd}
\]

probabilistic (norm one) clifford sections? 

left/right dirac operators

in lectures on clifford algebras - clifford analysis lecture, there is a generalization of cauchy integral formula