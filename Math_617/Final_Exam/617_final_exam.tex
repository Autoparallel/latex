\documentclass[leqno]{article}
\usepackage[utf8]{inputenc}
\usepackage[T1]{fontenc}
\usepackage{amsfonts}
%\usepackage{fourier}
%\usepackage{heuristica}
\usepackage{enumerate}
\author{Colin Roberts}
\title{MATH 617, Final Exam}
\usepackage[left=3cm,right=3cm,top=3cm,bottom=3cm]{geometry}
\usepackage{amsmath}
\usepackage[thmmarks, amsmath, thref]{ntheorem}
%\usepackage{kbordermatrix}
\usepackage{mathtools}
\usepackage{color,xcolor}

\theoremstyle{nonumberplain}
\theoremheaderfont{\itshape}
\theorembodyfont{\upshape:}
\theoremseparator{.}
\theoremsymbol{\ensuremath{\square}}
\newtheorem{proof}{Proof}
\theoremsymbol{\ensuremath{\square}}
\newtheorem{lemma}{Lemma}
\theoremsymbol{\ensuremath{\blacksquare}}
\newtheorem{solution}{Solution}
\theoremseparator{. ---}
\theoremsymbol{\mbox{\texttt{;o)}}}
\newtheorem{varsol}{Solution (variant)}

\newcommand{\tr}{\mathrm{tr}}
\newcommand{\R}{\mathbb{R}}
\newcommand{\N}{\mathbb{N}}
\newcommand{\Sets}{\mathcal{S}}
\newcommand{\Leb}{\mathcal{L}}


\usepackage{amssymb}
\usepackage{graphics}

\textheight=9.0in
\textwidth=6.5in
\oddsidemargin=0in
\topmargin=-0.50in

\pagestyle{empty}


\begin{document}

\begin{center}
  \textsc{\large ColoState ~~ Spring 2018 ~~ MATH 617 ~~ Final.PartII}
\end{center}

\begin{center}
  \textrm{Tue. 05/08/2018}
\end{center}

\vglue 0.10in

\bigskip
\noindent
\textsc{Name:} \underline{Colin Roberts\hglue 1.5in} ~~
\textsc{CSUID:} \underline{829773631\hglue 1.5in}

\vskip 0.10in
\noindent
\textit{% {\bf No} textbook, homework, notes, or any other references should be used.
Work independently.
Please write down all {\bf necessary} steps,
partial credit will be given if deserved. 
}

\large

\bigskip
\bigskip
\noindent
(20 points)
\textit{Problem 3}.
\quad
Let $ E $ be a subset of $ [0,1] $.  
Prove that $ E $ is Lebesgue measurable if and only if 
$$
  \lambda^*(E) = \mbox{sup}\{ \lambda(F): F \;\mbox{is closed and}\; F \subseteq E \}.
$$

\bigskip
\bigskip
\noindent
(20 points)
\textit{Problem 4}.
\quad
Let $ \lambda $ be the Lebesgue measure on the real line.
Consider a Lebesgue measurable subset $ E $ of $ [0,1] $
with the following property: 
$$
  \lambda(E \cap [a,b]) \ge c (b-a)
  \qquad
  \forall [a,b] \subseteq [0,1],
$$
where $ c>0 $ is a constant.  
Show that $ \lambda(E)=1 $.

\bigskip
\bigskip
\noindent
(20 points)
\textit{Problem 5}.
\quad
Prove or provide a counterexample for the following statement:
If $ f $ is absolutely continuous on $ [a,b] $,
$ g $ is continuous on $ [a,b] $,
and $ f'=g $ almost everywhere on $ [a,b] $,
then $ f'=g $ everywhere on $ [a,b] $.

\pagebreak

%%%%%%%%%%%%%%%%%%%%%%%%%%%%%%%%%%%%%%%%%%%%%%%%%%%%%%%%%%%%%%%%%%%%%%%%%%%%%%%%%%%%%%%%%%%%%%%%%%%%%%%%%%%%%%%%%%%%%
%%%%%%%%%%%%%%%%%%%%%%%%%PROBLEM%%%%%%%%%%%%%%%%%%%%%%%%%%%%%%%%%%%%%%%%%%%%%%%%%%%%%%%%%%%%%%%%%%%%%%%%%%%%%%%%%%%%%%%%%%%%%%%%%%%%%%%%%%%%%%%%%%%%%%%%%%%%%%%%%%%%%%%%%%%%%%%%%%%%%%%%%%%%%%%%%%%%%%%%%%%%%%%%%%%%%%%%%%%%%%%%%%%%%%%%%%

\noindent\textbf{Problem 3.} \quad
Let $E$ be a subset of $[0,1]$. Prove that $E$ is Lebesgue measurable if and only if
\[
\lambda^*(E)=\sup\{\lambda(F)~\colon~ F ~\textrm{is closed and}~ F\subseteq E\}.
\]

\noindent\rule[0.5ex]{\linewidth}{1pt}

\begin{proof}
This is an equivalent statement for the outer regularity of the Lebesgue measure.  For the forward direction, we suppose that $E\subseteq [0,1]$ is Lebesgue measurable.  Now, if $E$ is empty, then the statement is vacuously true since the $\emptyset$ contains no subsets. Specifically,
\[
0=\lambda^*(\emptyset)=\lambda^*(E)=\sup\{ \lambda(F) ~\colon~ F ~\textrm{is closed and}~ F\subseteq E\}.
\]
For $E$ nonempty, note that by Theorem 4.2.2. there exists $F_n\subseteq E$ such that $F_n$ is closed and
\[
\lambda^*(E\setminus F_n ) < \frac{1}{n}.
\] 
Note that $E\setminus F_n$ is Lebesgue measurable as well (since $F_n$ is a closed subset of $\R$) which means that
\[
\lambda^*(E\setminus F_n)=\lambda(E\setminus F_n),
\]
since the Lebesgue measurable sets are a $\sigma$-algebra. Then,
\begin{align*}
\lambda^*(E\setminus F_n)=\lambda^*(E\setminus F_n)&<\frac{1}{n}\\
\implies \lambda^*(E)-\frac{1}{n}&< \lambda^*(F_n).
\end{align*}
Taking $n\to \infty$ is equivalent to taking $\sup\{ \lambda(F) ~\colon~ F \textrm{~closed and~} F\subseteq E\}$, and we find that
\[
\lambda^*(E)\leq \sup\{ \lambda(F) ~\colon~ F \textrm{~closed and~} F\subseteq E\}.
\]
To see that $\lambda^*(E) \geq \sup\{ \lambda(F) ~\colon~ F \textrm{~closed and~} F\subseteq E\}$ just note that since for any closed (and hence Lebesgue measurable) $F\subseteq E$ we have that
\begin{align*}
\lambda^*(E)=\lambda(E)&\geq \lambda(F)\\
\implies  \lambda^*(E)&\geq \sup\{ \lambda(F) ~\colon~ F \textrm{~closed and~} F\subseteq E\}.
\end{align*}
Thus we have that 
\[
\lambda^*(E)=\sup\{ \lambda(F) ~\colon~ F \textrm{~closed and~} F\subseteq E\}.
\]

Now, for the converse, we suppose that
\[
\lambda^*(E)=\sup\{ \lambda(F) ~\colon~ F \textrm{~closed and~} F\subseteq E\}
\]
and show that $E$ is Lebesgue measurable. The above statement implies that for any $\epsilon>0$ we have some closed $F_\epsilon\subseteq E$ such that
\[
\lambda^*(E)-\lambda(F_\epsilon)<\epsilon.
\]
Then, since $F_\epsilon$ is Lebesgue measureable and $E$ is Lebesgue outer measurable we have that 
\begin{align*}
\lambda^*(E)-\lambda(F_\epsilon)=\lambda^*(E)-\lambda^*(F_\epsilon)= \lambda^*(E\setminus F_\epsilon)<\epsilon.
\end{align*}
Thus, by Theorem 4.2.2., we have that $E$ must be Lebesgue measurable.


\end{proof}



\pagebreak

%%%%%%%%%%%%%%%%%%%%%%%%%%%%%%%%%%%%%%%%%%%%%%%%%%%%%%%%%%%%%%%%%%%%%%%%%%%%%%%%%%%%%%%%%%%%%%%%%%%%%%%%%%%%%%%%%%%%%
%%%%%%%%%%%%%%%%%%%%%%%%%PROBLEM%%%%%%%%%%%%%%%%%%%%%%%%%%%%%%%%%%%%%%%%%%%%%%%%%%%%%%%%%%%%%%%%%%%%%%%%%%%%%%%%%%%%%%%%%%%%%%%%%%%%%%%%%%%%%%%%%%%%%%%%%%%%%%%%%%%%%%%%%%%%%%%%%%%%%%%%%%%%%%%%%%%%%%%%%%%%%%%%%%%%%%%%%%%%%%%%%%%%%%%%%%


\noindent\textbf{Problem 4.} \quad
Let $\lambda$ be the Lebesgue measure on the real line. Consider a Lebesgue measurable subset $E$ of $[0,1]$ with the following property:
\[
\lambda(E\cap [a,b]) \geq c (b-a) \quad \forall [a,b] \subseteq [0,1],
\]
where $c>0$ is a constant. Show that $\lambda(E)=1$.


\noindent\rule[0.5ex]{\linewidth}{1pt}


\begin{proof}
Define 
\[
f(x)=\int_a^x \chi_E d\lambda(t)
\]
and note that $f$ is differentiable almost everywhere. By our supposition, we then have that
\[
f'(x)=\lim_{h\to 0} \frac{1}{h}\int_x^{x+h} \chi_E d\lambda(t) \geq \lim_{h\to 0} \frac{ch}{h}=c>0
\]
almost everywhere.  Then, since we defined $f$ as the integral of $\chi_E$, $f'(x)=\chi_E(x)$ almost everywhere. Since we also showed $f'(x)>0$ almost everywhere, it must be that $f'(x)=1$ almost everywhere.  In particular, this means that $\chi_E(x)=1$ almost everywhere and so over $[a,b]$ the set in which $\chi_E(x)=0$ must be a null set. Ultimately, this means that $\lambda(E)=1$.
\end{proof}


\pagebreak

%%%%%%%%%%%%%%%%%%%%%%%%%%%%%%%%%%%%%%%%%%%%%%%%%%%%%%%%%%%%%%%%%%%%%%%%%%%%%%%%%%%%%%%%%%%%%%%%%%%%%%%%%%%%%%%%%%%%%
%%%%%%%%%%%%%%%%%%%%%%%%%PROBLEM%%%%%%%%%%%%%%%%%%%%%%%%%%%%%%%%%%%%%%%%%%%%%%%%%%%%%%%%%%%%%%%%%%%%%%%%%%%%%%%%%%%%%%%%%%%%%%%%%%%%%%%%%%%%%%%%%%%%%%%%%%%%%%%%%%%%%%%%%%%%%%%%%%%%%%%%%%%%%%%%%%%%%%%%%%%%%%%%%%%%%%%%%%%%%%%%%%%%%%%%%%

\noindent\textbf{Problem 5.} \quad
Prove or provide a counterexample for the following statement:
If $ f $ is absolutely continuous on $ [a,b] $,
$ g $ is continuous on $ [a,b] $,
and $ f'=g $ almost everywhere on $ [a,b] $,
then $ f'=g $ everywhere on $ [a,b] $.

\noindent\rule[0.5ex]{\linewidth}{1pt}

\begin{proof}
Define
\[
f(x)=\int_a^x g(t)d\lambda(t)
\]
and note that by this definition, $f'(x)=g(x)$ almost everywhere.  Now, since $g$ is continuous, for any $x_0\in (a,b)$ we have that for $h>0$ and some $z\in [x_0,x_0+h]\subseteq (a,b)$ (equivalently a $\tilde{z}\in [x_0-h,x_0]$) that
\begin{align*}
g(z)=\frac{1}{h} \int_{x_0}^{x_0+h} g(t)d\lambda(t),
\end{align*}
and equivalently
\[
g(\tilde{z})=\frac{1}{h}\int_{x_0-h}^{x_0} g(t)d\lambda(t).
\]
(Note that the above fact is critical for showing the remaining calculations.) Then, taking the limit as $h\to 0$ we have
\begin{align*}
f'(x_0)&=\frac{f(x_0+h)-f(x_0)}{h}\\
&=\lim\frac{1}{h}\int_{x_0}^{x_0+h} g(t)d\lambda(t)\\
&= g(x_0),
\end{align*}
and
\begin{align*}
f'(x_0)&=\frac{f(x_0-h)-f(x_0)}{h}\\
&=\lim\frac{1}{h}\int_{x_0}^{x_0-h} g(t)d\lambda(t)\\
&= g(x_0),
\end{align*}
which shows the limits from both sides agree and, since $x_0 \in (a,b)$ was arbitrary, that $f'(x)=g(x)$ everywhere in $(a,b)$. To see that $f'(x)=g(x)$ for $x=a$, just take $x_0=a$ above and $h>0$ that
\begin{align*}
f'(a)&=\frac{f(a+h)-f(a)}{h}\\
&=\lim\frac{1}{h}\int_{a}^{a+h} g(t)d\lambda(t)\\
&= g(a).
\end{align*}
and for $[b-h,b]$ with $h>0$, letting $h\to 0$ we have
\[
f'(b)=g(b).
\]
Thus, $f'(x)=g(x)$ for all $x\in [a,b]$.
\end{proof}

\pagebreak




\end{document}



