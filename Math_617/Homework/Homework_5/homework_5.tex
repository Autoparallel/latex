\documentclass[leqno]{article}
\usepackage[utf8]{inputenc}
\usepackage[T1]{fontenc}
\usepackage{amsfonts}
%\usepackage{fourier}
%\usepackage{heuristica}
\usepackage{enumerate}
\author{Colin Roberts}
\title{MATH 617, Homework 4}
\usepackage[left=3cm,right=3cm,top=3cm,bottom=3cm]{geometry}
\usepackage{amsmath}
\usepackage[thmmarks, amsmath, thref]{ntheorem}
%\usepackage{kbordermatrix}
\usepackage{mathtools}
\usepackage{color}

\theoremstyle{nonumberplain}
\theoremheaderfont{\itshape}
\theorembodyfont{\upshape:}
\theoremseparator{.}
\theoremsymbol{\ensuremath{\square}}
\newtheorem{proof}{Proof}
\theoremsymbol{\ensuremath{\square}}
\newtheorem{lemma}{Lemma}
\theoremsymbol{\ensuremath{\blacksquare}}
\newtheorem{solution}{Solution}
\theoremseparator{. ---}
\theoremsymbol{\mbox{\texttt{;o)}}}
\newtheorem{varsol}{Solution (variant)}

\newcommand{\tr}{\mathrm{tr}}
\newcommand{\R}{\mathbb{R}}
\newcommand{\N}{\mathbb{N}}
\newcommand{\Sets}{\mathcal{S}}
\newcommand{\Leb}{\mathcal{L}}


\usepackage{amssymb}
\usepackage{graphics}

\textheight=9.0in
\textwidth=6.5in
\oddsidemargin=0in
\topmargin=-0.50in

\pagestyle{empty}


\begin{document}

\begin{center}
  \textsc{\large ColoState ~~ Spring 2018 ~~ MATH 617 ~~ Assignment 5}
\end{center}

\begin{center}
  \textrm{Due Fri. 04/06/2018}
\end{center}

\vglue 0.10in

\bigskip
\noindent
\textsc{Name:} \underline{Colin Roberts\hglue 1.5in} ~~
\textsc{CSUID:} \underline{829773631\hglue 1.5in}

\vskip 0.15in

\bigskip
\noindent
(20 points) \textit{Problem 1}. \quad
Let $f(x)$ be a real-valued continuous function defined on $\R$. Prove that the inverse image $f^{-1}(B)$ of a Borel subset $B$ is also a Borel subset.

\bigskip
\bigskip
\noindent
(20 points) \textit{Problem 2}. \quad
Prove that if $g$ is absolutely continuous on $[a,b]$, then $g$ has bounded variation.

\bigskip
\bigskip
\noindent
(20 points) \textit{Problem 3}. \quad
Let $f$ be a real-valued increasing function on $[0,1]$ and $\int_0^1 f'd\lambda = f(1)-f(0)$. Prove that $f$ is absolutely continuous.

\bigskip
\bigskip
\noindent
(20 points) \textit{Problem 4}. \quad
Suppose that $F\colon [a,b] \to \R$ is increasing. Prove that there exist unique real-valued functions $G(x)$, $H(x)$ on $[a,b]$ satisfying the following
\begin{enumerate}[(i)]
\item $F(x)=G(x)+H(x)$, $\forall x \in [a,b]$;
\item $G(x)$ is absolutely continuous and $G(a)=F(a)$;
\item $H'(x)=0$ a.e. on $[a,b]$.
\end{enumerate}

\bigskip
\bigskip
\noindent
(20 points) \textit{Problem 5}. \quad
Suppose $F_n(x)$, $n\in \N$, is a sequence of absolutely continuous increasing functions on $[a,b]$ such that
\begin{enumerate}[(1)]
\item $F_n(a)=0$, $\forall n \in \N$;
\item $F'_n(x)$ is a decreasing sequence for almost every $x\in [a,b]$.
\end{enumerate}
Prove that
\begin{enumerate}[(i)]
\item $F_n(x)$ is a decreasing sequence for all $x \in [a,b]$;
\item If $\displaystyle{\lim_{n\to \infty} F_n(x)=F(x)}$ on $[a,b]$, then $F(x)$ is absolutely continuous.
\end{enumerate}



\pagebreak

%%%%%%%%%%%%%%%%%%%%%%%%%%%%%%%%%%%%%%%%%%%%%%%%%%%%%%%%%%%%%%%%%%%%%%%%%%%%%%%%%%%%%%%%%%%%%%%%%%%%%%%%%%%%%%%%%%%%%
%%%%%%%%%%%%%%%%%%%%%%%%%PROBLEM%%%%%%%%%%%%%%%%%%%%%%%%%%%%%%%%%%%%%%%%%%%%%%%%%%%%%%%%%%%%%%%%%%%%%%%%%%%%%%%%%%%%%%%%%%%%%%%%%%%%%%%%%%%%%%%%%%%%%%%%%%%%%%%%%%%%%%%%%%%%%%%%%%%%%%%%%%%%%%%%%%%%%%%%%%%%%%%%%%%%%%%%%%%%%%%%%%%%%%%%%%

\noindent\textbf{Problem 1.} \quad
Let $f(x)$ be a real-valued continuous function defined on $\R$. Prove that the inverse image $f^{-1}(B)$ of a Borel subset $B$ is also a Borel subset.

\noindent\rule[0.5ex]{\linewidth}{1pt}

\begin{proof}
Define the set
\[
\mathcal{A}=\{B\in \mathcal{B}_\R ~\colon ~ f^{-1}(B)\in \mathcal{B}_\R\},
\]
where $\mathcal{B}_\R$ denotes the Borel sets on $\R$.  By definition $\mathcal{A}\subseteq \mathcal{B}_\R$.  So, we need to show that $\mathcal{A}\supseteq \mathcal{B}_\R$. To see this, we show that $\mathcal{A}$ is a $\sigma$-algebra on subsets of $\R$ generated by open intervals and since $\mathcal{B}_\R$ is the smallest $\sigma$-algebra on $\R$ which is generated by open intervals, we will have that $\mathcal{A}\supseteq \mathcal{B}_\R$.  First, we have that continuity of $f$ implies that $f^{-1}$ pulls back open sets in $\mathcal{B}_\R$ to open sets.  Hence, we will have that $\mathcal{A}$ is generated by open sets.
\begin{enumerate}[(i)]
\item Note that $f^{-1}(\R)=\R$ since $f(x)$ is a continuous defined on all of $\R$ and so $\R\in \mathcal{A}$.  Then we also have that $f^{-1}(\emptyset)=\emptyset$ and so $\emptyset \in \mathcal{A}$.
\item Let $A\in \mathcal{A}$. This means $A=f^{-1}(A')$ for $A' \in \mathcal{B}_\R$.  Then we have that $A'^c \in \mathcal{B}_\R$
\[
A^c = \R \setminus A = \R \setminus f^{-1}(A')= f^{-1}(\R\setminus A')=f^{-1}(A'^c).
\]
So we have $A^c\in \mathcal{A}$.
\item Let $A_i\in \mathcal{A}$ for all $i\in \N$ which means that $A_i=f^{-1}(A_i')$ for $A_i'\in \mathcal{B}_\R$. Then, consider
\[
\bigcup_{i=1}^\infty A_i = \bigcup_{i=1}^\infty f^{-1}(A_i') = f^{-1}\left(\bigcup_{i=1}^\infty A_i'\right).
\]
Since $\bigcup_{i=1}^\infty A_i' \in \mathcal{B}_\R$, we have that $\bigcup_{i=1}^\infty A_i \in \mathcal{A}$.  
\end{enumerate}
These three items above show that $\mathcal{A}$ is a $\sigma$-algebra on $\R$ and thus $\mathcal{A}\supseteq \mathcal{B}_\R$ and so we have that $\mathcal{A}=\mathcal{B}_\R$, which proves the original statement.
\end{proof}



\pagebreak

%%%%%%%%%%%%%%%%%%%%%%%%%%%%%%%%%%%%%%%%%%%%%%%%%%%%%%%%%%%%%%%%%%%%%%%%%%%%%%%%%%%%%%%%%%%%%%%%%%%%%%%%%%%%%%%%%%%%%
%%%%%%%%%%%%%%%%%%%%%%%%%PROBLEM%%%%%%%%%%%%%%%%%%%%%%%%%%%%%%%%%%%%%%%%%%%%%%%%%%%%%%%%%%%%%%%%%%%%%%%%%%%%%%%%%%%%%%%%%%%%%%%%%%%%%%%%%%%%%%%%%%%%%%%%%%%%%%%%%%%%%%%%%%%%%%%%%%%%%%%%%%%%%%%%%%%%%%%%%%%%%%%%%%%%%%%%%%%%%%%%%%%%%%%%%%


\noindent\textbf{Problem 2.} \quad
Prove that if $g$ is absolutely continuous on $[a,b]$, then $g$ has bounded variation.

\noindent\rule[0.5ex]{\linewidth}{1pt}


\begin{proof}
Fix $\epsilon =1$ and we have that there exists a $\delta>0$ such that 
\[
\sum_{i=1}^n |g(b_i)-g(a_i)|<1,
\]
when $(a_i,b_i)$, $i=1,\dots,n$, are mutually disjoint subintervals of $[a,b]$ satisfying
\[
\sum_{i=1}^n (b_i-a_i)<\delta
\]
by the absolutely continuity of $g$.  This implies that if we take any subinterval $[c,d]$ of $[a,b]$ with $(d-c)<\delta$ that we have $V_c^d(g)< 1$ by absolute continuity. So we fix any partition of $[a,b]$ satsifying $\|P\|<\delta$ which yields
\[
V_a^b(g)=\sum_{i=1}^k V_{x_{i-1}}^{x_i} (g)< \sum_{i=1}^k 1=k.
\]
Hence, $g$ has bounded variation.
\end{proof}


\pagebreak

%%%%%%%%%%%%%%%%%%%%%%%%%%%%%%%%%%%%%%%%%%%%%%%%%%%%%%%%%%%%%%%%%%%%%%%%%%%%%%%%%%%%%%%%%%%%%%%%%%%%%%%%%%%%%%%%%%%%%
%%%%%%%%%%%%%%%%%%%%%%%%%PROBLEM%%%%%%%%%%%%%%%%%%%%%%%%%%%%%%%%%%%%%%%%%%%%%%%%%%%%%%%%%%%%%%%%%%%%%%%%%%%%%%%%%%%%%%%%%%%%%%%%%%%%%%%%%%%%%%%%%%%%%%%%%%%%%%%%%%%%%%%%%%%%%%%%%%%%%%%%%%%%%%%%%%%%%%%%%%%%%%%%%%%%%%%%%%%%%%%%%%%%%%%%%%

\noindent\textbf{Problem 3.} \quad
Let $f$ be a real-valued increasing function on $[0,1]$ and $\int_0^1 f'd\lambda = f(1)-f(0)$. Prove that $f$ is absolutely continuous.

\noindent\rule[0.5ex]{\linewidth}{1pt}

\begin{proof}
We wish to show that $f(x)=\int_0^x f' d\lambda + f(0)$. First, if $x=0$, then
\[
f(0)=\int_0^0 f'd\lambda + f(0) = f(0).
\]
And for $x=1$, we have
\[
f(1)=\int_0^1 f'  d\lambda +f(0)= f(1)-f(0)+f(0)=f(1).
\]
Now, suppose for a contradiction that for any $x\in (0,1)$ that we have $f(x)\neq \int_0^x f'(x)d\lambda - f(0)$. Since $f$ is an increasing function, we have that
\begin{align*}
\int_0^x f' d\lambda +f(0)<f(x). 
\end{align*}
However, this implies
\begin{align*}
\int_0^1 f'd\lambda = \int_0^x f' d\lambda + \int_x^1 f' d\lambda &< f(1)-f(0)
\end{align*}
which contradicts our original statement. Hence, we have that
\begin{align*}
f(x)=\int_0^x f'd\lambda +f(0)
\end{align*}
and since $f(x)$ is defined via an integral, we have that $f$ is absolutely continuous.
\end{proof}

\pagebreak



%%%%%%%%%%%%%%%%%%%%%%%%%%%%%%%%%%%%%%%%%%%%%%%%%%%%%%%%%%%%%%%%%%%%%%%%%%%%%%%%%%%%%%%%%%%%%%%%%%%%%%%%%%%%%%%%%%%%%
%%%%%%%%%%%%%%%%%%%%%%%%%PROBLEM%%%%%%%%%%%%%%%%%%%%%%%%%%%%%%%%%%%%%%%%%%%%%%%%%%%%%%%%%%%%%%%%%%%%%%%%%%%%%%%%%%%%%%%%%%%%%%%%%%%%%%%%%%%%%%%%%%%%%%%%%%%%%%%%%%%%%%%%%%%%%%%%%%%%%%%%%%%%%%%%%%%%%%%%%%%%%%%%%%%%%%%%%%%%%%%%%%%%%%%%%%

\noindent\textbf{Problem 4.} \quad
Suppose that $F\colon [a,b] \to \R$ is increasing. Prove that there exist unique real-valued functions $G(x)$, $H(x)$ on $[a,b]$ satisfying the following
\begin{enumerate}[(i)]
\item $F(x)=G(x)+H(x)$, $\forall x \in [a,b]$;
\item $G(x)$ is absolutely continuous and $G(a)=F(a)$;
\item $H'(x)=0$ a.e. on $[a,b]$.
\end{enumerate}

\noindent\rule[0.5ex]{\linewidth}{1pt}


\begin{proof}~
Since $F$ is increasing on $[a,b]$ we have that $F'\in L_1 [a,b]$.  So, we define
\[
G(x)=\int_a^x F'(x)d\lambda + F(a)
\]
and note that $G(a)=F(a)$ and that $G(x)$ is absolutely continuous by definition.  Now, we have that
\begin{align*}
H(x)&=F(x)-G(x)\\
&= F(x)-\int_a^x F'(x)d\lambda - F(a)\\
\implies H'(x)&=F'(x)-F'(x)=0 &&\textrm{almost everywhere since $F'(x)$ exists almost everywhere}.
\end{align*}
Note that $G(a)=F(a)$ implies that $H(a)=0$ and since $H'(x)=0$ almost everywhere, we have that $H(x)=0$ for all $x\in [a,b]$.

Now, to see that $G$ and $H$ are unique, suppose $\exists g, h$ satisfying the same criteria above but $g(x)\neq G(x)$, $g(x)\neq H(x)$, $h(x)\neq G(x)$, and $h(x)\neq H(x)$.  Then 
\begin{align*}
G(x)&=g(x)+h(x).
\end{align*}
Evaluating at $x=a$ implies that $h(a)=0$ and so $G(a)=g(a)=F(a)$. Again, $h'(x)=0$ almost everywhere and so $h(x)=0=H(x)$ and thus we have 
\[
G(x)=g(x).
\]
So $G$ and $H$ are unique.
\end{proof}

\pagebreak



%%%%%%%%%%%%%%%%%%%%%%%%%%%%%%%%%%%%%%%%%%%%%%%%%%%%%%%%%%%%%%%%%%%%%%%%%%%%%%%%%%%%%%%%%%%%%%%%%%%%%%%%%%%%%%%%%%%%%
%%%%%%%%%%%%%%%%%%%%%%%%%PROBLEM%%%%%%%%%%%%%%%%%%%%%%%%%%%%%%%%%%%%%%%%%%%%%%%%%%%%%%%%%%%%%%%%%%%%%%%%%%%%%%%%%%%%%%%%%%%%%%%%%%%%%%%%%%%%%%%%%%%%%%%%%%%%%%%%%%%%%%%%%%%%%%%%%%%%%%%%%%%%%%%%%%%%%%%%%%%%%%%%%%%%%%%%%%%%%%%%%%%%%%%%%%

\noindent\textbf{Problem 5.} \quad
Suppose $F_n(x)$, $n\in \N$, is a sequence of absolutely continuous increasing functions on $[a,b]$ such that
\begin{enumerate}[(1)]
\item $F_n(a)=0$, $\forall n \in \N$;
\item $F'_n(x)$ is a decreasing sequence for almost every $x\in [a,b]$.
\end{enumerate}
Prove that
\begin{enumerate}[(i)]
\item $F_n(x)$ is a decreasing sequence for all $x \in [a,b]$;
\item If $\displaystyle{\lim_{n\to \infty} F_n(x)=F(x)}$ on $[a,b]$, then $F(x)$ is absolutely continuous.
\end{enumerate}


\noindent\rule[0.5ex]{\linewidth}{1pt}

\begin{proof}~
\begin{enumerate}[(i)]
\item We have
\begin{align*}
F_n'(x)&\geq F_{n+1}'(x)\\
\implies \int_a^x F_n'(x)d\lambda &\geq \int_a^x F_{n+1}'(x) d\lambda\\
\implies F_n(x)-F_n(a)&\geq F_{n+1}(x)-F_{n+1}(a)\\
\implies F_n(x)&\geq F_{n+1}(x) && \textrm{since $F_n(a)=0$ $\forall n$.}
\end{align*}

\item Suppose we have $\lim_{n\to \infty}F_n(x)=F(x)$. First, since each $F_n$ is absolutely continuous, we have that for each $n$ and any collection of $m$ mutually disjoint intervals $[a_i,b_i]$ that if $\sum_{i=1}^m (b_i-a_i)<\delta$ then 
\[
\sum_{i=1}^m |F_n(b_i)-F_n(a_i)|< \frac{\epsilon}{3}
\]
Then note that $F_n(x)$ is a monotone sequence of absolutely continuous (and hence, continuous) functions and $[a,b]$ is compact which means that $F_n$ converges uniformly to $F$.  By uniform convergence, we have for some $N$ that for $n\geq N$
\[
|F(x)-F_n(x)|< \frac{\epsilon}{3m}.
\]
Now, using these same arbitrary intervals above and for $n\geq N$ we have
\begin{align*}
\sum_{i=1}^m |F(b_i)-F(a_i)|&=\sum_{i=1}^m |F(b_i)-F_n(b_i)+F_n(a_i)-F(a_i)-F_n(a_i)+F_n(b_i)|\\
&\leq \sum_{i=1}^m \left( |F(b_i)-F_n(b_i)|+|F(a_i)-F_n(a_i)|+|F_n(b_i)-F_n(a_i)|\right)\\
&< \frac{\epsilon}{3}+\frac{\epsilon}{3}+m\frac{\epsilon}{3m}\\
&= \epsilon.
\end{align*}
Hence, $F$ is absolutely continuous.
\end{enumerate}
\end{proof}

\pagebreak

\end{document}



