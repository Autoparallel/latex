\documentclass{article}
\usepackage[utf8]{inputenc}
\usepackage[left=2.5cm,right=2.5cm,top=2.5cm,bottom=2.5cm]{geometry}

%\title{IMA Personal Statement}
%\author{Colin Roberts}
%\date{February 2019}
\linespread{2}

\begin{document}


\noindent To those whom this may concern,

My name is Colin Roberts and I am a fourth year Ph.D. student in mathematics at Colorado State University (CSU) and Aiden Pullen is a current student of mine. Over the past two years, I have tought a year long sequence course titled ``Applied Mathematics for Chemists I/II". This course serves as a specialized mathematics course for chemistry students at CSU and we cover a broad amount of topics including ordinary and partial differential equations, sequences and series, multivariate calculus, and Fourier theory. Aiden joined me last semester for the first course and is continuing with me this semester in the sequel.

In the previous semester, Aiden was the top performer in the class. He completed every homework assignment and quiz with consistently high marks. Points missed on quizzes were corrected to their full extent when possible. This class also engaged students to take 15 minute one-on-one oral examinations where they had to explain their solutions to provided problems and answer new questions they had not seen prior to the exam. Aiden was well spoken and confident when we met virtually for these exams and managed to think through the questions he had not been able to prepare for. He showed that he has excellent intuition. During class sessions, he was also an active participant and was consistently engaged even through our virtual meetings.  This is a rarity in these days. I have only worked with Aiden for a bit over a semester but his hard and soft skills as well as curiosity make him an excellent candidate.

The courses in applied mathematics for chemists give the students an opportunity to learn about quantum mechanics as they learn mathematics. Aiden expressed specific interest in these topics and their applications. For example, he was interested in how quantum tunneling is realized in nuclear chemistry. I shared some resources with him and he asked follow up questions once he had read through them. He also has shared with me that he is particularly interested in radiochemistry and pursuing a graduate degree once he graduates. Given his success in the first portion of this challenging mathematics sequence, I can assume his ability to understand theoretical chemistry will be more than sufficient. He would make an excellent addition to the program.


\noindent -Colin Roberts

\end{document}