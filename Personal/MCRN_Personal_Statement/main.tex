\documentclass{article}
\usepackage[utf8]{inputenc}
\usepackage[left=3cm,right=3cm,top=3cm,bottom=3cm]{geometry}

%\title{IMA Personal Statement}
%\author{Colin Roberts}
%\date{February 2019}
\linespread{1.5}

\begin{document}

%\maketitle

%\begin{itemize}
%    \item Interdisciplanary
%    \item Calculus of variations and PDEs
%    \item Mechanics, electromagnetism, (topological) fluid dynamics, physics
%    \item Analysis, geometry, topology
%\end{itemize}

\noindent \textbf{MCRN summer school and academic year engagement program application:}\\

\noindent To those whom this may concern,

I am currently in my second year of a mathematics Ph.D program at Colorado State University.  My interests are primarily in differential geometry, functional analysis, and partial differential equations.  Over the two years of graduate school I have taken two courses in PDEs. Also, I am currently in a course in dynamical systems, and have spent much of my other time studying differential geometry.  I am applying for this summer school primarily due to my interests in the "tipping points in climate and weather systems" theme.

As an undergraduate, I majored in both mathematics and physics.  In my last year, I was able to take a graduate level physics course in chaos and nonlinear dynamics.  I found that this area of mathematics/physics was very interesting and seemed to have a large scope of application especially to weather.  My two years of graduate school have been mostly concentrated on pure mathematics in order to understand the framework behind physical theories. My largest interest is in physics applications within the fields of dynamical systems, PDEs and geometry.  This semester I am also working on a project on analyzing the dynamics of many classical systems such as the logistic map, Hen\'on map, and Lorenz attractor.  Specifically, I am trying understand the geometry and topology of the attractors as different parameters are varied. 

Originally, I persued an undergraduate degree in computer engineering. I enjoyed many of the classes I took on programming, but my calling was in physics and mathematics. Although I changed degree concentrations, I still happily wrote programs to solve PDEs that came up in physics or mathematics courses.  For undergraduate research, I wrote code to model light propagation in an ultracold gas. The project required me to create code to solve the Maxwell wave equation using the Crank-Nicolson method and make it all adaptable to initial gas and light properties. I enjoyed the project and this type of work is something I would be very excited to practice and learn more about.  I am very adaptable and would happily learn anything new that I would need to know.

Lastly, a large part of my interest in this project comes from my personal desire to understand our Earth's atmosphere and climate.  Of course, us humans are radically altering various levels of chemicals in our atmosphere and the effects should be better understood.  Any research in this area is important to pursue so that we can better take care of our planet. This area of research happens to align with my mathematical interests as well.  Analysis of PDEs and dynamical systems on interesting geometries is truly a beautiful area of study.\\


\noindent Best,

\noindent Colin Roberts

\end{document}
