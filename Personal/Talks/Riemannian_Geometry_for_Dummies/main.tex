\documentclass[UKenglish]{beamer}

\usepackage[utf8]{inputenx} % For æ, ø, å
\usepackage{csquotes}       % Quotation marks
\usepackage{microtype}      % Improved typography
\usepackage{amssymb}        % Mathematical symbols
\usepackage{mathtools}      % Mathematical symbols
\usepackage[absolute, overlay]{textpos} % Arbitrary placement
\setlength{\TPHorizModule}{\paperwidth} % Textpos units
\setlength{\TPVertModule}{\paperheight} % Textpos units
\usepackage{tikz}
\usetikzlibrary{overlay-beamer-styles}  % Overlay effects for TikZ

\AtBeginSection{\frame{\sectionpage}}
\AtBeginSubsection{\frame{\subsectionpage}}

\usepackage{hyperref}
\usepackage{svg}
%\usefonttheme{serif}

\usepackage{color, soul, xcolor} % Colored text and highlighting, respectively
\usepackage{tikz-cd} % For commutative diagrams
\usepackage{tikz-3dplot}
\usetikzlibrary{angles}
\RequirePackage{pgfplots}
\usepackage{mathtools}
\usepackage{answers}
\usepackage{setspace}
\usepackage{graphicx}
\usepackage{enumerate}
\usepackage{multicol}
\usepackage{mathrsfs}
\usepackage{amsmath,amsthm,amssymb}
\usepackage{marvosym,wasysym} %fucking smileys
\usepackage{float}
\usepackage{morefloats}
\usepackage{pgf,tikz}
\pgfplotsset{compat=1.15}
\usepackage{mathrsfs}
\usetikzlibrary{arrows}
\usepackage{subcaption}
\usepackage[most]{tcolorbox}
\tcbuselibrary{theorems}
\usepackage{fancyvrb}
\usepackage{longtable,booktabs}
\usepackage{stackrel}
\usepackage{animate}
\usepackage[percent]{overpic}
\definecolor{lighter_csu_green}{RGB}{60,133,77}
\newcommand\boldgreen[1]{\textcolor{lighter_csu_green}{\emph{\textbf{#1}}}}
\usepackage{MnSymbol}

%Commands
\newcommand{\R}{\mathbb{R}}
\newcommand{\opens}{\mathcal{O}}

%border matrix
\makeatletter
\newif\if@borderstar
\def\bordermatrix{\@ifnextchar*{%
\@borderstartrue\@bordermatrix@i}{\@borderstarfalse\@bordermatrix@i*}%
}
\def\@bordermatrix@i*{\@ifnextchar[{\@bordermatrix@ii}{\@bordermatrix@ii[()]}}
\def\@bordermatrix@ii[#1]#2{%
\begingroup
\m@th\@tempdima8.75\p@\setbox\z@\vbox{%
\def\cr{\crcr\noalign{\kern 2\p@\global\let\cr\endline }}%
\ialign {$##$\hfil\kern 2\p@\kern\@tempdima & \thinspace %
\hfil $##$\hfil && \quad\hfil $##$\hfil\crcr\omit\strut %
\hfil\crcr\noalign{\kern -\baselineskip}#2\crcr\omit %
\strut\cr}}%
\setbox\tw@\vbox{\unvcopy\z@\global\setbox\@ne\lastbox}%
\setbox\tw@\hbox{\unhbox\@ne\unskip\global\setbox\@ne\lastbox}%
\setbox\tw@\hbox{%
$\kern\wd\@ne\kern -\@tempdima\left\@firstoftwo#1%
\if@borderstar\kern2pt\else\kern -\wd\@ne\fi%
\global\setbox\@ne\vbox{\box\@ne\if@borderstar\else\kern 2\p@\fi}%
\vcenter{\if@borderstar\else\kern -\ht\@ne\fi%
\unvbox\z@\kern-\if@borderstar2\fi\baselineskip}%
\if@borderstar\kern-2\@tempdima\kern2\p@\else\,\fi\right\@secondoftwo#1 $%
}\null \;\vbox{\kern\ht\@ne\box\tw@}%
\endgroup
}
\makeatother

\usetheme{UiB}

%For easier reading
\setbeamersize{text margin left=40pt,text margin right=40pt}
\renewcommand{\baselinestretch}{1.3}


%% FONT STUFF
\usepackage{amsmath}
\usepackage{amsfonts}
\usefonttheme[onlymath]{serif}


\author{Colin Roberts}
\setbeamercolor{title}{fg=white} 
\title{Riemannian Geometry}
\setbeamercolor{subtitle}{fg=white} 
\subtitle{for Dummies}



\begin{document}


\section{Introduction}

\begin{frame}{}
	\vfill
	\large{Riemannian geometry} is the study of a \boldgreen{smooth manifold} $M$ along with a \boldgreen{Riemannian metric} $g$. 
	\vfill
\end{frame}

\begin{frame}{}
	\vfill
	The point of Riemmannian geometry is to generalize the differentiable and metric structure of $\R^n$.
	\vfill
\end{frame}

\begin{frame}{}
	\vfill
	We generalize to spaces that have interesting topology and geometry.
	\vfill
\end{frame}

\begin{frame}{}
	\vfill
	This will require us to rethink some notions we found ``easy" in $\R^n$.
	\vfill
\end{frame}

\begin{frame}{}
	\vfill
	But we will gain a very general framework for working with differentiable objects.
	\vfill
\end{frame}

\section{Motivation}

\begin{frame}{}
	\vfill
	Why study this in the first place?
	\vfill
\end{frame}

\begin{frame}{}
	\vfill
	Example: Partial differential equations (PDEs) on spaces that are not flat.
	\pause
	\begin{itemize}
		\item Fluid flow on Earth;
		\pause
		\item Electrical Impedence Tomography (EIT);
		\pause
		\item General relativity.
	\end{itemize}
	\vfill
\end{frame}

\begin{frame}{}
	\vfill
	Example: Optimization in interesting spaces.
	\pause
	\begin{itemize}
		\item Grassmannians;
		\pause
		\item Flags.
	\end{itemize}
	\vfill
\end{frame}

\section{Preliminaries}

\begin{frame}{}
	add more math text before/after pics so that people see some notation. More examples.
\end{frame}


\subsection{Smooth Manifolds}

\begin{frame}{}
\vfill
\textbf{\underline{Our To-Do List:}}
\begin{itemize}
	\item Start with a topological space $M$;
	\pause
	\item Look at open sets $U$ that cover $M$;
	\pause
	\item Construct local coordinates $\varphi$;
	\pause
	\item Show coordinate transition functions are smooth.
\end{itemize}
\vfill
\end{frame}

\begin{frame}{}
	Define the sphere as the set of points in $\R^3$... then say we'll mostly use this as an example so keep it in mind
\end{frame}

\begin{frame}{}
\vfill
\begin{figure}[H]
	\centering
	\def\svgwidth{0.75\columnwidth}
	\input{sphere.pdf_tex}
\end{figure}\vfill
\end{frame}

\begin{frame}{}
\vfill
\begin{figure}[H]
	\centering
	\def\svgwidth{.75\columnwidth}
	\input{sphere_charts.pdf_tex}
\end{figure}
\vfill
\end{frame}

\begin{frame}{}
\vfill
\begin{figure}[H]
	\centering
	\def\svgwidth{\columnwidth}
	\input{sphere_charts_maps.pdf_tex}
\end{figure}
\vfill
\end{frame}

\subsection{Vector Fields}

\begin{frame}{}
\vfill
\begin{figure}[H]
	\centering
	\def\svgwidth{0.75\columnwidth}
	\input{sphere_curve.pdf_tex}
\end{figure}
\vfill
\end{frame}

\begin{frame}{}
\vfill
\begin{figure}[H]
	\centering
	\def\svgwidth{0.75\columnwidth}
	\input{sphere_curve_tangent.pdf_tex}
\end{figure}
\vfill
\end{frame}

\begin{frame}{}
\vfill
\begin{figure}[H]
	\centering
	\def\svgwidth{\columnwidth}
	\input{pushforward.pdf_tex}
\end{figure}
\vfill
\end{frame}

\begin{frame}{}
\vfill
\begin{figure}[H]
	\centering
	\def\svgwidth{.75\columnwidth}
	\input{tangent_spaces.pdf_tex}
\end{figure}
\vfill
\end{frame}

\begin{frame}{}
\vfill
\begin{figure}[H]
	\centering
	\def\svgwidth{\columnwidth}
	\input{tangent_bundle.pdf_tex}
\end{figure}
\vfill
\end{frame}

\begin{frame}{}
\vfill
\begin{figure}[H]
	\centering
	\def\svgwidth{\columnwidth}
	\input{section.pdf_tex}
\end{figure}
\vfill
\end{frame}

\subsection{Example}

\begin{frame}{}
	Example of charts and maps for sphere as well as a vector field.  Do spherical coordinates
\end{frame}

\begin{frame}{}
\vfill
\begin{figure}[H]
	\centering
	\def\svgwidth{\columnwidth}
	\input{spherical_coordinates.pdf_tex}
\end{figure}
\vfill
\end{frame}

\begin{frame}{}
\vfill
\begin{figure}[H]
	\centering
	\def\svgwidth{\columnwidth}
	\input{spherical_coordinates_tangent_vectors.pdf_tex}
\end{figure}
\vfill
\end{frame}

\section{Riemannian Geometry}

\begin{frame}{}
\vfill
\textbf{\underline{Our To-Do List:}}
\begin{itemize}
	\item Build an inner product on the tangent space $T_pM$;
	\pause
	\item Have the inner product vary smoothly as we vary the point $p$;
	\pause
	\item Define this as our Riemannian metric tensor field $g$;
	\pause
	\item Extract geometrical and analytical qualities of the underlying manifold $M$.
\end{itemize}
\vfill
\end{frame}


\subsection{Riemannian Metric}

\begin{frame}{}
\vfill
$g_{ij}(x) = \varphi^*(x)e_i \cdot \varphi^*(x)e_k = (\partial_i \varphi)\cdot (\partial_j \varphi) = \sum_{k=1}^m \frac{\partial \varphi_k}{\partial x_i}\frac{\partial \varphi_k}{\partial x_j}$
\vfill
\end{frame}

\begin{frame}{}
connection covariant derivative interpretation, covariant derivative in spherical coordinates,second covariant matrix and covariant laplacian from hessian? Compatability with riemannian metric picture. Show a geodesic in the coordinates between cities or something?
\end{frame}


\begin{frame}{}
	From minimization of length/energy. Both are good to mention.	Geodesic equation
	\[
	\nabla_{\dot{\gamma}} \dot{\gamma}=0
	\]
	equivalent to
	\[
	\ddot{x}^l + \dot{x}^j\dot{x}^k \Gamma_{jk}^l=0
	\]
	which is saying that the only "acceleration" of the curve comes from the geometry it lies on. When flat space, $\Gamma_{jk}^l=0$ and we have $\ddot{x}=0$.
\end{frame}

\begin{frame}{}
	volume and integration
\end{frame}



\section{Applications}
 navier stokes on the Earth. Riemannian metric as a conductivity?








\section{Conclusions}


\end{document}



%%% MAYBE USEFUL
%\begin{frame}{}
%	\vfill
%	A \boldgreen{topological space} $M$ is a set of points with a collection of subsets $\opens$ that we define to be open. These sets satisfy \pause
%	\begin{itemize}
%		\item $\emptyset,M \in \opens$;
%		\pause
%		\item Arbitrary unions of open sets are open;
%		\pause
%		\item Finite intersections of open sets are open.
%	\end{itemize}
%	\vfill
%\end{frame}
%
%\begin{frame}{}
%	\vfill
%Just do diffeomorphisms and forget this
%	\begin{itemize}
%		\item A \boldgreen{homeomorphism} is a continuous bijection $f\colon M \to N$ with continuous inverse $f^{-1}$.  
%		\pause
%		\item We say $M$ and $N$ are homeomorphic.
%	\end{itemize}
%	\vfill	
%\end{frame}
%
%\begin{frame}{}	
%	ignore boundary 
%	\vfill
%	A \boldgreen{manifold} (with boundary) is a space that locally looks like $\R^n$ (or $\R^{n^+}$). 
%	\pause	
%	\[
%	\downarrow
%	\]
%	A \boldgreen{manifold} $M$ (with boundary) is a topological space such that each open set in $\opens$ is homeomorphic to $\R^n$ (or $\R^{n^+}$).
%	\vfill
%\end{frame}
%
%\begin{frame}{}
%	\vfill
%	
%\end{frame}
%
%\begin{frame}{}
%	
%\end{frame}
%
%\begin{frame}{Manifolds}
%	\begin{figure}[H]
%    \centering
%    \begin{overpic}[width=0.8\textwidth]{Figures/coordinate_charts_manifold.png}
%    \put (39,61) {\Large$U_\alpha$}
%    \put (70,57) {\Large$U_\beta$}
%    \put (12,65) {\LARGE$M$}
%    \put (20,35) {\Large$\varphi_\alpha$}
%    \put (73,30) {\Large$\varphi_\beta$}
%    \put (22,3) {\Large$\R^m$}
%    \put (60,3) {\Large$\R^m$}
%    \put (38,32) {\Large$\varphi_{\alpha \beta}$}
%    \put (38,18) {\Large$\varphi_{\beta \alpha}$}
%    \end{overpic}
%\end{figure}
%\end{frame}