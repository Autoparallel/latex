\documentclass[UKenglish]{beamer}


\usepackage[utf8]{inputenx} % For æ, ø, å
\usepackage{csquotes}       % Quotation marks
\usepackage{microtype}      % Improved typography
\usepackage{amssymb}        % Mathematical symbols
\usepackage{mathtools}      % Mathematical symbols
\usepackage[absolute, overlay]{textpos} % Arbitrary placement
\setlength{\TPHorizModule}{\paperwidth} % Textpos units
\setlength{\TPVertModule}{\paperheight} % Textpos units
\usepackage{tikz}
\usetikzlibrary{overlay-beamer-styles}  % Overlay effects for TikZ

\AtBeginSection{\frame{\sectionpage}}

\usepackage{hyperref}
\usepackage{svg}
\usefonttheme{serif}

\usepackage{color, soul, xcolor} % Colored text and highlighting, respectively
\usepackage{tikz-cd} % For commutative diagrams
\usepackage{tikz-3dplot}
\usetikzlibrary{angles}
\RequirePackage{pgfplots}
\usepackage{mathtools}
\usepackage{answers}
\usepackage{setspace}
\usepackage{graphicx}
\usepackage{enumerate}
\usepackage{multicol}
\usepackage{mathrsfs}
\usepackage{amsmath,amsthm,amssymb}
\usepackage{marvosym,wasysym} %fucking smileys
\usepackage{float}
\usepackage{morefloats}
\usepackage{pgf,tikz}
\pgfplotsset{compat=1.15}
\usepackage{mathrsfs}
\usetikzlibrary{arrows}
\usepackage{subcaption}
\usepackage[most]{tcolorbox}
\tcbuselibrary{theorems}
\usepackage{fancyvrb}
\usepackage{longtable,booktabs}
\usepackage{stackrel}
\usepackage{animate}
\usepackage[percent]{overpic}
\definecolor{lighter_csu_green}{RGB}{60,133,77}
\newcommand\boldgreen[1]{\textcolor{lighter_csu_green}{\emph{\textbf{#1}}}}

%Commands
\newcommand{\R}{\mathbb{R}}
\newcommand{\opens}{\mathcal{O}}

%border matrix
\makeatletter
\newif\if@borderstar
\def\bordermatrix{\@ifnextchar*{%
\@borderstartrue\@bordermatrix@i}{\@borderstarfalse\@bordermatrix@i*}%
}
\def\@bordermatrix@i*{\@ifnextchar[{\@bordermatrix@ii}{\@bordermatrix@ii[()]}}
\def\@bordermatrix@ii[#1]#2{%
\begingroup
\m@th\@tempdima8.75\p@\setbox\z@\vbox{%
\def\cr{\crcr\noalign{\kern 2\p@\global\let\cr\endline }}%
\ialign {$##$\hfil\kern 2\p@\kern\@tempdima & \thinspace %
\hfil $##$\hfil && \quad\hfil $##$\hfil\crcr\omit\strut %
\hfil\crcr\noalign{\kern -\baselineskip}#2\crcr\omit %
\strut\cr}}%
\setbox\tw@\vbox{\unvcopy\z@\global\setbox\@ne\lastbox}%
\setbox\tw@\hbox{\unhbox\@ne\unskip\global\setbox\@ne\lastbox}%
\setbox\tw@\hbox{%
$\kern\wd\@ne\kern -\@tempdima\left\@firstoftwo#1%
\if@borderstar\kern2pt\else\kern -\wd\@ne\fi%
\global\setbox\@ne\vbox{\box\@ne\if@borderstar\else\kern 2\p@\fi}%
\vcenter{\if@borderstar\else\kern -\ht\@ne\fi%
\unvbox\z@\kern-\if@borderstar2\fi\baselineskip}%
\if@borderstar\kern-2\@tempdima\kern2\p@\else\,\fi\right\@secondoftwo#1 $%
}\null \;\vbox{\kern\ht\@ne\box\tw@}%
\endgroup
}
\makeatother

\usetheme{UiB}

%For easier reading
\setbeamersize{text margin left=40pt,text margin right=40pt}
\renewcommand{\baselinestretch}{1.3}

\author{Colin Roberts}
\setbeamercolor{title}{fg=white} 
\title{Riemannian Geometry}
\setbeamercolor{subtitle}{fg=white} 
\subtitle{for Dummies}



\begin{document}


\section{Introduction}

\begin{frame}{}
	\vfill
	\large{Riemannian geometry} is the study of a \boldgreen{smooth manifold} $M$ along with a \boldgreen{metric tensor field} $g$. 
	\vfill
\end{frame}

\begin{frame}{}
	\vfill
	The point of Riemmannian geometry is to generalize the differentiable and metric structure of $\R^n$.
	\vfill
\end{frame}

\begin{frame}{}
	\vfill
	We generalize to spaces that have interesting topology and geometry.
	\vfill
\end{frame}

\begin{frame}{}
	\vfill
	This will require us to rethink some notions we found ``easy" in $\R^n$.
	\vfill
\end{frame}

\begin{frame}{}
	\vfill
	But we will gain a very general framework for working with differentiable objects.
	\vfill
\end{frame}

\section{Motivation}

\begin{frame}{}
	\vfill
	Why study this in the first place?
	\vfill
\end{frame}

\begin{frame}{}
	\vfill
	Example: Partial differential equations (PDEs) on spaces that are not flat.
	\pause
	\begin{itemize}
		\item Fluid flow on Earth;
		\pause
		\item Electrical Impedence Tomography (EIT);
		\pause
		\item General relativity.
	\end{itemize}
	\vfill
\end{frame}

\begin{frame}{}
	\vfill
	Example: Optimization in interesting spaces.
	\pause
	\begin{itemize}
		\item Grassmannians;
		\pause
		\item Flags.
	\end{itemize}
	\vfill
\end{frame}

\section{Preliminaries}

\begin{frame}{}
\begin{figure}[H]
	\centering
	\def\svgwidth{0.75\columnwidth}
	\input{sphere.pdf_tex}
\end{figure}
\end{frame}

\begin{frame}{}
\begin{figure}[H]
	\centering
	\def\svgwidth{.75\columnwidth}
	\input{sphere_charts.pdf_tex}
\end{figure}
\end{frame}

\begin{frame}{}
\begin{figure}[H]
	\centering
	\def\svgwidth{\columnwidth}
	\input{sphere_charts_maps.pdf_tex}
\end{figure}
\end{frame}

\begin{frame}{}
\begin{figure}[H]
	\centering
	\def\svgwidth{0.75\columnwidth}
	\input{sphere_curve.pdf_tex}
\end{figure}
\end{frame}

\begin{frame}{}
\begin{figure}[H]
	\centering
	\def\svgwidth{0.75\columnwidth}
	\input{sphere_curve_tangent.pdf_tex}
\end{figure}
\end{frame}

\begin{frame}{}
	\vfill
	A \boldgreen{topological space} $M$ is a set of points with a collection of subsets $\opens$ that we define to be open. These sets satisfy \pause
	\begin{itemize}
		\item $\emptyset,M \in \opens$;
		\pause
		\item Arbitrary unions of open sets are open;
		\pause
		\item Finite intersections of open sets are open.
	\end{itemize}
	\vfill
\end{frame}

\begin{frame}{}
	\vfill
Just do diffeomorphisms and forget this
	\begin{itemize}
		\item A \boldgreen{homeomorphism} is a continuous bijection $f\colon M \to N$ with continuous inverse $f^{-1}$.  
		\pause
		\item We say $M$ and $N$ are homeomorphic.
	\end{itemize}
	\vfill	
\end{frame}

\begin{frame}{}	
	ignore boundary 
	\vfill
	A \boldgreen{manifold} (with boundary) is a space that locally looks like $\R^n$ (or $\R^{n^+}$). 
	\pause	
	\[
	\downarrow
	\]
	A \boldgreen{manifold} $M$ (with boundary) is a topological space such that each open set in $\opens$ is homeomorphic to $\R^n$ (or $\R^{n^+}$).
	\vfill
\end{frame}

\begin{frame}{}
	\vfill
	
\end{frame}

\begin{frame}{}
	
\end{frame}

\begin{frame}{Manifolds}
	\begin{figure}[H]
    \centering
    \begin{overpic}[width=0.8\textwidth]{Figures/coordinate_charts_manifold.png}
    \put (39,61) {\Large$U_\alpha$}
    \put (70,57) {\Large$U_\beta$}
    \put (12,65) {\LARGE$M$}
    \put (20,35) {\Large$\varphi_\alpha$}
    \put (73,30) {\Large$\varphi_\beta$}
    \put (22,3) {\Large$\R^m$}
    \put (60,3) {\Large$\R^m$}
    \put (38,32) {\Large$\varphi_{\alpha \beta}$}
    \put (38,18) {\Large$\varphi_{\beta \alpha}$}
    \end{overpic}
\end{figure}
\end{frame}




\section{Applications}









\section{Conclusions}


\end{document}