%%%%%%%%%%%%%%%%%%%%%%%%%%%%%%%%%%%%%%%%%%%%%%%%%%%%%%%%%%%%%%%%%%%%%%%%%%%%%%%%%%%%
% Document data
%%%%%%%%%%%%%%%%%%%%%%%%%%%%%%%%%%%%%%%%%%%%%%%%%%%%%%%%%%%%%%%%%%%%%%%%%%%%%%%%%%%%
\documentclass[12pt]{article}
%%%%%%%%%%%%%%%%%%%%%%%%%%%%%%%%%%%%%%%%%%%%%%%%%%%%%%%%%%%%%%%%%%%%%%%%%%%%%%%%%%%%




%%%%%%%%%%%%%%%%%%%%%%%%%%%%%%%%%%%%%%%%%%%%%%%%%%%%%%%%%%%%%%%%%%%%%%%%%%%%%%%%%%%%
% Packages
%%%%%%%%%%%%%%%%%%%%%%%%%%%%%%%%%%%%%%%%%%%%%%%%%%%%%%%%%%%%%%%%%%%%%%%%%%%%%%%%%%%%
\usepackage{color, soul, xcolor} % Colored text and highlighting, respectively
\usepackage{tikz-cd} % For commutative diagrams
\usepackage{mathtools}
\usepackage{answers}
\usepackage{setspace}
\usepackage{graphicx}
\usepackage{enumerate}
\usepackage{multicol}
\usepackage{mathrsfs}
\usepackage[margin=.75in]{geometry} 
\usepackage{amsmath,amsthm,amssymb}
\usepackage{marvosym,wasysym} %fucking smileys
%%%%%%%%%%%%%%%%%%%%%%%%%%%%%%%%%%%%%%%%%%%%%%%%%%%%%%%%%%%%%%%%%%%%%%%%%%%%%%%%%%%%




%%%%%%%%%%%%%%%%%%%%%%%%%%%%%%%%%%%%%%%%%%%%%%%%%%%%%%%%%%%%%%%%%%%%%%%%%%%%%%%%%%%%
% Shortcuts
%%%%%%%%%%%%%%%%%%%%%%%%%%%%%%%%%%%%%%%%%%%%%%%%%%%%%%%%%%%%%%%%%%%%%%%%%%%%%%%%%%%%
% Number systems
\newcommand{\N}{\mathbb{N}}
\newcommand{\Z}{\mathbb{Z}}
\newcommand{\C}{\mathbb{C}}
\newcommand{\R}{\mathbb{R}}
\newcommand{\Q}{\mathbb{Q}}

% Operators/functions
\newcommand{\id}{\mathrm{Id}}
\DeclareMathOperator{\sech}{sech}
\DeclareMathOperator{\csch}{csch}
%%%%%%%%%%%%%%%%%%%%%%%%%%%%%%%%%%%%%%%%%%%%%%%%%%%%%%%%%%%%%%%%%%%%%%%%%%%%%%%%%%%%




%%%%%%%%%%%%%%%%%%%%%%%%%%%%%%%%%%%%%%%%%%%%%%%%%%%%%%%%%%%%%%%%%%%%%%%%%%%%%%%%%%%%
% Environments
%%%%%%%%%%%%%%%%%%%%%%%%%%%%%%%%%%%%%%%%%%%%%%%%%%%%%%%%%%%%%%%%%%%%%%%%%%%%%%%%%%%%
% Italic font
\newtheorem{theorem}{Theorem}[part]
\newtheorem{lemma}{Lemma}[part]
\newtheorem{corollary}{Corollary}[part]
\newtheorem{axiom}{Axiom}[part]

% Plain font
\theoremstyle{definition}
\newtheorem{definition}{Definition}[part]
\newtheorem{definitions}{Definitions}[part]
\newtheorem{example}{Example}[part]
\newtheorem{remark}{Remark}[part]
\newtheorem{solution}{Solution}[part]
\newtheorem{problem}{Problem}[part]
\newtheorem{answer}{Answer}[part]
\newtheorem{question}{Question}[part]
\newtheorem{exercise}{Exercise}[part]
%%%%%%%%%%%%%%%%%%%%%%%%%%%%%%%%%%%%%%%%%%%%%%%%%%%%%%%%%%%%%%%%%%%%%%%%%%%%%%%%%%%%
 
 
 
%%%%%%%%%%%%%%%%%%%%%%%%%%%%%%%%%%%%%%%%%%%%%%%%%%%%%%%%%%%%%%%%%%%%%%%%%%%%%%%%%%%%
% Beginning of document
%%%%%%%%%%%%%%%%%%%%%%%%%%%%%%%%%%%%%%%%%%%%%%%%%%%%%%%%%%%%%%%%%%%%%%%%%%%%%%%%%%%%
\title{Mathematical Physics Lab: Lagrangian Formalism}
\author{Colin Roberts}


\begin{document}
\maketitle

\part{Lagrangians in Classical Mechanics}

\textcolor{blue}{
\begin{itemize}
    \item Add definition for configuration space (use Arnold chapter).
    
\end{itemize}

}

\section*{Introduction}
Classical mechanics of $n$ objects is written down as a set of single differential equations
\[
\Dot{\mathbf{p}}\coloneqq \sum_{i=1}^n \frac{d\mathbf{p}_i}{dt}e_i=\mathbf{F}_i(\mathbf{x}_i,t),
\]
\textcolor{blue}{Consider writing $\dot{p}$ as a 1 form}
where $\mathbf{p}_i(t)$ is the \emph{momentum} of the $i^\textrm{th}$ particle, $\mathbf{x}_i$ is the position of the $i^\textrm{th}$ particle, and $\mathbf{F}_i(\mathbf{x}_i,t)$ is the \emph{force} applied to the $i^\textrm{th}$ particle at the time $t$ and a given position $\mathbf{x}_i$.  

\begin{remark}
For now, assume that $\mathbf{x}_i \in \R^3$ for subtle reasons we will soon get to. 
\end{remark}

Recall that 
\[
\mathbf{p}=m\mathbf{v},
\]
where $m$ is the \emph{mass}, $\mathbf{v}=\frac{d\mathbf{x}}{dt}$ is the \emph{velocity}, and $\mathbf{x}$ is the \emph{position}. When mass is constant (which is the case for many systems) and we consider only a single object, we can write:
\[
m\Ddot{\mathbf{x}}=\mathbf{F}(\mathbf{x},t).
\]
In other words, we have the usual
\[
F=ma
\]
that we have all seen from an introductory physics course.

In many systems, we have a more structure.  One that we can impose is that the force applied is \emph{conservative}.  

\begin{definition}
Recall the \textbf{work} $\int_\gamma \mathbf{F}\cdot d\mathbf{l}$, where $\gamma$ is a path in a region $\Omega \subset \R^3$. Intuitively, this is the force that is working in the direction the object is moving.  It represents a change in the \emph{kinetic energy} of the object.
\end{definition}

\begin{definition}
A force $\mathbf{F}(\mathbf{x},t)$ is \textbf{conservative} if the work is path independent.
\end{definition}

\begin{theorem}[Existence of a Scalar Potential Field]
If $\mathbf{F}$ is conservative, then $\exists V(\mathbf{x})$ satisfying 
\[
\nabla V(\mathbf{x})=\mathbf{F}(\mathbf{x}).
\]
In other words, in $\R^3$, we have the following statements equivalent to the above:
\begin{itemize}
    \item $\mathbf{F}$ is not dependent on time.  That is, $F(\mathbf{x},t)=F(\mathbf{x}).$
    \item $\mathbf{F}$ is \emph{curl free}. That is to say, 
    \[
    \nabla \times \mathbf{F}(\mathbf{x})=0.
    \]
\end{itemize}
\end{theorem}

\begin{proof}
Clearly if $\mathbf{F}=\nabla V(\mathbf{x})$ then we have that $\mathbf{F}$ is not time dependent and that $\nabla \times \mathbf{F}=\nabla \times (\nabla V)=0$.   

Now, we have that stokes theorem implies that for a closed curve $C$ and a surface $S$ bounded by the curve $C$ that
\[
\oint_C \mathbf{F}\cdot d\mathbf{l} = \int_S \nabla \times \mathbf{F}\cdot d\mathbf{S}=0.
\]
This means that $\nabla \times \mathbf{F}\equiv 0$ on the region $\Omega$. This can be extended to a global function by patching together regions.

We then define:
\[
V(\mathbf{x})=\int_0^\mathbf{x} \mathbf{F}\cdot d\mathbf{l},
\]
which does not depend on a path taken.  Instead, $V$ only takes the distance between points into consideration.
\end{proof}

\begin{remark}
The fact that a curl free force corresponds to a scalar potential field relies on the fact that $\R^3$ is simply connected.  This can all be made even more formal using de Rham cohomology and exterior differentiation as opposed to a vector formulation. I'm not going into this for this talk.
\end{remark}

\begin{example}
Ignoring masses and constants (choose units where they disapear) we can write that the Newtonian gravitational force field on a mass located at the origin is 
\[
\mathbf{F}(\mathbf{x})=\frac{1}{\|\mathbf{x}\|^2}.
\]
Make the change of coordinates to spherical coordinates so that
\[
\mathbf{F}(\mathbf{x})=\mathbf{F}(r, \theta, \phi)=\frac{1}{r^2}.
\]
Then, in these coordinates, 
\[
\nabla \times \mathbf{F}=0,
\]
and clearly $\mathbf{F}$ is independent of time as well as the angular variables.  So we define
\begin{align*}
V(r)&=\int_0^r \frac{1}{r^2}dr\\
&= \frac{-1}{r}.
\end{align*}
\end{example}

\begin{problem}
Recall $\nabla$ in spherical coordinates and verify $\nabla \times \mathbf{F}=0$.
\end{problem}

\section{The Lagrangian Framework}
The motivation for the Lagrangian framework is to extend classical mechanics to more general spaces.  For example, this formulation allows one to look at classical mechanics in $\R^n$, or (curved) spacetime, or for one to work with fields instead of particles.  Classical mechanics itself does not generalize as nicely to quantum mechanics compared to the Lagrangian formulation.

In calculus, you can optimize functions over their domain by investigating their derivatives.  The main analogous result we will care about is the following.

\begin{theorem}[First Derivative Test for Extrema]
Let $f\colon \Omega \subset \R^n \to \R$ be sufficiently smooth. Then if $f$ attains an extreme value at $x_0$, $\nabla f(x_0)=0.$
\end{theorem}

Let us build a new framework.  

\begin{definition}[Lagrangian of a System]
The \textbf{Lagrangian} of a system of $n$ objects is
\[
\mathcal{L}(\mathbf{x}_1,\dots,\mathbf{x}_n,\mathbf{p}_1,\dots,\mathbf{p}_n,t)\coloneqq T(\mathbf{p}_1,\dots,\mathbf{p}_n)-V(\mathbf{x}_1,\dots,\mathbf{x}_n)
\]
where $T$ is the kinetic energy and $V$ is the potential (energy). We can also swap $\mathbf{p}$ for $\Dot{\mathbf{x}}$
\end{definition}

\begin{definition}[Action (functional) of a Particle]
The \textbf{action} of a particle following a path $\gamma \colon (t_0,t_f) \colon \R^n$ is
\[
s[\mathbf{\gamma}]\coloneqq \int_{t_0}^{t_f} \mathcal{L}(\mathbf{\gamma},\mathbf{\Dot{\gamma}},t)dt
\]
\end{definition}

\begin{remark}
You can think of action as the amount of energy cost an object encounters when traversing a path. 
\end{remark}

\begin{axiom}[Least Action Principle]
The trajectory of an object minimizes the action.
\end{axiom}

The following theorem is an analogous result to the necessary conditions for optimization in Calc 1.

\begin{theorem}[Euler-Lagrange Equations]
Choosing generalized coordinates for a object $\mathbf{q}(t)$ (position) and $\mathbf{\Dot{q}}(t)$ (velocity), an object trajectory minimizing the action must satisfy the Euler-Lagrange equation
\[
\frac{\partial\mathcal{L}}{\partial \mathbf{q}}-\frac{d}{dt}\frac{\partial \mathcal{L}}{\partial \Dot{\mathbf{q}}}=0.
\]
\end{theorem}

The next bit of these notes will be concerned with building up the proof to this statement.

\textcolor{blue}{Give enough info to prove this.}


\section{Applications}
\textcolor{blue}{Noether's theorem, an example with electromagnetic fields.}

\section{Problems}
\textcolor{blue}{Lagrangian for $2$ masses and $3$ springs. Some moderately complicated system that would suck to do using force diagrams.}

\end{document}