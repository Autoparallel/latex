\documentclass{article}
\usepackage[utf8]{inputenc}
\usepackage[left=3cm,right=3cm,top=3cm,bottom=3cm]{geometry}

%\title{IMA Personal Statement}
%\author{Colin Roberts}
%\date{February 2019}
\linespread{1.5}

\begin{document}

%\maketitle

%\begin{itemize}
%    \item Interdisciplanary
%    \item Calculus of variations and PDEs
%    \item Mechanics, electromagnetism, (topological) fluid dynamics, physics
%    \item Analysis, geometry, topology
%\end{itemize}

\noindent \textbf{Biology, Analysis, Geometry, Energies, Links [bagel19]: A Program on Low-dimensional Topology, Geometry, and Applications}\\

\noindent Dear organizers,\\

I am currently in my second year of a mathematics Ph.D program at Colorado State University (CSU) working with Clayton Shonkwiler. My interests are in differential geometry, functional analysis, and partial differential equations.  Over the last two years, I have taken two courses in PDEs, a course in variational methods, and have spent much of my other time studying differential geometry.  As an undergraduate, I was a double major in mathematics and physics.  Most of my interests in mathematics are topics that are motivated by physics.

While in graduate school, I have learned about how functional analysis is used in infinite dimensional optimization and the analysis of PDEs.  I think that the study of knot energies sounds fascinating and seems to combine these topics nicely.  The topological fluid dynamics, geometric curvature energies, and discretization of geometric flows all sound interesting based on my interests in geometry and topology.  As I have been reading through papers and books, I have found the areas of topological fluid dynamics seems especially enticing. Basically, anything from Arnol'd has captivated me! Recently, I have also stumbled upon a question about a geometric flow that seeks to minimize some type of ``width functional" of planar curves, where the minimizers would be curves of constant widths.  This has required me to start learning new material on discretization in order to simulate this. It has been fun challenge and would be very happy to learn more about these techniques.


Attending this session would give me the opportunity to see the current state of research in the very fields that I am interested in as well as introduce me to some new material that I have yet to explore.  This exposure would help me find specific topics I would like to work on for a Ph.D thesis.  Getting to meet the active researchers and other graduate students interested in the same fields would be great since it is mostly Clayton with interests in these problems at CSU.\\


\noindent Best,

\noindent Colin Roberts

\end{document}
