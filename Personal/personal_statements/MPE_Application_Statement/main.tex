\documentclass{article}
\usepackage[utf8]{inputenc}

\title{MPE20 Application Statement}
\author{Colin Roberts}
\date{}

\begin{document}

\maketitle

In July of 2019, I joined the Mathematics and Climate Research Network (MCRN) group in a two week workshop organized by Christopher Jones. I heard about this workshop via SIAM since, as a member, I receive these helpful notifications! Together as a large group, we learned about mathematics related to the Earth's climate and weather.  At the end of the two week session, we broke into smaller groups where we would separate to go work on a specific concentration.  Our smaller group consists of Erik Van Vleck, Sarah Iams, Noah Marshall, Rose Crocker, Juniper Glass-Klaiber, and Aishah Albarakati and our interests are in dimension reduction and data assimilation. Christopher Jones will be organizing a mini symposium on \emph{Dynamics and Data in Models of Climate Processes} where our group will give a talk titled \emph{Model and Data Reduction Techniques for Data Assimilation}.

As of now, I am in my third year of a Ph.D. program at Colorado State University.  My personal interests revolve around mathematical physics, Riemannian geometry, and applications to real world problems. I absolutely love problems motivated by physics and approaching them with an abstract mathematical toolbox. Specifically, I have found that the questions related to both weather and climate bring about some beautifully interesting mathematics while of course being extremely pertinent for society as a whole.  Being able to engage in more research about our planet would benefit me greatly as I will be able to discover other physical problems that are currently being studied. By knowing other research areas, I can broaden my view of applied mathematics which could help spark Ph.D. research or generate ideas for me to consider after earning my degree.

I strongly believe that mathematicians have the ability to make a large impact in understanding our world and thus we can push for the the changes that are necessary for preserving our planet.  Joining this conference and being surrounded with others with similar goals allows us all to make real progress.  There, I will be able to meet new colleagues who are interested in the mathematics of planet Earth.  Some may be able to provide me guidance in what I may be able to do in the future or they could be future collaborators.  At the very least, it gives me chance to see the friendly faces that make up this community! I am excited to be an active participant in this conference and share the new techniques our group is developing.


\end{document}
