\documentclass{article}
\usepackage[utf8]{inputenc}
\usepackage[left=2.5cm,right=2.5cm,top=2.5cm,bottom=2.5cm]{geometry}

%\title{IMA Personal Statement}
%\author{Colin Roberts}
%\date{February 2019}
\linespread{1.05}

\begin{document}

%\maketitle

%\begin{itemize}
%    \item Interdisciplanary
%    \item Calculus of variations and PDEs
%    \item Mechanics, electromagnetism, (topological) fluid dynamics, physics
%    \item Analysis, geometry, topology
%\end{itemize}

%\noindent \textbf{Los Alamos Machine Learning:}\\

\noindent To those whom this may concern,

%reorder paragraphs? Better introduction and conclusion?

I am currently in my fourth year of a mathematics Ph.D program at Colorado State University where I also earned a bachelor's degree in general mathematics and physics. My current thesis project is in applying techniques from Clifford analysis and geometric algebra towards physical and geometrical inverse problems such as the Electrical Impedance Tomography problem. When I saw this project announced, I became very keen on exploring the topics in Focus 2 (physics-informed machine learning). Not only is it abundantly clear that machine learning is a widely applicable area of ongoing research, but Focus 2 concentrates on improving the real world simulations as well as developing techniques to integrate constraints from symmetries and conservation laws; both of which I have relevant experience and knowledge in. Similarly, this will all require a strong background in linear and multilinear algebra which I have gathered over many years through taking and teaching courses as well as through my own research.

Since the focus of my thesis is in pure mathematics, I look for other research opportunities in applied math to broaden my horizons. This also gives me a chance to provide a unique perspective to those outside of my fields of interest. During the summers of 2019 and 2020 I participated with the American Institute of Mathematics and the Mathematics Climate Research Network on projects in Data Assimilation (DA) and COVID-19 modeling. The DA project has continued since 2019 and has led our group to submitting a paper. DA is a technique fundamentally used to improve the accuracy of a mathematical model using real world data. A Bayesian DA scheme compares state of a dynamical system to data in order to generate a posterior probability distribution that informs the modeler on how to more optimally update the state of the system. Our group chose to address a major challenge in DA by developing a technique that is usable for high dimensional nonlinear problems (e.g., discretized PDEs) without requiring a Gaussian assumption on the posterior distribution. Our contribution was to combine various dimension reduction techniques (i.e., techniques derived from principal component analysis) with particle filters which allowed us to successfully perform assimilation with the shallow water equations on a fine mesh with sparse measurement. Through the success of this project, I have been able to learn from other members and develop my mathematical knowledge as well as my abilities to collaborate, program, and write. Happily, we plan to continue working together in related projects.

The flow of knowledge between physics and mathematics is awe inspiring to me as I find new insights in physics from geometry and vice-versa. This love has led me to my thesis topic in PDEs and Riemannian geometry. Ultimately I have found geometric algebra and Clifford analysis to serve as a unified language in their simple descriptions of physics. Those fields build upon (multi)linear algebra and pour out rich geometric interpretations of standard linear theory.  With geometric algebra, one can faithfully realize all symmetry groups and, for example, find that symmetries of electromagnetism are realized neatly inside Pauli's spacetime algebra. Further, geometric algebra itself provides a wonderfully intuitive and succinct framework for studying the geometry of vector spaces with inner products of arbitrary signature. This framework also has computational benefits for performing vector space operations since in many cases it removes the need of matrix computations (e.g., using quaternions to describe rotations). Clifford analysis is built by strapping smooth structures to geometric algebras and it is also a refinement of harmonic analysis and differential forms. In this setting, we find additional powerful results which can lead to novel solutions to old problems. As a proponent of these simply stated and computationally cheap theories, I firmly believe that they can improve physics based machine learning techniques by reducing total complexity and providing new insight. Of course, I am also excited to share this knowledge with others!

At the moment my experience with machine learning is minimal but I am extremely confident that my mathematical background will prove ample. Moreover, my skills as an instructor will allow me to better communicate and share my knowledge with colleagues. As I near graduation, I will be preferentially searching for job positions in the technology sector where experts in machine learning are strongly desirable and this fellowship serves as concrete evidence that my mathematical abilities are applicable in industry. As well, my network of friends and colleagues with aligned interests will grow which puts me in a better position to find a vocation one I graduate. I truly believe this experience will prove to be invaluable and I will graciously accept if I am given the chance. Thank you for your time!\\


\noindent -Colin Roberts

\end{document}

Returning to the previous example, Maxwell's equations distill into a single Dirac equation where the aforementioned symmetries arise.
